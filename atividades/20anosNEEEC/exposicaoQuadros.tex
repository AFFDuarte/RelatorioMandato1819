% ============================================
% # Exposição dos Carros da Queima das Fitas #
% ============================================

\subsection{Exposição dos Carros da Queima das Fitas}

No corredor do piso 2, desde a torre T até à Torre R, estava uma exposição de algumas fotos de um concurso de fotografia que teve lugar há uns anos atrás.

Tais fotos não tinham grande interesse nem se relacionavam de nenhuma forma com o Departamento e curso em si. Ao ler a Bíblia do \acrshort{neeec}, anterior a esta, descobrimos que uma das ideias do mandato desse ano era fazer uma exposição dos carros da Queima das Fitas do nosso curso.

Ao sabermos disto fomos imediatamente falar com o professor Humberto, Diretor do \acrshort{deec}, para que nos autorizasse a substituir a exposição que lá estava pelas fotos dos carros. O professor adorou a ideia pelo que pusemos mãos à obra.
Demorou um pouco mais do que esperávamos, uma vez que tínhamos de fazer um pouco de trabalho de investigação, ao tentar descobrir membros dos carros mais antigos e pedir-lhes as fotos em boa qualidade, tendo sido possível ir até 2011. 

Após a recolha de todos os momentos fotográficos enviámos tudo por e-mail para a reprografia do DARQ (Departamento de Arquitetura). No entretanto fomos comprar quadros ao Leroy Merlin, despesa essa que foi dividida 50-50 pelo Departamento e NEEEC.
A colocação de todos os quadros foi algo que deu algum trabalho, mas após algumas horas ficou tudo pronto a tempo e horas da inauguração. E assim, no dia 3 de abril de 2018, foi inaugurada essa exposição, com a presença do Diretor do Departamento, Humberto Jorge, e do Presidente da \acrshort{dg}, Alexandre Amado.

Dias depois da exposição ter sido inaugurada, e após fazer sucesso, várias pessoas dos antigos carros queriam as fotos mudadas, ou porque não apareciam todos os membros desse carro na foto ou porque alguém tinha ficado feio. Assim, tivemos que imprimir as novas fotos e recoloca-las de novo nos quadros. 
Por fim, achamos que foi uma boa maneira de homenagear os nossos estudantes e ao mesmo tempo dinamizar o nosso Departamento, com algo novo e diferente.

\subsection{Inauguração dos Espaços de Estudo}

Com as remodelações dos vários espaços de estudo realizadas no presente mandato, faltava a sua devida inauguração. Sendo estas remodelações um marco naquilo que são as principais missões do \acrshort{neeec} pareceu-nos bastante adequado aproveitar a existência desta celebração para, a ela, associar a inauguração destes espaços contando com a presença do Diretor do Departamento, Humberto Jorge, e do Presidente da \acrshort{dg}, Alexandre Amado.