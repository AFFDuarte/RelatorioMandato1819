% ================================
% # Gala Ohms D'Ouro             #
% ================================

\subsection{Gala Ohms D'Ouro}

A VI edição da Gala Ohms d’Ouro decorreu no dia 23 de março de 2018. Nesta edição atribuímos um tema à gala pois o Núcleo de estudantes comemorava o seu vigésimo aniversário. Sendo este o tema do ano, durante a gala houve vários momentos dedicados ao mesmo. 

A gala decorreu na Tertúlia de Eventos, tal como no ano transato, local este que foi ideal para o tema escolhido, pois era um espaço bastante acolhedor, e que proporcionou as devidas decorações para o tema. 

Querendo manter a seriedade do tema e um pouco de humor tivemos como apresentadores o José Pedro Pereira e o Tiago Baltazar, que conseguiram desempenhar a sua função na perfeição. 

Por fim, esta edição contou com uma after-party no local do evento, com um pack de bebidas que podia ser comprado previamente ou durante a gala.

Para a criação deste evento foi criada uma comissão organizadora composta pelas seguintes áreas de trabalho: coordenação, imagem, divulgação, prémios, site, história, protocolo, logística, decorações, tesouraria e secretaria. Adicionalmente, mais perto do evento foram adicionadas as áreas do jornal e da apresentação. Todas estas áreas ficaram com um ou dois responsáveis nomeados com exceção da história que tiveram três responsáveis.

\subsubsection{Coordenação}
Os Coordenadores deste evento foram o Presidente do Núcleo, João Bento, e a Coordenadora das Saídas Profissionais, Vânia Silva. Uma vez que a equipa esteve bastante bem dividida, a função dos Coordenadores passou, essencialmente por verificar se tudo estava bem orientado, gerir as reuniões da Gala e tapar algum buraco quando necessário. Entre os dois Coordenadores houve também uma divisão das áreas de atuação ficando, por exemplo, a Vânia mais responsável por toda a parte de decoração e o João responsável por toda a parte da história do Núcleo. Adicionalmente a coordenação ficou também responsável pelas encomendas de lonas e prémios, algo que, no entanto, poderia ter sido atribuída, sem qualquer problema, ao responsável da logística.

\subsubsection{Imagem}
O Coordenador de Imagem do \acrshort{neeec}, Moisés Dias, nomeou um membro do seu Pelouro para ficar inteiramente responsável pela imagem da gala, Marco Silva. Esta divisão foi excelente tendo permitido que a imagem da gala estivesse sempre pronta a horas. O Marco contou ainda com a ajuda do João Ferreira nas alturas de maior trabalho, principalmente, nos dias mais próximos à gala. A imagem foi responsável por criar o design dos prémios, o logótipo dos 20 anos, o design para as lonas da celebração do aniversário, toda a imagem da gala (cartazes, publicações na internet, entre muitos outros) e o design para envelopes, nomes das mesas, lista de pessoas por mesas, etc.

No futuro recomendamos vivamente a criação do símbolo da gala em esferovite (3D) que poderá estar em exibição à entrada do Departamento e no palco da gala.

\subsubsection{Divulgação}
Foi nomeado um membro do Núcleo, André Duarte, para ficar responsável por toda o plano de divulgação da gala. O André, trabalhou sempre em colaboração com a imagem e com a coordenação e era responsável por todas as publicações nas redes sociais. Esta foi uma aposta extremamente ganha, tendo permitido resultados excelentes na divulgação da gala e, apesar de ser uma pessoa com poucas tarefas a fazer na organização da gala, é um cargo que recomendamos a sua manutenção, sem acumulação de cargos.

\subsubsection{Prémios}
Este ano foi nomeado um responsável pelos prémios mas esta é uma área que poderá estar adjudicada ao responsável pela logística. Para esta área foi necessário fazer os boletins de votos, contactar os nomeados (algo que foi feito, e corretamente, pelo responsável do protocolo), decidir a imagem e a quem encomendar os prémios (algo que pode ser feito pelo responsável da logística) e contar os votos, algo que teve se ser feito, segundo o regulamento, por um aluno, um professor e um funcionário não nomeados, pelo que também poderá ser algo gerido pela logística.
Por fim, o responsável da logística, Zé Pedro, criou o novo regulamento de atribuição de prémios, documento esse que permitiu uma clarificação verdadeira dos prémios a atribuir e uma transparência total dos votos contabilizados, após várias suspeitas em relação a resultados de edições anteriores. Uma das alterações substanciais deste regulamento foi a definição de que quem venceu os prémios nos ano anterior não o poderia ganhar este ano. Este regulamento foi importantíssimo para o bom funcionamento da gala e recomendamos a continuação da sua utilização no futuro, lutando sempre por uma publicação cada vez mais pública dos resultados.

\subsubsection{Site}
O site da gala era um problema já antigo, estando o mesmo por concluir há vários anos. Assim, foram nomeados dois responsáveis pelo site da gala, João Dinis Sanches Ferreira e Elvis Borges, que construiram, de raiz, um novo site para a gala (neeec.pt/ohmsdouro). De forma a que este site fosse útil introduziram-se várias informações importantes nele como a história das galas anteriores e os nomeados, vencedores e resultados detalhados da presente edição. Um problema grave que este site costuma ter é que o mesmo acaba por não estar pronto aquando do lançamento das informações. É importante perceber que o público só visita o site, em regra, uma vez pelo que é essencial que o mesmo já esteja disponível o mais cedo possível com todas as informações. Também as inscrições foram feitas através deste site, o que foi muito bom para uma imagem coesa do evento.

A equipa do site queixou-se também, com bastante razão, do facto de estarem muito dependentes de outras equipas para obter as informações a meter no site o que provocou alguns atrasos.

Este site encontra-se agora, finalmente, online e recomendamos vivamente que organizações futuras continuem a apostar na existência do mesmo. Aliado ao facto de este apresentar alguma qualidade, não negando a necessidade de algumas melhorias, recomendamos que este continue a ser utilizado e atualizado. Caso achem que o mesmo deva ser reformulado, recomendamos vivamente que mantenham o atual site ativo e atualizado até disporem de uma versão definitiva pronta a colocar online, com todas as informações constantes do mesmo.

\subsubsection{História}
Dada a necessidade de fazer o site da gala e a celebração do 20º aniversário do Núcleo foi criada uma equipa exclusivamente dedicada a explorar a história do Núcleo e das edições anteriores da gala. Esta equipa era composta pela Ana Calhau, pelo Guilherme Roque e pela Elisabete Santos.

No que toca à história da gala, cada Coordenador de cada edição da gala emitiu um texto sobre cada edição, que foi colocado no site. Foi também possível recuperar os vencedores de todas as edições mas esta informação nunca chegou a ser publicada, algo que recomendamos vivamente a ser feito no futuro. Por fim, foram recuperadas as fotos da primeira edição, que foram divulgadas no Facebook da gala.

No que toca à história do \acrshort{neeec}, a mesma encontra-se descrita no resto do presente capítulo sobre os 20 anos do Núcleo.

\subsubsection{Protocolo e Patrocínios}
Esta área foi criada pela primeira vez este ano e teve como responsável o César Pereira. O principal objetivo era criar um responsável por enviar todos os convites, notificar os nomeados, convidar professores e empresas, entregar convites quer aos alunos, quer aos professores entre outros. Esta ação foi um sucesso, tendo sido centrada numa só pessoa, ações que anteriormente teriam de se responsabilidade da coordenação.

Adicionalmente, o César foi também responsável pela área dos patrocínios, tarefa executada em dezembro, janeiro e fevereiro, tendo o protocolo tido mais trabalho em fevereiro e março. Estas duas funções funcionaram de forma extremamente positiva. No futuro aconselhamos a uma manutenção dos patrocínios, que este ano foram de 150€ + IVA e consistiram na divulgação das empresas no facebook da gala e no press conference da mesma, e à elaboração de pacotes que promovam mais receitas para a gala de forma a compensar o prejuízo da mesma, garantindo sempre que não se perde a identidade da gala que não se baseia, de todo, em patrocínios. Algo importante seria também fazer com que as empresas se envolvessem na gala, estando presentes na mesma, algo que este ano, mesmo com a oferta de entradas à mesma, não resultou.

\subsubsection{Logística}

O responsável pela logística da gala foi o Steve Sintra. Este é, sem dúvida, o Pelouro mais importante da gala, tendo este sido o responsável pela reserva do espaço, pela estrutura de toda a gala, pelas ofertas, já tradicionais, como os charutos e as rosas dados no final da gala, pela compra de material diverso, pela montagem de tudo no dia da gala e pela resolução de problemas ao longo da mesma. Adicionalmente, o Steve foi responsável pela contagem dos votos tendo apresentado uma atitude excelente de respeito para com os resultados. Assim, só ele, o professor Peixoto e a Aurora Gaspar sabiam os vencedores da gala pelo que todas as entregas de prémios foram surpresa para todos na gala. Durante a gala, houve um pequeno problema com o som dos microfones pelo que o Cristiano Alves contactou rapidamente um amigo que, com a ajuda do Steve, rapidamente colocaram colunas mais potentes dentro da sala sem que quase ninguém reparasse no que se estava a passar. Contudo, este é um ponto a ter em conta no futuro, principalmente para que as colunas estejam bem colocadas e para que ninguém mexa nas configurações feitas na mesa de som.

\subsubsection{Decorações}

As decorações são uma parte importante da montagem da gala para que, com baixos custos, a gala apresente um elevado ambiente de glamour. Assim, este ano, os responsáveis por esta área foram a Vânia Silva e a Elisabete Santos. Foi adquirida fita dourada o que, em conjunto, com os materiais disponibilizados pela Tertúlia de Eventos bem como com os cartões a enumerar os nomes das mesas e quem se deve sentar na mesma, criaram um ambiente bastante elegante e adequado às cores da gala. A imagem teve também um papel preponderante na decoração da gala uma vez que o press conference foi feito por essa equipa e foi essencial, uma vez que se encaixava numa área grande de elevada visibilidade apresentando uma imagem muito bonita.

\subsubsection{Jornal}
Com o aproximar da data celebrativa, surgiu uma ideia, por parte do Vice-Presidente do \acrshort{neeec}, João Martins, de se criar um jornal. Entretanto, o João Bento sugeriu a criação de um jornal dedicado em exclusivo aos 20 anos da gala pelo que o João Martins quis avançar com a ideia. Desta forma, pegando nas informações fornecidas pela equipa da história, criou um texto sobre toda a história do Núcleo bem como, em conjunto com o Marco Silva, criou a imagem do jornal. Foi também pedido ao João Bento um editorial para o jornal e o Ivo Frazão fez a revisão final do texto. Contudo, a revisão final do jornal acabou por ser feita de forma muito tardia (na madrugada anterior ao dia da gala) pelo que durante o dia da gala foi necessário estar-se a imprimir e agrafar os jornais, tarefa que só se conseguiu concluir com a ajuda do Diretor do Departamento, professor Humberto Jorge, já perto das 20 horas.

\subsubsection{Apresentação}
Ainda em janeiro foram escolhidos os apresentadores da gala: José Pedro Silva, apresentador repetente do ano anterior e Tiago Baltazar. Esta escolha teve como objetivo garantir um ambiente divertido à gala não perdendo o foco de glamour da mesma. Os apresentadores estiveram mais que à altura do desafio e o facto de terem sido escolhidos tão cedo permitiu uma preparação completa dos guiões de apresentação. Notou-se também que, como é normal pela experiência, o Zé Pedro se apresentou muito mais à vontade na apresentação da gala.

Uma coisa importante no guião da gala deste ano é que voltámos a um modelo intermédio entre o anterior e o de há dois anos. Assim, as entradas foram servidas em ambiente de boas-vindas e, após as pessoas se sentarem, ocorreu o discurso de boas vindas do Presidente do \acrshort{neeec} e do Diretor do \acrshort{deec}. De seguida, foi servida a comida e seguiram-se umas palavras dos antigos dirigentes do \acrshort{neeec}, enquanto eram trocados os pratos principais pelas sobremesas. Após o serviço das sobremesas procedeu-se à entrega dos vários prémios, intercalados com a entrega dos prémios honorários aos bombeiros, dos prémios dos melhores alunos, do prémio honorário, do prémio especial 20 anos ao professor Humberto Jorge e de umas palavras do Vice-Presidente da \acrlong{dg} da Associação Académica de Coimbra.

\subsubsection{Tesouraria e Secretaria}

O Tesoureiro do \acrshort{neeec}, Ivo Frazão, foi também o Tesoureiro deste evento tendo sido o responsável, em conjunto com o Secretário do \acrshort{neeec}, Miguel Antunes, por todas as inscrições, emissões e receções de faturas às empresas e aos participantes da gala, entre outros. Ambos foram também responsáveis por fazer as atas das reuniões da Comissão Organizadora da Gala, algo que consideramos essencial para que seja mantida toda a informação. De ressalvar que foi criado um formulário de inscrições embutido no site da gala que, com a ajuda do Google Scripts, apresentou vários automatismos que facilitaram bastante quer as inscrições quer as informações enviadas aos participantes. Recomendamos vivamente a reutilização deste formulário no futuro. Os pagamentos foram todos feitos por transferência bancária algo que só apresentou duas reclamações e que facilitou imenso o trabalho de secretaria, nomeadamente o pagamento do jantar bem como a declaração de contas.

\subsubsection{Considerações Finais}

Na nossa opinião, a qualidade desta gala cresceu imenso. O espaço utilizado para o evento apresentou todas as necessidades para termos um evento de elevada categoria e a organização da equipa proporcionou um evento muito bom. Existem, no entanto, algumas coisas que podem ser sempre melhoradas:
\begin{itemize}
\item Apesar do aumento e melhoria da divulgação, a gala não teve um aumento do número de inscritos, quando comparado com a gala do ano anterior. Comparando com outras edições, nomeadamente a terceira edição em que chegou a haver 130 inscritos, parece-nos que deve ser investido bastante trabalho no aumento do número de participantes.
\item Este ano contratámos uma fotógrafa à parte, algo que nos parece ter sido a escolha mais acertada pois é possível exigir um trabalho mais profissional. Contudo, esta decisão acarreta custos mais elevados. Apesar do valor pago à fotografa ter sido baixo (tendo em conta os valores de fotógrafos no geral) a qualidade das fotos ficou muito aquém da qualidade desejada de um profissional na área pelo que no futuro recomendamos mais atenção a esta situação. De notar que, apesar de tudo, as fotos deste ano são as melhores de todas as edições da gala existentes até ao momento.
\item Dados os atrasos, principalmente com a realização do jornal, a Direção do \acrshort{neeec} acabou por chegar já muito tarde ao evento, tendo sido o Miguel Antunes, Secretário do \acrshort{neeec}, o primeiro a chegar. Apesar de nos terem informado que tal não era costume, houve várias pessoas a chegar à hora marcada (19h30), ou até antes, pelo que é imperativo que já exista alguém do Núcleo disponível para receber as pessoas a essa hora.
\item Nesta edição da gala, vários pormenores foram pensados ao detalhe o que deu uma imagem muito positiva: os certificados enrolados para os melhores alunos que não se encontravam em fim de ciclo, o jornal em cada lugar, o plano de divulgação, a montagem das lonas, a entrega dos charutos e rosas em tabuleiros, entre outros. No futuro, recomendamos uma manutenção destes detalhes que tornam o evento perfeito.
\item No site ficam a faltar as imagens das galas, informação sobre a gala de 2018 (o texto está feito) e os vencedores das edições anteriores. Seria interessante inserir esta informação.
\item Os patrocínios devem ser aumentados em futuras edições para permitir uma gala mais sustentável. Na nossa opinião, deve ser também promovida uma presença das empresas no dia da gala (não para se promoverem mas sim para interagirem com a comunidade).
\item O convite de antigos dirigentes trouxe um ambiente interessante para a gala. No futuro recomendamos vivamente que o convite a ex dirigentes do Núcleo seja bastante forte de forma a conseguir ter na gala uma comunidade grande do Núcleo.
\item Este ano o serviço foi em buffet em vez de ser servido à mesa. No geral, parece ter sido uma mudança muito positiva permitindo, pelo mesmo preço, haver vários pratos.
\item No final da gala foi enviado um questionário de satisfação que permitiu obter várias opiniões sobre a gala. Aconselhamos vivamente a análise deste inquérito aquando do início da organização da próxima edição.
\item Alguns professores gostam bastante de estar presentes neste evento mas confirmam a sua presença após algumas visitas pessoais aos seus gabinetes e demasiado em cima da hora. É muito importante arranjar forma de evitar isto pois tal provoca vários problemas na logística da gala.
\item A gala tem-se realizado na última sexta-feira de aulas, antes da Páscoa. Contudo, na nossa opinião, esta data deve ser alterada uma vez que há muitas pessoas (quer alunos, quer professores) que nessa sexta-feira não estão em Coimbra pois vão de férias ou têm compromissos. Outra coisa que também achamos importante é ter atenção à data do VIII Badaladas, organizado pela Quantunna, evento que se tem realizado sobreposto com a Gala nos últimos dois anos.
\item Apesar da gala ter dado prejuízo, de referir que a qualidade da mesma é a base do evento (o que se provou quando se tentou diminuir a qualidade da mesma em 2016 e, a partir desse ano, as pessoas deixaram de ter interesse em vir à Gala, sendo agora muito difícil recuperar a imagem da mesma). Assim, entendemos que poderão ter de ser aplicadas algumas poupanças, por exemplo verificando se é possível baixar o preço da refeição servindo menos comida, quer para diminuir os custos, quer para diminuir o PVP aplicado aos estudantes, mas deve-se sempre ter em atenção a qualidade da gala.
\item O bar aberto é um assunto a estudar dado que o modelo aplicado este ano voltou a não resultar. Deve-se analisar se, de facto, existe algum problema em o preço já incluir bar aberto para todos os participantes.
\item A existência de uma after-party num dado bar como o NB, por exemplo, foi algo que foi excluído nesta edição mas que consideramos importante existir pois apesar de, havendo after-party no local, haver poucas pessoas que vão para a discoteca, é importante haver um local oficial para ir após a gala, mantendo o grupo unido.
\end{itemize}