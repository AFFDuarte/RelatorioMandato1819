% =================================
% # Arranjo dos Espaços de Estudo #
% =================================

\subsubsection{Arranjo dos Espaços de Estudo}

Ao longo dos últimos têm existido inúmeras queixas, fundamentadas, pela falta de condições nos espaços de estudo do \acrshort{deec}. Apesar de existirem vários espaços abertos 24 horas (as varandas do piso 3, a sala de estudo do piso 6 e a sala T.4.2, para além da sala T.4.3, antes da criação do \acrlong{cp}), estes espaços não dispunham de tomadas elétricas e internet e tinham o seu espaço completamente desorganizado havendo faltas de cadeiras, etc. No sentido de lutar para que houvessem espaços de estudo com qualidade para os nossos estudantes, este ano, decidimos meter mãos à obra, falar com a Direção do \acrshort{deec} e renovar completamente os espaços já existentes, pondo de parte a inércia da Manutenção do Departamento.

Começamos com o arranjo do piso 3, um espaço bastante usado durante o dia, mas que apresentava vários problemas, como falta de pontos de eletricidade e lâmpadas e cadeiras que faziam muito barulho quando arrastadas ou outras que não existiam.

Inicialmente fizemos novas montagens elétricas nas divisórias, de modo a ter um interruptor maior e mais resistente que o anterior, de forma a que não fosse fácil que este se partisse, mesmo com o uso, como acontecia com os anteriores. A seguir afixamos extensões às laterais das divisórias, ficando cada lado das mesmas com 3 pontos de eletricidade, ou seja 6 pontos por cada 4 pessoas.

Após essa instalação, mudamos todas as cadeiras de forma a uniformizar o layout do espaço e a garantir que todos os lugares dispunham de uma cadeira. No total, este espaço, ficou com 40 lugares disponíveis.

Ficou um espaço muito bom para estudar sendo, no entanto, frio durante o Inverno mas infelizmente este é um problema que não tem resolução aparente dadas as condições do edifício.

Passamos à renovação da Biblioteca do Piso 6, onde o problema era outro, completamente diferente, o barulho, aliado à desorganização.

Numa fase inicial, ainda no verão, começámos por fazer um mapa da sala e criámos uma fila de mesas junto à parede e a respetiva eletrificação, através de calha, garantindo 3 pontos de luz para cada 2 lugares. Também no verão, rompemos com o contrato da máquina de vending, tendo a mesma sido removida.

Em outubro, mudamos o layout da sala:
\begin{itemize}
\item Parte de estudo em grupo – Antes do vidro.
\item Parte de estudo individual – Após o vidro.
\end{itemize}

Na parte de estudo em grupo agrupamos pares de mesas ficando um total de 5 grupos de mesas, ou seja, 20 lugares ao todo, nesta parte, com extensões elétricas para todos.

Na parte de estudo individual encostámos uma fila de mesas à parede, de uma ponta à outra da sala, e colocando uma calha ao longo de toda essa fila, proporcionando 3 pontos de eletricidade para cada 2 pessoas. No meio da sala colocamos 2 filas de mesas frente a frente nas quais ficou a faltar a implementação da última fase do projeto.

No total ficamos com cerca de 80 lugares nesta parte totalizando 43\% mais lugares em toda a sala que anteriormente.

Passado alguns meses, em março, após a chegada das divisórias compradas pelo \acrshort{neeec} e pagas pelo \acrshort{deec}, procedemos à fase final da renovação, na qual colocamos as divisórias entre as 2 filas de mesas do centro. Depois da colocação destas afixámos extensões em cada divisória, ficando com o mesmo rácio, de tomadas por pessoa, que as mesas da parede.

Pedimos ao \acrshort{gri} para implementar “lag” e bloquear jogos na rede de Internet da biblioteca, de forma a que as pessoas usassem a sala para estudo e não para fins lúdicos (que ajudam a criar barulho) contudo, após a UGF, esta implementação caiu.

A ideia das divisórias surgiu como forma de diminuir o barulho, para que as pessoas não falassem com a pessoa em frente. Este problema melhorou de facto mas existem momentos em que continua a verificar-se barulho na sala e a solução não parece estar à vista, visto que mesmo tentando sensibilizar as pessoas a fazerem silêncio o problema persiste. Contudo, sempre que alguém manda calar os restantes a sala fica em silêncio absoluto acabando por ter menos barulho quando está cheia do que quando está "a meio".

Na sala de estudo T.4.2 tirámos as mesas cinzentas que lá estavam e pusemo-las no arrumo B1, uniformizando o design do espaço, apenas com mesas e cadeiras de madeira. Prendemos as mesas que estavam encostadas à parede à mesma e as restantes entre si. Adicionalmente, colocámos um mapa da disposição correta da sala para que sempre que haja eventos a sala seja colocada de forma correta. No total ficamos com 32 lugares nesta sala tendo a mesma sido também alvo de eletrificação e arranjos pontuais como o facto de se ter prendido o quadro e as luzes e substituídas as lâmpadas fundidas do teto.

Quer na sala T.4.2 como nas varandas do Piso 3 é muito comum os estudantes fumarem nos jardins anexos e, não tendo local onde se sentar, levam as cadeiras de madeira para a rua fazendo com que estas se estraguem e deixem de haver cadeiras em todos os lugares. De forma bem sucedida, para prevenir isto, colocámos cadeiras de plástico, próprias para rua, em todos os locais de fumo.

Em todos os espaços de estudo foram colocados avisos que incentivam ao silêncio, ao fecho das luzes, ao facto de não se dever jogar naqueles locais e publicidade à plataforma de queixas logísticas do \acrshort{neeec}. Estes avisos foram colocados em todos os locais de estudo (1 aviso para cada 2 lugares) tendo sido um sucesso, dado a sua componente mais amiga.
