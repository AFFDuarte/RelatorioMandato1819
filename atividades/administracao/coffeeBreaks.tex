% ========================
% # Coffee-Breaks        #
% ========================

\subsubsection{Coffee-Breaks}

O Pelouro da Administração tem uma forte ligação com todos os outros estando presente, nomeadamente, quando existem coffee-breaks nas atividades. Estes constituem uma benesse de reduzido custo e fácil organização sendo muito utilizados nos workshops do Pelouro das Saídas Profissionais e Formação. Para os organizar há que ter, no entanto, alguns cuidados para que os mesmos corram da melhor forma possível.
É então necessário fazer os seguintes passos:
\begin{enumerate}
    \item Comunicar com o responsável da atividade, de modo a saber: horas, local e número de participantes esperado.
    \item Combinar se é necessário ou não ajuda logística para a realização dos Coffee-Breaks ou se é apenas necessário deixar todo o material pronto a levar na sala do Núcleo.
\end{enumerate}

Material sempre necessário:
\begin{itemize}
    \item Pratos e copos descartáveis;
    \item Guardanapos;
    \item Várias garrafas de água para os oradores do evento.
\end{itemize}

Quanto à comida basta fazer uma estimativa a olho, tendo em conta o número de participantes colocando-se, habitualmente, bolachas, biscoitos, águas e sumos. Caso haja disponibilidade financeira pode-se aproveitar as várias máquinas de café existentes para servir café e servir alguns doces, vindo de pastelarias, e frutas. Com o \acrshort{ene3} e o Bot Olympics, este ano, tivemos uma boa injeção de alimentos da Dancake, pelo que não foi necessário comprar nada ao longo do ano, exceto as garrafas de água. Contudo, ficou em falta o serviço de café e fruta.

Um ponto importante, relacionado com o desperdício, é que as sobras são expostas no Núcleo, com um aviso a dizer que se podem comer, desaparecendo tudo em menos de 24h.