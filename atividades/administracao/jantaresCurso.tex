% ===============================
% # Jantares de Curso           #
% ===============================

\subsubsection{Jantares de Curso}

Os Jantares de Curso são a maior atividade de convívio de todos os estudantes do \acrshort{deec}, por razões bastante óbvias. Normalmente é o Núcleo que organiza os primeiros jantares de cada semestre e os carros organizam os jantares da serenata da Latada e da Queima das Fitas.

Este ano a Latada foi adiantada uma semana, fazendo com que não existisse jantar de serenata, uma vez que se tinha feito jantar de curso uma semana antes e seriam já demasiados jantares num curto espaço de tempo. Para além disso, deixou de haver febrada dos carros antes do Mega-Convívio, portanto o Núcleo achou por bem deixar dividir o jantar com os carros, ao abrigo de um contrato.

O acordo estipulado entre o Núcleo e os Carros contemplava não só a divisão de tarefas, como também, a divisão do lucro e eventuais multas/reduções de lucro.
As tarefas ficaram distribuídas da seguinte forma:
\begin{enumerate}
\item Contactar o local do jantar – Carro do Pedro Marçal
\item Fazer o cartaz e divulgar o evento – Carro da Carlos Abegão
\item Fazer a escala e verificar o cumprimento da mesma – \acrshort{neeec}
\item Controlar as entradas no jantar – \acrshort{neeec}
\end{enumerate}

Para ser justo para todos, caso existissem faltas à escala, a entidade responsável pela pessoa que faltou veria a sua percentagem de lucros reduzida em 1\% e esta seria distribuída equitativamente pelas restantes entidades. Caso existisse uma falta coberta por outro membro da entidade a percentagem seria reduzida em 0,5\%. Esta última medida foi implementada tendo em conta que havia muitas membros de um dos carros que pertenciam também ao Núcleo, representando as duas entidades.

O jantar foi no pavilhão da Palmeira e não correu como o esperado. Basicamente se quiséssemos encher o jarro tínhamos de ir à banca para que os empregados nos enchessem a mesma. O mesmo se aplicava aos finos, em que só se podia pedir 2 de cada vez.
Quanto ao serviço faltou comida a várias pessoas e o pavilhão não tinha as condições que um restaurante dispõe. Não obstante, o ambiente foi bom e correu tudo bem, com exceção de uma cadeira partida por um aluno, que nem se apercebeu de tal, e a organização teve de pagar imediatamente, algo que não é costume noutros locais até porque a mobília utilizada era de fraca qualidade.

O jantar do 2º semestre foi por conta do Núcleo, em exclusivo, portanto deu menos dores de cabeça. Fez-se uma banca no bar para as inscrições, com uma folha de registo de todo o dinheiro que entrava e saía da caixa. No entanto, houve alguns problemas com o dinheiro uma vez que havia inscrições pagas na banca do bar e outras pagas no núcleo e as pessoas não entendiam a necessidade de separar os registos de ambas as caixas.
O local escolhido foi o Restaurante da Torre do Arnado e no geral correu bastante bem, tirando o facto de que tínhamos marcado o local para 130 pessoas e só se inscreveram 88, o que já é hábito nos jantares do 2º semestre, que têm bastante menos adesão. Uma vez que houve um erro na observação do excel, informou-se a senhora do restaurante, no dia, que estavam inscritas 120 pessoas pelo que no ato de pagamento, a senhora cobrou 10€ dado o enorme rombo de clientes versus a quantidade de comida comprada e empregados contratados para o serviço. Assim este jantar deu um lucro reduzido novamente.

Recomendamos a manutenção destes jantares dado serem uma atividade tradicional. Contudo, aconselhamos a uma gestão melhor dos eventos de forma a que possam ter uma saldo mais positivo e um custo mais reduzido para os participantes.

