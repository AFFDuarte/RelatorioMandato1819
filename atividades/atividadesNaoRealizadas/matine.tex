% ==========================
% # Matiné                 #
% ==========================

\subsection{Matiné}

Esta atividade não estava no plano de atividades e foi sugerida por alguns membros do Pelouro, acabando por não se realizar por falta de tempo. Para esta atividade sugeriu-se utilizar uma tarde para passar filmes de forma gratuita na sala de convívio, para todos os alunos. Os filmes seriam escolhidos aleatoriamente, de uma lista de sugestões feita pelos membros do Pelouro da Cultura (podendo esta estar aberta aos alunos) e transmitidos com a ajuda do projetor na sala de convívio, utilizando o material nela existente (cadeiras, sofás, entre outros). Foi também dada a sugestão de haver oferta ou venda de pipocas durante a transmissão do filme.

Esta atividade é uma boa ideia mas poderá não ter grande adesão por parte dos alunos. A realizar-se, teria de ser durante a tarde, provavelmente quarta feira, mas muitos alunos vão para casa assim que terminam as aulas, ou durante a noite, o que não é muito convidativo pois os alunos teriam de ir para o Departamento após o jantar, algo que raramente acontece. Outro problema desta atividade são os próprios filmes. Hoje em dia com tanta forma de ver filmes (cinemas, Netflix, pirataria, etc.) a maior parte das pessoas já viu os filmes mais recentes e os filmes mais antigos não despertam muito curiosidade na maioria dos alunos. Contudo, é uma atividade que deve ser muito bem pensada pois até pode despertar mais interesse noutros anos letivos e, caso seja realizada e até tenha boa adesão, pode ser algo que se poderá repetir, por exemplo, uma vez por mês.
