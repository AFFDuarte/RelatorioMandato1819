% ==========================
% # Obras                  #
% ==========================

\subsection{Obras}

Como referido ao longo deste relatório, várias são as obras que a Direção do \acrshort{deec} quis fazer mas que acabaram por ainda não ser feitas devido a demoras nos processos contratuais que envolvem a faculdade:
\begin{itemize}
\item Construção de um arrumo no fundo do corredor do piso 4: dada a necessidade do \acrshort{neeec} ter um espaço para arrumar as suas coisas e o facto da solução possível (arrumo do piso 3A) não ser a ideal, uma vez que aquele local é necessário para a manutenção do Departamento, a Direção do \acrshort{deec} pretende construir um arrumo ao fundo do corredor do piso 4, aproveitando uma zona que, neste momento, é espaço perdido. Assim, pretendia-se construir uma parede, com uma porta e uma puxada de luz e eletricidade para dentro deste novo arrumo. É de realçar que esta obra encontra-se submetida à faculdade, estando agora à espera de aprovação para avançar.
\item Arranjo do chão da sala de convívio e da sala do Núcleo
\item Arranjo da eletrificação da sala de convívio: o circuito que diz respeito às lâmpadas da sala de convívio encontra-se altamente defeituoso e está, desde o início do mandato, à espera de uma retificação por parte da manutenção do \acrshort{deec}. Este circuito teve uma alteração provisória a meio do mandato, feita pelo professor Humberto Jorge, para impedir que o quadro viesse constantemente a baixo mas continuamos à espera que a manutenção do \acrshort{deec} termine a solução definitiva.
\item Pintura da sala do Núcleo e da sala de convívio: em 2016, a Direção em posse na altura pintou o Núcleo. Contudo, ao alteramos algumas coisas na parede do Núcleo passou-se a distinguir bastante bem as zonas recém pintadas das restantes até porque a cor com que o Núcleo foi pintado foi branco puro e não a cor creme que o Departamento tem em todas as paredes. Por este mesmo motivo, uma vez que a Direção do \acrshort{deec} sofreu uma reprimenda da arquiteta, não nos foi autorizado pintar a parede do Núcleo tendo ficado prometida a pintura da sala do Núcleo e da sala de convívio, algo que não chegou a ser feito devido, novamente, às burocracias contratuais da \acrshort{fctuc}.
\item Colocação de parede de vidro na sala de estudo do piso 6: aquando das reformulações deste espaço, ficou pendente a colocação de uma parede de vidro na zona que separa a zona de estudo em grupo da zona de estudo individual. Com esta novidade pretende-se reduzir o barulho que passa de uma zona para outra. Novamente, esta obra tem o aval da Direção do \acrshort{deec} e da arquiteta e encontra-se à espera de resposta da faculdade.
\item Colocação do quadro de cortiça na torre T, piso 4: esta compra está aprovada pela Direção do \acrshort{deec} e pela arquiteta e encontra-se à espera de resposta da faculdade.
\item Arranjo dos estores: como já referido, aguarda-se pela vinda da empresa que coloca este tipo de estores.
\item Colocação de estores na Antiga Biblioteca: esta compra está aprovada pela Direção do \acrshort{deec} e pela arquiteta e encontra-se à espera de resposta da faculdade.
\end{itemize}

Além disto, existem outras situações gerais que ficam também por fazer:
\begin{itemize}
\item Arranjo do parque de estacionamento do piso 6 bem como das estradas do Polo 2: esta situação foi discutida em reunião de núcleos Polo 2 tendo obtido o forte apoio do \acrshort{nei} e do \acrshort{neec} mas não obteve o apoio do \acrshort{neemaac}. Desta forma, ficou decidido falar-se com a faculdade para que se pudesse corrigir esta situação mas ficou também definido que seria João Machado, representante dos estudantes na Assembleia da Faculdade e Presidente do \acrshort{neemaac}, quem iria falar do assunto na Assembleia da Faculdade algo que, pelo menos que saibamos, nunca ocorreu.
\item Organização do arrumo B1 para se poder utilizar por parte dos carros da Queima das Fitas: esta foi de facto uma obra que foi iniciada mas que não se encontra totalmente concluída uma vez que não permite uma utilização conjunta pelo \acrshort{deec}, pelo \acrshort{neeec}, pelos carros e por outras associações estudantis. É de notar que a forma como a obra foi feita exige uma reformulação uma vez que a Direção do \acrshort{deec} não pretende armazenar produtos de febradas como grelhadores naquele local uma vez que estes materiais costumam sujar em demasia os espaços por onde passam. De notar também que o arrumo possui imenso material que deve ser deitado fora, sendo necessário, em conjunto com a Direção do \acrshort{deec}, verificar qual o material a deitar fora.
\end{itemize}