% ==========================
% # Placares Cortiça       #
% ==========================

\subsection{Placares Cortiça}

Desde o início do mandato a Direção do \acrshort{neeec} e a Direção do \acrshort{deec} entraram em conversações para rever os vários locais de afixação de cartazes existentes no \acrshort{deec}. Havia vários problemas identificados.

No piso 4, na zona da torre T, junto às escadas que vêm diretamente do piso 2, existia um placar de cortiça contudo no bar também só existia um o que não chegava, de todo, para a enorme quantidade de cartazes que estavam sempre lá afixados. Ambos os placares se encontravam em elevado estado de deterioração com buracos extremamente grandes pelo que os dois placares tiveram de ser reparados levando uma camada de cortiça nova na parte da frente e na parte de trás. Por sua vez, o placar da torre T foi deslocado para a zona do bar para que, neste local, houvesse uma maior área para publicidade.

É de salientar que nos arrumos do \acrshort{deec} existem vários placares iguais a estes, em perfeito estado, contudo a Direção do \acrshort{deec} decidiu não os alocar para este fim uma vez que esses placares são frequentemente utilizados para exposições no \acrshort{deec} ou fora deste e convém que se mantenham com o estado de conservação que possuem.

Após a reformulação da sala de estudo do piso 6, conversou-se com a Direção do \acrshort{deec} para que se pudesse ter uma área de publicidade na entrada do piso 6, local de elevada passagem no \acrshort{deec}. Contudo, nunca se avançou com uma versão definitiva, tendo sido, no entanto, algo que ficou apenas por fazer mas que inclui o aval de ambas as entidades.

Já no piso 4, na zona da torre T, decidiu-se colocar um placar de cortiça fixo na parede, à semelhança do que se passa na zona do elevador da torre do bar do piso 2. Contudo, esta obra teve de ser levada à arquiteta do Departamento pelo que se encontra extremamente atrasada.

É de salientar que todos os espaços publicitários, com exceção do espaço da entrada do piso 2 (sala de convívio, bar e zona do elevador) foram organizados com sinalética que indica as várias áreas de publicidade que podem existir e quais as condições para se poder afixar cartazes.