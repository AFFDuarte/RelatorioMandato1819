% ==========================
% # Venda de Cartolas      #
% ==========================

\subsection{Venda de Cartolas}

Apesar de não ser tradição no \acrshort{neeec}, há alguns núcleos que costumam vender os kits de finalistas (cartolas, bengalas e bandas) aos seus colegas, o que, juntamente com uma proposta recebida por parte do Filipe Cavaleiro, em representação da WonnaBCreative, nos fez ponderar sobre a possibilidade do \acrshort{neeec} também realizar essa mesma venda.

Após alguma deliberação, decidimos não avançar com essa venda dos kits de finalistas. As razões que nos levaram a essa deliberação foram:
\begin{itemize}
\item O número de finalistas que existem no nosso curso anualmente tende a ser um número relativamente reduzido (apesar de não termos números oficiais, seguimos pela intuição que tínhamos sobre a possibilidade desse número), pelo que dificilmente teríamos algum lucro significativo com a venda, principalmente comparando com o trabalho associado (não seria muito, mas era mais uma fonte de trabalho numa altura em que a energia para trabalhar começava já a escassear).
\item Dado o desconhecimento sobre o número possível de vendas, não saberíamos quantos kits teríamos que verdadeiramente comprar e procurámos evitar que houvesse stock a transitar de um ano para o outro.
Para além dessa deliberação inicial, analisando em retrospetiva a situação, apercebemo-nos também que teríamos que ter em conta quais as necessidades de cartolas negras, sendo um número ainda mais difícil de obter, e saber os tamanhos das cartolas que cada pessoa iria querer, que variam significativamente.
\end{itemize}

Se procurarem realizar esta tarefa no futuro, é importante que:
\begin{itemize}
\item Saibam se é possível devolver o stock que sobre;
\item Apenas se comprem as cartolas após se saber os tamanhos da cabeça das pessoas que as teriam de vir comprar ao \acrshort{neeec}.
\end{itemize}