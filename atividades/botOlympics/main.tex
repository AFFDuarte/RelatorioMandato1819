% ========================
% # Bot Olympics         #
% ========================

\section{Bot Olympics}

\subsection{Introdução}

O Bot Olympics é a maior competição de robótica da \acrlong{uc}, co-organizada pelo \acrshort{neeec} e pelo \acrfull{cr}.

É um evento que junta alunos de todas as Engenharias bem como estudantes do ensino secundário que competem num ambiente de aprendizagem, espírito de equipa e onde a capacidade de resolução de problemas é fundamental para concretizar o objetivo final da competição.

O Bot Olympics fornece também uma componente formativa aos participantes através de workshops intensivos de programação, Arduino e robótica que são essenciais para a competição, mas acima de tudo para a consolidação de conhecimentos na área de Engenharia Eletrotécnica e de Computadores.
Esta competição serve também, principalmente para o público não universitário, como um meio de divulgação \textit{in loco} do curso de Engenharia Eletrotécnica e de Computadores, dando a conhecer uma parte prática do mesmo, principalmente na área de automação.

\subsubsection{Missão}

\begin{itemize}
\item Criar um bom ambiente de competição propício à interação entre os participantes de forma a fortalecer o bom espírito entre as equipas;
\item Dotar os participantes de conhecimentos necessários para a programação e construção de um robot através de workshops intensivos de Arduino e Robótica;
\item Colocar os participantes em contacto mais direto com a área da robótica.
\end{itemize}

\subsection{Atividades}

O evento iniciou-se no dia 23 de fevereiro pelas 11h da manhã com uma sessão de abertura onde estiveram presentes todos os participantes, membros da Direção do evento e ainda o Diretor do Departamento de Engenharia Eletrotécnica e de Computadores, no qual se fez uma breve apresentação da competição bem como do Departamento e do curso.

No decorrer do dia os participantes receberam a devida formação para atingir os objetivos da competição, a saber, Workshop de Arduino, no qual aprenderam a programar e funcionamento do microcontrolador usado nos robots da competição, e Workshop de Robótica, onde puderam aprender alguns conceitos essenciais para navegação de robots móveis em ambiente semelhante ao da competição, incorporando os conhecimentos do Workshop anterior. O Departamento foi devidamente preparado para a logística associada a todo o evento, nomeadamente dormidas, alimentação e banhos. O segundo dia foi dedicado à preparação das equipas para a competição, no qual puderam trabalhar nos robots que iriam utilizar na final, desenvolver algoritmos e realizar testes com todas as condições necessárias para o efeito.

Durante todo o período de preparação e trabalho nos robots as equipas tiveram à sua disposição uma equipa de mentores, com formação prévia, para os auxiliar sempre que necessário.

O último dia consistiu em dois momentos: no momento da manhã, após se terem deslocado para o local da final, as equipas puderam fazer todos os ajustes e teste finais antes da prova ao algoritmo desenvolvido e ao robot; o momento da tarde, que se iniciou pelas 14h30, foi dedicado à competição entre as equipas e à posterior entrega dos prémios às equipas vencedoras.

\subsection{Disposições Finais}

A quarta edição do Bot Olympics foi, sem dúvida, um grande passo na história desta competição. Apesar das várias dificuldades sentidas na organização do evento, pelo enorme aumento da logística associado ao mesmo, tanto pelo crescimento do número de participantes, como pelo aumento da complexidade do evento, nomeadamente com a final a ocorrer num espaço distinto, consideramos que os objetivos iniciais do evento foram claramente atingidos e ultrapassados. O facto de termos presentes escolas de vários distritos do país e não só de Coimbra, de termos equipas de várias escolas diferentes e de termos uma final extremamente competitiva, mostraram o claro interesse das pessoas pelo tema da robótica. O Bot Olympics apresenta agora vários caminhos abertos para um crescimento sustentável nas próximas edições, que esperamos que transmita a um leque ainda maior de pessoas o interesse pela robótica.