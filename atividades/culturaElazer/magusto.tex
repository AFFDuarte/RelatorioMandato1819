% ================================
% # Magusto                      #
% ================================

\subsubsection{Magusto}

Este evento já decorre de forma tradicional todos os anos, permitindo o convívio entre alunos, professores e funcionários do \acrshort{deec}, onde são distribuídas castanhas e jeropiga. Para além disso contou-se com a realização de torneio de sueca e matraquilhos, a decorrer em simultâneo, da responsabilidade do Pelouro de Desporto.
\ifthenelse{\boolean{biblia}}
{ % TRUE
Este é um evento que, por ser gratuito e divertido, atrai bastantes participantes.
A mudança de dia do evento (quinta feira, em vez de quarta como era habitual) fez com que a sua adesão fosse ainda mais elevada, mas prejudicou os torneios devido à ocorrência em simultâneo de aulas e frequências.
}
{ % FALSE
}

Há que ter em atenção a forma como são assadas castanhas: como não são assadas pelo Núcleo, mas sim por pela Padaria de São João convém definir um prazo um bocado mais cedo para assar de forma a ter as castanhas prontas a servir à hora de início do evento. Outro aspeto a ter em consideração é como se obtém as castanhas: caso haja possibilidade de obter gratuitamente, apanhando, é de aproveitar, ao invés de se gastar dinheiro a comprar. Há que ter muita atenção à limpeza do espaço: o evento tem sido realizado na esplanada do bar pelo que é de maior importância que no final do evento esteja tudo limpo, nomeadamente que sejam varridas as cascas de castanhas, algo que é difícil por anoitecer cedo.
