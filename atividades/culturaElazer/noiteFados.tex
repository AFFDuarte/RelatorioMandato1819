% ================================
% # Noite de Fados               #
% ================================

\subsubsection{Noite de Fados}

Realizado em conjunto com o \acrshort{nei}, este evento contou com uma febrada e atuações de uma tuna (Quantunna) e de um grupo de fados (Capas ao Luar) oferecendo assim uma noite diferente aos estudantes de ambos os departamentos: (\acrshort{deec} e \acrshort{dei}).

Sendo este um evento organizado com um outro Núcleo por vezes torna-se difícil a comunicação entre ambos. Foi feita uma divisão de tarefas por núcleos, o foi uma boa ideia, mas devido a diferentes formas de trabalho certos aspetos do evento deixaram muito a desejar (por exemplo, a divulgação do evento, particularmente no nosso Departamento, pois os cartazes – apesar da constante insistência - foram feitos, entregues e divulgados pelo \acrshort{nei} demasiado perto do evento, sem que o \acrshort{neeec} pudesse dar a sua opinião relativamente à imagem e eventuais erros, o que se traduziu numa maior adesão por parte dos alunos do \acrshort{dei}). A escala de trabalhos deve ser divulgada com alguma antecedência para que não haja problemas de preenchimento.

Para além da febrada e das atuações, colocou-se a mesa de matraquilhos do \acrshort{neeec} para quem quisesse jogar, exceto durante as atuações das tunas e grupos de fado, devido ao barulho. Os jogos foram gratuitos durante toda a noite o que atraiu muita gente. Esta situação é um pouco “sensível” pois, caso os matraquilhos fossem pagos (como normalmente são) sempre era uma fonte de rendimento para o Núcleo, mas, por outro lado, corre-se o risco de, caso seja pago, o número de interessados não ser tão grande eliminando, então, essa fonte de rendimento. Consideramos que isto é algo que se deva manter, principalmente gratuitamente, ou até mesmo criar uma pequena competição/torneio (com por exemplo, febras e/ou finos como prémio) pois é algo que atrai pessoal. De reforçar que, caso se mantenham os matraquilhos, estes devem estar bloqueados durante as atuações.

Apesar de todos os problemas já referidos, o evento correu bem e permitiu oferecer aos alunos de ambos os departamentos uma noite diferente.
