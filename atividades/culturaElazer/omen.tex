% ================================
% # Omen                         #
% ================================

\subsubsection{HP by Omen University Challenge}

Este evento é, habitualmente, realizado nos finais de maio. Consiste numa competição, de inscrição gratuita, de FIFA onde os vencedores podem ganhar vales de compras entre 25€ e 100€ e acesso à final da competição, realizada em Lisboa. Este evento é independente do \acrshort{neeec} sendo organizado pela E2Tech e, portanto, toda a logística do evento é assegurada pela mesma. O \acrshort{neeec} apenas fica encarregue de retirar tudo o que se encontra na sala de convívio (ou do local escolhido para o realizar) e abrir as portas para os funcionários da empresa transportarem e montarem o material de manhã, pelas 8h.

Deixamos algumas sugestões para futuras realizações:
\begin{itemize}
\item No presente ano de 2017-2018, foi pedido ao \acrshort{neeec}, pelos organizadores, uma lista com alguns nomes só para “encher inscrições”, uma vez que uma das empresas envolvidas exige números mínimos. Caso tal seja voltado a pedir, deve-se ter em atenção aos nomes colocados pois, ao contrário do que foi dito pela organização, os nomes são chamados em “voz alta” para assinalar a presença dos participantes, podendo criar alguns problemas caso haja nomes de pessoas que não se inscreveram e se encontrem presentes na sala, como ocorreu;
\item Tal como dito anteriormente, o evento é totalmente gratuito e isto é algo que se deve considerar manter pois cobrar inscrição pode fazer com que muitos participantes acabem por não se inscrever (até porque existem alguns participantes que não são alunos do \acrshort{deec}, vindo de fora apenas para competir);
\item Apesar de, nos últimos anos, o evento se ter realizado em maio, a data pode ser repensada uma vez que as eliminatórias começam em fevereiro. Para tal, basta contactar a empresa dinamizadora no início do ano letivo para se estabelecer uma nova data e assim trazer mais participantes para o evento;
\item É importante manter alguma atividade a decorrer em paralelo com o evento de forma a atrair pessoas para o mesmo ou escolher uma data em que o Departamento esteja cheio. No ano passado, a data escolhida foi 24 de maio, uma quarta-feira, última de aulas, em que o Departamento estava vazio mas a realização de uma febrada dos Carros da Queima das Fitas do ano seguinte nos Jardins do NEEEC fez com que o evento estivesse sempre movimento. Este ano a data escolhida foi semelhante, 23 de maio, última quarta-feira de aulas do semestre, estando novamente o Departamento vazio mas, como desta vez não existia nenhuma atividade a decorrer em simultâneo, o evento teve vários momentos, na altura de uso livre, em que estava vazio.
\end{itemize}