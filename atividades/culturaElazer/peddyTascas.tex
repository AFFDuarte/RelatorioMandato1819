% ================================
% # Peddy Tascas                 #
% ================================

\subsubsection{Peddy Tascas} \label{subsubsec:atividades-cultura-peddy}

Este evento é realizado, habitualmente, pela altura das Festas Académicas (Festa das Latas e Imposição de Insígnias e Queima das Fitas), mas no presente ano letivo, dado a falta de adesão em anos anteriores, realizou-se apenas uma edição, nas semanas antes da Queima das Fitas. Neste evento os participantes, organizados em equipas de quatro a cinco elementos, visitam diversas tascas e “pontos \acrshort{neeec}” onde têm uma bebida que têm de consumir, ganhando assim pontos. Nesta edição, os prémios foram bilhetes gerais e pontuais para a Queima das Fitas, o que atraiu mais pessoas ao evento. Durante o percurso foram colocados alguns “Pontos \acrshort{neeec}”. Estes consistiam em locais onde membros do Núcleo serviam uma bebida aos membros das equipas (uma bebida diferente por ponto) e faziam pequenos desafios onde as equipas, caso bem sucedidas a completar os desafios, poderiam ganhar mais pontos.

É um evento de maior escala e como tal a ajuda, não só da equipa do Pelouro, mas também de todos os membros do Núcleo é essencial (são necessários acompanhantes para as equipas, pessoas para estarem nos “Pontos \acrshort{neeec}”, entre outros). É também muito importante prever algumas situações para que não haja problemas, como por exemplo, garantir que nos “Pontos \acrshort{neeec}” não acaba bebida (comprar sempre a mais do que o previsto). Caso sobre bebida de algum “Ponto \acrshort{neeec}”, esta pode ser sempre utilizada no ponto de chegada de forma a dar a possibilidade dos participantes obterem mais pontos.

O pagamento das tascas deve ser efetuado perto do fim do evento, tendo em atenção os horários de funcionamento das mesmas.

Deixamos agora algumas sugestões para futuras realizações:
\begin{itemize}
    \item Tal como referido anteriormente há que garantir que não há falta de material, nomeadamente bebidas, copos, etc;
    \item Distribuir pelas tascas, antes do evento começar, uma folha de presenças com as equipas de forma a garantir que, caso seja feito o pagamento da tasca antes de uma ou mais equipas terem passado pela mesma, não lhes seja recusada bebida, algo que acontece frequentemente; 
    \item Ter em atenção a todos os pedidos impostos pelas tascas e caso não seja dito nada, perguntar e garantir que não surgem imprevistos (horas de encerramento, algumas tascas começam a servir refeições a partir de certa hora e as equipas têm de passar todas por lá antes da hora imposta, etc.);
    \item Ter em atenção aos acompanhantes das equipas. É impossível garantir que não bebam, mas ter em atenção que, obviamente, não podem beber da mesma forma que os participantes para que não haja problemas de qualquer natureza (seja no preenchimento de tabelas de pontuação, seja problemas com as pessoas, e assim);
    \item Caso haja atividades a serem realizadas, ter o material sempre preparado e em boas condições;
    \item Ter preparado mais do que dois percursos (ou então ter em conta o número de equipas inscritas) de forma a garantir, ou pelo menos prevenir, que uma ou mais equipas se cruze com outra podendo atrasar ou criar mau ambiente de funcionamento na tasca;
    \item Criar uma forma de serem as tascas a registar os pontos para evitar que haja equipas a ser favorecidas pelo elemento que as acompanha, ainda para mais quando há prémios tão aliciantes, como houve este ano.
\end{itemize}

Apesar do trabalho e eventuais problemas que possam surgir, este é um evento que atrai muitos participantes (principalmente pela proximidade com os eventos da Queima das Fitas ou Latada) e que todos gostam de participar.