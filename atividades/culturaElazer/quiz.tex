% ================================
% # Quiz                         #
% ================================

\subsubsection{Quiz Cultural}

O quiz cultural foi um evento realizado aquando do Mês Solidário, onde equipas de dois elementos competiram ao responder a perguntas de vários temas (Cinema/Séries, Música, Geografia e Cultura Geral). As equipas responderam (numa “placa” de resposta fornecida) às perguntas apresentadas (através de um PowerPoint) e receberam/perderam pontos de acordo com as suas respostas. No final da competição, a equipa com maior pontuação foi a vencedora.

A placa de resposta foi feita por um elemento do Pelouro e era constituída por uma pequena placa de cartão e papel cavalinho, forrada a papel autocolante, onde os participantes podiam escrever com marcadores as suas respostas, mostrar as mesmas e, por fim, apagar com um papel e assim sucessivamente.

O evento realizou-se no auditório A.5.2 do \acrshort{deec} que, tendo em conta o número de participantes (no presente ano letivo, contou-se com dez participantes), foi um bom local pois conseguiu-se colocar todas as equipas e ainda espaçá-las. Caso o número de participantes fosse superior, este evento decorreria melhor numa sala (por exemplo a T.4.1 ou T.4.4 do \acrshort{deec}) ou até mesmo num dos auditórios de maiores dimensões. 

O feedback dos participantes foi extremamente positivo: gostaram da atividade e divertiram-se. Para muitos, a hora de realização do evento (quarta-feira às 18h) foi um problema pelo que esta deveria ter ocorrido mais cedo.

É necessário ter em atenção algumas situações: rever muito bem as perguntas e respostas para que não haja problemas com respostas erradas. Usar uma tabela de pontuações mais “automática” para que quem esteja a apontar as mesmas não perca muito tempo a confirmar os pontos a atribuir ou retirar às equipas (exemplo: em vez de apontar o número de pontos que a equipa obteve numa pergunta, colocar apenas um A (acertou) ou um E (errou) e definir os seus valores no excel).

Este é um evento que costuma ter pouca adesão mas que costuma ser muito divertido para quem participa pelo que a sua realização nos parece importante mas deve ser feita uma aposta maior na sua divulgação.
