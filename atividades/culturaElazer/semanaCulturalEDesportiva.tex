% ================================
% # Semana Cultural e Desportiva #
% ================================

\subsubsection{Semana Cultural e Desportiva}

Pela primeira vez, foi realizada um Semana Desportiva e Cultural (“NEEEC Sports \& Culture Week”) organizada em parceria com o Pelouro do Desporto. Durante esta semana (segunda, terça, quarta e quinta) foram propostas diversas atividades aos alunos do \acrshort{deec}, tanto de cariz desportivo como de cariz mais cultural/lúdico.

No primeiro dia, realizou-se uma pequena feira composta pelas diversas secções culturais e desportivas da \acrshort{aac} para demonstrar a oferta e o que nelas se faz. Após várias tentativas de comunicação com todas as secções culturais só se obteve resposta de duas (Grupo Ecológico e SOS Estudante) e apenas uma, Grupo Ecológico, esteve presente.

No segundo dia, do final da manhã até meio da tarde, contou-se com a presença do Simology, um simulador realista de condução. O simulador foi colocado na sala de convívio e requeria uma inscrição prévia com um custo de 2€. Devido ao desconhecimento do que era o Simology, as inscrições até ao dia foram muito poucas (cerca de cinco). Uma vez montado, os alunos demonstraram bastante interesse na atividade e muitos inscreveram-se na hora de participar fazendo com que se tivesse de prolongar por mais uma hora o aluguer do material, continuando a haver fila na hora de fecho. Foi uma atividade um bocado dispendiosa e que trouxe um pouco de prejuízo mas foi sem dúvida a atividade, por parte do Pelouro da Cultura, que mais curiosidade e participação teve por parte dos alunos, ao longo de todo o ano. Os inscritos tiveram a oportunidade de experimentar o simulador (com recurso a óculos de realidade virtual) e no final era apontado o tempo que os participantes demoravam a percorrer a pista. O inscrito com melhor tempo (i.e. com o menor tempo) teve direito a um prémio.

De tarde, realizou-se um torneio de FIFA 18. Com inscrição prévia, os participantes competiam entre si de forma aleatória numa série de jogos para descobrir quem seria o vencedor (o torneio foi “gerido” através do site https://challonge.com/). Estava também proposto um torneio de Street Fighter IV que acabou por não se realizar por falta de inscrições para o mesmo.

No terceiro dia, realizou-se uma tarde de jogos de tabuleiro, de participação gratuita. Feita na sala de convívio, os alunos tinham à sua disponibilização diversos jogos de tabuleiro (Monopólio, Party \& Co., Trivial Pursuit, Mikado, Uno, Damas, Xadrez, entre outros) que podiam usufruir. O tempo de duração do evento foi diminuído (inicialmente duas horas, passou para uma hora) porque não houve participantes.

No quarto dia propôs-se a realização de um novo Quiz, à semelhança do anterior. O evento não foi realizado devido à falta de inscrições suficientes (apenas uma equipa se inscreveu).

No geral, sendo um evento novo os alunos não demonstraram grande interesse, o que se traduziu em muito poucas inscrições nos eventos. Apesar disso, consideramos ser um evento bastante interessante e com grande potencial no futuro, se mais divulgado e colocado numa data melhor. Os torneios de videojogos convém serem de jogos que os participantes conheçam e tenham interesse (algo que se verificou ao haver inscrições no torneio de FIFA e não no torneio de Street Fighter). Os jogos de tabuleiro não parecem ter grande interesse por parte dos alunos (algo que é comprovado durante o resto do ano, onde os alunos nos seus tempos livres não requisitam alguns dos jogos oferecidos pelo \acrshort{neeec} (Damas, Xadrez, Cartas, etc.) mas requisitam bastante as raquetes de ténis de mesa, por exemplo).
