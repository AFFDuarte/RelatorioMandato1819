% ========================
% # Visita à Alta        #
% ========================

\subsubsection{Visita à Alta }

A visita à alta realizou-se no dia 27 de setembro com o objetivo de integrar a praxe com uma visita organizada pelo \acrshort{neeec} à parte histórica da cidade de Coimbra.

Existiu, no início, um contratempo no percurso predefinido pela equipa do \acrshort{neeec}. Como apenas existia uma pessoa referente a essa equipa a acompanhar a praxe, e não estava 100\% integrada com o programa da atividade, foi influenciado pelo grupo de Doutores que acompanhavam os caloiros o que provocou um desvio no percurso e atrasos na Visita ao Pátio das Escolas.

Este evento é uma oportunidade de mostrar aos caloiros um pouco da cidade de Coimbra, nomeadamente a zona circundante da \acrlong{uc}. Enquadrado numa situação de praxe académica funciona melhor pois ajuda no transporte e guia dos caloiros. Há que tentar que comece cedo de forma a que os alunos possam visitar todos os locais.
Sugestões para futuras realizações:
\begin{itemize}
\item Contactar antecipadamente a Biblioteca Geral para confirmar o número de pessoas;
\item Chegando à porta férrea dividir os alunos por grupos de forma a facilitar as visitas;
\item Tentar fazer \textit{guide lines} com informações sobre os locais a visitar;
\item Ter mais do que um fotógrafo;
\item Criar um género de Peddy Paper ao invés de uma visita ao grupo permitindo que as visitas sejam feitas em grupos mais pequenos;
\item Aproveitar o verão para organizar melhor o evento.
\end{itemize}