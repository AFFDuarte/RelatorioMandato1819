% ================================
% # Descida ao Rio               #
% ================================

\subsubsection{Descida ao Rio}

A descida ao rio é um evento que todos os anos se tem tentado organizar mas que não se tem realizado por ter falta de adesão. Em anos anteriores, o evento era organizado com o Stand Up Paddle Coimbra mas, este ano, decidimos fazê-lo com a empresa O Pioneiro do Mondego sendo então uma descida em kayak desde Penacova até Torres do Mondego. Uma vez que a atividade é organizada pela empresa, toda a logística da mesma é bastante simples. É então importante insistir na divulgação do evento. O evento tem um custo muito elevado pelo que sugerimos algumas coisas para edições futuras:
\begin{itemize}
\item Caso esteja de acordo com a política financeira do Núcleo, uma parte do custo (que este ano foi de 17€) poderá ser suportado pelo Núcleo (isto ia contra os princípios financeiros que nos regemos este ano, uma vez que se tratava de uma atividade recreativa, pelo que não o fizémos) para que o preço para os sócios do NEEEC seja mais reduzido;
\item O evento poderia ser realizado com todos os núcleos do Pólo 2 permitindo assim um maior número de inscritos, menor número de pessoas de cada núcleo e a ter que ir e, potencialmente, um menor custo por participante. Contudo, pela nossa experiência em eventos conjuntos do Polo 2, tal poderia não ser, no entanto, uma boa ideia.
\end{itemize}

A divulgação do evento foi feita andas da Queima das Fitas pelo que houve imenso tempo para as pessoas pensarem no evento. O facto do evento ser tão próximo da Queima faz com que as pessoas tenham pouco dinheiro para a atividade mas é difícil mudá-la de data devido às condições climatéricas.

No dia do evento foi necessário levar o dinheiro para pagar à empresa sendo essencial a presença de um elemento da equipa do Pelouro. Com 7 participantes realizou-se a descida do mondego em kayak, finalmente, após tantos anos de insistência. Estava uma dia de muito sol e calor o que fez com que a atividade fosse ainda melhor. Obteve-se um bom feedback dos participantes e todos gostaram.

Em futuras edições, nomeadamente se a adesão for maior, poderá envolver-se esta atividade junto de outras que promovam algumas receitas que impeçam esta atividade ser tão cara.
