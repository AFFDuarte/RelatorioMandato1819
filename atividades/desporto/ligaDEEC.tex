% ================================
% # Liga DEEC                    #
% ================================

\subsubsection{Liga DEEC}

Como é tradicional, realizou-se um torneio de futsal que dá acesso à Liga Polo 2. A este torneio optámos por chamar Liga \acrshort{deec} e não Liga Polo 2 uma vez que o evento é exclusivamente organizado pelo \acrshort{neeec} e não tem sequer as mesmas regras da Liga Polo 2 (de notar, que só o NEEC e o NEEA é que denominam as suas eliminatórias como Liga Polo 2). Este evento realizou-se como costume, em novembro, tendo coincidido com um período muito conturbado do Pelouro pelo que foi organizado pela Direção em conjunto com o Pelouro e executado pelo Pelouro.

Decidiu-se delimitar de imediato o número de equipas a 8 e permitir um número de elementos entre 6 e 8 em cada equipa. Fez-se também um regulamento completamente novo e detalhadamente pensado em reunião de Pelouro, o qual foi muito positivo para a execução do torneio. Recomendamos a restrição do número de equipas logo de início pois tal permite delinear o torneio à partida no regulamento e na organização do mesmo. Pelo mesmo motivo, recomendamos, no futuro, a que se limite o número de elementos das equipas não permitindo equipas de tamanhos diferentes, o que trouxe problemas (embora simples).

As inscrições tinham um custo de 5€ por pessoa e o preço foi anunciado como preço individual e não de equipa para não assustar as pessoas. Quanto à logística das inscrições, as mesmas esgotaram e houve equipas inscritas a mais. Foi cumprido o regulamento e dada primazia a quem exerceu o pagamento em primeiro lugar o que, por se tratar de um fim de semana, deu privilégio a quem pagou por transferência, provocando amuos nas equipas excluídas (foi excluída uma equipa e outra desistiu). De realçar, no entanto, que as inscrições demoraram ainda alguns dias para começar tendo sido fechadas apenas no sábado antes do evento.

Para a logística do torneio, preparou-se um excel que ordenava automaticamente as equipas pelos grupos (fazia o sorteio) e dava para depois inserir os detalhes de cada jogo. Este excel facilitou imenso o trabalho durante o torneio e o mesmo encontra-se disponível para ser usado em futuras edições e adaptado a outras modalidades. Existiu também uma ficha de jogo que, precisa de várias melhorias no futuro (nomeadamente: não há cantos no futsal, o ponto dos jogadores que não compareceram ou que chegaram atrasados deve estar na mesma tabela de amarelos e vermelhos para que seja de fácil preenchimento e dois espaços para árbitros de mesa).

Em relação ao evento em si:
\begin{itemize}
\item Foi criada uma escala que contemplava apanha-bolas, ajuda logística além dos árbitros o que possibilitou dar uma qualidade muito grande ao evento. O apanha-bola bola é uma pessoa essencial e deve estar sempre escalado.
\item Venda de panikes, sumos e águas: a partir da segunda noite, uma vez que na primeira noite muitas pessoas foram comprar estes elementos a outros lados, passou a estar disponível para venda alguns produtos alimentares o que permitiu ganhar algum dinheiro para compensar as elevadas despesas do evento.
\item Música: nos intervalos e no início e fim da noite havia sempre música o que animava muito o evento.
\item Início de cada noite: o início de cada noite tinha sempre atrasos (após o primeiro dia, o atraso era maior na organização no que nas equipas) o que fez com que ficássemos todas as noites após a hora final (23h30). A partir da meia noite, o segurança queria-se ir embora não se importando de esperar no primeiro dia, mas ficando visivelmente chateado nos restantes com destaque na final que ameaçou os presentes de os fechar dentro da escola uma vez que a hora já excedia demais.
\item Os jogos tiveram durações maiores nas meias finais e nas finais o que provocou um atraso muito grande nesse dia, apesar de se realizarem 4 jogos e não 6 como nos outros dias.
\item Foi criada pelos membros do Pelouro uma faixa da Liga \acrshort{deec} o que foi engraçado para as fotos e não acarretou custos uma vez que foi feito de forma manual, contudo, com a chuva teve de ir para o lixo.
\item Final: a final foi transmitida em direto, teve música dos campeões e a entrega de medalhas foi feita em estilo pódio utilizando, para isso, as escadas existentes junto ao campo. Não existiu troféu, mas foi entregue a grade de cerveja o que permitiu uma festa muito engraçada dos vencedores.
\end{itemize}
