% ========================
% # Desporto             #
% ========================

\section{Desporto}

No presente ano letivo, a equipa do Pelouro do Desporto foi constituída por 8 elementos. Após a realização de várias atividades onde o Coordenador Geral do Pelouro, André Soares, não se apresentou como responsável pelo Pelouro, a Direção do \acrshort{neeec} decidiu demiti-lo e tomou a gestão do Pelouro, provisoriamente, até à conclusão da realização da final da Liga \acrshort{deec}. Assim, o Pedro Henriques foi convidado a assumir as funções de Coordenador Geral tendo, no entanto, deparado-se com um plano de atividades por realizar que não tinha sido feito por si. Este Pelouro acabou, no entanto, por ser bastante ativo tendo-se realizado várias atividades novas nomeadamente uma semana desportiva e cultural, um passeio de bicicleta e a descida do rio. Por fazer, ficou a visita ao estádio cidade de Coimbra e à Academia Briosa XXI bem como a organização de uma ida a um jogo da Académica/OAF. No que toca à organização interna, o Pelouro esteve sempre bastante dependente do CG bem como da Direção mas todos os membros do Pelouro estiveram sempre disponíveis a ajudar em tudo, contudo, verificou-se que o número elevado de membros neste Pelouro (8) acabou por dificultar mais o trabalho do mesmo do que facilitar.

\subsection{Atividades}

% ====================================
% # Transmissão dos Jogos da Seleção #
% ====================================

\subsubsection{Transmissão dos Jogos da Seleção}

No mês de junho foram transmitidos os jogos da seleção nacional de futebol, para a Taça das Confederações, Portugal vs Rússia (2ª jornada da fase de grupos) e Portugal vs Chile (meias finais). O jogo da 1ª e 3ª jornadas não foram transmitidos uma vez que pelas suas horas e/ou dias da semana em que calhavam não havia público suficiente no DEEC para se justificar a realização da atividade.

O local escolhido para o evento foi a sala de convívio, mudando-se a disposição da sala completamente: pôs-se filas de cadeiras ao longo de toda a sala com um projetor apontado para a parede da reprografia. Tiveram de se por panos escuros nas janelas mais próximas do projetor, uma vez que as cortinas deixavam passar muita luz, impedindo a possibilidade de se ver a transmissão decentemente.

Para criar uma dinâmica gira durante o jogo fez-se uma cachorrada onde se venderam minis frescas, cachorros e águas, sendo que a zona de trabalho era junto à parede do Núcleo, atrás da transmissão.

\ifthenelse{\boolean{biblia}}
{ % TRUE
Em termos logísticos foi bastante fácil organizar o evento tendo sido a parte mais trabalhosa a cachorrada pois requer que haja pessoas sempre disponíveis para servir as pessoas, ou seja uma escala, o que é sempre complicado numa época de exames.
}
{ % FALSE
}

O evento teve imensa adesão, tendo em conta que estávamos em plena época de exames e o Departamento não tinha muita gente. Ao longo do evento, dada a falta de cadeiras que havia na sala de convívio (situação entretanto resolvida com as cadeiras de jardim) foi necessário ir buscar várias cadeiras às salas de aula da torre T.

% ================================
% # Caloiros VS Doutores         #
% ================================

\subsubsection{Caloiros VS Doutores}

No âmbito da receção ao caloiro, o Pelouro do Desporto decidiu fazer um jogo de futsal 7 x 7 de Caloiros vs Doutores. Este evento é já muito habitual noutros núcleos pelo que achámos por bem trazê-lo para o nosso Núcleo.

O evento realizou-se numa quarta-feira à noite, após uma praxe, no mesmo dia da visita à alta. De realçar que, devido à praxe, existiu uma claque no evento o que dinamizou muito o mesmo. É também de realçar que a praxe não esteve muito disponível para manter as atividades praxísticas desde o final da visita à alta até ao início do jogo o que provocou uma claque mais reduzida, não deixando, no entanto, de ser muito divertida. No futuro, aconselhamos a uma maior coordenação das atividades com a praxe.

O jogo de futsal 7 x 7 teve um limite de 14 jogadores por equipa e duas partes com a duração de 30 minutos, cada. O campo escolhido foi o campo de Santa Cruz tendo o aluguer do mesmo um custo de 42 euros. A inscrição teve um custo de 2 euros por pessoa para cobrir a despesa do campo. De realçar que as inscrições não esgotaram, mas ficaram muito perto disso (13 inscritos em cada equipa).

No futuro é necessária uma maior organização interna do Pelouro para a atividade, principalmente no que diz respeito ao cumprimento da escala (que para este evento foi muito simples) e à verificação de pagamentos (os problemas que ocorreram, não voltariam a ocorrer com as modificações entretanto feitas no que toca à gestão de inscrições entretanto feitas no \acrshort{neeec}). Destacamos que o evento é muito divertido, agradou a todos os participantes, mas que só tem piada se for feito com a praxe. Foi também uma ideia nossa, mas que acabou por não avançar por não ter viabilidade financeira nem logística, realizar um pequeno convívio com finos junto ao jogo, algo que se for mais bem pensado poderá ser feito no futuro.


% ===================================
% # Torneio de Sueca e Matraquilhos #
% ===================================

\subsubsection{Torneio de Sueca e Matraquilhos}

Com a realização do tradicional magusto, responsabilidade do Pelouro da Cultura e Lazer, é habitual fazer-se um torneio de sueca. Este ano, decidimos também fazer um torneio de matraquilhos. Para tal, foi feito e divulgado um regulamento para cada torneio, essenciais para o bom funcionamento do mesmo, e foram anunciados prémios, que consistiam em finos, para os vencedores.

O evento acabou por ter vários problemas uma vez que este se realizou a uma quinta-feira, havendo aulas e frequências em simultâneo, o que dificultou a presença de participantes e pessoas para trabalhar na atividade. Por sua vez, o CG do Pelouro decidiu atrasar em duas horas o início do torneio para tentar ter mais inscritos. Assim, quem podia à hora marcada deixou de poder e quem não podia não soube da alteração pelo que também não participou. Ambos os torneios acabaram por se realizar com poucas equipas (cerca de 3) tendo o mesmo prolongado-se até muito tarde, muito para além da hora prevista para o fim da escala, fazendo com que membros do Pelouro tivessem de se revezar entre si para conseguir terminar o evento. Por sua vez, o CG, meia hora após o início dos torneios, decidiu ir para Viseu sem avisar ninguém, tendo deixado dois membros do Pelouro encarregue dos dois eventos, tendo sido, portanto, demitido no dia seguinte. Um dos membros do Pelouro era também participante do torneio que estava a vigiar o que dificultou, ainda mais, a situação.

Uma vez que o magusto é feito nos jardins do bar e anoitece cedo em novembro, teve ainda de se arranjar soluções, à pressa, para mudar de local quando anoiteceu. Ambos os torneios, apesar de terem poucos participantes, terminaram numa altura em que já quase ninguém estava no departamento, havendo inclusive desclassificações porque equipas precisavam de se ir embora. Em suma, este foi um evento super desorganizado que, obviamente, deve ter continuidade mas deve ser organizado de forma muito mais ponderada.

% ================================
% # Liga DEEC                    #
% ================================

\subsubsection{Liga DEEC}

Como é tradicional, realizou-se um torneio de futsal que dá acesso à Liga Polo 2. A este torneio optámos por chamar Liga \acrshort{deec} e não Liga Polo 2 uma vez que o evento é exclusivamente organizado pelo \acrshort{neeec} e não tem sequer as mesmas regras da Liga Polo 2 (de notar, que só o NEEC e o NEEA é que denominam as suas eliminatórias como Liga Polo 2). Este evento realizou-se como costume, em novembro, tendo coincidido com um período muito conturbado do Pelouro pelo que foi organizado pela Direção em conjunto com o Pelouro e executado pelo Pelouro.

Decidiu-se delimitar de imediato o número de equipas a 8 e permitir um número de elementos entre 6 e 8 em cada equipa. Fez-se também um regulamento completamente novo e detalhadamente pensado em reunião de Pelouro, o qual foi muito positivo para a execução do torneio. Recomendamos a restrição do número de equipas logo de início pois tal permite delinear o torneio à partida no regulamento e na organização do mesmo. Pelo mesmo motivo, recomendamos, no futuro, a que se limite o número de elementos das equipas não permitindo equipas de tamanhos diferentes, o que trouxe problemas (embora simples).

As inscrições tinham um custo de 5€ por pessoa e o preço foi anunciado como preço individual e não de equipa para não assustar as pessoas. Quanto à logística das inscrições, as mesmas esgotaram e houve equipas inscritas a mais. Foi cumprido o regulamento e dada primazia a quem exerceu o pagamento em primeiro lugar o que, por se tratar de um fim de semana, deu privilégio a quem pagou por transferência, provocando amuos nas equipas excluídas (foi excluída uma equipa e outra desistiu). De realçar, no entanto, que as inscrições demoraram ainda alguns dias para começar tendo sido fechadas apenas no sábado antes do evento.

Para a logística do torneio, preparou-se um excel que ordenava automaticamente as equipas pelos grupos (fazia o sorteio) e dava para depois inserir os detalhes de cada jogo. Este excel facilitou imenso o trabalho durante o torneio e o mesmo encontra-se disponível para ser usado em futuras edições e adaptado a outras modalidades. Existiu também uma ficha de jogo que, precisa de várias melhorias no futuro (nomeadamente: não há cantos no futsal, o ponto dos jogadores que não compareceram ou que chegaram atrasados deve estar na mesma tabela de amarelos e vermelhos para que seja de fácil preenchimento e dois espaços para árbitros de mesa).

Em relação ao evento em si:
\begin{itemize}
\item Foi criada uma escala que contemplava apanha-bolas, ajuda logística além dos árbitros o que possibilitou dar uma qualidade muito grande ao evento. O apanha-bola bola é uma pessoa essencial e deve estar sempre escalado.
\item Venda de panikes, sumos e águas: a partir da segunda noite, uma vez que na primeira noite muitas pessoas foram comprar estes elementos a outros lados, passou a estar disponível para venda alguns produtos alimentares o que permitiu ganhar algum dinheiro para compensar as elevadas despesas do evento.
\item Música: nos intervalos e no início e fim da noite havia sempre música o que animava muito o evento.
\item Início de cada noite: o início de cada noite tinha sempre atrasos (após o primeiro dia, o atraso era maior na organização no que nas equipas) o que fez com que ficássemos todas as noites após a hora final (23h30). A partir da meia noite, o segurança queria-se ir embora não se importando de esperar no primeiro dia, mas ficando visivelmente chateado nos restantes com destaque na final que ameaçou os presentes de os fechar dentro da escola uma vez que a hora já excedia demais.
\item Os jogos tiveram durações maiores nas meias finais e nas finais o que provocou um atraso muito grande nesse dia, apesar de se realizarem 4 jogos e não 6 como nos outros dias.
\item Foi criada pelos membros do Pelouro uma faixa da Liga \acrshort{deec} o que foi engraçado para as fotos e não acarretou custos uma vez que foi feito de forma manual, contudo, com a chuva teve de ir para o lixo.
\item Final: a final foi transmitida em direto, teve música dos campeões e a entrega de medalhas foi feita em estilo pódio utilizando, para isso, as escadas existentes junto ao campo. Não existiu troféu, mas foi entregue a grade de cerveja o que permitiu uma festa muito engraçada dos vencedores.
\end{itemize}


% ================================
% # \acrshort{neeec} VS Profs               #
% ================================

\subsubsection{NEEEC VS Profs}

O \acrshort{neeec} vs PROFS é um jogo de futebol de salão, 5x5, pertencente ao mês solidário, onde os elementos do \acrshort{neeec} defrontam os professores do \acrshort{deec} num espírito de convívio tendo a atividade uma inscrição simbólica de um bem alimentar. Este evento é restrito aos elementos do \acrshort{neeec} pois caso contrário traria um défice ainda maior de alunos perante professores.

O convite aos professores este ano foi alargado aos funcionários sendo que o Eng. Maia e o Tito manifestaram bastante interesse em ir mas não puderam, por motivos pessoais. É muito importante partilhar bem este evento junto dos professores e garantir que os emails são entregues. Como em tudo, o melhor é fazer o convite pessoalmente, garantindo assim mais entradas. Alguns professores como o professor Peixoto, o professor Marco e o professor Crisóstomo costumam marcar sempre presença.

Este ano, o jogo decorreu no campo da Escola Secundária Infanta D. Maria, no mesmo local da Liga DEEC durante uma hora. Este evento é muito interessante contudo a data (final de semestre, em época de avaliações) bem como as condições climatéricas (frio) afastam um pouco os participantes. Contudo, quem vai acaba por adorar a experiência pois é possível ter um contacto mais informal com a comunidade, o que é sempre divertido.

% ================================
% # Semana Cultural e Desportiva #
% ================================

\subsubsection{Semana Cultural e Desportiva}

Pela primeira vez, foi realizada um Semana Desportiva e Cultural (“NEEEC Sports \& Culture Week”) organizada em parceria com o Pelouro da Cultura e Lazer. Durante esta semana (segunda, terça, quarta e quinta) foram propostas diversas atividades aos alunos do \acrshort{deec}, tanto de cariz desportivo como de cariz mais cultural/lúdico. Este era um evento completamente novo no \acrshort{neeec} pelo que foi organizado sem bases em atividades anteriores, não se sabendo o que se esperar do evento. 

No primeiro dia, de manhã, realizou-se uma feira com várias secções da \acrshort{aac}. Após um email enviado para todas as secções desportivas da \acrshort{aac}, obtivemos apenas resposta de uma (Secção de Halterofilismo). Pela tarde o objetivo era realizar um torneio de ping-pong na sala de convívio, mas com a falta de participantes aliada à falta de pessoas na sala de convívio a essa hora, o torneio acabou por ser cancelado. Por último, para o fim da tarde estava marcada uma pequena corrida (5km) mas como começou a chover e o número de participantes também era baixo a corrida também foi cancelada.

No segundo dia, terça feira, o Pelouro de Desporto não fez nenhuma atividade.

No terceiro dia, marcado para a noite estava um torneio de basket 3x3, este só contava com uma equipa inscrita, então este torneio também foi cancelado. 

Para finalizar, no último dia, quinta-feira, foi organizado a última atividade, um torneio de bowling, que contou com cerca de 9 participantes que não pertenciam ao núcleo. Este torneio foi bem sucedido tendo tido um feedback positivo dos participantes e reunido vários interessados que não puderam participar por já ter começado o torneio.

A semana em geral é um conceito interessante e que deve ser repetido no futuro, sendo obviamente repensada. Cada torneio deve ser organizado com mais detalhe, com regras explícitas. É também importante que existam prémios aliciantes e que estes sejam divulgados. O evento foi realizado no início de março mas já muito próximo do início das avaliações pelo que, para que seja bem sucedido, deve ser recolocado numa altura com melhores condições meteorológicas e menos avaliações, talvez no início de um semestre.

% ================================
% # Passeio de Bicicleta         #
% ================================

\subsubsection{Passeio de Bicicleta à Figueira da Foz}

Ao longo desta mandato, surgiu a ideia de se fazer um passeio de bicicleta entre Coimbra e a Figueira da Foz. A ideia seria partir do parque verde de manhã e ir pelas estradas junto ao Mondego até à Figueira. De seguida almoçaria-se por lá e voltaria-se a Coimbra de comboio. Esta atividade foi marcada para o dia 1 de maio mas acabou por ser adiantada para 25 de abril para que a Queima das Fitas pudesse organizar os Hunger Games a 1 de maio, como tinha pedido, evento que depois acabou por anunciar para 2 de maio e realizar a 4 de maio.
\ifthenelse{\boolean{biblia}}
{ % TRUE
O passeio teve muito pouca adesão mas acabou por se realizar como um evento simples, entre amigos.
}
{ % FALSE
}

É de notar que este evento teve alguma adesão e comentários no Facebook pelo que se recomenda fazer de novo esta iniciativa com três ressalvas importantes: divulgá-la com muito mais antecedência para permitir aos inscritos que moram fora de Coimbra terem tempo para reparar as suas bicicletas e trazê-las; proporcionar a possibilidade de aluguer de bicicletas, algo que após uma pesquisa no Google se percebe que é fácil de conseguir; o facto do passeio ter sido até à Figueira não é de todo um obstáculo físico, contudo, para a generalidade das pessoas é encarado como um objetivo impossível de alcançar pelo que, se calhar, é recomendável que o passeio seja até um local mais próximo de Coimbra.

% ================================
% # Descida ao Rio               #
% ================================

\subsubsection{Descida ao Rio}

A descida ao rio é um evento que todos os anos se tem tentado organizar mas que não se tem realizado por ter falta de adesão. Em anos anteriores, o evento era organizado com o Stand Up Paddle Coimbra mas, este ano, decidimos fazê-lo com a empresa O Pioneiro do Mondego sendo então uma descida em kayak desde Penacova até Torres do Mondego. Uma vez que a atividade é organizada pela empresa, toda a logística da mesma é bastante simples. É então importante insistir na divulgação do evento. O evento tem um custo muito elevado pelo que sugerimos algumas coisas para edições futuras:
\begin{itemize}
\item Caso esteja de acordo com a política financeira do Núcleo, uma parte do custo (que este ano foi de 17€) poderá ser suportado pelo Núcleo (isto ia contra os princípios financeiros que nos regemos este ano, uma vez que se tratava de uma atividade recreativa, pelo que não o fizémos) para que o preço para os sócios do NEEEC seja mais reduzido;
\item O evento poderia ser realizado com todos os núcleos do Pólo 2 permitindo assim um maior número de inscritos, menor número de pessoas de cada núcleo e a ter que ir e, potencialmente, um menor custo por participante. Contudo, pela nossa experiência em eventos conjuntos do Polo 2, tal poderia não ser, no entanto, uma boa ideia.
\end{itemize}

A divulgação do evento foi feita andas da Queima das Fitas pelo que houve imenso tempo para as pessoas pensarem no evento. O facto do evento ser tão próximo da Queima faz com que as pessoas tenham pouco dinheiro para a atividade mas é difícil mudá-la de data devido às condições climatéricas.

No dia do evento foi necessário levar o dinheiro para pagar à empresa sendo essencial a presença de um elemento da equipa do Pelouro. Com 7 participantes realizou-se a descida do mondego em kayak, finalmente, após tantos anos de insistência. Estava uma dia de muito sol e calor o que fez com que a atividade fosse ainda melhor. Obteve-se um bom feedback dos participantes e todos gostaram.

Em futuras edições, nomeadamente se a adesão for maior, poderá envolver-se esta atividade junto de outras que promovam algumas receitas que impeçam esta atividade ser tão cara.
