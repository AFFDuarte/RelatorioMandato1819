% ================================
% # Semana Cultural e Desportiva #
% ================================

\subsubsection{Semana Cultural e Desportiva}

Pela primeira vez, foi realizada um Semana Desportiva e Cultural (“NEEEC Sports \& Culture Week”) organizada em parceria com o Pelouro da Cultura e Lazer. Durante esta semana (segunda, terça, quarta e quinta) foram propostas diversas atividades aos alunos do \acrshort{deec}, tanto de cariz desportivo como de cariz mais cultural/lúdico. Este era um evento completamente novo no \acrshort{neeec} pelo que foi organizado sem bases em atividades anteriores, não se sabendo o que se esperar do evento. 

No primeiro dia, de manhã, realizou-se uma feira com várias secções da \acrshort{aac}. Após um email enviado para todas as secções desportivas da \acrshort{aac}, obtivemos apenas resposta de uma (Secção de Halterofilismo). Pela tarde o objetivo era realizar um torneio de ping-pong na sala de convívio, mas com a falta de participantes aliada à falta de pessoas na sala de convívio a essa hora, o torneio acabou por ser cancelado. Por último, para o fim da tarde estava marcada uma pequena corrida (5km) mas como começou a chover e o número de participantes também era baixo a corrida também foi cancelada.

No segundo dia, terça feira, o Pelouro de Desporto não fez nenhuma atividade.

No terceiro dia, marcado para a noite estava um torneio de basket 3x3, este só contava com uma equipa inscrita, então este torneio também foi cancelado. 

Para finalizar, no último dia, quinta-feira, foi organizado a última atividade, um torneio de bowling, que contou com cerca de 9 participantes que não pertenciam ao núcleo. Este torneio foi bem sucedido tendo tido um feedback positivo dos participantes e reunido vários interessados que não puderam participar por já ter começado o torneio.

A semana em geral é um conceito interessante e que deve ser repetido no futuro, sendo obviamente repensada. Cada torneio deve ser organizado com mais detalhe, com regras explícitas. É também importante que existam prémios aliciantes e que estes sejam divulgados. O evento foi realizado no início de março mas já muito próximo do início das avaliações pelo que, para que seja bem sucedido, deve ser recolocado numa altura com melhores condições meteorológicas e menos avaliações, talvez no início de um semestre.