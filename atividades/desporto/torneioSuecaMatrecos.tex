% ===================================
% # Torneio de Sueca e Matraquilhos #
% ===================================

\subsubsection{Torneio de Sueca e Matraquilhos}

Com a realização do tradicional magusto, responsabilidade do Pelouro da Cultura e Lazer, é habitual fazer-se um torneio de sueca. Este ano, decidimos também fazer um torneio de matraquilhos. Para tal, foi feito e divulgado um regulamento para cada torneio, essenciais para o bom funcionamento do mesmo, e foram anunciados prémios, que consistiam em finos, para os vencedores.

O evento acabou por ter vários problemas uma vez que este se realizou a uma quinta-feira, havendo aulas e frequências em simultâneo, o que dificultou a presença de participantes e pessoas para trabalhar na atividade. Por sua vez, o CG do Pelouro decidiu atrasar em duas horas o início do torneio para tentar ter mais inscritos. Assim, quem podia à hora marcada deixou de poder e quem não podia não soube da alteração pelo que também não participou. Ambos os torneios acabaram por se realizar com poucas equipas (cerca de 3) tendo o mesmo prolongado-se até muito tarde, muito para além da hora prevista para o fim da escala, fazendo com que membros do Pelouro tivessem de se revezar entre si para conseguir terminar o evento. Por sua vez, o CG, meia hora após o início dos torneios, decidiu ir para Viseu sem avisar ninguém, tendo deixado dois membros do Pelouro encarregue dos dois eventos, tendo sido, portanto, demitido no dia seguinte. Um dos membros do Pelouro era também participante do torneio que estava a vigiar o que dificultou, ainda mais, a situação.

Uma vez que o magusto é feito nos jardins do bar e anoitece cedo em novembro, teve ainda de se arranjar soluções, à pressa, para mudar de local quando anoiteceu. Ambos os torneios, apesar de terem poucos participantes, terminaram numa altura em que já quase ninguém estava no departamento, havendo inclusive desclassificações porque equipas precisavam de se ir embora. Em suma, este foi um evento super desorganizado que, obviamente, deve ter continuidade mas deve ser organizado de forma muito mais ponderada.