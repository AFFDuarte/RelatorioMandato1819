% ==========================
% # Assembleias de Núcleos #
% ==========================

\subsubsection{Assembleias de Núcleos}

\paragraph{Ida às \acrfullpl{an}}

As \acrfullpl{an} são na nossa opinião importantes para a tomada de decisões dentro do Conselho Inter-Núcleos, apesar de estarem a perder tempo de debate sobre assuntos essenciais aos Núcleos como a Pedagogia e as Saídas Profissionais. Este ano, enfrentámos um ano de revisão do Regulamento Interno da Assembleia de Núcleos e das distribuição das Verbas da Queima das Fitas referente aos Núcleos de Estudantes, tendo sempre uma posição ativa neste como noutros assuntos que afetam mais diretamente os nossos sócios.

As \acrfullpl{an} são convocadas mensalmente pelo Presidente da \acrlong{dg}, cujas datas são decididas, normalmente, na \acrshort{an} anterior. A representação do Núcleo é feita pelo Presidente e por quem este quiser que o acompanhe, tendo sido hábito ser o Vice-Presidente a acompanhá-lo. Contudo, em \acrshortpl{an} em que se saiba que irão ser discutidos tópicos de determinadas áreas deverão ser levadas as pessoas respetivas. Esta não deixa de ser uma tarefa difícil uma vez que em cada \acrshort{an} existem vários assuntos a debater.

\paragraph{\acrfullpl{an} no \acrshort{deec}}

De há poucos anos para cá, as \acrfullpl{an} realizam-se em locais diferentes sendo um Núcleo anfitrião em cada mês. No final destas \acrshortpl{an} é também costume haver um barril que é pago pelo Núcleo anfitrião. Quando há uma \acrshort{an} no Polo 2, o barril é pago por todos os núcleos. Desta forma, desde o início do ano, houve algum interesse em realizar uma no \acrshort{deec}, que foi a primeira realizada neste Departamento.

Para trazer a \acrshort{an} para o \acrshort{deec} foi preciso esperar que passasse algum tempo desde a última realizada no Polo 2, propor o local na assembleia anterior e esperar ter o consenso de todos. Desta forma só foi possível realizar a \acrshort{an} no \acrshort{deec}, em fevereiro. Este é um evento que trás prejuízo para o Núcleo (valor do barril a dividir por 7 núcleos, o que neste caso, é 9,14€ o que, dado o seu propósito, vai um pouco contra a política do Núcleo, mas achámos por bem ter o \acrshort{deec} representado através desta receção. De notar que o valor dos barris não costuma ser pago pelos outros núcleos estando, no final deste mandato, ainda por receber o valor de todos os núcleos exceto do \acrshort{nei} e do \acrshort{neemaac}, núcleos que por terem recebido assembleias nos seus departamentos saldaram automaticamente as contas connosco.

A \acrshort{an} realizou-se na sala T4.1 com a habitual disposição em "U". A dimensão da sala adequou-se bastante ao evento, havendo espaço para todos se sentirem confortáveis. Contudo, o eco da mesma prejudicou bastante a sonoridade, sendo necessário a Coordenadora da assembleia apelar ao silêncio para se conseguir fazer ouvir. Foi criada uma rede wireless pois naquela sala a rede é a do \acrshort{deec} e os restantes estudantes, de fora do \acrshort{deec}, não tinham acesso à mesma. Esta rede foi muito positiva para o desenrolar dos trabalhos. O Núcleo disponibilizou também extensões para todos, tendo sido verificado no final se a contagem inicial e final batiam certo, para evitar roubos que já aconteceram noutras ocasiões. No final, decorreu o habitual convívio no jardim do Núcleo tendo havido uma pequena preocupação para manter tudo arrumado pelo que no dia seguinte tudo se encontrava em ordem para a normal atividade do \acrshort{neeec}. 
