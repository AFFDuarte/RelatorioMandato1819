% ==========================
% # Manifestações          #
% ==========================

\subsubsection{Manifestações}

\paragraph{Basta}

Após sucessivos problemas na qualidade do ensino superior, a \acrshort{dg} entendeu levar a Assembleia Magna uma moção para que, por ocasião das comemorações do dia do estudante, houvesse uma manifestação com o principal intuito de denunciar o aumento de propinas e taxas suportadas pelas famílias, o desrespeito pelo papel dos estudantes na governação das instituições de ensino e o desinvestimento generalizado no setor.

Esta manifestação foi marcada para o dia 21 de março, uma quarta-feira, tendo tido uma elevada campanha de mobilização em toda a \acrshort{aac} nomeadamente, através da divulgação no Facebook e dos núcleos. A cada Núcleo foi dada a oportunidade de pintar uma faixa própria e exclusiva, que depois pode levar na manifestação, com uma reivindicação que entendesse. Uma vez que se tinha realizado recentemente um fórum pedagógico onde foi muito discutida a falta de capacidade, por parte do \acrshort{deec}, na contratação de professores, o mote levado pelo \acrshort{neeec} para a manifestação foi “A \acrshort{uc} não vai para a frente com um corpo docente insuficiente”.

Esta manifestação correu bem de uma forma geral tendo tido uma adesão significativa. De realçar, no entanto, que os membros do \acrshort{neeec} não se viram muito identificados na manifestação tendo estado presentes apenas os cinco elementos da Direção e cerca de 5 Coordenadores/Colaboradores do \acrshort{neeec}. A pintura da faixa foi algo simples de se fazer, mas, devido ao calendário de avaliações em vigor nessa semana e ao facto da mesma ter de ser pintada na \acrshort{dg}, foi algo que o Presidente do Núcleo teve de fazer sozinho pois mais nenhum membro tinha disponibilidade para colaborar. De salientar que, na nossa opinião, o facto da manifestação ter um tema muito basto, fez com que não se focassem todas as forças num só mote perdendo assim a força e credibilidade da manifestação.

\paragraph{25 de abril}

No dia 25 de abril existe uma manifestação popular à qual a \acrshort{aac} se costuma juntar. Este ano, uma vez que a manifestação se sobrepôs com outros eventos da própria académica, como um jogo da OAF e uma atividade do Núcleo, o Passeio de Bicicleta, não foi possível o \acrshort{neeec} estar representado. De notar, no entanto, que a adesão dos núcleos e dos próprios da \acrshort{dg} parece ter sido fraca tendo o feedback dos presentes sido também ele muito fraco.