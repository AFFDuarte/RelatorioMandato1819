% ==========================
% # Tomada de Posse        #
% ==========================

\subsubsection{Tomada de Posse}

A tomada de posse dos novos órgãos gerentes do \acrshort{neeec} inclui alguns detalhes protocolares quer pela tradição quer pelo facto de envolver dois mandatos distintos. Em 2017, os convites e a marcação da data foi feita pela Presidente cessante do \acrshort{neeec}, a imagem foi feita pelo CG da Imagem possante, a divulgação de cartazes e dos eventos no Facebook foi feita pelo Presidente possante e a cachorrada foi organizada pelo Administrador e pelo Tesoureiro possantes. Como é tradição, a tomada de posse decorreu na mesma tarde que a tomada de posse do \acrshort{nei} sendo que, este ano decorreu em primeiro lugar a tomada de posse do \acrshort{neeec} e, de seguida, decorreu a do \acrshort{nei} (tradicionalmente, a ordem das mesmas alterna de ano para ano). Também como é costume, quem fez a tomada de posse em primeiro lugar recebeu a cachorrada conjunta nas suas instalações pelo que esta decorreu nos jardins do \acrshort{neeec}. Existiu ainda uma tentativa da Direção cessante em organizar a tomada de posse do \acrshort{neemaac} no mesmo dia mas tal não foi possível devido a incompatibilidades horárias entre todos os envolvidos.

Em relação à marcação da data e hora esta tem de ser feita em conjunto pelos membros da Direção anterior que queiram assistir à cerimónia e pelos membros possantes. É também necessário verificar a disponibilidade do \acrshort{nei} bem como da Direção do \acrshort{deec}. Este ano, a tomada de posse decorreu a 6 de junho de 2017. Para confirmação da data e hora é necessário informar a \acrshort{dg} que indicará a disponibilidade do Presidente ou de um dos Vice-Presidentes estarem presentes na cerimónia.

A tomada de posse é marcada para uma determinada hora, mas os elementos da \acrshort{dg} chegam constantemente atrasados em mais de meia hora. Este facto é muito mau visto pelos professores e pelos alunos que não têm de estar presentes na cerimónia e não estão habituados a este tipo de situações e provoca também atrasos na programação do dia. Em 2017, o Diretor do Departamento (professor Hélder Araújo) foi-se embora antes da cerimónia começar devido ao atraso existente sendo necessário chamar o Vice-Diretor do Departamento, à pressa, para se poder começar a cerimónia. Desta forma, é aconselhável em edições futuras marcar uma hora com a \acrshort{dg} e apenas informar os professores de uma hora mais tardia. Deve-se ter, no entanto, alguns cuidados com a data a divulgar em público pois os alunos do \acrshort{deec} chegam a horas mas os elementos de outros núcleos, como já sabem como costuma funcionar este tipo de coisas, chegam também atrasadíssimos. Quando a tomada de posse do \acrshort{neeec} ocorre imediatamente após a do \acrshort{nei} (o que não foi o caso de 2017) deve-se ter em atenção que, caso as anteriores atrasem (o que sucede quase sempre) a tomada de posse do \acrshort{neeec} atrasará também o que, novamente, não é bem entendido por professores e alunos.

Antes da tomada de posse começar, a equipa da área dos núcleos solicitou o preenchimento de alguns documentos com algumas informações. Para facilitar o processo (e para que os documentos ficassem logo preenchidos evitando, assim, atrasos nos processos seguintes), o Secretário do Núcleo levou consigo um documento onde tinha os nomes completos, números de estudante e cartão de cidadão de todos os membros da lista. Nos documentos solicitados foi também solicitado o horário do Núcleo (foi introduzido o novo horário que entrou em vigor a partir de setembro, que já havia sido combinado) e os estudantes com direito a estatuto. Neste último campo, é solicitada a lista dos estudantes a receber estatuto pela primeira vez e outra lista dos estudantes que já têm estatuto e que deverão continuar a ter. Aqui, preenchemos, erradamente o documento pois nos estudantes que já têm estatuto devem ser inseridos todos os membros do mandato anterior (o estatuto dura dois anos) e nos estudantes a receber estatuto pela primeira vez devem ser colocados os membros do novo mandato que não tiveram estatuto de dirigente associativo no mandato que termina.

Em 2017, como costume, foram colocados pequenos papéis a identificar o nome e os cargos dos intervenientes da mesa (pela ordem Presidente cessante, Coordenador de curso do MiEEC, Presidente da \acrshort{dg}, Diretor do \acrshort{deec} e Presidente possante). Estes papéis são adequados, mas devem ser feitos em tamanho maior para que toda a gente na sala os consiga ler, o que não aconteceu. Ao contrário de alguns anos anteriores, não existiu nenhum ramo de flores, nada projetado na tela (por exemplo, o símbolo do \acrshort{neeec} ou uma imagem da tomada de posse) nem sistema de som (dada a acústica da sala não parece ser necessário). Este ano também não houve bandeira da académica devido a uma falha da \acrshort{dg} pelo que se teve de tapar o objeto que costuma simular o púlpito com uma capa preta (este problema não deverá voltar a acontecer uma vez que o \acrshort{neeec} tem agora um púlpito próprio que pode utilizar na cerimónia – caso este púlpito não esteja adequado poderá também solicitar o púlpito da \acrshort{dg}). Foi também acrescentada uma garrafa de água por cada interveniente da mesa.

Quanto à disposição dos lugares, os membros que iriam tomar posse foram sentados no auditório pela ordem que vão tomar posse (é chamado, em primeiro lugar, o último vogal da Mesa do Plenário, percorre-se toda a lista da Mesa do Plenário e repete-se o processo para a Direção). Na Tomada de Posse de 2018, também os suplentes deverão tomar posse, sendo chamados pela mesma ordem.

A cerimónia é presidida pelo Presidente da \acrshort{dg} tendo o seguinte protocolo:
\begin{enumerate}
\item Abertura por parte do Presidente da \acrshort{dg}
\item Discurso do Diretor do \acrshort{deec}
\item Discurso do Coordenador de Curso do \acrshort{mieec}
\item Discurso do Presidente Cessante
\item Tomada de Posse dos novos elementos
\item Discurso do Presidente Possante
\item Discurso do Presidente da \acrshort{dg}
\item F-R-A por parte do novo Presidente do \acrshort{neeec}
\end{enumerate}

No final da cerimónia, é costume nos Núcleos, haver um pequeno lanche com um Porto de honra o que não sucedeu este ano devido às condições económicas do Núcleo. No entanto, recomendamos que tal situação seja resolvida nas próximas edições deste evento. Esta atividade é independente da cachorrada no final do dia.

\ifthenelse{\boolean{biblia}}
{ % TRUE
No final do dia decorreu uma cachorrada, devidamente explicada em \ref{subsec:cachorradaTomadaPosse}.
}
{ % FALSE
}