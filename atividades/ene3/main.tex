% ========================
% # \acrshort{ene3}                 #
% ========================

\section{ENE3}

\subsection{Introdução}

O \acrfull{ene3} é um evento anual organizado por e para estudantes da área, tendo como principal objetivo juntar num espaço alunos vindos de todo o país e proporcionando um vasto leque de atividades no âmbito da Engenharia Eletrotécnica.

O evento conta já com oito edições, fundado em Coimbra em 2007, e esteve de volta à cidade-mãe, pela terceira vez, de 5 a 8 de setembro de 2017.

\subsubsection{Missão}

A missão da Comissão Organizadora para esta edição do \acrshort{ene3} consistiu, essencialmente, nos seguintes pontos:
\begin{itemize}
\item Promover a visibilidade de projetos académicos e profissionais, fortalecendo o exercício da profissão;
\item Realizar um encontro dotado de todas as vertentes que o curso abrange e que o evento permite explorar;
\item Criar um ambiente propício à interação espontânea de participantes entre si e com oradores;
\item Facilitar o acesso de pessoas de todas as regiões ao evento, trazendo-o novamente a Coimbra (por ficar perto do centro do país, sendo quase equidistante das várias regiões, existe uma maior facilidade para estudantes de qualquer canto do país estarem presentes);
\item Atrair a atenção do mercado nacional de Engª Eletrotécnica para a importância de um Encontro Nacional.
\item Trazer ao \acrshort{ene3} temas de conhecimento emergentes, oradores nacionais e internacionais de elevada qualidade técnica e oferecer sessões paralelas para que os participantes possam escolher assistir a palestras/workshops, bem como alargar o leque de oferta de sessões práticas.
\end{itemize}

\subsection{Atividades}

A atividade começou no dia 5, terça-feira, com a chegada dos participantes durante toda a manhã e consequente sessão de abertura onde estiveram presentes os participantes do evento, o Diretor do Departamento de Engenharia Eletrotécnica e de Computadores, um dos Vice-Diretores da Faculdade de Ciência e Tecnologia, o Presidente do \acrshort{neeec}, o Vice-Chair do \acrshort{ieeeuc} e um dos Coordenadores Gerais do evento. Após o almoço, dividimos os participantes em vários grupos para que estes realizassem uma visita aos laboratórios de investigação sediados no Departamento de Engenharia Eletrotécnica e de Computadores bem como aos espaços comuns deste edifício. Seguiu-se uma deslocação até às instalações da Ordem dos Engenheiros da Região Centro onde, durante a restante parte da tarde, os participantes puderam acompanhar o Workshop Game Changers. Neste evento, foi possível contar com várias sessões com mesas redondas, workshops e talks de várias empresas diferentes e da própria Ordem dos Engenheiros. O dia terminou com uma febrada nos Jardins da Associação Académica de Coimbra seguidas de algumas atividades para que os participantes se conhecessem melhor entre si e ainda uma apresentação musical de grupos de tunas e fados.

No segundo dia, a manhã foi preenchida com cinco workshops diferentes. Cada um dos participantes pode escolher um dos temas tendo assim a oportunidade de aprofundar os seus conhecimentos em Machine Learning, KiCAD, validação de software crítico, desenho CAD ou setor elétrico. Após o almoço, a tarde seguiu com talks a cargo da WEG, da \acrfull{oe} e da CELFINET, três dos nossos maiores patrocinadores. O final da tarde foi dedicado a uma feira de emprego que contou com mais de uma dezena de stands de empresas enquanto em paralelo decorriam workshops na área da Data Science, Setor energético e desenho de PCB’s. Os participantes puderam visitar a feira e, de seguida, acompanhar um dos workshops disponíveis. A feira contou ainda com várias atividades de dinamização como a oferta de gelados, jogos de matrecos e ping pong o que proporcionou uma tarde muito agradável a todos os participantes enquanto conheciam um pouco mais do mundo empresarial. O dia findou com um quiz de cultura geral, engenharia eletrotécnica e perguntas dedicadas ao \acrshort{ene3}, onde os participantes puderam competir entre si, sendo os vencedores premiados com visitas a empresas patrocinadoras.

No terceiro dia, a manhã começou novamente com uma panóplia de 5 workshops, desta vez dedicados às linguagens WEB, ao Bitalino, à Internet of Things, à programação de robôs usando ROS e à Eletrónica e Microcontroladores. Ainda antes do almoço, os participantes puderam ouvir uma palestra da EDP sobre o tema Indústria 4.0. A tarde teve início com uma das sessões mais cativantes do evento: a mesa redonda “Do Sonho à Realidade” trouxe ao \acrshort{ene3}, João Rafael Koehler, Paulo Marques e João Bernardo Parreira numa sessão dirigida por João Barreto onde o empreendedorismo e a inovação foram as palavras de ordem. De seguida, os participantes foram até ao Pátio das Escolas onde foram divididos por equipas e iniciaram um peddy tascas e peddy paper pelos locais mais emblemáticos da cidade, peddy esse que terminou numa febrada no Campo de Santa Cruz, já no final da noite.

O último dia contou com várias sessões de recrutamento e com a sessão de encerramento onde foram discutidos, com todos os participantes, os detalhes do evento e onde este se deveria realizar no ano seguinte.

\subsection{Disposições Finais}

Ao sabermos que tínhamos a nosso cargo a organização de um evento de tão grande dimensão logística como este, abraçámos este projeto tendo como objetivos principais a dinamização do evento e o aumento da qualidade técnica do mesmo. Chegamos ao fim com o sentimento de trabalho realizado. O evento foi a atividade que mais patrocínios contou na história do \acrshort{neeec}, tendo sido um excelente meio de aproximação a inúmeras empresas. Contou também com a presença do maior número de faculdades diferentes desde que o \acrshort{ene3} foi criado, o que nos deixa muito felizes. Um dos objetivos, o aumento do número de participantes, ficou um pouco aquém do desejável porque, apesar de termos sido uma das edições com maior número de participantes ficámos aquém do nosso objetivo de 200 participantes, o que achamos que é mais que exequível num evento como este.