% ========================
% # F3E                  #
% ========================

\section{F3E}

A segunda edição da \acrlong{f3e} realizou-se nos dias 25 e 26 de setembro de 2017, data escolhida pelo facto de ser a altura em que os estudantes finalistas entregam as suas teses, finalizando o seu curso, e começam à procura de emprego. 

Os moldes da feira foram semelhantes ao ano anterior, tendo, no entanto, havido algumas alterações que passamos agora a descrever.

O primeiro dia foi inteiramente dedicado às \textit{soft skills}, com os workshops de LinkedIn (lecionados pela Galileu), os workshops de CV (lecionados pela Flag) e as simulações de entrevista de emprego (lecionados pela Sónia Teles do Gabinete das Saídas Profissionais da \acrshort{aac}).

O segundo dia foi dedicado às empresas, onde cada uma tinha uma banca no corredor do piso 4, e, caso o pacote o incluísse, poderia fazer uma palestra temática, recrutar e ter acesso aos currículos dos participantes do evento. A criação de um dia único de bancas foi um molde que correu bastante bem. No final do dia decorreu ainda um Lanche Networking na Antiga Biblioteca. O facto deste lanche ser num espaço fechado, associado ao facto da hora coincidir com a hora de desmontar as bancas fez com que o lanche não tivesse a adesão nem a dinamização desejadas contudo, recomendamos a manutenção desta atividade e, acima de tudo, a sua dinamização em futuras edições. Por exemplo, poderá ser feito no bar a uma hora de maior movimento. Este tipo de atividade é, sem dúvida, o melhor meio de contacto entre alunos e empregadores.

A sessão de empreendedorismo esteve, este ano, a cargo do João Parreira, do Cristiano Alves e do João Freitas numa sessão moderada pela Tânia Covas, da \acrshort{dits}. Esta sessão foi um sucesso mas foi divulgada muito em cima da hora podendo ter tido ainda uma adesão muito maior.

Logisticamente, este ano não criámos um parque de estacionamento para as empresas no Piso 6. Em substituição, demos-lhes acesso ao estacionamento na garagem do departamento que tem, através de elevador, acesso direto ao local da feira. Uma vez que o site do \acrshort{neeec} já se encontrava concluído, criámos uma página dedicada à \acrshort{f3e}, contudo, esta página foi extremamente fraca e já feita apenas após a feira, uma lacuna extremamente grave pois os sites são importantíssimos para as empresas.

Nesta edição, voltámos a investir bastante na publicidade e dinamização do evento tendo a divulgação do mesmo sido iniciada a meio do mês de agosto. Contudo, o facto de termos o \acrshort{ene3} no início do mesmo mês que a \acrshort{f3e} fez com que houvesse alguma sobreposição na divulgação dos dois eventos ainda para mais quando os temas, nomeadamente patrocínios, \textit{soft skills}, entre outros, eram exatamente os mesmos em ambas as atividades. Nos dias que antecederam a feira foi colocada uma lona de grandes dimensões no corrimão das escadas entre o \acrshort{dei} e o \acrshort{deec} que publicitava a feira e os seus patrocinadores. Esta lona foi um sucesso dado que publicitava o evento a todos os que entravam no Polo 2 pela entrada mais junto ao rio, contudo deve ser colocada, em futuras edições, mais cedo.

Este ano, no que toca a \textit{soft skills}, teve temas muito semelhantes ao ano anterior mas, no futuro, recomendamos que a feira seja dinamizada para que não caia na normalidade e deixe de ser um evento tão frequentado como é habitualmente. Note-se que o facto da feira ser vocacionada para todos os curso de engenharia e não apenas para o MiEEC faz com que esta tenha um público muito maior e diverso do que seria normal noutra situação.

Deixamos ainda, uma sugestão que foi dada por uma empresa: numa futura edição, poderia existir um boletim de selos, ou algo similar, para que os estudantes fossem visitar as bancas das empresas de modo a que não haja a possibilidade de algumas estarem cheias e outras vazias. A ideia seria as empresas terem na sua posse um conjunto de selos e os estudantes interessados teriam um boletim. Quando estes fossem a cada uma das banca recebiam um selo no seu boletim e no final juntavam-se todos os boletins preenchidos e fazia-se um sorteio, o estudante que ganhasse o sorteio teria direito a um prémio, que pode ser arranjado através de um patrocínio.

\subsection{Primeira Edição (2016)}

Por não existir nenhum relatório da primeira edição desta atividade e uma vez que as decisões tomadas na primeira edição de cada atividade são, quase sempre, as que melhor definem os eventos, deixamos aqui o relatório da 1ª edição da \acrshort{f3e}.

\paragraph{Missão}

Há muitos anos que se falava em ter uma feira de emprego no \acrshort{deec}. A feira de emprego da \acrshort{uc} é centralizada não tendo nenhuma área dedicada à engenharia e a feira de emprego do \acrfull{isec} cresce a olhos vistos de ano para ano sendo uma forte atividade no local. Por sua vez, a ligação do \acrshort{deec} às empresas escasseia cada vez mais sendo importante reverter esta situação. Não existem feiras semelhantes no Polo 2 pelo que nasceu assim a ideia de criar esta feira aberta a todo o Polo 2, com empresas patrocinadoras que apoiam o evento/o Núcleo de forma a que possamos dar um evento de ainda melhor qualidade. 

\paragraph{Escolha das datas}

Decisão: Sendo a primeira edição da feira optámos por colocar a feira no final do mês de setembro. Nesta data, os alunos ainda não têm trabalhos para fazer nem estão dedicados ao estudo para as avaliações mostrando mais disponibilidade. Por sua vez, a maioria dos finalistas, acabaram de entregar a tese e encontram-se à procura de emprego. Optámos então pelos dias 27 e 28 de setembro de 2016, uma terça e quarta-feira. Nesta altura também não existe concorrência no que toca a feiras de emprego de outros núcleos, da \acrshort{aac}, do \acrshort{isec} ou da \acrshort{uc} como acontece ao longo de todo o 2º semestre. 

Resultado: A data foi sem dúvida uma aposta positiva apresentando apenas o seguinte problema: o mandato começa em junho pelo que a feira tem de ser logo trabalhada a partir do início do mandato, o que faz com que seja totalmente planeada no verão possibilitando assim um trabalho melhor. Contudo em setembro existem muitas atividades do Núcleo em si o que faz com que a feira fique muito dependente do Pelouro das Saídas Profissionais, em exclusivo. O mês de agosto também é mau para estabelecer contactos com empresas pelo que os contactos têm de ser estabelecidos logo em julho e insistidos em agosto mas já numa fase de finalização. Em suma, tendo em conta todos estes aspetos, a data é perfeita e, na nossa opinião, não deve ser alterada nas próximas edição.  

\paragraph{Estrutura do Programa}

Decisão: Optámos por ter uma feira de emprego exposta no piso 4 durante os dois dias da feira e várias atividades a decorrer em simultâneo, como workshops e sessões com empresas. A feira realizou-se numa terça por ser um dia muito movimentado do \acrshort{deec} e numa quarta por ser um dia em que as pessoas não têm aulas pelo que estão mais disponíveis. 

Resultado: A terça feira foi um tremendo sucesso tendo tido imensos participantes quer propositados quer curiosos que estavam no Departamento e acabaram por visitar a feira. A quarta-feira revelou-se um fiasco. Estando no início do semestre houve um convívio na noite anterior pelo que o Departamento em si encontrava-se completamente vazio. As empresas não gostaram desse dia uma vez que quase ninguém as visitou e chegaram a circular rumores pelos feirantes que a Latada já tinha começado, o que era mentira. As palestras tiveram poucas pessoas comparando com terça-feira. Na nossa opinião, o dia de quarta-feira deve ser retirado da feira pois não tem qualquer impacto.  

\paragraph{Público-alvo}

Decisão: Uma vez que a área de engenharia eletrotécnica é muito abrangente achámos que a feira não devia ter só como público alvo os alunos do \acrshort{deec} mas sim da área das engenharia no geral. Desta forma aumentamos o público alvo o que nos trás mais participantes, agrada as empresas e aumenta a qualidade do evento. 

Resultado: O aumento do público alvo foi um sucesso sendo que mais de 60\% dos participantes eram de fora do \acrshort{deec}.  

\paragraph{Divulgação}
Esta foi uma das apostas mais fortes da nossa edição. A nível de redes sociais optámos por criar uma página de Facebook independente do Núcleo uma vez que o público-alvo é todo o Polo 2 e não apenas o \acrshort{deec}. A divulgação teve de começar muito cedo, a meio de agosto para que o conceito começasse a entrar na cabeça das pessoas. Em setembro, produzimos panfletos que foram entregues na banca dos caloiros. Distribuímos durante vários dias na semana antes da feira os panfletos pelo bar do \acrshort{deec}, pelas cantinas e por outros pontos de passagem dos outros departamentos. Fomos também aos outros núcleos e pedimos a afixação de cartazes. Por fim, divulgámos a feira através do InforEstudante chegando assim a toda a \acrshort{fctuc} facilmente e contámos com o apoio do Diário As Beiras tendo saído duas notícias e um anúncio neste jornal, de forma gratuita. A divulgação no jornal não capta muito público mas é essencial para a apresentação da feira à comunidade geral, de forma a se saber que ela existe, e de forma a divulgar as empresas. As empresas adoraram este fator. 

Sugestões: Ficou a faltar divulgação no polo 1 e no \acrshort{isec} o que pode aumentar o público a vir. No que toca a media, ficámos apenas pel’As Beiras o que foi mau. Fica a faltar a construção de um site, devido à inexistência do site do \acrshort{neeec} o que é muito negativo pois este é um fator muito importante para as empresas e é também uma forma muito mais organizada de dispor a informação. 

\paragraph{Disposição do espaço}

Optámos por colocar as empresas ao longo de todo o corredor do piso 4 o que se revelou um autêntico sucesso. Este é um espaço de passagem obrigatório para quase todos os membros do Departamento obrigando assim as pessoas a visitar a feira. 

As sessões com as empresas realizaram-se na Antiga Biblioteca criando assim um espaço próximo com os participantes e fazendo com que a sala tenha sempre um efeito de sala cheia. 

Os workshops realizaram-se no Laboratório de Apoio Informático 1, localizado em frente à sala de convívio por ter vários computadores (atualmente este laboratório já não existe). 

Sugestões: Não temos grandes alterações a sugerir em relação à disposição do espaço uma vez que todas as atividades funcionaram muito bem nos locais em que se realizaram. 

\paragraph{Logística}

Em todo o Departamento foi colocada sinalética indicando os espaços da feira o que foi ótimo para as pessoas que visitavam a feira e não eram do Departamento. 

Foi criado um espaço de estacionamento exclusivo para as empresas no estacionamento do piso 6 permitindo que as mesmas entrassem logo com os seus objetos pelo elevador da torre do bar. 

Nas salas foi sempre identificado o nome da sala e o que lá ia decorrer e nas bancas do corredor, foi identificado o nome da empresa que ali se encontrava. 

As mesas utilizadas para as bancas foram as da Antiga Biblioteca o que dava um ar mais apresentável às bancas. Foram disponibilizados, de raiz, 2 cadeiras por cada mesa. 

Houve uma rede wireless dedicada ao evento e também uma conta para os PCs. Falhou no entanto a rede junto das bancas, algo essencial para as empresas poderem trabalhar. Houve também distribuição de eletricidade para todas as bancas. 

A todos os oradores foram fornecidas águas e houve também um patrocínio da Delta havendo assim oferta de café a todos, na Antiga Biblioteca. 

Houve ainda uma banca de distribuição de jornais As Beiras localizada junto às bancas. 

\paragraph{Formações e sessões com empresas}

Os workshops foram dedicados ao currículo, ao LinkedIn e à simulação de entrevistas, todos dados pela Sónia Teles do Gabinete das Saídas Profissionais da \acrshort{aac}. Os workshops tiveram todos muito pouco tempo de duração e, como estavam todos seguidos, provocaram atrasos sucessivos o que não foi bom. Por sua vez, devido à enorme adesão que os workshops tiveram, foram criadas sessões extra, não tão bem pensadas no que toca a horários tendo provocado os problemas referidos, durante o decorrer do evento. A formadora também apresentou alguns atrasos o que foi negativo na logística do tempo. 

Sugestão: Aumentar o intervalo entre as sessões, aumentar a duração das mesmas e separar as sessões sobre currículo e LinkedIn. 