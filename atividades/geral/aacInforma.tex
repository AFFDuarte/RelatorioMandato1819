% ========================
% # AAC (in)Forma        #
% ========================

\subsubsection{AAC (in)Forma}

Todos os anos a \acrshort{dg} leva a cabo sessões que têm como objetivo formar os núcleos de estudantes recém-empossados para o trabalho a levar a cabo durante o ano.

No ano de 2017, a \acrshort{dg} organizou o \acrshort{aac} (in)Forma em duas sessões, iguais, numa sexta-feira e num sábado (30 de junho e 1 de julho, respetivamente) sendo que os elementos dos NE’s podiam ir às sessões dos temas que lhes interessavam em qualquer um dos dois dias. Após votação no \acrshort{cin} anterior, a sessão de sexta-feira decorreu no Departamento de Física e a sessão de sábado decorreu no Departamento de Mecânica (ambos os departamentos foram propostos pelos NE’s que deles tomam conta pelo que, em futuras edições, caso funcione da mesma forma, o \acrshort{deec} poderá albergar este evento caso o \acrshort{neeec} se disponha a tal na \acrshort{an}).

As sessões propostas pela \acrshort{dg} para este \acrshort{aac} (in)Forma foram as seguintes:
\begin{itemize}
\item Política Educativa
\item Órgãos de Governo
\item Ação Social
\item Administração e Tesouraria
\item \acrshort{gape}
\item Pedagogia
\item Saídas Profissionais
\item Relações Externas
\item Cultura
\item Desporto
\item Relações Internacionais
\item Comunicação e Imagem
\item Intervenção Cívica
\item Novos Estatutos da \acrshort{aac}
\end{itemize}

Após o CIN, a Direção do \acrshort{neeec} informou de imediato os seus CGs para que estes investigassem junto das suas equipas sobre quem queria ou não ir às sessões, não exigindo nenhum mínimo de presenças sobre estes mas insistindo na importância de ter, pelo menos, os CGs presentes. Alguns pelouros, nomeadamente a Pedagogia, tiveram uma adesão enorme à atividade indo quase toda a equipa à sessão da Pedagogia e mais de metade da mesma à sessão do \acrshort{gape}, mas tal não se verificou muito nas restantes sessões. Estiveram presentes nas sessões das áreas que representam os CGs da Pedagogia e \acrshort{gape}, das Relações Externas e Comunicação, da Cultura e Lazer e da Imagem e ainda o Presidente e o Secretário da Mesa do Plenário. A CG das Saídas Profissionais e Formação não pode estar presente, mas assegurou de imediato a presença de um dos seus Coordenadores, enquanto que o CG do Pelouro do Desporto não esteve presente nem providenciou ninguém para estar presente. O Administrador do Núcleo também não esteve presente na sessão da sua área não tendo providenciado ninguém do seu Pelouro para o substituir, baseando-se na presença dos restantes elementos da Direção para lhe transmitir a informação. Da Direção do Núcleo, o Presidente, o Vice-Presidente e o Tesoureiro estiveram presentes em todas as sessões (exceto quando estas eram sobrepostas em que estes tiveram de se dividir entre si).

Quanto às sessões no geral, o tempo e qualidade das mesmas bem como o facto da orgânica da \acrshort{dg} não ser exatamente igual à do \acrshort{neeec} não possibilitam que este evento sirva, nem de perto nem de longe, como evento único de formação para os elementos do \acrshort{neeec}. Acontece ainda que existem sessões muito boas, como a da Comunicação, na qual não esteve presente o CG da área em questão pelo que este perdeu uma oportunidade de formação, e sessões, como a das Saídas Profissionais, de muito fraca qualidade não permitindo qualquer tipo de formação aos elementos que estiveram presentes. Por sua vez, temas como os órgãos de governo, em que quase nenhum elemento do \acrshort{neeec} sabe do assunto, não têm adesão pelo que o tema continua sem ser sabido pela equipa. Por sua vez, de notar que a sessão sobre os estatutos da \acrshort{aac} foi de elevada qualidade e foi fulcral quer para a Direção bem como para a Mesa do Plenário poderem aplicar os estatutos nos diversos regulamentos que tiveram de redigir, aumentando assim a qualidade dos regulamentos redigidos e também a real e fácil aplicação dos mesmos à orgânica do \acrshort{neeec}.

Posto isto, a nossa sugestão para o futuro é:
\begin{itemize}
\item Manter a presença do \acrshort{neeec} neste tipo de eventos e garantir a presença obrigatória quer dos CGs, quer dos Colaboradores.
\item Elaborar formações internas (profundas e aplicadas à orgânica do \acrshort{neeec}) sobre estes temas logo no início do mandato para que todos os elementos do \acrshort{neeec} possam saber as informações.
\item Incentivar as diversas equipas do \acrshort{neeec} a tomar conhecimento dos temas das outras equipas de forma a que estas percebam que as equipas do \acrshort{neeec} não são isoladas entre si e que entendam a orgânica do \acrshort{neeec}.
\item Relatar/relembrar à \acrshort{dg}, antes do início das edições deste tipo de eventos, o que correu mal na(s) última(s) para que se evite voltar a ter sessões sem qualidade.
\end{itemize}
