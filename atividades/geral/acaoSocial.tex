% ========================
% # Ação Social          #
% ========================

\subsubsection{Ação Social}

\paragraph{Doação aos Bombeiros}

Aquando dos incêndios do verão de 2017 gerou-se em Portugal uma onda solidária que tinha como objetivo a doação de alimentos aos bombeiros que combatiam os incêndios. Esta foi uma campanha que rapidamente se alastrou a toda a comunidade. O \acrshort{neeec} divulgou então a sua disponibilidade para receber bens na sala do Núcleo. Contudo, provavelmente devido aos já elevados pontos de recolha, a campanha quer em junho, quer em outubro foi um fracasso.

Em dezembro decidimos dedicar o mês solidário à causa dos incêndios tendo angariado bens alimentares, bens materiais e dinheiro. Os bens alimentares e materiais acabaram por ter de se doar à Casa de Infância Elísio de Moura, como é costume noutros anos, uma vez que as quantidades já angariadas para a causa dos incêndios era demasiado grande. Por sua vez, o dinheiro angariado no lanche solidário foi doado a um fundo dedicado ao apoio à reabilitação das áreas agrícolas destruídas.

Para esta campanha, foram instalados caixotes na entrada do \acrshort{deec}, no piso 2, na zona do bar e na sala de convívio. A campanha foi muito divulgada junto dos professores. Todas as atividades de dezembro tinham também entrada gratuita acompanhada da necessidade de se doar um bem alimentar. Desta forma, conseguiu-se uma recolha muito expressiva sendo necessário um carro completamente cheio para transportar os bens.


\paragraph{The Street Store}

Como é habitual todos os anos, a \acrfull{dg} e a \acrfull{sddh} realizaram mais uma edição "The Street Store" onde durante um fim de semana, num conceito muito semelhante a uma loja vulgar, disponibilizaram roupa, calçado, refeições, cuidados de higiene, alguns serviços, entre outras coisas, a população sem-abrigo e carenciada da cidade de Coimbra, proporcionando, também, momentos de convívio e entretenimento.

Dessa forma, solicitaram a ajuda dos núcleos de estudantes para a recolha de bens para a mesma. Desta forma, o \acrshort{neeec} aceitou o convite bastando para isso fazer a habitual divulgação e colocar os caixotes na zona do bar, entrada do \acrshort{deec} e sala de convívio, à semelhança das restantes campanhas solidárias feitas ao longo do ano.

De notar que para obtermos uma maior recolha de bens, enviámos email para os professores sensibilizando-os da campanha em questão. Foram preenchidos cerca de quatro caixotes, sendo que no último dia da campanha membros da \acrshort{dg} vieram ao Núcleo recolher os mesmos. O \acrshort{neeec} cedeu ainda alguns bens alimentares resultantes de sobras de febradas e afins que já não deveriam ser utilizados nas próximos eventos, dado os seus prazos de validade, mas que eram ainda bons para a altura em que se iria realizar a Street Store.

Esta é, sem dúvida, uma atividade que recomendamos o \acrshort{neeec} a participar.
