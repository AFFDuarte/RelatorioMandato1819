% ========================
% # Barraca da Latada    #
% ========================

\subsubsection{Barraca da Latada}

Como é habitual todos os anos, o \acrshort{neeec} teve presente uma barraca durante todas as noites da Festa das Latas no recinto da mesma.

Em 2017, a Festa das Latas contou com várias alterações tendo havido uma reunião a 19 de setembro para apresentação das mesmas. Nessa reunião, os núcleos propuseram a estrutura da tenda dos núcleos com o palco a meio, virado para o rio e o sorteio de apenas uma bebida em vez de três, como proposto pela \acrshort{dg}, propostas essas que foram todas aceites. A \acrshort{dg} informou também que haveria algumas secções culturais e desportivas que fariam parte da tenda dos núcleos havendo assim um total de 34 barracas em vez de apenas 26 e passando a tenda a chamar-se de "Tenda da Academia".

Na Assembleia de Núcleos de 24 de setembro, foi decidida a bebida do nosso Núcleo. Para tal, foi feito um sorteio em que os primeiros a sair poderiam escolher a bebida e seriam os últimos a escolher o local da barraca. Dado o aumento do número de barracas, cada bebida poderia ser escolhida duas vezes para que houvesse bebidas para todos. No sorteio, o \acrshort{neeec} ficou a meio, entre o lugar 20 e o 25, não tendo dificuldade em escolher a bebida (após reunião de Direção, levávamos 5 hipóteses ordenadas de escolha: Safari, Blue Corazon, Absinto, Vinho Tinto/Branco, Amêndoa Amarga e Moscatel). É de referir que todas as bebidas podem ser misturadas com os sumos disponíveis que cada Núcleo entender. Tivemos a mesma bebida que Medicina. Após um dia de experiência em que todos os membros do Núcleo puderam dar a sua opinião, optámos por misturar a bebida com sumo de laranja, ananás ou coca-cola sendo que a mistura com coca-cola foi a que obteve mais preferência (75\%) enquanto que a mistura com laranja foi um verdadeiro fracasso, não tendo sido sequer vendida na última noite. O preço de cada bebida começou por ser de 1 senha = 2 bebida, 2 senhas = 5 bebidas, mas, logo após a primeira noite mudou para apenas 1 senha = 2 bebidas (2 senhas = 4 bebidas, etc), dado o sucesso da bebida e o facto do Núcleo de medicina estar a cobrar 1 senha = 1 bebida na primeira noite, algo que depois também alterou.

As noites do parque da Latada iniciaram-se numa quarta-feira tendo sido anunciado que as barracas estariam prontas para montar a partir da manhã de segunda-feira. Contudo, só foi possível começar a montar na terça-feira à hora de almoço sendo que a barraca do \acrshort{neeec} nem existia ainda a essa hora ficando em pé apenas pelas 17 horas. Tal provocou uma corrida em contra relógio para a montagem da mesma adicionada ao facto de, nessa noite ser a Serenata e ninguém estar disponível para trabalhar na barraca a partir das 21h. Na quarta-feira a barraca já ficou pronta muito em cima da hora de jantar, havendo jantar do Núcleo, e sem extintor nem tabela de preços afixada. Desta forma, todos os pormenores do interior da barraca tiveram de ser feitos já com o recinto aberto.

A construção e design da barraca da latada foi algo bastante trabalhoso e físico, tendo, o design e conceção do mesmo, ficado a cargo do Pelouro da Imagem. Graças à bebida escolhida (Savana, marca branca de Safari), decidimos fazer uma escultura em esferovite com a forma da garrafa, onde o logo da garrafa dizia ELECTRO e tinha um símbolo do \acrshort{neeec} a acompanhar. Pintamos também um placar com a palavra SAFARI, onde o “I” se fazia parecer com um raio e que, com a ajuda do Clube de Robótica, conseguimos fazer com que piscasse, dando também a entender que a palavra escrita era SAFAR, algo que foi alvo de alguma sátira durante a divulgação. A placa que é usual utilizar na barraca, feita pelo Clube de Robótica, que diz ELECTRO com fita de LED’s, foi novamente utilizada, dando um efeito bastante original no contexto das outras barracas.

É ainda de notar a colaboração com o Cristiano Alves para a parte eletrónica da barraca: em julho houve uma reunião para decidir o layout da mesma tendo ficado decidida a versão final do mesmo até ao final de agosto, após análise financeira de todas as hipóteses em cima da mesa. Contudo, o Cristiano, devido à falta de tempo, atrasou-se imenso tendo havido apenas um jogo, que não funcionava a 100\% (jogo do penalti) e mais uma letra iluminada nas placas. Quer o jogo, quer as letras luminosas só ficaram a funcionar na madrugada do primeiro dia o que era também desnecessário. É também de referir que as baterias que alimentavam a barraca foram adquiridas este ano e se encontram nos arrumos do \acrshort{neeec}.

Existiu um jogo, o jogo do penalti, que contabilizava o tempo que uma pessoa demorava a beber uma bebida. Caso a pessoa batesse o recorde imposto até ao momento ganhava uma bebida. O jogo teve elevado sucesso, mas pouca visibilidade uma vez que o mecanismo era minúsculo e só quem estava lá perto é que notava que o jogo existia. É também de notar que o mecanismo avariou a meio da latada, passando o jogo a ser manual e a ter resultados totalmente incertos o que provocou um normal desinteresse das pessoas. Também de referir que nem todos os escalados na barraca, sabiam trabalhar com o jogo uma vez que não foi possível ter o jogo antes da festa começar para se poder treinar. Desta forma, aconselhamos a uma muito maior dinamização dos jogos e desafios existentes no futuro e um maior planeamento da logística dos mesmos.

A barraca tem ainda algumas regras impostas pela ASAE:
\begin{itemize}
    \item É obrigatória a existência de um extintor na barraca. Esta edição, solicitámos o extintor à Direção do Departamento, que nos autorizou a sua cedência, contudo, um acidente no primeiro dia da barraca fez com que o mesmo se abrisse e se esvaziasse ainda no Departamento, o que foi um verdadeiro caos para limpar. Como tal, existe agora um extintor cedido pela Daniela Temudo no arrumo do \acrshort{neeec} que pode ser utilizado exclusivamente para estes fins mas que se encontra fora do prazo (é de ter em conta que as barracas são de facto altamente inflamáveis pelo que a existência de um extintor faz de facto muito sentido). É ainda de notar que o extintor, ficando bem escondido, não costuma ser roubado. Com medo que tal acontecesse, na primeira noite, um dos Coordenadores trouxe o extintor e foi barrado à porta pelos seguranças pelo que, no dia seguinte teve de se ir à PSP para resolver o assunto e reaver o extintor.
    \item É obrigatório o uso de luvas: esta medida, embora correta uma vez que a barraca se torna altamente imunda, é puramente impraticável. Contudo a presença das luvas é obrigatória e foi vigiada na primeira noite por membros da COFL17. Assim, deve ser comprado um número muito reduzido de luvas e colocado na barraca.
    \item É também obrigatória a colocação de um decreto de lei sobre bebidas alcoólicas e é também necessário colocar os preços das senhas. Aconselhamos a que, uma vez que já existe uma máquina encadernadora no Núcleo, estas informações sejam colocadas na barraca em avisos plastificados para que durem todas as noites. Aconselhamos também a uma boa exposição do preço e ainda a uma maior exposição das regras do jogo, caso este exista.
\end{itemize}

Passamos agora a algumas considerações sobre a logística da barraca:
\begin{itemize}
	\item Estrutura: algo essencial é um espaço para armazenamento de bens (casacos, malas, etc), uma boa estrutura para partir o gelo e colocá-lo dentro dos copos, um bom local para armazenamento das garrafas, local para armazenamento do lixo e local para preparar as bebidas. Para tal, deve-se planear atempadamente a estrutura interior da barraca para a aquisição de madeiras tendo, no entanto, em conta que só se sabe como é o interior da barraca (ou seja, onde estão os pilares da mesma) após se lá ir pela primeira vez, o que infelizmente ocorre muito em cima do início da festa.
    \item Caso a bebida seja apenas uma, aconselhamos à utilização de alguns garrafões (não muito mais que dois) para a criação da bebida. Caso o número de bebidas seja maior, aconselhamos o mesmo, mas recomendamos a utilização moderada dos mesmos para não haver sobras. Este ano utilizámos três dispensadores, um para cada bebida, o que nos foi extremamente útil, provocando apenas pequenos problemas quando acabava a preparação da bebida em momentos de muita afluência.
    \item A receita da bebida não deve ser 50-50 nem perto disso pois tal provoca bastante prejuízo e não realça uma melhoria da qualidade da bebida que faça as pessoas consumir mais. Infelizmente, chega-se a um ponto da noite em que as pessoas pagam já só para beber sumo e acham isso normal. Contudo, o facto de a receita ser de proporções diferentes é muito chato para a reposição dos depósitos enquanto estes ainda não estão vazios. É também necessário ter atenção a quem pede a bebida sem gelo pois a quantidade de bebida é quase o dobro.
    \item Chão: aconselhamos à inserção de paletes de madeira em todo o chão por dois motivos: evitar tanta poeira a ser levantada bem como evitar as pessoas estarem tão baixas dentro do balcão.
    \item Madeira: a compra de madeira para a montagem da barraca acaba por sair muito caro pelo que aconselhamos à maior reutilização de materiais possíveis. Na Leroy Merlin é também possível apresentar o \acrshort{neeec} e comprar assim madeira já cortada para outros fins que está na zona de corte e que fica mesmo muito mais barata que as placas novas.
    \item Escala: nesta edição, à semelhança de anos anteriores, a COFL cedeu 5 entradas por noite para a barraca. Daqui temos a referir vários aspetos:
    \begin{itemize}
        \item Ao contrário de anos anteriores, criámos uma escala em que havia 4 turnos (22h – 00h; 00h – 02h; 02h – 04h e 04h-06h com duas pessoas, cada um, e havia um turno adicional da 01h às 05h. Esta medida pareceu-nos extremamente positiva uma vez que permitiu uma maior ordem, contudo não é assim tão descabido, como nos parecia, haver turnos seguidos, por exemplo na primeira parte da noite ou na última parte da noite (22h - 03h / 03h - 06h). Adicionalmente, o recinto nunca abria às 22h em ponto e o primeiro turno era quase sempre suprimido porque ninguém chegava antes das 23h/23h30. Até às 00h é absolutamente desnecessário haver mais do que uma pessoa na barraca, contudo, quando a mesma começa a ter afluência, o trabalho cresce de forma exponencial. Este primeiro turno serve para reposição inicial de stock e quando o mesmo não começa antes da meia noite, já há algum stress com isso.
        \item Os turnos eram compostos por uma pessoa da Direção e um Coordenador. Na nossa opinião, a presença de um elemento da Direção é essencial. O terceiro turno, existente entre as 01h e as 05h era também composto pela Direção o que, poderia ser exagerado numas noites, mas noutras revelou-se fulcral.
        \item Todas as pessoas escaladas devem ter noção da responsabilidade que têm pois as faltas à escala provocam uma enorme confusão. Este ano, a escala também foi feita de forma a ter as pessoas mais dinâmicas e desenrascadas distribuídas pelas várias noites pelo que, a falta destas, numa das noites de maior afluência provocou problemas absolutamente desnecessários.
        \item A Direção, por estar na escala todos os dias, cansa-se mesmo muito com a barraca pelo que se deve ter isso em conta aquando da elaboração da escala pois todos os elementos (este ano, principalmente no caso do Presidente e do Vice-Presidente) acabam por querer aproveitar um dia de recinto para estar com os amigos e depois acabam por nem conseguir prestar atenção à barraca ou à noite de convívio.
        \item Existem alguns Coordenadores Gerais que não gostam de trabalhar na barraca e alguns Colaboradores que sentem pena de não o poder fazer. Contudo, devido ao reduzido número de turnos e pulseiras deve-se ter isso em conta. Este ano, resolveu-se a situação facilmente, uma vez que os 6 Coordenadores Gerais não eram suficientes para cobrir todos os turnos.
        \item Existem vários Coordenadores e Colaboradores que cumprem a escala com rigor mas não trabalham nem mais um minuto para além da mesma pelo que se tem de ter cuidado com quem se atrasa. É também preciso cuidado com quem se disponibiliza a ficar sozinho, por pensar que há pouco trabalho, e que, quando está sozinho, surge uma enchente de afluência, algo que é muito frequente, e, como tal, não consegue dar vazão à mesma.
        \item Noite dos velhos: em muitos núcleos costuma existir uma noite em que trabalham os elementos do Núcleo que já não fazem parte do mesmo, algo que não implementámos por não nos ter sido transmitido, mas que, alguns elementos da Direção anterior acabaram por dizer que gostavam que tivesse acontecido.
    \end{itemize}
\end{itemize}

Abordamos agora a parte financeira da barraca: o pagamento das bebidas é feito com senhas que têm um custo de 2 euros. Estas podem servir para o que cada Núcleo entender (exemplo: 1 senha = 100 bebidas ou 2 senhas = 1 bebida ou 10 senhas = 20 bebidas, etc). Os copos, os sumos e as bebidas têm de ser adquiridos à \acrshort{dg} e podem ficar de uma noite para a outra. É, no entanto, de evitar que a noite dê prejuízo devido a um elevado número de stock na barraca (é possível levantar e devolver bebidas até às 5 da manhã em ponto, momento em que a barraca de fornecimento das bebidas fecha).

\ifthenelse{\boolean{biblia}}
{ % TRUE
    É também recebido algum dinheiro direto na barraca o que é ilegal, pelo que se deve ter algum cuidado com isso, havendo algumas pessoas que tentam comprar bebidas de propósito para descobrir quais os núcleos em ilegalidade. É de notar que sobre as receitas da barraca é cobrado IVA, o que é normal, pelo que ao se receber dinheiro por fora não se tem de pagar o IVA. Contudo, este dinheiro é injustificável pelo que o ideal é que cubra apenas as despesas que existem com a barraca. Caso haja dinheiro a mais recebido é também possível ir comprar senhas e entregá-las. Contudo, é preciso ter cuidado para que as senhas sejam entregues na noite correta.
}
{ % FALSE

}
É também de salientar que os copos têm dois tamanhos podendo-se fazer variações através disso (exemplo de tal, é o Núcleo de informática que o faz e tem lucros elevadíssimos). Note-se que os copos podem ser reutilizados (e pode-se promover uma campanha para tal) promovendo assim mais uma poupança, contudo, a tentativa de os trazer para casa e lavar correu mal este ano quer pela elevada sujidade em que eles estavam quer porque isso implica que a pessoa que saiu às 6h os traga, os lave e os volte a levar ao recinto no dia seguinte, antes da abertura do mesmo.

Este ano, devido ao facto de o Presidente da secção de futebol ser o professor Manuel Crisóstomo, fomos também convidados a tomar conta da escala da dita barraca. Os Colaboradores do \acrshort{neeec} não manifestaram, de todo, interesse em preencher a escala tendo o \acrshort{neeec} desligado dessa barraca, uma vez que essa não era, de todo, uma competência sua. No entanto, a escala da mesma foi ocupada pela Direção antiga do \acrshort{neeec} e por alguns Coordenadores do \acrshort{neeec} atual que eram muito próximos da mesma. A barraca não foi decorada, teve aberta nuns dias e fechada noutros e toda a gente percebeu que as pessoas que lá “trabalhavam” queriam apenas entrada e bebida grátis pelo que tal provocou vários comentários muito negativos quer dentro da \acrshort{dg}, quer dentro da Assembleia de Núcleos trazendo uma imagem negativa do \acrshort{neeec}. Contudo, na generalidade percebeu-se também que o \acrshort{neeec} atual não esteve envolvido neste assunto e na descredibilização e aproveitamento da barraca mas situações como esta são de evitar ao máximo.