% ========================
% # Dádiva de Sangue     #
% ========================

\subsubsection{Dádiva de Sangue}

Ao longo do ano, o Serviço de Sangue dos \acrshort{chuc} entra em contacto com o \acrshort{neeec} e com o \acrshort{deec} para solicitar apoio para recolhas de sangue em duas ocasiões: em outubro para fazer uma recolha em novembro e em fevereiro para fazer uma recolha em março.

O contacto é estabelecido por email onde enviam um pedido de salas sugerindo a utilização das salas T.4.2 e T.4.3 (por terem sido as utilizadas na última edição). Este email é enviado quer para a Direção do \acrshort{neeec}, quer para a do \acrshort{deec} bem como para as listas de emails da secretaria e de todos os docentes. Desta forma, a pessoa que deve responder acaba por ser uma coisa ambígua. No email são também enviados dois cartazes: um que publicita a atividade e outro que indica quem pode ou não doar sangue. O cartaz que divulga a atividade é extremamente básico e não tem o símbolo do \acrshort{neeec}, mas tem o seu nome, e tem o símbolo da \acrshort{uc} ao invés do símbolo do \acrshort{deec}. Uma vez que estas informações se baseiam nas atividades anteriores, poderão ser solicitadas alterações bastando responder ao email com as indicações a alterar, sem qualquer problema. Apesar dos cartazes fornecidos serem maus esteticamente e irem contra as nossas normas de imagem, dado o volume de trabalho, decidimos não fazer nenhum cartaz da nossa autoria para divulgar esta atividade. Contudo, foram colocados inúmeros cartazes no departamento e feita uma elevada campanha nas redes sociais para esta iniciativa.

No presente ano, respondemos a ambos os emails e informámos o responsável pelo \acrlong{cp}, João Ferreira, para a necessidade de fechar a sala do mesmo.

Para a montagem das salas, é feita uma vistoria no dia anterior, sendo feita uma limpeza à sala e tudo o resto é montado no dia seguinte pela equipa dos CHUC, sem qualquer problema. Desta forma, basta permitir o acesso às salas e todo o trabalho é feito pela equipa. No final do dia basta ir à sala para voltar a colocar as mesas no sítio correto pois toda a sala já se encontra limpa pela equipa dos CHUC.

Estas iniciativas tiveram sempre bastante adesão nunca havendo momentos mortos na sala. Uma vez que é uma atividade extremamente consolidada, cujo processo é sempre igual em cada edição, não trás qualquer complicação e ajuda muito quem necessita pelo que recomendamos a manutenção e reforço do apoio do \acrshort{neeec} a este tipo de iniciativas.