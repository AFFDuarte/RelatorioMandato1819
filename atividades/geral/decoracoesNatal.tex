% ========================
% # Decorações de Natal  #
% ========================

\subsubsection{Decorações de Natal}

Na altura do natal, o \acrshort{deec} costuma ter algumas decorações natalícias. Estas costumavam localizar-se na sala de convívio, dinamizadas pelo \acrshort{deec} e na entrada do \acrshort{deec}, dinamizadas pelas senhoras da Secretaria. Contudo desde o mandato anterior, o \acrshort{neeec} tem tomada uma posição importante nesta área fazendo assim com que a decoração natalícia do Departamento seja de maior dimensão, alargando-se a mais locais.

As decorações deste ano foram montadas de forma a que no dia útil após o dia 1 de dezembro estivesse tudo montado e iluminado. Desta forma, a equipa do Núcleo reuniu-se na sala de convívio no domingo, 3 de dezembro. Estas decorações demoraram algumas horas a ser montadas principalmente pelo facto de haver vários trabalhos manuais a fazer (recorte de estrelas e afins). Houve ainda um problema com o facto de haver muitas pessoas para trabalhar e pouco trabalho para fazer (ou que pudesse ser feito em simultâneo) o que provocou alguma confusão desnecessária.

Este ano, decidimos colocar decorações em vários locais onde os estudantes costumam estar presentes: a sala de estudo do piso 6 (árvore de natal), a zona do bar (árvore no exterior e mangueira luminosa no corrimão do corredor), a entrada do Departamento (a árvore de natal do Departamento e umas estrelas gigantes colocadas na parede, penduradas do piso 4), a sala de convívio (com uma árvore de natal e iluminação em vários locais), a sala do Núcleo (com uma pequena árvore de natal) e por todas as janelas de vidro do Departamento umas estrelas de papel recortadas.

No jardim do bar criou-se também uma árvore de natal com luzes de exterior utilizando rede de sombra e iluminação LED. Esta árvore necessitou de algum esforço logístico para ser montada uma vez que tinha de ter vários arames a segurá-la e instalação elétrica protegida da chuva. Contudo, durante o mês de dezembro houve uma intempérie e as bases da árvore soltaram-se todas ficando a mesma toda estragada, tendo sido desmantelada ainda antes do meio do mês.

Adicionalmente foram colocadas várias mensagens de boas festas espalhadas pelo Departamento, principalmente nas portas de maior passagem e acesso a esplanadas, algo que diferenciou a decoração deste ano.

A árvore de natal do piso 2, habitualmente montada pelas funcionárias da Secretaria, foi montada pelos elementos do \acrshort{neeec} (já após a montagem geral das decorações), a pedido das mesmas. A desmontagem foi feita, no entanto, por elas uma vez que a árvore não foi desmontada pelo \acrshort{neeec}, propositadamente, uma vez que nos encontrávamos em época de exames e havia muito poucas pessoas disponíveis para trabalhar.

No futuro, recomendamos a continuação desta atividade, nestes moldes, pois provoca um ambiente muito familiar no Departamento e facilmente se adapta a todos as atividades do mês solidário do \acrshort{neeec} bem como a toda a dinâmica de boas festas que possam ser criadas.
