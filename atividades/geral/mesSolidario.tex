% ========================
% # Mês Solidário        #
% ========================

\subsubsection{Mês Solidário}

O mês solidário foi um evento realizado em dezembro de 2017, proposto pelo Presidente do Núcleo tendo em conta a sua realização e sucesso em anos anteriores, o qual consistiu na recolha de bens e na organização de um lanche solidário, de maneira a ajudar instituições necessitadas, focando-nos principalmente em quem tivesse sido afetado pelos incêndios de outubro.

Durante todo o mês de dezembro foram colocados vários caixotes em pontos estratégicos como o bar, a sala de convívio e a entrada do Departamento para que se fizesse a recolha de bens alimentares, vestuário, livros, material escolar, entre outros.

Foi ainda definido que todos os eventos, nomeadamente os workshops e o torneio NEEEC vs Profs, realizados no mês solidário teriam entrada gratuita acompanhada da necessidade de se doar um bem alimentar. A \acrshort{rga} realizada nesse mês não teve essa condição, uma vez que tal seria ilegal, mas os participantes foram incentivados a trazer também um bem.

Realizamos um lanche solidário, na sala de convívio, no qual cada elemento do Núcleo ofereceu algo e as receitas desse mesmo lanche reverteram totalmente para a causa ajudada. A responsável pelo mês solidário, Ana Calhau, realizou um formulário perguntando o que cada elemento poderia trazer para o lanche, e posteriormente foi elaborada uma lista tendo em conta as respostas dadas, atribuindo um bolo ou alguns litros de sumo a cada um. Mais tarde, nas vésperas do evento alguns elementos comunicaram a não possibilidade da entrega do que lhes foi destinado tendo contribuído com outra coisa. O lanche teve afluência média, tendo tido a presença de alguns alunos, professores e funcionários que muitas vezes contribuíram dando donativos maiores para além da compra da comida.

Os bens alimentares e materiais foram doados à Casa de Infância Elísio de Moura, já ajudada anteriormente, uma vez que a solidariedade gerada para com os afetados pelos incêndios era já muito grande. O dinheiro angariado foi doado a um fundo dedicado ao apoio à reabilitação das áreas agrícolas destruídas em Tondela.
