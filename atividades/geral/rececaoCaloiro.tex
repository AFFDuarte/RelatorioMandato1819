% ========================
% # Receção ao Caloiro   #
% ========================

\subsubsection{Receção ao Caloiro}

O dia de receção ao caloiro, este ano, começou, como é habitual, pelas 07h59 na escadaria em frente ao \acrshort{deec} com a tradicional praxe, onde os doutores conhecem os novos caloiros e os caloiros conhecem os velhos doutores.

Finda a praxe os caloiros recolheram o seu saco com brindes (este ano incluía uma t-shirt do Polo II, uma bolsa para telemóvel do \acrshort{deec}, publicidade à Escola de Condução referida em \ref{parcerias_escola}, publicidade a outras entidades que deram brindes ao Polo 2 e panfletos com informações como as atividades do \acrshort{neeec} para a Receção ao Caloiro, a \acrshort{f3e}, o mapa do \acrshort{deec}, o mapa do Polo 2 e os horários dos SMTUC) e, de seguida, dirigiram-se para o auditório A4 onde tiveram a sua primeira aula, a aula fantasma. Este ano a aula fantasma foi dada pela Solange Silva. Apesar da mesma ter corrido bem, sugerimos que no futuro se escolha alguém mais cativante e intimidante. Esta aula terminou meia hora mais cedo do que era previsto o que foi muito negativo uma vez que a ideia seria a aula terminar e logo, de seguida, dar-se início à sessão de apresentação com os professores que decorreu no mesmo local.

De seguida seguiu-se a cerimónia de boas vindas em que estiveram presentes o Diretor do \acrshort{deec}, o Presidente do \acrshort{neeec} e todos os professores regentes das quatro cadeiras do primeiro ano.

Após esta sessão, os caloiros foram divididos em grupos para visitar as associações estudantis sediadas no nosso Departamento (\acrshort{neeec}, Clube de Robótica e \acrshort{best}) (o facto de se visitar o \acrshort{neeec} e o \acrshort{best} bem como o facto de haver visitas de manhã foi uma novidade este ano que nos pareceu extremamente vantajosa). Também como novidade tivemos o professor Manuel Crisóstomo, Presidente da \acrfull{sf}, a falar sobre a mesma enquanto era apresentada a sala de convívio aos caloiros.

De seguida, todos os caloiros foram levados para o jardim junto à sala do \acrshort{neeec} onde decorreu a típica febrada do primeiro dia de aulas. Este ano decidimos mudar o layout da febrada, usando como espaço de trabalho parte do jardim, que vem desde o anexo até à árvore, de modo a que a fila para comprar senhas não impedisse a passagem das pessoas e houvesse mais espaço útil. A febrada em si correu bastante bem, a escala estava toda preenchida e foi toda cumprida. O método de rasgar senhas e trocar a caixa de hora em hora foi implementado e foi uma mais valia para controlar borlas contudo, o facto de alguns membros não estarem habituados a este sistema fez com que o mesmo não fosse cumprido pelo que, a partir de certa hora, após o sistema ter sido quebrado, foi impossível controlar qualquer borla. Após um almoço praxístico, os caloiros puderam assistir a uma pequena atuação da Quantunna, nos jardins do \acrshort{neeec}.

Depois de todos os caloiros comerem foram divididos novamente em grupos (os mesmos da manhã) para fazerem a habitual visita ao Departamento e a inscrição nas turmas práticas. A visita englobou vários laboratórios do Instituto de Sistemas e Robótica e do Instituto de Telecomunicações, ficando em falta o INESC. A inscrição nas turmas práticas foi um dos momentos que pior correu neste dia principalmente devido ao facto de os caloiros descobrirem que se conseguiam inscrever nas turmas mais cedo do que era suposto e desrespeitarem totalmente as indicações dos responsáveis do \acrshort{neeec}. Isto levou a que se informasse a responsável da Secretaria do \acrshort{deec} e que todas as inscrições fossem anuladas. Este facto desencadeou uma conversa entre os responsáveis do \acrshort{neeec} e a Direção do \acrshort{deec} sobre o facto de as inscrições nas turmas serem pré-feitas pelo que deverá ter sido o último ano em que as inscrições foram feitas pelos caloiros. Houve também outros problemas relacionados com problemas informáticos, nomeadamente com as contas de cada um dos caloiros, problemas que têm acontecido todos os anos. Todos estes problemas atrasaram bastante o decorrer das atividades durante a tarde.