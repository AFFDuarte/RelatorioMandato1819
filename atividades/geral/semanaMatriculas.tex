% =========================
% # Semana das Matrículas #
% =========================

\subsubsection{Semana das Matrículas}

Durante a semana das matrículas, organizada pela \acrshort{uc} em conjunto com a \acrshort{aac}, os núcleos (e outras organizações como, por exemplo, o BEST) têm uma banca no Átrio das Químicas, junto à Faculdade de Medicina do Polo 1. No caso dos núcleos, os caloiros, teoricamente, são encaminhados, no final do processo de matrícula, para a banca do Núcleo que diz respeito ao seu curso. É de notar, que este processo não ocorre da forma mais ordeira pelo que muitos dos caloiros visitam a banca antes de entrarem no processo de matrículas e outros (muitos) saem do processo de matrículas e já não se dirigem à banca do Núcleo pois estão cansados ou querem ir embora com medo das praxes que poderão ser sujeitos. É também de realçar que, cada vez mais, os estudantes matriculam-se na internet não indo à banca das matrículas.

Para assegurar a gestão da banca do \acrshort{neeec} foi criada uma escala para todos os dias com início às 8h45 e término após as 17h. De notar que no primeiro dia a hora para montagem das bancas era às 7h45, horário esse que não foi cumprido pelo \acrshort{neeec}, e ainda bem, uma vez que pelas 9h, hora em que chegámos, as bancas ainda não estavam no Átrio, disponíveis para ser montadas. Após o primeiro dia, as mesas e cadeiras ficam no átrio das químicas pelo que é apenas recolhido todo o material das bancas que pode ser levado ou pode ficar guardado no \acrshort{nedf}. De realçar que o espaço é bastante ventoso pelo que é necessário prender muito bem o roll up do Núcleo para este não voar nem se estragar. Este ano, fizemos um formulário auxiliar que permitia aos Colaboradores presentes na banca fazerem um questionário seguido aos caloiros. Nesse formulário, era questionado o nome do caloiro (havia uma lista com todos aqueles que entraram na 1ª fase), eram explicadas as informações sobre a comunicação do \acrshort{neeec} (Facebook, Instagram e site. Recomendava-se ainda uma visita ao Facebook do Somos Polo 2. De seguida, explicava-se o programa das semanas de receção ao caloiro e era entregue um flyer aos caloiros com todo o programa destas semanas. Nesta apresentação, aproveitou-se para apresentar as \acrshortpl{rga} e a \acrshort{f3e}, atividades importantes mas não tão direcionadas para os caloiros, de forma a que estes fossem tomando uma ideia do que o \acrshort{neeec} fazia. De seguida, era apresentado o jantar de curso estando já abertas as inscrições. Desta forma, os pais podiam logo apoiar financeiramente a inscrição do evento. Era também apresentada a Latada. Por fim, davam-se mais alguns papeis com informações úteis como um mapa do Polo 2, uma pequena explicação sobre as cantinas, bares e residências dos \acrshort{sasuc}, um mapa do \acrshort{deec}, os horários dos SMTUC para as linhas 34 e 38. Era também apresentada de novo a \acrshort{aac} (os caloiros já passam por uma banca da \acrshort{aac} no interior do circuito). Este ano existia também uma grelha da \acrshort{dg} sobre Desporto Universitário que deveria ser preenchida pelos caloiros. Por fim, falava-se da praxe tentando acalmar qualquer medo que houvesse. Para finalizar, era apresentado o mega convívio do polo 2 e tentava-se fazer a venda de pulseiras para o evento (aqui, muitos doutores falavam negativamente do evento prejudicando a venda das pulseiras, o que é de evitar de todo).

Este novo modelo da banca correu bastante bem tendo a impressão do material sido positiva pois os caloiros, ao levarem com tanta informação, absorvem muito pouco pelo que, ao levarem os papéis, podem “rever a matéria em casa”. A implementação de um inquérito online também foi muito útil e conduziu os membros do Núcleo a fazer as questões corretas aos caloiros. Nota-se também uma taxa muito mais elevada de respostas comparando com o questionário escrito à mão e não há qualquer problema com a “descodificação de letras”. Há, no entanto, dois problemas a referir: o primeiro é o facto da internet (quer por wifi, quer por dados móveis) ser fraquíssima naquele local e nem toda a gente ter dados móveis, o que dificultou, às vezes, o preenchimento e o facto de algumas pessoas preferirem fazer as suas próprias questões pela ordem que entendem esquecendo-se sempre de perguntar vários pormenores e prejudicando, assim, a experiência do caloiro.

É de notar que todas as informações sobre esta semana são dadas pelo Pelouro do \acrshort{gape} da \acrshort{dg} muito em cima da hora pelo que o \acrshort{neeec} deve-se precaver atempadamente para esta situação. Também de realçar que, devido à realização do \acrshort{ene3} em Coimbra, o \acrshort{neeec} tinha em stock um elevado número de mapas que pode usar para este evento. Contudo, não é fácil arranjar estes mapas pois quem os distribui – Turismo do Centro – já os aloca para a \acrshort{aac} que depois faz a distribuição dentro do circuito das matrículas, não dispondo o \acrshort{neeec} de mapas para dar na banca ou nos kits.

\ifthenelse{\boolean{biblia}}
{ % TRUE
Existe ainda uma pequena tradição na madrugada de Domingo para Segunda desta semana, altura em que os núcleos do Polo 2 (bem como os do Polo 1) colocam uma faixa alusiva ao Polo 2, algo que é proibido pela reitoria e como tal tem de ser feito à pressa e ilicitamente. Na nossa opinião, esta atividade não tem interesse nenhum deixando apenas as pessoas ainda mais cansadas para uma semana que já se avizinha difícil. De realçar também que, este ano, não se conseguiu colocar a faixa de noite pelo que se teve a acabar de colocar na manhã do primeiro dia, não tendo havido qualquer problema com isso, exceto o facto de se ter estado acordado até às 05h para nada.
}
{ % FALSE
}