% ============================
% # Venda de Kits de Caloiro #
% ============================

\subsubsection{Venda de Kits de Caloiro}

Tradicionalmente, o \acrshort{neeec} costuma vender, por altura da latada, os kits de caloiro que incluem o penico, a chupeta, o apito e as meias. No início deste mandato, tínhamos em stock 22 kits sobrantes do mandato anterior. Adicionalmente, entrámos em contacto com a loja "O Caloiro", no final de agosto, para encomendarmos mais 50 kits. O preço praticado por esta loja foi de 6,3€ por unidade. Foi também sondada a hipótese de se comprar kits de caloiro em conjunto com outros núcleos do Polo 2 para obter preços mais vantajosos, o que não foi feito uma vez que, para além do \acrshort{neeec}, apenas o \acrshort{neemaac} também se interessou em fazer venda de kits e, pelo historial de confusão logística na venda de coisas em comum bem como pelo facto de uma encomenda conjunta entre estes dois núcleos não trazer um preço mais vantajoso, optou-se por fazer encomendas separadas. Contudo, no futuro, poderemos recomendar uma encomenda conjunta se houver um controlo grande e organizado na parte logística da distribuição dos kits, quer pela facilidade em obter melhores preços como pela facilidade em trocar kits caso se vendam mais ou menos kits num local do que noutro em relação ao que era esperado.

Para este ano, foi também acordado com o \acrshort{neemaac} a venda de kits a 7,5€ o que não foi cumprido pela parte deles (venderam a 7€), não tendo, no entanto, causado qualquer impacto negativo quer a um Núcleo, quer a outro pois ambos esgotaram o stock.

A venda dos kits, este ano, foi anunciada a 28 de setembro (quinta-feira do jantar de curso) sendo que, nesse momento, ainda não estavam disponíveis os kits de caloiro deste ano mas apenas os antigos. A venda começou a ser feita de forma calma tendo a maior parte dos kits sido vendidos na terça da serenata e na quarta seguinte. Se tivesse havido mais kits, mais teriam sido vendidos mas não muitos mais pelo que, consideramos que 72 kits foram uma quantidade muito boa e a manter (no ano anterior foram comprados 100 kits e sobraram 22 pelo que os números são, de facto, parecidos).
