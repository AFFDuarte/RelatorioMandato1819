% ========================
% # Camisolas de Curso   #
% ========================

\subsubsection{Camisolas de Curso}

Uma das tradições do \acrshort{neeec}, ao contrário de alguns cursos em que esta responsabilidade é dos Carros da Queima, é fazer o design e a venda de camisolas de curso, cujo design é feito pela equipa da Imagem. Esta é uma tarefa um pouco complicada visto não ser possível agradar a toda a gente. Este ano, questionámos a Direção sobre este assunto logo no início do mandato, sabendo que a venda seria feita no início do segundo semestre contudo, apenas em dezembro, alocámos recursos a esta tarefa. Ainda em novembro, em reunião de Coordenadores Gerais, decidiu-se o tipo de camisola a criar, tendo-se optado por uma sweat. Inicialmente, o CG do pelouro solicitou a todos os membros que propusessem uma ideia mas acabou por não surgir nenhuma. Já em janeiro, o CG elaborou algumas ideias que colocou no Slack para que todos os membros do Núcleo pudessem opinar. Aqui houve imensas opiniões contrárias e alguma inércia na cedência de opiniões pelo que a tarefa ficou bastante complicada. No final, num dia em que a camisola tinha de ficar desenhada, acabou por surgir uma ideia muito simples apenas com o nome do curso na parte da frente e um símbolo de perigo de eletricidade na parte de trás da camisola que, após apresentada, reuniu o consenso de todos. A cor, contudo, não teve consenso tendo-se optado por encomendar duas cores diferentes (25 unidades de cada). A encomenda das camisolas foi feita à Singular Print, tendo-se obtido um preço agradável por unidade (15€) e tendo sido feita uma venda em conjunto (uma camisola custava 20€ enquanto que duas, de cores diferentes, custavam 35€ apenas). A qualidade do tecido das camisolas foi muito bom, contudo, o desenho final foi bastante diferente do pedido, embora tal não tenha sido detetado por nós aquando do envio da maquete para confirmação pelo que não foi possível reclamar. A venda de camisolas teve início no primeiro dia de aulas do segundo semestre tendo sido um fracasso de vendas havendo, no final do mandato, ainda várias camisolas por vender e tendo a iniciativa dado prejuízo ao Núcleo. Adicionalmente foram encomendados poucos tamanhos S e XL, tendo estes esgotado logo, e demasiados L pelo que é importante rever as quantidades encomendadas no futuro (foram encomendados 20\% de XL; 5\% de S; 40\% de L e 35\% de M).