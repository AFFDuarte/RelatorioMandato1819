% ========================
% # Imagem               #
% ========================

\section{Imagem}

\subsection{Introdução}

A equipa da Imagem do \acrshort{neeec} tem como função fazer toda a divulgação de eventos em formas gráficas (cartazes e vídeos, maioritariamente). A divisão de trabalhos ao longo dos anos acabou por ser feita, principalmente, apenas entre 3 membros uma vez que um dos membros se demonstrou sempre muito indisponível para trabalhar, outro dos membros acabou por mudar de Pelouro e outro dos membros não apresentou tanta proatividade no seu trabalho. Nos eventos de grande dimensão, \acrshort{ene3}, \acrshort{f3e}, Bot Olympics e Gala Ohms d'Ouro, houve sempre um membro responsável pela coordenação da imagem desses eventos: o João Ferreira ficou responsável pela \acrshort{f3e} e pelo Bot Olympics, enquanto o Marco Silva foi o responsável pela Gala Ohm’s de Ouro. Quanto à organização do \acrshort{ene3}, a equipa deste pelouro, do mandato 2017/2018, entrou um pouco mais tarde, visto esta área do evento estar a ser coberta na sua maioria pelo Rui Silva e Afonso Cheung, Coordenadores do Pelouro da Imagem em 2015/2016 e 2017/2018, respetivamente.

Uma vez que a utilização de ferramentas apropriadas para o trabalho deste pelouro é fundamental, logo no início do mandato o CG do Pelouro organizou uma reunião que teve como objetivo ministrar uma formação na ferramenta Illustrator para todos os membros da equipa e, de seguida, pediu a cada um dos membros que fizesse um cartaz para a transmissão dos jogos da Seleção de Portugal na Taça das Confederações. Desta forma, foi possível ver logo o domínio das ferramentas por cada um dos membros e as suas formas de trabalhar.

Adicionalmente, foi criado um template que seria suposto funcionar em todos os cartazes de eventos, mas que, como não correu como o esperado, passou a ser utilizado para fazer promoção de todos os workshops. Acabou por se criar também um template para a promoção das \acrshortpl{rga} e templates para mais meios de divulgação que serão falados adiante. Este Pelouro foi também responsável por ministrar um workshop de Photoshop e outro de Illustrator, ferramentas utilizadas no nosso trabalho.

\subsection{Atividades}

% ========================
% # \acrshort{ene3}                 #
% ========================

\subsubsection{ENE3}

Organizar o \acrshort{ene3} foi um grande desafio, visto ser o primeiro grande evento do ano e a fasquia do mesmo ser elevadíssima pela grandeza do evento. Todo o nosso trabalho consistiu em pegar em trabalho já feito e adapta-lo ao que fosse preciso. Uma vez que o logótipo e o tema já haviam sido criados, a partir dele, fizemos imagens para as publicações nas redes sociais, toda a imagem gráfica durante o evento, sinalética, faixas publicitárias, o palco do evento e o vídeo final oficial do evento, entre outros.

% ========================
% # F3E                  #
% ========================

\subsubsection{F3E}

Organizar a \acrshort{f3e} foi trabalhoso e desafiante por ter sido o primeiro evento de grande dimensão, exclusivo do presente mandato. A divulgação das empresas criou alguns problemas na criação da imagem devido às imensas regras de conjugação de cores que estas entidades impõem, não sendo compatíveis com o grande leque de cores deste evento acabando por ser criado um template de divulgação que colocava os logótipos das empresas em fundo branco, algo que facilitou o trabalho, a partir daí. Fizeram-se posters, credenciais, sinalética e uma lona que sofreu variadas alterações devido ao constante aumento do número de patrocinadores, tendo a sua execução terminado de forma tardia.

% ========================
% # Bot Olympics         #
% ========================

\subsubsection{Bot Olympics}

Este ano a organização da 4ª Edição do Bot Olympics deram um grande desafio ao Pelouro da Imagem do \acrshort{neeec} pois pretenderam mudar toda a linha gráfica do evento. Ao inicio isto parecia desnecessário mas acabámos por compreender que tal se devia à nova forma que o evento apresentou, marcando uma diferença em relação aos eventos anteriores. Para isto, foram propostas algumas tentativas de um logótipo novo, tendo depois de umas 5 ou 6 propostas surgido o logótipo final. A partir daí criou-se uma imagem de fundo alusiva também ao evento que deveria ser utilizada em todas as imagens alusivas ao mesmo. Fizeram-se posters, t-shirts, medalhas, sinalética e toda a cobertura fotográfica do evento. No fim fez-se um vídeo que resumiu o evento, e foi partilhado nas redes sociais. A captação de imagens ao longo do evento foi algo importante para que, em futuras edições, exista uma maior divulgação do mesmo.

% ========================
% # Ohms D'Ouro          #
% ========================

\subsubsection{Ohms D'Ouro}

Este ano realizou-se a VI edição da Gala Ohms d’Ouro de uma forma especial dado que este seria também um momento de comemoração dos 20 anos \acrshort{neeec}. Assim sendo, houve uma melhoria substancial em todos os aspetos da gala em especial na imagem tendo havido uma maior aposta em vídeos promocionais em prol de um maior número de inscrições. Estes vídeos foram pensados de forma a chamar mais gente para a gala tendo sido então efetuado um vídeo com fotos das edições anteriores, um vídeo estilo “Teaser” para uma primeira apresentação da edição deste ano da gala e um vídeo final para divulgação dos apresentadores. Este ano existiu também uma enorme ligação entre a imagem e a comunicação da gala o que permitiu que todas as publicações de divulgação estivessem prontas atempadamente evitando assim atrasos ou erros nas publicações, o que foi bastante benéfico para que tudo fluísse naturalmente. Devido ao facto de terem existido patrocinadores na gala foi também necessário a criação de um novo “Press Conference” para que os mesmos patrocinadores pudessem ser divulgados. Foram também criados modelos de apresentações em PowerPoint para que os nomeados da gala fossem apresentados pelos apresentadores da mesma, de forma correta.
Este é um evento que, sem dúvida, tem tudo para continuar a crescer no entanto é importante a continuação da aposta em vídeos promocionais, dado que estes têm mais impacto que uma simples imagem.


% ===========================
% # Workshop de Photoshop   #
% ===========================

\subsubsection{Workshop de Photoshop}

Foi ministrado, pelo Marco Silva, um workshop que superou todas as expetativas no que diz respeito ao número de inscrições, tendo sido um dos workshops com o maior número de pessoas a participar neste mandato.

Este workshop teve uma pequena introdução teórica onde foi explicado o que era edição de imagem “Bitmap” e que tipo de resolução/modo de imagem é que deveríamos escolher mediante o facto de se estávamos a criar conteúdo para a Web ou conteúdo para ser impresso.

Depois disto estava então prevista a resolução de 5 exercícios práticos sendo que só 4 deles é que foram resolvidos devido a atrasos que aconteceram durante o decorrer do workshop.

Estes atrasos deveram-se, essencialmente, ao facto de muitas pessoas não terem levado o Photoshop já instalado no seu computador pessoal tendo sido, por isso, necessário perder tempo do workshop para fazer circular pela sala discos de armazenamento com a instalação do Photoshop, é então de realçar a importância de neste tipo de workshops ter sempre vários discos de armazenamento com o programa em causa prontos para que este atraso possa ser minimizado. Outra das dificuldades deste workshop prendeu-se com o facto de o projetor utilizado não ter a melhor resolução/qualidade o que impossibilitou, por vezes, que os formandos conseguissem acompanhar o que o formador estava a explicar o que também contribuiu para o atraso do workshop.

% ===========================
% # Workshop de Illustrator #
% ===========================

\subsubsection{Workshop de Illustrator}

Foi ministrado, pelo Moisés Dias, um workshop sobre a ferramenta Illustrator. Este segundo Workshop foi baseado em imagem vetorial, com uma audiência de cerca de duas dezenas pessoas onde foi explicada a utilidade da ferramenta. Começando com uma explicação das diferenças entre imagem vetorial e bitmap e uma passagem pelas ferramentas do programa, seguiu-se um exercício muito simples, onde os participantes teriam que desenhar objetos simples. A partir daí foram introduzidas as noções de Layer, transparência, cor e stroke, editor de texto e a ferramenta do PathFinder. Logo depois fez-se um exercício onde foi fornecida uma pasta com imagens bitmap. Os participantes teriam que fazer a conversão delas para formato vetorial onde poderiam mover essa imagem e edita-la ao seu gosto. Por fim fez-se um exercício mais complexo, onde o objetivo era desenhar um logótipo, e aí foram utilizadas imensas ferramentas, como pathfinder, dropshadow, pen tool, entre outros. Em suma, foi um evento muito produtivo e com bastante adesão. Um evento que, sem dúvida, deverá ser repetido sempre que possível.

% ===========================
% # Template                #
% ===========================

\subsubsection{Template}

A criação de um template foi algo sugerido pela Direção no inicio do mandato, visto ser uma mais valia para o trabalho da equipa no contexto de retirar imenso trabalho na criação de uma imagem diferente para cada evento. Foi apresentado um modelo que foi aceite de imediato, mas ao fim de uns poucos eventos que foram divulgados utilizando-o, chegou-se à conclusão que todos os cartazes eram idênticos de mais. Decidiu-se então cancelar a utilização do template para todos os eventos, passando este a ser utilizado em todos os cartazes de Workshops. Esta acabou por ser uma ideia que não foi mais abordada pois achamos importante manter o interesse deste pelouro que é usar a veia criativa dos membros para a elaboração dos cartazes.

% ===========================
% # Agenda Mensal           #
% ===========================

\subsubsection{Agenda Mensal}

No inicio do ano letivo, com o propósito de divulgar a receção ao caloiro foi criada uma espécie de calendário, onde estavam lá expostos os eventos que iriam ocorrer nos meses de setembro e outubro relativos à receção ao caloiro. Algo que correu bastante bem pelo que se sugeriu fazer isso todos os meses. Essa ideia foi um pouco posta de parte durante o mandato visto ser preciso ter um grande planeamento pelo que os eventos teriam que ser todos marcados até ao final do mês anterior, e não poderiam ser alterados, caso contrário esse cartaz ficaria errado. No mês de abril decidiu-se testar a criação de um template da agenda mensal que iria sair com os eventos de maio. Assim sendo, a agenda de maio e junho foram publicadas tendo alcançado bastantes pessoas no facebook. A mesma foi impressa em A2 e afixada na sala de convívio e no bar. Recomendamos que no futuro seja divulgada nas televisões do departamento e que esta iniciativa seja mantida, dado o seu sucesso.

% ========================
% # Camisolas de Curso   #
% ========================

\subsubsection{Camisolas de Curso}

Uma das tradições do \acrshort{neeec}, ao contrário de alguns cursos em que esta responsabilidade é dos Carros da Queima, é fazer o design e a venda de camisolas de curso, cujo design é feito pela equipa da Imagem. Esta é uma tarefa um pouco complicada visto não ser possível agradar a toda a gente. Este ano, questionámos a Direção sobre este assunto logo no início do mandato, sabendo que a venda seria feita no início do segundo semestre contudo, apenas em dezembro, alocámos recursos a esta tarefa. Ainda em novembro, em reunião de Coordenadores Gerais, decidiu-se o tipo de camisola a criar, tendo-se optado por uma sweat. Inicialmente, o CG do pelouro solicitou a todos os membros que propusessem uma ideia mas acabou por não surgir nenhuma. Já em janeiro, o CG elaborou algumas ideias que colocou no Slack para que todos os membros do Núcleo pudessem opinar. Aqui houve imensas opiniões contrárias e alguma inércia na cedência de opiniões pelo que a tarefa ficou bastante complicada. No final, num dia em que a camisola tinha de ficar desenhada, acabou por surgir uma ideia muito simples apenas com o nome do curso na parte da frente e um símbolo de perigo de eletricidade na parte de trás da camisola que, após apresentada, reuniu o consenso de todos. A cor, contudo, não teve consenso tendo-se optado por encomendar duas cores diferentes (25 unidades de cada). A encomenda das camisolas foi feita à Singular Print, tendo-se obtido um preço agradável por unidade (15€) e tendo sido feita uma venda em conjunto (uma camisola custava 20€ enquanto que duas, de cores diferentes, custavam 35€ apenas). A qualidade do tecido das camisolas foi muito bom, contudo, o desenho final foi bastante diferente do pedido, embora tal não tenha sido detetado por nós aquando do envio da maquete para confirmação pelo que não foi possível reclamar. A venda de camisolas teve início no primeiro dia de aulas do segundo semestre tendo sido um fracasso de vendas havendo, no final do mandato, ainda várias camisolas por vender e tendo a iniciativa dado prejuízo ao Núcleo. Adicionalmente foram encomendados poucos tamanhos S e XL, tendo estes esgotado logo, e demasiados L pelo que é importante rever as quantidades encomendadas no futuro (foram encomendados 20\% de XL; 5\% de S; 40\% de L e 35\% de M).

% ========================
% # Organogramas         #
% ========================

\subsubsection{Organogramas}

Como indicado na secção \ref{subsec:organogramasPiso2}, foi-nos solicitado, pelos órgãos gerentes dos \acrshort{deec}, um organograma com informação sobre a Direção do Departamento e dos seus serviços administrativos, ao qual juntámos também o organograma dos Delegados de ano e Coordenadores de Curso. Para tal, foi feito um template único onde bastou alterar a imagem de fundo e a respetiva atenuação para elaborar os dois organogramas.

% ================================
% # \acrshort{neeec} Informa#
% ================================

\subsubsection{NEEEC Informa}

O \acrshort{neeec} Informa é uma template que foi criado com o objetivo de informar a comunidade estudante, docente e administrativa do \acrshort{deec}. Este template pode ser colocado onde for necessário (Facebook, Instagram e televisão do \acrshort{deec}, por exemplo) e permite que seja transmitida alguma mensagem textual associada a uma imagem, tornando-a assim mais cativante. Apesar deste template estar feito em illustrator, o que exige a utilização do programa, o mesmo é acompanhado de um tutorial e dos tipos de letra necessários para que qualquer pessoa o possa editar. Pretende-se que, com o uso do template, se altere a imagem e a cor consoante a publicação em questão.

A ideia deste template surgiu aquando do lançamento dos horários, informação que foi divulgada através de um simples link, e foi já usado para informar que o fecho da sala de convívio para o HP Omen University Challenge bem como do fecho da sala de estudo da T.4.2 para a realização das eleições do \acrshort{neeec}. Ambas as publicações foram publicadas no MiEEC/\acrshort{uc} e permitiram uma elevada interação com as publicações o que nos leva a crer que o uso deste template deve ser massificado para todas as publicações deste tipo. A dimensão do template permite também a sua colocação nas televisões do departamento.

% ================================
% # Hall of Fame                 #
% ================================

\subsubsection{Hall Of Fame}

Existem vários estudantes do \acrshort{deec} que se destacam nas mais diversas áreas desde o Desporto, à Música, entre muitas outras. Desta forma, surgiu a ideia de criar um template que permitisse a divulgação dessas mesmas pessoas, dos respetivos talentos e de uma pequena descrição das mesmas. Esta ideia pretendia ser divulgada nos insta stories do \acrshort{neeec}, na televisão do \acrshort{deec} e no site do \acrshort{neeec} e teria uma periodicidade quinzenal, durante o período de aulas, o que permitiria destacar 14 pessoas por ano (valor extremamente baixo tendo em conta as várias pessoas do Departamento). Após as primeiras publicações, incentivadas pela Direção do \acrshort{neeec}, a ideia seria divulgar o meio de comunicação do Núcleo que permitisse aos estudantes sugerir outros colegas a divulgar em futuras edições.

Esta iniciativa não chegou a ser concretizada uma vez que o Pelouro da imagem emitiu um template para isto apenas no mês de maio e da parte da Direção, por não haver nenhum responsável por esta iniciativa, não houve disponibilidade para pesquisar informação sobre as pessoas a divulgar nas primeiras edições. Mais ainda, o facto da iniciativa ser lançada já no final das aulas, faria com que as primeiras edições fossem muito poucas pelo que a iniciativa iria parecer ter sido lançada só porque sim. No futuro, recomendamos bastante esta iniciativa pois parece-nos uma ideia excelente para divulgar os estudantes do \acrshort{deec}.

% ========================
% # Imagens Polo 2       #
% ========================

\subsubsection{Imagens Polo 2}

Para quase todos os eventos organizados pelo Polo 2, a imagem selecionada é feita por todos os núcleos sendo depois votada qual a que será utilizada. Por sua vez, quase todas as imagens do \acrshort{nei} costumam vencer existindo um elevado número de recursos alocados para este tipo de concursos. Este ano concorremos à imagem da T-Shirt do Caloiro, à do Mega Convívio e à da Mega Febrada, e pela primeira vez a nossa imagem foi escolhida para a Mega Febrada. A partir daqui foi necessário emitir toda a imagem para este evento acabando isto por ser um presente "envenenado".