% ===========================
% # Workshop de Illustrator #
% ===========================

\subsubsection{Workshop de Illustrator}

Foi ministrado, pelo Moisés Dias, um workshop sobre a ferramenta Illustrator. Este segundo Workshop foi baseado em imagem vetorial, com uma audiência de cerca de duas dezenas pessoas onde foi explicada a utilidade da ferramenta. Começando com uma explicação das diferenças entre imagem vetorial e bitmap e uma passagem pelas ferramentas do programa, seguiu-se um exercício muito simples, onde os participantes teriam que desenhar objetos simples. A partir daí foram introduzidas as noções de Layer, transparência, cor e stroke, editor de texto e a ferramenta do PathFinder. Logo depois fez-se um exercício onde foi fornecida uma pasta com imagens bitmap. Os participantes teriam que fazer a conversão delas para formato vetorial onde poderiam mover essa imagem e edita-la ao seu gosto. Por fim fez-se um exercício mais complexo, onde o objetivo era desenhar um logótipo, e aí foram utilizadas imensas ferramentas, como pathfinder, dropshadow, pen tool, entre outros. Em suma, foi um evento muito produtivo e com bastante adesão. Um evento que, sem dúvida, deverá ser repetido sempre que possível.