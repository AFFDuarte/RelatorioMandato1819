% ===========================
% # Workshop de Photoshop   #
% ===========================

\subsubsection{Workshop de Photoshop}

Foi ministrado, pelo Marco Silva, um workshop que superou todas as expetativas no que diz respeito ao número de inscrições, tendo sido um dos workshops com o maior número de pessoas a participar neste mandato.

Este workshop teve uma pequena introdução teórica onde foi explicado o que era edição de imagem “Bitmap” e que tipo de resolução/modo de imagem é que deveríamos escolher mediante o facto de se estávamos a criar conteúdo para a Web ou conteúdo para ser impresso.

Depois disto estava então prevista a resolução de 5 exercícios práticos sendo que só 4 deles é que foram resolvidos devido a atrasos que aconteceram durante o decorrer do workshop.

Estes atrasos deveram-se, essencialmente, ao facto de muitas pessoas não terem levado o Photoshop já instalado no seu computador pessoal tendo sido, por isso, necessário perder tempo do workshop para fazer circular pela sala discos de armazenamento com a instalação do Photoshop, é então de realçar a importância de neste tipo de workshops ter sempre vários discos de armazenamento com o programa em causa prontos para que este atraso possa ser minimizado. Outra das dificuldades deste workshop prendeu-se com o facto de o projetor utilizado não ter a melhor resolução/qualidade o que impossibilitou, por vezes, que os formandos conseguissem acompanhar o que o formador estava a explicar o que também contribuiu para o atraso do workshop.