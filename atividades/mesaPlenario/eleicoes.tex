% ==================================
% # Eleições                       #
% ==================================

\subsubsection{Eleições}

O processo de eleições sofreu inúmeras alterações em relação ao ano anterior, tendo em conta os novos estatutos da \acrshort{aac}. Assim, o processo eleitoral passou a necessitar da seguinte sequência:
\begin{itemize}
\item É necessário elaborar o regulamento eleitoral e marcar a \acrshort{rga} para discussão e aprovação do mesmo até 8 dias antes do prazo de fim para entrega das candidaturas.
\item A entrega das candidaturas terá de ser feita até ao final de março ou abril (caso as eleições sejam em abril ou maio, respetivamente) na Secretaria da \acrshort{aac}.
\item As candidaturas devem ser feitas na Secretaria da \acrshort{aac}, em formulário próprio, onde são indicados os dados de todos os efetivos e suplentes das listas candidatas e onde os mesmos têm de assinar.
\item O prazo entre o fim da entrega das listas e a data das eleições tem de ser de, pelo menos, duas semanas.
\item Não por imposição dos estatutos ou do regulamento interno mas por conselho do conselho fiscal, deve existir um dia de reflexão entre a data do fim da campanha eleitoral e o início das eleições.
\item As eleições podem decorrer durante 1 ou 2 dias e as urnas devem abrir e fechar no horário que ficar estipulado no regulamento interno, que deve ser decidido entre todos.
\item A tomada de posse deve ser marcada após as eleições e deve decorrer até ao dia 15 de junho.
\item Na tomada de posse, além dos membros efetivos, também os membros suplentes assinam a ata de tomada de posse. Desta forma, caso haja alguma demissão, a subida dos novos membros é feita de forma automática.
\item No início do processo eleitoral, foi dito por parte do Presidente do \acrshort{cf} que o Presidente da Comissão Eleitoral teria de assinar uma ata de tomada de posse junto da Secretaria da \acrshort{aac}. Contudo, o membro observador do \acrshort{cf} para estas eleições disse que tal não era necessário pelo que o Presidente da Comissão Eleitoral nunca chegou a assinar nenhuma ata de tomada de posse.
\end{itemize}

Após a discussão e aprovação do regulamento eleitoral em \acrshort{rga} (26/04/2018), a versão final foi enviada para o \acrfull{cf} via email no dia 30/04/2018 para que este órgão pudesse dar o aval necessário. Por parte do \acrshort{cf} nunca foi exigida ou sugerida nenhuma alteração ao regulamento após este envio, pelo que a Comissão Eleitoral tomou esta atitude como um sinal de aprovação do mesmo. Mesmo antes da referida \acrshort{rga}, entrou em contacto com o Presidente da Comissão Eleitoral (Presidente da Mesa do Plenário) um membro do \acrshort{cf} destinado a observar e fiscalizar todo o processo eleitoral do \acrshort{neeec}. Com este elemento foram sendo esclarecidas algumas dúvidas especificas relacionadas com o processo eleitoral a fim de assegurar que este fosse o mais justo e imparcial possível e que cumprisse todas as normas. Foi entregue uma lista candidata na Secretaria da \acrshort{aac} assim como no email da Mesa do Plenário (segundo o regulamento eleitoral, a candidatura era apenas válida caso fossem feitos estes dois passos). A respetiva lista respeitou o período de campanha e enviou, como regulamentado, os vários cartazes organogramas e panfletos para o mail da Mesa do Plenário e para o \acrshort{cf} para sua homologação. A lista enviou também as informações dos vários elementos que assegurariam o funcionamento da mesa de voto no dia das eleições. Para além destes a Comissão Eleitoral assegurou sempre outro membro presente na mesa de voto, alternando entre o Presidente da Comissão Eleitoral, Rui Silva, e o delegado, César Pereira. Os turnos tiveram a duração de 2 horas. Existiu apenas uma urna onde foram colocados os votos para a Direção do \acrshort{neeec} e para a Mesa do Plenário do \acrshort{neeec}, com boletins individuais para cada órgão.

No dia anterior às eleições (23/05/18) ao final do dia, foi comunicado ao Presidente da Comissão Eleitoral que o membro observador do \acrshort{cf} seria outro em vez do elemento com que este tinha mantido contacto previamente. Na nossa opinião, aqui começaram as falhas de falta de rigorosidade por parte do \acrshort{cf}. No dia das eleições (24/05/18), a Comissão Eleitoral preparou toda a sala onde se realizaram as votações (T4.2), imprimiu e afixou indicações de local de voto, e cartazes para informar a comunidade que a urna estaria aberta desde as 9h às 19h deste dia. À hora de abertura, com grande espanto nosso, foi comunicado via chamada telefónica ao Presidente da Comissão Eleitoral, que o membro observador do \acrshort{cf} estaria atrasado e só chegaria às 9h30h ao local das eleições. Tendo em conta que os boletins de voto, cadernos eleitorais, atas de abertura/fecho, folhas de descarga, envelopes e urna estariam ao encargo do \acrshort{cf}, as eleições não puderam iniciar à hora regulamentada. Além disso, após a chegada deste elemento perto da hora comunicada pelo mesmo, verificou-se que os cadernos eleitorais emitidos pelo \acrshort{cf}, não estariam completos e teriam até falhas de impressão que omitiam os dados de alguns estudantes. Devido a este problema, o Presidente da Comissão Eleitoral foi contactado pelo Presidente do \acrshort{cf} que o informou que não tinha forma de trazer ao local das eleições novos cadernos eleitorais, pelo que teríamos de ser nós, Comissão Eleitoral, a encontrar uma forma de os imprimir. A solução encontrada foi a impressão dos cadernos na Secretaria do \acrshort{deec}. Com todos estes percalços por parte do \acrshort{cf}, as eleições só começaram perto das 11 horas, quase duas horas depois do inicio estipulado no regulamento. 

Durante as votações, tudo correu como previsto e de acordo com o estipulado. O observador do \acrshort{cf} apareceu de 2 em 2 horas na sala para confirmar que tudo corria segundo regulamentado, mas não podemos deixar de dizer que este demonstrou uma atitude de desleixo. Esta atitude revelou, na opinião do Presidente da Comissão Eleitoral, uma grande falta de rigorosidade em todo o processo e afirma que este elemento da comissão esteve presente aparentemente contrariado e apenas por obrigação. 

Por fim, as urnas encerraram às 19 horas como regulamentado e procedeu-se à contagem dos votos cujos resultados foram os seguintes:
Direção: 
Lista E: 94 votos; 
Brancos: 0 votos; 
Nulos: 1 voto; 

Mesa do plenário: 
Lista E: 92 votos; 
Brancos: 2 votos; 
Nulos: 1 votos. 
