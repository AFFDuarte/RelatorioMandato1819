% ==================================
% # Regulamento Eleitoral          #
% ==================================

\subsubsection{Regulamento Eleitoral}

A proposta de regulamento eleitoral é obrigatória para que se aprove esse mesmo regulamento garantindo assim que as eleições das estruturas do \acrshort{neeec} para o mandato seguinte se efetuem dentro de todas as regras devidas e inerentes ao regulamento interno do \acrshort{neeec} e aos estatutos da \acrshort{aac}.

De igual forma ao processo feito pela mesa para a revisão do regulamento interno, foi feito um esforço para que a comunidade participasse na construção de proposta ao regulamento eleitoral, explicando a sua razão desta ser feita nas \acrshort{rga}s, incluindo espaços de discussão das propostas nas mesmas e utilizando os canais de divulgação do Núcleo para avisar a comunidade da importância de tal regulamento. De igual forma a proposta surgiu principalmente do trabalho mútuo entre a Mesa do Plenário e a Direção do Núcleo em reuniões extraordinárias onde se teve por base de construção os novos estatutos da \acrshort{aac}, a proposta de normas de funcionamento da Assembleia Magna, o novo regulamento interno do \acrshort{neeec}, o anterior regulamento eleitoral, regulamentos eleitorais mais atualizados de outros núcleos e propostas individuais que cada membro presente nestas reuniões foi colocando. Nas \acrshort{rga}s também existiram propostas e alterações de outros estudantes membros do \acrshort{neeec}.

Consideramos que o regulamento eleitoral aprovado estabelece as normas essenciais ao ato eleitoral e estabelece com clareza a estrutura das listas candidatas e da comissão eleitoral de forma a que o funcionamento das eleições e de assuntos a si inerentes corram da melhor forma e que este regulamento sirva de melhor base para os regulamentos de mandatos futuros, tratando-se de eleições disputadas ou não.