% ==================================
% # Revisão do Regulamento Interno #
% ==================================

\subsubsection{Revisão do Regulamento Interno}

Uma vez que durante este mandato, dada a criação dos novos estatutos da \acrshort{aac}, foi necessário proceder à criação da proposta de um novo regulamento interno. Para tal, a Mesa do Plenário fez esforços continuados para que a comunidade participasse ativamente neste processo. Estes esforços consistiram na exposição e clarificação da revisão do regulamento nas várias \acrshort{rga}s, na divulgação das \acrshort{rga}s e da importância de revisão do regulamento à comunidade através dos canais de comunicação do Núcleo. Para além disso, foram incluídos espaços de discussão aberta nas \acrshort{rga}s sobre os assuntos relativos a esta revisão e feita a abertura de um período de audição pública aberto à comunidade onde os membros puderam fazer novas propostas ou sugerir alterações à proposta já concebida. Apesar destes esforços consideramos que a maioria dos estudantes do \acrshort{mieec} não teve um papel ativo na construção do novo regulamento interno por falta de interesse ou até desconhecimento do assunto em causa. A proposta construída, atual regulamento interno, surgiu principalmente dos esforços de trabalho conjuntos entre a Mesa do Plenário e a Direção do Núcleo que realizaram várias reuniões extraordinárias, de longa duração, onde foi feita a discussão exaustiva dos artigos e pontos que deveriam constar no novo regulamento interno, tendo por base principal os estatutos, na parte referente ao funcionamento dos núcleos, o antigo regulamento interno do \acrshort{neeec}, alguns regulamentos de outros núcleos mais atualizados, a aplicação do regulamento à realidade do \acrshort{neeec} e inúmeras propostas individuais feitas pelos membros presentes ao longo da discussão. Nas \acrshort{rga}s também existiram propostas e alterações de outros estudantes membros do \acrshort{neeec}.

Este esforço coletivo culminou, a nosso ver, num documento que estabelece objetivamente e explicitamente as estruturas do \acrshort{neeec} e as suas normas de funcionamento de forma a garantir que a atual e as futuras direções possam identificar com clareza os seus deveres e responsabilidades para com a comunidade e para com a \acrshort{aac}, garantindo também o melhor funcionamento, bem-estar e subsistência do \acrshort{neeec} e da sua comunidade. Consideramos também que este trabalho facilitará os mandatos que, no futuro, procedam às próximas revisões uma vez que abordam mais detalhadamente todos os assuntos que concernem às estruturas do Núcleo.