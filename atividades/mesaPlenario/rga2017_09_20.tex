% ==================================
% # RGA 20/09/2017                 #
% ==================================

\subsubsection{RGA 20/09/2017 (1ª RGA Ordinária)}

Na primeira \acrshort{rga} a Mesa do Plenário foi dirigida por Rui Silva e a ata foi redigida por César Pereira. Da ordem de trabalhos fizeram parte a discussão de problemas relativos à disciplina computação gráfica, a discussão do mapa de avaliações do primeiro semestre e a aprovação do plano de atividades. Nesta reunião estiveram presentes 39 pessoas, sendo a reunião com maior adesão do mandato e, não havendo registos que o contrariem, de toda a história do \acrshort{neeec}. Esta grande adesão, na nossa opinião, relacionou-se com a presença dos alunos de primeiro ano que vieram tomar conhecimento do funcionamento do plenário de Núcleo e essencialmente devido ao primeiro ponto referido na ordem de trabalhos, uma vez que os estudantes se encontravam seriamente preocupados com os modos de funcionamento da disciplina.

Existiram inúmeras queixas referentes ao ano letivo de 2016/2017 que foram apresentadas quer à pedagogia do \acrshort{neeec} quer ao corpo pedagógico do \acrshort{deec}. Os alunos queriam ver os problemas corrigidos antes do início do funcionamento da disciplina no 2º semestre deste ano letivo. De forma a dar resposta a estes problemas existiu uma discussão aberta entre os estudantes presentes com os membros do Pelouro da pedagogia do \acrshort{neeec} de forma a que se formulasse um documento representativo das queixas dos estudantes. Este documento teve a finalidade de ser enviado para a Direção dos departamentos responsáveis pelo funcionamento da cadeira, dando a conhecer as reclamações dos estudantes e incentivando os responsáveis a encontrar soluções.

Os outros pontos da ordem de trabalhos não suscitaram tanta discussão e foram mais breves. Nesta \acrshort{rga} foi também apresentado pela Direção do \acrshort{neeec} o plano de atividades para o presente mandato que foi aprovado por unanimidade.