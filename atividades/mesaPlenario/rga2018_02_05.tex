% ==================================
% # RGA 05/02/2018                 #
% ==================================

\subsubsection{RGA 05/02/2018 (2ª RGA Extraordinária)}

Nesta \acrshort{rga} a Mesa do Plenário foi dirigida por Afonso Cheung e a ata foi redigida por Rui Gouveia. Da ordem de trabalhos fizeram parte a resolução dos problemas relacionados com a disciplina de computação gráfica. Nesta reunião estiveram presentes 16 estudantes e a mesma teve a presença extraordinária do professor Jorge Batista (Coordenador do \acrshort{mieec}) e do professor José Teixeira (professor da disciplina Computação Gráfica). De forma a garantir a imparcialidade desta reunião a mesa, achou por bem, que os membros que a fossem dirigir não tivessem estado inscritos no ano letivo anterior nesta disciplina de forma a que não existisse conflito de interesses tendo o Rui Silva sido substituído pelo Vice-Presidente da Mesa e o César Pereira substituído pelo 2º Vogal da Mesa.

Esta reunião consistiu numa discussão aberta entre alunos e professores, onde se tentou chegar a um consenso entre os dois para melhorar o funcionamento da disciplina e encontrar as razões das várias queixas. Deste diálogo resultaram promessas por parte do professor que as regras da disciplina iam ter mais rigor, tendo o mesmo finalizado o seu discurso apelando à honestidade e ao bom funcionamento da disciplina. A pedagogia do \acrshort{neeec} afirmou também que estaria atenta ao funcionamento da cadeira através do novo sistema de Delegados de Ano que pretende dar melhor e mais rápida resposta a este tipo de problemas. O professor Jorge Batista informou também que existiria a criação de um concelho de coordenação que poderá ajudar, no futuro, a resolver de forma célere possíveis problemas deste género.

Ainda nesta reunião, o Presidente do \acrshort{neeec}, João Bento, propôs o adiamento da aprovação do Regulamento Interno do Núcleo para o mês de março, por faltar receber da \acrshort{dg} os Regulamentos de Administração e Gestão Financeira e o de Secretaria, proposta essa que foi aprovada por unanimidade.