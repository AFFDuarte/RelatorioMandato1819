% ==================================
% # RGA 14/03/2018                 #
% ==================================

\subsubsection{RGA 14/03/2018 (3ª RGA Extraordinária)}

Nesta \acrshort{rga} a Mesa do Plenário foi dirigida por Rui Silva e a ata foi redigida por Rui Gouveia. Da ordem de trabalhos fizeram parte a discussão e aprovação da revisão do Regulamento Interno do \acrshort{neeec}, a discussão e aprovação da revisão ao Regulamento de Delegados do \acrshort{mieec}/UC, a discussão e aprovação da moção de agradecimento ao professor doutor Humberto Jorge, entre outros assuntos. Nesta reunião estiveram presentes 24 estudantes sendo, no entanto, de ressalvar que vários foram entrando e saindo havendo momentos em que o quórum para a realização da \acrshort{rga} não estava reunido pelo que a reunião teve de ser temporariamente suspensa.

O Presidente da Mesa Rui Silva introduziu o primeiro ponto explicando a necessidade da revisão do regulamento, esclareceu que a proposta de revisão do regulamento que seria discutida foi a disponibilizada para consulta à comunidade no dia 9/03/2018 e referiu que no período de audição publica disponibilizado e publicitado não foi apresentada qualquer proposta de alteração ao regulamento por parte da comunidade. Posto isto o Presidente do \acrshort{neeec} introduziu a discussão e alteração de alguns pontos da proposta de regulamento. Após a vasta discussão dos vários pontos que compõem a proposta foram aprovadas várias alterações sugeridas por diversos membros do plenário presentes na reunião. O documento foi aprovado por unanimidade, com a inclusão das alterações referidas, passando este a ser o novo regulamento interno do \acrshort{neeec}, após aprovação do \acrlong{cf}.

No ponto seguinte foi discutido o regulamento dos delegados de ano do \acrshort{mieec}/UC, regulamento já apresentado e aprovado na \acrshort{rga} de 6/12/2017, uma vez que após reflexão, seria necessário fazer algumas adaptações ao regulamento visando o melhor funcionamento do órgão. A introdução às propostas de alterações foi feita pelo Presidente do \acrshort{neeec}, João Bento, na falta do Coordenador da Pedagogia do \acrshort{neeec}, Carlos Simões. Após a apresentação, vários membros da comunidade discutiram os vários pontos e propuseram alterações que culminaram numa proposta conjunta que foi aprovada por unanimidade tornando-se assim o novo regulamento dos delegados de ano do \acrshort{mieec}/UC.

Passou-se à discussão e aprovação da moção de agradecimento ao professor doutor Humberto Jorge, proposta apresentada por João Bento, que explicou a sua razão dizendo que o professor tem sido um elemento fulcral no relacionamento que existe entre o \acrshort{neeec} e o \acrshort{deec} e no sucesso de muitas iniciativas quer deste mandato, quer de toda a história dos 20 anos do \acrshort{neeec}. Após discussão e intervenção de vários estudantes que pretendiam assinar a iniciativa esta proposta foi aprovada por unanimidade.

No final da reunião discutiu-se um ponto extraordinário, apresentado por João Bento, uma vez que, segundo sugestões anónimas feitas pelos canais de comunicação ao Núcleo, deveria existir uma moção de agradecimento aos bombeiros a reconhecer o seu trabalho pelo trágico ano de incêndios que afetou a região no ano de 2017, a proposta de moção foi aprovada. Depois disto os estudantes presentes apelaram a que existisse também uma moção de agradecimento a todas as entidades do \acrshort{deec} e às direções passadas que também contribuíram para o bom relacionamento do Departamento com o \acrshort{neeec}. Esta proposta de moção foi também aprovada.