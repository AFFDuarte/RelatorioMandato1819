% ==================================
% # RGA 29/05/2018                 #
% ==================================

\subsubsection{RGA 29/05/2018 (3ª RGA Ordinária)}

Nesta \acrshort{rga} a Mesa do Plenário foi dirigida por Rui Silva e a ata foi redigida por César Pereira. Da ordem de trabalhos fizeram parte a discussão e aprovação do relatório de atividades e contas do \acrshort{neeec}, a aprovação do inventário do Núcleo, a marcação das eleições dos Delegados de Ano para o ano letivo 2018/2019, a aprovação da avaliação dos membros colaboradores do Núcleo no presente mandato, a aprovação de todas as atas das \acrshortpl{rga} do mandato, entre outros assuntos. Nesta reunião estiveram presentes 34 estudantes tendo esta sido a 2ª \acrshort{rga} com maior adesão do mandato e, não havendo registos que o contrariem, a 2ª com maior adesão de toda a história do \acrshort{neeec}.

O Presidente da mesa, Rui Silva introduziu o primeiro ponto, dando a palavra a João Bento, Presidente do \acrshort{neeec}, Ivo Frazão, Tesoureiro do \acrshort{neeec} e José Pedro Silva, Administrador do \acrshort{neeec}, para estes apresentarem respetivamente o Relatório de Mandato e Atividades, o Relatório de Contas e o Inventário do \acrshort{neeec}. Rui Silva ressalvou ainda que todos os documentos apresentados encontravam-se em forma de rascunho uma vez que estavam extremamente elaborados, quando comparados com mandatos anteriores, pelo que a versão final dos mesmos deveria ser apresentada no primeiro plenário do mandato seguinte. Todos os documentos foram aprovados por unanimidade.

Rui Silva passou então a palavra a Pedro Cavaleiro, Coordenador eleito da Pedagogia do \acrshort{neeec} para que este propusesse as datas de eleições dos Delegados de Ano de 2018/2019. Após alguma discussão, foram aprovadas as datas de dezoito de outubro para o delegado do primeiro ano e a vinte e sete de setembro para os restantes anos.

Findo o ponto anterior, João Bento apresentou o documento que estipula as regras de certificação dos colaboradores do \acrshort{neeec} e, tendo em conta as mesmas, quais os membros a ser certificados. As regras e os membros certificados foram aprovados também por unanimidade tendo João Bento informado também que os membros envolvidos nas Comissões Organizadoras e no voluntariado do \acrshort{ene3}, \acrshort{f3e} e Bot Olympics teriam direito a Suplemento ao Diploma por parte da \acrshort{fctuc}.

Por fim, Rui Silva introduziu o último ponto, explicando a necessidade de se aprovar, de agora em diante, todas as atas do Plenário anterior, dadas as novas regras estipuladas pelo Regulamento Interno do \acrshort{neeec}. Assim todas as atas dos anteriores cinco plenários foram aprovadas por unanimidade.

Para finalizar, nos outros assuntos, João Bento deu umas palavras finais congratulando a Mesa do Plenário e toda a equipa do \acrshort{neeec} pelo trabalho desenvolvido ao longo do mandato. Criticou também algumas ações do \acrshort{cf} nomeadamente com o decorrer das eleições para o \acrshort{neeec} onde, na sua opinião, ficou em causa o trabalho exímio produzido pela Mesa do Plenário.
Rui Silva, Presidente da Mesa do Plenário, finalizou a reunião agradecendo a todos os que compareceram nas \acrshortpl{rga} ao longo do ano, permitindo a realização das mesmas, à sua equipa nomeadamente ao César, ao Rui e ao Afonso e à Direção do \acrshort{neeec}. Comprometeu-se ainda a enviar os devidos documentos para o \acrshort{cf} com a maior brevidade possível.