% ==============================
% # Apadrinhamentos de Erasmus #
% ==============================

\subsubsection{Apadrinhamento de Erasmus}

O Buddy Program (ou Apadrinhamento de Erasmus) foi uma atividade que o Pelouro de pedagogia tentou implementar este ano a pensar nos estudantes internacionais que vêm para Coimbra fazer programas de Erasmus.

Para realizarmos esta atividade tínhamos 2 necessidades principais:
\begin{enumerate}
\item Saber quem é que realmente vinha frequentar o nosso departamento (alunos estrangeiros);
\item Recrutar alunos do DEEC, voluntários, com a função de acompanhar os novos estudantes de modo a tornar o seu semestre uma experiência empolgante e memorável. Para isso, teriam de acompanhá-los durante a sua estadia na nossa cidade e integrá-los na vida académica e nas tradições da \acrshort{uc}, garantindo o contacto dos mesmos com os órgãos corretos de apoio ao estudante.
\end{enumerate}

Para isso foi pedido à chefe da Secretaria, Maria João Cavaleiro, um registo dos novos estudantes internacionais que iram frequentar o \acrshort{mieec} em 2017/2018. De seguida, enviámos um mail a perguntar aos mesmos se estes queriam integrar-se neste novo projeto. Para os estudantes do \acrshort{deec} foi solicitado o preenchimento de um formulário com diversas questões de forma a conseguirmos fazer as relações entre cada estudante internacional e cada estudante português de forma o mais eficiente e confortável possível.

De ambos os lados não obtivemos respostas o que dificultou a execução do projeto, tendo levado ao seu cancelamento.

A receção dos estudantes internacionais no DEEC é um lacuna grave existente no nosso curso há já vários anos pelo que nos parece importante voltar a pensar em soluções neste sentido num futuro muito próximo, tendo também em atenção os possíveis relacionamentos quer com o \acrshort{dri}, quer com o Coordenador de Mobilidade do DEEC, já referidos em \ref{mobilidade_relacionamentos}.