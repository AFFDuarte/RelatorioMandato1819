% ==============================
% # Delegados de Ano           #
% ==============================

\subsubsection{Delegados de Ano}

A maior parte dos problemas pedagógicos são de pequena dimensão e começam a surgir nas cadeiras, são falados entre os amigos, passam para o Facebook, para as conversas de café e só chegam à Pedagogia quando o problema é já muito grande e difícil de controlar. Por esta razão o Pelouro de pedagogia teve a ideia de criar uma estrutura, os Delegados de Ano e, assim, tentar resolver este problema.

O objetivo principal desta iniciativa é que os problemas cheguem à pedagogia o mais rápido possível para que possamos entrar em ação atempadamente e dedicar o nosso tempo a iniciativas de maior dimensão e abrangência geral.

Desta forma, foram criados delegados de ano para cada ano de licenciatura, para cada ramo de mestrado e, após sugestão do Cristiano Alves numa \acrshort{rga}, aprovada pelos presentes, um delegado representante dos alunos com situações especiais atribuídas. Isto porque, à partida, esses delegados terão maior proximidade com os problemas das cadeiras e com a opinião dos alunos em relação aos mesmos, criando, assim, uma linha de comunicação viável e rápida o que irá proporcionar uma resolução célere.

Para isso tivemos de definir um conjunto de normas e orientações que tem como objetivo organizar este projeto, um regulamento.

\paragraph{Regulamento inicial}

Para realizarmos o regulamento tivemos que nos basear num regulamento já existente dos Delegados de Ano de Farmácia o que facilitou muito tanto a organização do documento bem como o que era necessário documentar para que não houvesse grandes dúvidas no momento de realização do projeto. Para trabalhar neste regulamento dividimos o Pelouro em várias subequipas:
\begin{enumerate}
\item Pesquisa e recolha de informação;
\item Escrever o documento;
\item Verificação e correção do documento.
\end{enumerate}

Com esta organização não tivemos grandes problemas na realização do documento pois todos os elementos sabiam a sua função e havia um grande espírito de entreajuda o que facilitou sempre a resolução dos eventuais problemas que pudessem aparecer.

O documento foi depois apresentado numa \acrshort{rga}, a 6 de dezembro, tendo sido revisto e aprovado por unanimidade. Também nessa \acrshort{rga} ficaram marcadas as eleições para os Delegados do ano letivo 2017/2018, delegados esses que só puderam entrar em funções em fevereiro de 2018, logo após às eleições dos mesmos.

\paragraph{Revisão do regulamento}

Em março foi feita a revisão do regulamento interno do \acrshort{neeec}. Aí, decidiu-se que o regulamento de delegados de ano passaria a estar estatutariamente previsto e aproveitou-se a ocasião para rever novamente o mesmo. Desta forma foi possível alterar algumas coisas que tinham corrido mal até então e definir uma estrutura mais estável para o ano letivo seguinte.

\paragraph{Eleições}

Este ano, por se tratar de um período de implementação, as eleições dos delegados de ano realizaram-se no início do 2º semestre e os seus mandatos terminarão com as eleições dos próximos delegados, no início do ano letivo 2018/2019. O objetivo dos mandatos se prolongarem ainda no início de cada ano letivo prende-se com o facto de ser necessário um feedback dos mesmos aquando da realização dos horários e da marcação de avaliações. A eleição dos delegados de ano ocorreu a 15 de fevereiro de 2018.

O período de divulgação eleitoral, pelo regulamento, deverá ocorrer durante dois dias consecutivos, devendo sempre haver um dia de reflexão entre o final da campanha e o dia das eleições. Desta forma, o período de divulgação, estava previsto para os dias 12 e 13 de fevereiro. Contudo, não houve campanha nestes dias embora um dos delegados, depois eleito, o pretende-se fazer. É muito importante, no futuro, informar todos os candidatos destas datas.

Os candidatos a Delegado de Ano deveriam manifestar o seu interesse mediante apresentação de candidatura ao \acrshort{neeec} com um mínimo de 7 dias antes da data da eleição, ou seja, neste caso, até ao dia 8 de fevereiro de 2018.

Todas as informações relativas aos prazos e datas constavam do regulamento de delegados de ano e foram ditas na \acrshort{rga} de dezembro. Por ser a primeira vez que estas eleições ocorriam, foi feita divulgação das informações mais importantes junto dos alunos do \acrshort{mieec} para que houvesse candidatos. Em nenhum dos casos houve eleições disputadas. Contudo, no 2º ano, tal só não aconteceu pois um dos candidatos entregou a sua candidatura já fora do prazo tendo a mesma sido anulada de imediato. No ramo de automação não houve candidatos o que trouxe inúmeros problemas no futuro, nunca se tendo conseguido arranjar nenhum. No ramo de computadores também não houve candidatos mas, após algumas conversações foi possível nomear um candidato, conforme previsto no regulamento.

\paragraph{Decorrer do projeto}

A Coordenação do Curso mostrou-se muito interessada nesta iniciativa parecendo que tal fosse correr muito bem. Contudo, o atraso com as eleições, um desleixo muito grande do nosso Pelouro perante este grupo, associada à falta de métodos laborais dos vários delegados bem como à falta de um contacto rápido e informal entre todos (por exemplo, uma conversa no WhatsApp) fez com que este projeto não corresse da melhor forma. Em março houve a primeira, e única, reunião entre a pedagogia do NEEEC, a coordenação de curso e os delegados, reunião que serviu para esclarecer algumas duvidas ainda existentes entre os novos delegados e algumas sugestões para o futuro do projeto e do próprio curso mas rapidamente acabou por se transformar num mini fórum pedagógico, não tendo sido estabelecidas formas de trabalho para o resto do semestre.

No futuro, consideramos que esta é uma iniciativa a manter e a melhorar bastante sendo importante haver formas estabelecidas de contacto e trabalho entre todos. É também muito importante utilizar a opinião dos delegados para a elaboração de horários e marcação de avaliações. Por fim, é de relembrar que os delegados de ano são agora uma estrutura do \acrshort{neeec} prevista no Regulamento Interno pelo que a sua não execução não fundamentada poderá levar à instauração de processos sobre a Direção do \acrshort{neeec}, por parte do \acrshort{cf}.
