% ==============================
% # Fórum Pedagógico           #
% ==============================

\subsubsection{Fórum Pedagógico}

O fórum pedagógico é um espaço de reflexão e debate sobre tudo o que se passou no \acrshort{mieec}/UC no último semestre onde todos os alunos e docentes podem colocar as suas dúvidas, sugestões, reclamações, etc. São também discutidos tópicos previamente escolhidos com maior relevo na presente época. É de referir que, muito provavelmente, este evento é o único momento do ano em que os alunos têm ao dispor uma sessão informal com a presença de professores e alunos, coordenada pela Pedagogia do \acrshort{neeec} e pela Coordenação de Curso, onde poderão debater estes tópicos da forma o mais sincera possível.
Durante o mandato foram feitos dois fóruns pedagógicos:
\begin{enumerate}
\item No primeiro:
	\begin{enumerate}
	\item Foi realizado no dia 11/out/2017, pelas 14:30 na biblioteca antiga no piso 5;
	\item Para cada fórum é preciso reservar a biblioteca ou o local onde se deseja realizar o evento, e levar o projetor do núcleo;
	\item É importante não esquecer de convidar todos os professores para o fórum pedagógico com, no mínimo, uma semana de antecedência fazendo um lembrete nos dias que antecedem;
    \item É também importante ir falar pessoalmente com os professores fazendo-lhes ver a importância da sua adesão e explicar que se estes não vão é também normal que os alunos não vão;
	\item Neste fórum os tópicos principais de discussão foram:
    	\begin{enumerate}
    	\item Reestruturação do \acrshort{mieec} 
		\item Os resultados dos inquéritos pedagógicos 
		\item A aprovação de um documento sobre a cadeira de Mecânica e Ondas
    	\end{enumerate}
	\end{enumerate}
\item No segundo:
	\begin{enumerate}
	\item Realizado no dia 7/03/18, no mesmo local e horas do anterior;
	\item Neste fórum os tópicos principais de discussão foram:
		\begin{enumerate}
		\item Inquéritos pedagógicos onde houve uma adesão histórica;
        \item Apresentação dos Delegados de Ano eleitos;
        \item A assiduidade às aulas, onde foi apresentado um power point com o ponto de vista de um aluno para melhorar a assiduidade dos alunos, sendo este um dos maiores problemas que o \acrshort{mieec} apresenta.
		\end{enumerate}
	\end{enumerate}
\end{enumerate}





