% ==========================
% # Inquéritos Pedagógicos #
% ==========================

\subsubsection{Inquéritos Pedagógicos}

Os inquéritos pedagógicos são a forma mais direta que o Pelouro da pedagogia tem de recolher informação dos estudantes sobre o que pensam das mais variadas situações que possam estar a acontecer ou tenham acontecido durante o semestre.

Pelos inquéritos podemos receber dúvidas, sugestões, reclamações, etc. para mais tarde expor os resultados no local mais correto, os Fóruns Pedagógicos, em frente aos alunos, professores, Direção do \acrshort{deec} e coordenação do \acrshort{mieec} e assim fazer diferença e tornar o nosso curso melhor para nós e para os que irão frequentá-lo no futuro.

Para realizar os inquéritos a equipa teve de se organizar de forma a que fossem colocadas as perguntas certas e de forma correta de forma a recolher a informação que pretendíamos. Uma vez que estamos a colocar questões a alunos de engenharia que, habitualmente, não dão pouca importância a estes eventos por falta de interesse ou por acharem que é uma perda de tempo, é essencial colocar as que questões de forma a que as respostas deem origem a respostas curtas mas concisas.

Para dividir a equipa, duas pessoas ficaram de fazer os inquéritos e, mais tarde, outras duas ficaram de analisar as respostas para levar ao Fórum Pedagógico.

As datas para estes inquéritos foram as seguintes:
\begin{itemize}
\item Os inquéritos foram publicados dia 29/07/2017. Contudo a divulgação feita para os mesmos foi extremamente fraca o que, conjugado com as férias, resultou num fraco número de respostas. No futuro recomendamos vivamente que os inquéritos sejam lançados no final de junho (não antes pois muitos problemas das cadeiras surgem na época de avaliações);
\item A 19/09/17, altura em que as aulas começaram, foi feito um reforço na divulgação para que o número de respostas fosse mais positivo;
\item 13/02/18 foram publicados os inquéritos para se obter informações sobre o primeiro semestre. Estes foram divulgados no site do núcleo, entretanto criado e, desta vez, tiveram imensa divulgação e uma campanha de comunicação associada, o "Sabias que...?". Nesta campanha foram divulgados vários exemplos onde a Pedagogia já interviu nos últimos anos através de questões que foram publicadas no instagram do NEEEC. Assim, foram colocados exemplos como "SABIAS QUE foi graças aos inquéritos pedagógicos que os trabalhos práticos de STR foram alterados no passado ano letivo? Não deixes a tua opinião ser tapada, preenche já os inquéritos disponíveis em https://neeec.pt/apoio-ao-estudante/inqueritos/." Com esta campanha, os inquéritos obtiveram 8 dezenas de respostas, número que consideramos escasso mas que foi um dos maiores da história do NEEEC.
\end{itemize}