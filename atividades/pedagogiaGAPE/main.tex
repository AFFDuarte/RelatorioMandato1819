% ========================
% # Pedagogia e GAPE     #
% ========================

\section{Pedagogia e GAPE}

\subsection{Introdução}

O Pelouro da Pedagogia e \acrshort{gape} do \acrshort{neeec} teve, desde o início, o propósito de responder a todas as dúvidas dos estudantes do \acrshort{mieec} e sanar os problemas que foram aparecendo nas cadeiras, ao longo do ano. Desde o início, todos os elementos da equipa apresentaram uma enorme disponibilidade para ajudar em todas as situações que foram aparecendo e, sobretudo, mostraram interesse. Por tudo isto que foi dito, liderar este grupo foi muito fácil, houve sempre um ambiente confortável, boa comunicação interna, motivação para realizar qualquer tarefa atribuída e todos aceitaram as suas responsabilidades. Como é natural, foram cometidos vários erros mas foi-se aprendendo com os membros e houve um trabalho conjunto para que estes não se repetissem.

Ao longo do ano, consoante o que tínhamos planeado fazer, os membros da equipa foram divididos em equipas que tinham tarefas diferentes atribuídas. As equipas iam variando ao longo do ano consoante as equipas que eram necessárias. Dessa forma foi possível realizar todas as tarefas de forma mais rápida e apresentar trabalho com a melhor qualidade possível. No final, as equipas que numa dada altura do ano não tinham trabalho ajudavam as outras, corrigindo as suas eventuais falhas ou erros.

\subsection{Atividades}

% ==========================
% # Inquéritos Pedagógicos #
% ==========================

\subsubsection{Inquéritos Pedagógicos}

Os inquéritos pedagógicos são a forma mais direta que o Pelouro da pedagogia tem de recolher informação dos estudantes sobre o que pensam das mais variadas situações que possam estar a acontecer ou tenham acontecido durante o semestre.

Pelos inquéritos podemos receber dúvidas, sugestões, reclamações, etc. para mais tarde expor os resultados no local mais correto, os Fóruns Pedagógicos, em frente aos alunos, professores, Direção do \acrshort{deec} e coordenação do \acrshort{mieec} e assim fazer diferença e tornar o nosso curso melhor para nós e para os que irão frequentá-lo no futuro.

Para realizar os inquéritos a equipa teve de se organizar de forma a que fossem colocadas as perguntas certas e de forma correta de forma a recolher a informação que pretendíamos. Uma vez que estamos a colocar questões a alunos de engenharia que, habitualmente, não dão pouca importância a estes eventos por falta de interesse ou por acharem que é uma perda de tempo, é essencial colocar as que questões de forma a que as respostas deem origem a respostas curtas mas concisas.

Para dividir a equipa, duas pessoas ficaram de fazer os inquéritos e, mais tarde, outras duas ficaram de analisar as respostas para levar ao Fórum Pedagógico.

As datas para estes inquéritos foram as seguintes:
\begin{itemize}
\item Os inquéritos foram publicados dia 29/07/2017. Contudo a divulgação feita para os mesmos foi extremamente fraca o que, conjugado com as férias, resultou num fraco número de respostas. No futuro recomendamos vivamente que os inquéritos sejam lançados no final de junho (não antes pois muitos problemas das cadeiras surgem na época de avaliações);
\item A 19/09/17, altura em que as aulas começaram, foi feito um reforço na divulgação para que o número de respostas fosse mais positivo;
\item 13/02/18 foram publicados os inquéritos para se obter informações sobre o primeiro semestre. Estes foram divulgados no site do núcleo, entretanto criado e, desta vez, tiveram imensa divulgação e uma campanha de comunicação associada, o "Sabias que...?". Nesta campanha foram divulgados vários exemplos onde a Pedagogia já interviu nos últimos anos através de questões que foram publicadas no instagram do NEEEC. Assim, foram colocados exemplos como "SABIAS QUE foi graças aos inquéritos pedagógicos que os trabalhos práticos de STR foram alterados no passado ano letivo? Não deixes a tua opinião ser tapada, preenche já os inquéritos disponíveis em https://neeec.pt/apoio-ao-estudante/inqueritos/." Com esta campanha, os inquéritos obtiveram 8 dezenas de respostas, número que consideramos escasso mas que foi um dos maiores da história do NEEEC.
\end{itemize}

% ==============================
% # Apadrinhamentos de Erasmus #
% ==============================

\subsubsection{Apadrinhamento de Erasmus}

O Buddy Program (ou Apadrinhamento de Erasmus) foi uma atividade que o Pelouro de pedagogia tentou implementar este ano a pensar nos estudantes internacionais que vêm para Coimbra fazer programas de Erasmus.

Para realizarmos esta atividade tínhamos 2 necessidades principais:
\begin{enumerate}
\item Saber quem é que realmente vinha frequentar o nosso departamento (alunos estrangeiros);
\item Recrutar alunos do DEEC, voluntários, com a função de acompanhar os novos estudantes de modo a tornar o seu semestre uma experiência empolgante e memorável. Para isso, teriam de acompanhá-los durante a sua estadia na nossa cidade e integrá-los na vida académica e nas tradições da \acrshort{uc}, garantindo o contacto dos mesmos com os órgãos corretos de apoio ao estudante.
\end{enumerate}

Para isso foi pedido à chefe da Secretaria, Maria João Cavaleiro, um registo dos novos estudantes internacionais que iram frequentar o \acrshort{mieec} em 2017/2018. De seguida, enviámos um mail a perguntar aos mesmos se estes queriam integrar-se neste novo projeto. Para os estudantes do \acrshort{deec} foi solicitado o preenchimento de um formulário com diversas questões de forma a conseguirmos fazer as relações entre cada estudante internacional e cada estudante português de forma o mais eficiente e confortável possível.

De ambos os lados não obtivemos respostas o que dificultou a execução do projeto, tendo levado ao seu cancelamento.

A receção dos estudantes internacionais no DEEC é um lacuna grave existente no nosso curso há já vários anos pelo que nos parece importante voltar a pensar em soluções neste sentido num futuro muito próximo, tendo também em atenção os possíveis relacionamentos quer com o \acrshort{dri}, quer com o Coordenador de Mobilidade do DEEC, já referidos em \ref{mobilidade_relacionamentos}.

% ==============================
% # Mapa de Avaliações         #
% ==============================

\subsubsection{Marcação de Avaliações}

A marcação de avaliações foi feita de forma diferente nos dois semestre uma vez que o Coordenador de Curso foi alterado no final do primeiro semestre.

No 1º semestre, foi feita uma reunião onde a Pedagogia esteve presente onde os professores foram dizendo onde pretendiam fazer as avaliações. Uma vez que a reunião se sobrepôs com a realização do ENE3, apenas estiveram presentes dois membros do Pelouro o que dificultou bastante o trabalho do mesmo dada a quantidade de trabalho e rapidez com que ocorre a reunião. O maior problema destas reuniões de marcação de exames é que os professores não estão todos presentes o que dificulta muito a organização de uma época de frequências que agrade a todos. Após estas reuniões é depois necessária analisar de novo o mapa e fazer sugestões de alteração envolvendo a coordenação de curso, os professores e os alunos. Este processo acaba depois por se arrastar bastante ao longo do tempo.

No 2º semestre a coordenação de curso quis mudar o estilo de avaliação e, para isso, foram criadas duas zonas onde se concentraram todas as avaliações permitindo assim aos alunos terem mais tempo para ir às aulas e depois terem um tempo apenas se focarem no estudo. Estas zonas foram criadas na semana antes e na semana depois das férias da páscoa e nas duas semanas após a Queima das Fitas. No final do ano foi feito um inquérito para saber a opinião dos estudantes tendo este tido, até ao momento, 66 respostas onde a maioria dos alunos diz que não concorda com a medida. Os inquéritos dispunham depois de uma forma de elaborar comentários havendo neste várias respostas importantes a analisar.

Para além da forma como é feita a marcação do mapa de avaliações temos que ter em conta a opinião dos alunos, por isso tem que se publicar nos sítios adequados o mapa e definir uma data final para deixar de ser possível modificar o mapa. Caso haja alguma sugestão de alteração tem que se falar com o professor em questão, mantendo sempre o coordenador de curso a par da situação de modo a verificar se a alteração pode ser feita.

% ==============================
% # Fórum Pedagógico           #
% ==============================

\subsubsection{Fórum Pedagógico}

O fórum pedagógico é um espaço de reflexão e debate sobre tudo o que se passou no \acrshort{mieec}/UC no último semestre onde todos os alunos e docentes podem colocar as suas dúvidas, sugestões, reclamações, etc. São também discutidos tópicos previamente escolhidos com maior relevo na presente época. É de referir que, muito provavelmente, este evento é o único momento do ano em que os alunos têm ao dispor uma sessão informal com a presença de professores e alunos, coordenada pela Pedagogia do \acrshort{neeec} e pela Coordenação de Curso, onde poderão debater estes tópicos da forma o mais sincera possível.
Durante o mandato foram feitos dois fóruns pedagógicos:
\begin{enumerate}
\item No primeiro:
	\begin{enumerate}
	\item Foi realizado no dia 11/out/2017, pelas 14:30 na biblioteca antiga no piso 5;
	\item Para cada fórum é preciso reservar a biblioteca ou o local onde se deseja realizar o evento, e levar o projetor do núcleo;
	\item É importante não esquecer de convidar todos os professores para o fórum pedagógico com, no mínimo, uma semana de antecedência fazendo um lembrete nos dias que antecedem;
    \item É também importante ir falar pessoalmente com os professores fazendo-lhes ver a importância da sua adesão e explicar que se estes não vão é também normal que os alunos não vão;
	\item Neste fórum os tópicos principais de discussão foram:
    	\begin{enumerate}
    	\item Reestruturação do \acrshort{mieec} 
		\item Os resultados dos inquéritos pedagógicos 
		\item A aprovação de um documento sobre a cadeira de Mecânica e Ondas
    	\end{enumerate}
	\end{enumerate}
\item No segundo:
	\begin{enumerate}
	\item Realizado no dia 7/03/18, no mesmo local e horas do anterior;
	\item Neste fórum os tópicos principais de discussão foram:
		\begin{enumerate}
		\item Inquéritos pedagógicos onde houve uma adesão histórica;
        \item Apresentação dos Delegados de Ano eleitos;
        \item A assiduidade às aulas, onde foi apresentado um power point com o ponto de vista de um aluno para melhorar a assiduidade dos alunos, sendo este um dos maiores problemas que o \acrshort{mieec} apresenta.
		\end{enumerate}
	\end{enumerate}
\end{enumerate}







% ==============================
% # Sessão sobre Erasmus       #
% ==============================

\subsubsection{Sessão sobre Erasmus}

No final de novembro, altura próxima do início das inscrições para o programa Erasmus, foi feita uma palestra informativa sobre este programa de mobilidade, onde o professor responsável pelo programa, Paulo Coimbra, deu uma apresentação para dar a conhecer o programa e responder às dúvidas dos alunos, que estavam interessados. Para além disso, o Diogo Abreu, aluno que este ano esteve a estudar em Bolonha, esteve presente na sessão via Skype, dando o seu ponto de vista e contando o que estava a presenciar no estrangeiro, as dificuldades que teve, as experiências, etc.

Esta sessão deveria ter contado com a presença de estudantes estrangeiros, que estão a estudar cá em Coimbra atualmente, mas tal não foi possível. Adicionalmente, o professor Paulo Coimbra gostaria de, à semelhança do ano anterior, fazer uma sessão numa sala com computadores onde os alunos pudessem fazer as suas candidaturas e esclarecer as suas dúvidas contudo, por o evento ter sido feito antes das candidaturas estarem abertas, tal acabou por se realizar noutra data, sem a colaboração do NEEEC. Esta colaboração teria sido interessante principalmente no que toca à divulgação da atividade.

% ==============================
% # Delegados de Ano           #
% ==============================

\subsubsection{Delegados de Ano}

A maior parte dos problemas pedagógicos são de pequena dimensão e começam a surgir nas cadeiras, são falados entre os amigos, passam para o Facebook, para as conversas de café e só chegam à Pedagogia quando o problema é já muito grande e difícil de controlar. Por esta razão o Pelouro de pedagogia teve a ideia de criar uma estrutura, os Delegados de Ano e, assim, tentar resolver este problema.

O objetivo principal desta iniciativa é que os problemas cheguem à pedagogia o mais rápido possível para que possamos entrar em ação atempadamente e dedicar o nosso tempo a iniciativas de maior dimensão e abrangência geral.

Desta forma, foram criados delegados de ano para cada ano de licenciatura, para cada ramo de mestrado e, após sugestão do Cristiano Alves numa \acrshort{rga}, aprovada pelos presentes, um delegado representante dos alunos com situações especiais atribuídas. Isto porque, à partida, esses delegados terão maior proximidade com os problemas das cadeiras e com a opinião dos alunos em relação aos mesmos, criando, assim, uma linha de comunicação viável e rápida o que irá proporcionar uma resolução célere.

Para isso tivemos de definir um conjunto de normas e orientações que tem como objetivo organizar este projeto, um regulamento.

\paragraph{Regulamento inicial}

Para realizarmos o regulamento tivemos que nos basear num regulamento já existente dos Delegados de Ano de Farmácia o que facilitou muito tanto a organização do documento bem como o que era necessário documentar para que não houvesse grandes dúvidas no momento de realização do projeto. Para trabalhar neste regulamento dividimos o Pelouro em várias subequipas:
\begin{enumerate}
\item Pesquisa e recolha de informação;
\item Escrever o documento;
\item Verificação e correção do documento.
\end{enumerate}

Com esta organização não tivemos grandes problemas na realização do documento pois todos os elementos sabiam a sua função e havia um grande espírito de entreajuda o que facilitou sempre a resolução dos eventuais problemas que pudessem aparecer.

O documento foi depois apresentado numa \acrshort{rga}, a 6 de dezembro, tendo sido revisto e aprovado por unanimidade. Também nessa \acrshort{rga} ficaram marcadas as eleições para os Delegados do ano letivo 2017/2018, delegados esses que só puderam entrar em funções em fevereiro de 2018, logo após às eleições dos mesmos.

\paragraph{Revisão do regulamento}

Em março foi feita a revisão do regulamento interno do \acrshort{neeec}. Aí, decidiu-se que o regulamento de delegados de ano passaria a estar estatutariamente previsto e aproveitou-se a ocasião para rever novamente o mesmo. Desta forma foi possível alterar algumas coisas que tinham corrido mal até então e definir uma estrutura mais estável para o ano letivo seguinte.

\paragraph{Eleições}

Este ano, por se tratar de um período de implementação, as eleições dos delegados de ano realizaram-se no início do 2º semestre e os seus mandatos terminarão com as eleições dos próximos delegados, no início do ano letivo 2018/2019. O objetivo dos mandatos se prolongarem ainda no início de cada ano letivo prende-se com o facto de ser necessário um feedback dos mesmos aquando da realização dos horários e da marcação de avaliações. A eleição dos delegados de ano ocorreu a 15 de fevereiro de 2018.

O período de divulgação eleitoral, pelo regulamento, deverá ocorrer durante dois dias consecutivos, devendo sempre haver um dia de reflexão entre o final da campanha e o dia das eleições. Desta forma, o período de divulgação, estava previsto para os dias 12 e 13 de fevereiro. Contudo, não houve campanha nestes dias embora um dos delegados, depois eleito, o pretende-se fazer. É muito importante, no futuro, informar todos os candidatos destas datas.

Os candidatos a Delegado de Ano deveriam manifestar o seu interesse mediante apresentação de candidatura ao \acrshort{neeec} com um mínimo de 7 dias antes da data da eleição, ou seja, neste caso, até ao dia 8 de fevereiro de 2018.

Todas as informações relativas aos prazos e datas constavam do regulamento de delegados de ano e foram ditas na \acrshort{rga} de dezembro. Por ser a primeira vez que estas eleições ocorriam, foi feita divulgação das informações mais importantes junto dos alunos do \acrshort{mieec} para que houvesse candidatos. Em nenhum dos casos houve eleições disputadas. Contudo, no 2º ano, tal só não aconteceu pois um dos candidatos entregou a sua candidatura já fora do prazo tendo a mesma sido anulada de imediato. No ramo de automação não houve candidatos o que trouxe inúmeros problemas no futuro, nunca se tendo conseguido arranjar nenhum. No ramo de computadores também não houve candidatos mas, após algumas conversações foi possível nomear um candidato, conforme previsto no regulamento.

\paragraph{Decorrer do projeto}

A Coordenação do Curso mostrou-se muito interessada nesta iniciativa parecendo que tal fosse correr muito bem. Contudo, o atraso com as eleições, um desleixo muito grande do nosso Pelouro perante este grupo, associada à falta de métodos laborais dos vários delegados bem como à falta de um contacto rápido e informal entre todos (por exemplo, uma conversa no WhatsApp) fez com que este projeto não corresse da melhor forma. Em março houve a primeira, e única, reunião entre a pedagogia do NEEEC, a coordenação de curso e os delegados, reunião que serviu para esclarecer algumas duvidas ainda existentes entre os novos delegados e algumas sugestões para o futuro do projeto e do próprio curso mas rapidamente acabou por se transformar num mini fórum pedagógico, não tendo sido estabelecidas formas de trabalho para o resto do semestre.

No futuro, consideramos que esta é uma iniciativa a manter e a melhorar bastante sendo importante haver formas estabelecidas de contacto e trabalho entre todos. É também muito importante utilizar a opinião dos delegados para a elaboração de horários e marcação de avaliações. Por fim, é de relembrar que os delegados de ano são agora uma estrutura do \acrshort{neeec} prevista no Regulamento Interno pelo que a sua não execução não fundamentada poderá levar à instauração de processos sobre a Direção do \acrshort{neeec}, por parte do \acrshort{cf}.


% ==============================
% # Situação de CG 			   #
% ==============================

\subsubsection{Situação de Computação Gráfica}

No decorrer da Unidade Curricular de \acrfull{cg}, cadeira obrigatória do plano curricular dos estudantes do ramo de Computadores no 4º ano do \acrfull{mieec} e opcional aos estudantes dos restantes ramos do curso, foram detetados vários problemas pelos alunos que a frequentaram, incidindo estes, essencialmente, sobre o docente da cadeira.

Os alunos decidiram reportar à equipa do Pelouro da Pedagogia estes problemas, tendo esta promovido contacto junto de todos os estudantes que frequentaram a Unidade Curricular de forma a procurar obter o maior conjunto de informações possível, resultando, então, num documento que foi apresentado numa \acrshort{rga} para aprovação dos alunos de forma a dar continuidade ao processo.

Após aprovação, o Pelouro enviou um email para a Direção do DEEC e para o professor para terem conhecimento do documento e dos problemas expostos pelos alunos.

Às dezassete horas do dia nove de novembro de 2017, teve lugar uma reunião no gabinete 3A.17 do \acrshort{deec}, onde estiveram presentes o Professor José Carlos Teixeira, docente regente da unidade curricular em questão no ano letivo de 2016/2017, o Coordenador-geral do Pelouro de Pedagogia do \acrshort{neeec}, Carlos Simões, o André  Duarte, em representação do Conselho Pedagógico, o João Martins, em  representação da Direção do \acrshort{neeec}, e três membros do Pelouro de Pedagogia do \acrshort{neeec} – João Ferreira, Francisco Veiga e Pedro Cavaleiro. Foram realizadas duas atas, uma por parte do docente, outra por parte do Pelouro da Pedagogia do \acrshort{neeec}.

Nesta reunião foram discutidos todos os pontos expostos no documento, onde o professor e o Pelouro defendeu os seus pontos de vista.

Depois de várias reuniões com o professor José Teixeira, docente de CG, onde ele argumentou e se defendeu do documento que elaboramos com as queixas dos alunos, este em contrapartida elaborou um documento onde apresenta tudo o que acha incorreto ou que não aconteceu.

O Pelouro de pedagogia achou que a forma de resolver esta situação de forma célere e eficaz foi convocar à Mesa do Plenário uma \acrshort{rga} onde estivessem presentes todos os órgãos ligados diretamente a este problema, professor, alunos, Direção do \acrshort{deec} e o Pelouro de pedagogia de forma a arranjarmos uma solução conjunta.

Para isso e para os alunos conseguirem argumentar os pontos que o professor discordava, enviamos um mail com o documento do professor para que os alunos pudessem analisar e elaborar a sua opinião para depois apresentarem o seu ponto de vista na \acrshort{rga}.

Por fim foi realizada a \acrshort{rga} onde se resolveram os problemas e se discutiram alguns tópicos para a edição seguinte da cadeira. No entanto, é de ressalvar que nesta \acrshort{rga} poucos foram os alunos que conseguiram falar abertamente sobre o assunto o que dificultou o diálogo.

% ==============================
% # Situação de MO 			   #
% ==============================

\subsubsection{Situação de Mecânica e Ondas}

Por sugestão do docente responsável pela unidade curricular de \acrfull{mo}, o Professor Fernando Sampaio dos Aidos, foi feito um estudo com vista à melhoria das normas de funcionamento da cadeira.

Nesse sentido, o Pelouro de Pedagogia do \acrshort{neeec} conduziu um inquérito junto dos alunos inscritos em Mecânica e Ondas no ano letivo de 2016/2017, com vista a recolher as suas opiniões acerca de possíveis melhorias a serem consideradas nas edições futuras da unidade curricular, em particular, o processo de avaliação que foi alvo de uma análise cuidada.

O inquérito de participação foi divulgado via Facebook, no grupo de alunos do \acrshort{mieec}. Realizaram-se duas publicações apelando à participação ativa dos alunos elegíveis. Adicionalmente, o Professor Sampaio dos Aidos apelou também à participação dos alunos durante as aulas.

Esta iniciativa constituiu uma oportunidade única, onde cada aluno pôde expressar a sua opinião de forma anónima, com vista a otimizar o funcionamento da unidade curricular nas suas edições futuras. Pela natureza desta oportunidade, esperava-se uma taxa de participação bastante elevada, o que não se veio a verificar. No entanto, os resultados recolhidos foram apresentados ao docente e este teve-os em consideração no planeamento da edição seguinte da disciplina.


% ====================================
% # Resolução de Problemas por Email #
% ====================================

\subsubsection{Resolução de Problemas por Email}

Ao longo do ano, a maior parte dos problemas são tratados maioritariamente por email por serem de fácil resolução e por possibilitarem um tratamento mais rápido e eficaz.

Quer nas redes sociais, quer pessoalmente, sempre que se verificavam comentários sobre problemas pedagógicos, os alunos foram incentivados a enviar sempre um email de forma a que o assunto fosse sempre tratado pelos meios oficiais e ficasse tudo arquivado para se saber como proceder em certos casos, no futuro.

\paragraph{Controlo}

Uma vez que as notas do exame de época especial de Controlo não foram disponibilizadas antes de setembro, altura das matrículas, os alunos não se poderiam inscrever no novo ano. Assim os mesmos estariam também impossibilitados de fazer os horários. Desta forma, o Pelouro entrou em contacto com os serviços de gestão académica e foi possível arranjar uma solução para que os alunos se pudessem inscrever nas turmas na hora correta. Adicionalmente, foi feita pressão para que as notas saíssem o mais rapidamente possível.

\paragraph{VPC}

As notas de \acrfull{vpc} não foram lançadas com a antecedência de três dias seguidos antes da data marcada para a realização do exame de recurso. Dessa forma os alunos inscritos na cadeira têm o direito a uma nova prova de avaliação, cabendo aos serviços de gestão académica a marcação dessa mesma prova tendo em conta o calendário dos alunos. 

Estes dois exemplos são uma amostra dos vários problemas que foram resolvidos através da troca de alguns emails bastando entrar em contacto com os serviços, os professores e os alunos para que todos fiquem agradados com a situação e os problemas fiquem sanados.
