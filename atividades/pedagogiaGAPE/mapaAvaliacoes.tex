% ==============================
% # Mapa de Avaliações         #
% ==============================

\subsubsection{Marcação de Avaliações}

A marcação de avaliações foi feita de forma diferente nos dois semestre uma vez que o Coordenador de Curso foi alterado no final do primeiro semestre.

No 1º semestre, foi feita uma reunião onde a Pedagogia esteve presente onde os professores foram dizendo onde pretendiam fazer as avaliações. Uma vez que a reunião se sobrepôs com a realização do ENE3, apenas estiveram presentes dois membros do Pelouro o que dificultou bastante o trabalho do mesmo dada a quantidade de trabalho e rapidez com que ocorre a reunião. O maior problema destas reuniões de marcação de exames é que os professores não estão todos presentes o que dificulta muito a organização de uma época de frequências que agrade a todos. Após estas reuniões é depois necessária analisar de novo o mapa e fazer sugestões de alteração envolvendo a coordenação de curso, os professores e os alunos. Este processo acaba depois por se arrastar bastante ao longo do tempo.

No 2º semestre a coordenação de curso quis mudar o estilo de avaliação e, para isso, foram criadas duas zonas onde se concentraram todas as avaliações permitindo assim aos alunos terem mais tempo para ir às aulas e depois terem um tempo apenas se focarem no estudo. Estas zonas foram criadas na semana antes e na semana depois das férias da páscoa e nas duas semanas após a Queima das Fitas. No final do ano foi feito um inquérito para saber a opinião dos estudantes tendo este tido, até ao momento, 66 respostas onde a maioria dos alunos diz que não concorda com a medida. Os inquéritos dispunham depois de uma forma de elaborar comentários havendo neste várias respostas importantes a analisar.

Para além da forma como é feita a marcação do mapa de avaliações temos que ter em conta a opinião dos alunos, por isso tem que se publicar nos sítios adequados o mapa e definir uma data final para deixar de ser possível modificar o mapa. Caso haja alguma sugestão de alteração tem que se falar com o professor em questão, mantendo sempre o coordenador de curso a par da situação de modo a verificar se a alteração pode ser feita.