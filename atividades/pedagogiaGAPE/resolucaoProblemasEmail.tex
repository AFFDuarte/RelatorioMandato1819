% ====================================
% # Resolução de Problemas por Email #
% ====================================

\subsubsection{Resolução de Problemas por Email}

Ao longo do ano, a maior parte dos problemas são tratados maioritariamente por email por serem de fácil resolução e por possibilitarem um tratamento mais rápido e eficaz.

Quer nas redes sociais, quer pessoalmente, sempre que se verificavam comentários sobre problemas pedagógicos, os alunos foram incentivados a enviar sempre um email de forma a que o assunto fosse sempre tratado pelos meios oficiais e ficasse tudo arquivado para se saber como proceder em certos casos, no futuro.

\paragraph{Controlo}

Uma vez que as notas do exame de época especial de Controlo não foram disponibilizadas antes de setembro, altura das matrículas, os alunos não se poderiam inscrever no novo ano. Assim os mesmos estariam também impossibilitados de fazer os horários. Desta forma, o Pelouro entrou em contacto com os serviços de gestão académica e foi possível arranjar uma solução para que os alunos se pudessem inscrever nas turmas na hora correta. Adicionalmente, foi feita pressão para que as notas saíssem o mais rapidamente possível.

\paragraph{VPC}

As notas de \acrfull{vpc} não foram lançadas com a antecedência de três dias seguidos antes da data marcada para a realização do exame de recurso. Dessa forma os alunos inscritos na cadeira têm o direito a uma nova prova de avaliação, cabendo aos serviços de gestão académica a marcação dessa mesma prova tendo em conta o calendário dos alunos. 

Estes dois exemplos são uma amostra dos vários problemas que foram resolvidos através da troca de alguns emails bastando entrar em contacto com os serviços, os professores e os alunos para que todos fiquem agradados com a situação e os problemas fiquem sanados.