% ==============================
% # Situação de CG 			   #
% ==============================

\subsubsection{Situação de Computação Gráfica}

No decorrer da Unidade Curricular de \acrfull{cg}, cadeira obrigatória do plano curricular dos estudantes do ramo de Computadores no 4º ano do \acrfull{mieec} e opcional aos estudantes dos restantes ramos do curso, foram detetados vários problemas pelos alunos que a frequentaram, incidindo estes, essencialmente, sobre o docente da cadeira.

Os alunos decidiram reportar à equipa do Pelouro da Pedagogia estes problemas, tendo esta promovido contacto junto de todos os estudantes que frequentaram a Unidade Curricular de forma a procurar obter o maior conjunto de informações possível, resultando, então, num documento que foi apresentado numa \acrshort{rga} para aprovação dos alunos de forma a dar continuidade ao processo.

Após aprovação, o Pelouro enviou um email para a Direção do DEEC e para o professor para terem conhecimento do documento e dos problemas expostos pelos alunos.

Às dezassete horas do dia nove de novembro de 2017, teve lugar uma reunião no gabinete 3A.17 do \acrshort{deec}, onde estiveram presentes o Professor José Carlos Teixeira, docente regente da unidade curricular em questão no ano letivo de 2016/2017, o Coordenador-geral do Pelouro de Pedagogia do \acrshort{neeec}, Carlos Simões, o André  Duarte, em representação do Conselho Pedagógico, o João Martins, em  representação da Direção do \acrshort{neeec}, e três membros do Pelouro de Pedagogia do \acrshort{neeec} – João Ferreira, Francisco Veiga e Pedro Cavaleiro. Foram realizadas duas atas, uma por parte do docente, outra por parte do Pelouro da Pedagogia do \acrshort{neeec}.

Nesta reunião foram discutidos todos os pontos expostos no documento, onde o professor e o Pelouro defendeu os seus pontos de vista.

Depois de várias reuniões com o professor José Teixeira, docente de CG, onde ele argumentou e se defendeu do documento que elaboramos com as queixas dos alunos, este em contrapartida elaborou um documento onde apresenta tudo o que acha incorreto ou que não aconteceu.

O Pelouro de pedagogia achou que a forma de resolver esta situação de forma célere e eficaz foi convocar à Mesa do Plenário uma \acrshort{rga} onde estivessem presentes todos os órgãos ligados diretamente a este problema, professor, alunos, Direção do \acrshort{deec} e o Pelouro de pedagogia de forma a arranjarmos uma solução conjunta.

Para isso e para os alunos conseguirem argumentar os pontos que o professor discordava, enviamos um mail com o documento do professor para que os alunos pudessem analisar e elaborar a sua opinião para depois apresentarem o seu ponto de vista na \acrshort{rga}.

Por fim foi realizada a \acrshort{rga} onde se resolveram os problemas e se discutiram alguns tópicos para a edição seguinte da cadeira. No entanto, é de ressalvar que nesta \acrshort{rga} poucos foram os alunos que conseguiram falar abertamente sobre o assunto o que dificultou o diálogo.