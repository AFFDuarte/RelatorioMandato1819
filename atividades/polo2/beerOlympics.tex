% ========================
% # Beer Olympics        #
% ========================

\subsubsection{Beer Olympics}

O Beer Olympics é um evento composto por várias modalidades que envolvem sempre jogos com cerveja desde corridas de sacos, beerpong, etc.

A primeira fase deste evento é realizado em conjunto com o \acrshort{nei}. Cada Núcleo é responsável pelas provas das equipas que o representam. Para além das provas, há também uma febrada, da responsabilidade de ambos os núcleos, a decorrer em simultâneo. Esta parceria é algo que achamos que se deva manter pois não há público suficiente, na altura em que se costuma realizar o evento, para dinamizar uma febrada única por Núcleo. Apesar de ser um bocado mais complicado organizar eventos com pessoas que têm métodos de trabalho diferentes, no final acaba por haver mais pessoas a ajudar e até mesmo, caso seja necessário, realizar provas com os dois núcleos. No presente ano letivo, apenas uma, de três equipas inscritas para competir pelo \acrshort{nei}, compareceu, acabando por competir com as equipas do \acrshort{neeec} (apesar disso, esta equipa garantiu passagem direta para a final, competindo pelo \acrshort{nei}). Também este ano, uma das equipas competiu pelo \acrshort{neeec} na primeira fase e acabou a competir a fase final pelo \acrshort{nei} (com a autorização dos restantes núcleos organizadores) pois um dos elementos do grupo era aluno do \acrshort{dei}.

A segunda fase, a final do evento, é realizado em conjunto com todos os núcleos do Polo 2, onde cada Núcleo leva as equipas com melhor classificação para competir contra as equipas dos outros núcleos.

É um evento que até atraí alguns participantes, mas, no presente ano letivo, não atraiu muitas pessoas à febrada. É preciso ter em atenção às compras, particularmente na compra da cerveja porque tecnicamente são dois eventos (febrada e a competição) a decorrer. É necessário também comunicação com o Núcleo “coorganizador” para definir o que cada um precisa de tratar (material para o jogos, compras, equipamentos necessários para a febrada – grelhador, máquina de finos, etc.).

É preciso ter em atenção às equipas inscritas e garantir que o respetivo pagamento das mesmas foi efetuado. Habitualmente feito no parque de estacionamento do \acrshort{deec}, este ano decorreu no cimo das escadas entre os dois departamentos. Este não é o melhor espaço para se realizar o evento (piso irregular, proximidade com estrada). O estacionamento do Departamento é, sem dúvida, o melhor espaço para se realizar pois é de maiores dimensões e tem mais condições (apesar da proximidade com a estrada, há um muro que ajuda a proteger um bocado o espaço e as pessoas). Apesar de ser o melhor espaço, trás algumas complicações e prejudica bastante a vida das pessoas pois tem de se fechar o estacionamento, o que é uma enorme falta de educação para quem não participa no evento e necessita de ir para o Departamento. Caso este espaço não esteja disponível, o melhor local é o espaço em frente ao \acrshort{dei} (espaçoso e seguro). Caso não haja outra opção que o parque de estacionamento do \acrshort{deec}, é imperativo fazer um aviso prévio a toda a comunidade que o mesmo se vai encontrar encerrado o dia todo, contudo, na nossa opinião, essa opção nunca deve ser tomada. No final do evento, independentemente da sua localização, há que garantir que o espaço é deixado limpo (principalmente os copos de plástico usados).

No presente ano letivo, a final do evento foi realizada no \acrshort{dem}, onde toda a organização do evento foi assegurada pelos coordenadores do evento (Tiago Caniço, David Pereira e Diogo Oliveira) algo que trouxe aspetos positivos, mas também negativos. O facto de não nos termos de preocupar com a sua organização é claramente positivo pois é menos trabalho que temos, mas ao mesmo tempo acabamos por sentir que o evento não é “nosso”, mas sim do Núcleo que assegurou a organização. A escala para esta fase, foi muito pouco preenchida pelo \acrshort{neeec}, mas os que se comprometeram a ajudar, cumpriram a escala (alguns tendo de fazer turnos extras por falta de comparência de outros núcleos).

No atual ano letivo, uma das equipas apuradas desistiu de participar à última da hora. Foi então contactada uma nova equipa, por ordem de classificação, que aceitou participar. Esta situação é algo que no futuro deve ser prevenida (ou pelo menos tentar prevenir), contactando as equipas com antecedência (ou até mesmo na manhã do dia da competição) para confirmar a sua presença e ter prontas eventuais substituições.
