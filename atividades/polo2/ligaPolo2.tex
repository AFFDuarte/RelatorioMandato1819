% ========================
% # Liga Polo 2          #
% ========================

\subsubsection{Liga Polo 2}

A Liga Polo 2 é um evento desportivo já realizado há vários anos onde competem as várias equipas apuradas dos torneios internos de cada Núcleo. De forma a que seja possível a realização de oitavos de final, cada Núcleo envia duas equipas e os dois núcleos com mais inscritos nos seus torneios internos enviam três equipas. Este é logo um sistema com problemas à partida pois não se sabe, inicialmente, quais os núcleos com mais inscritos, não se podendo garantir aos participantes quem é ou não apurado de imediato, algo que, na nossa opinião, é ridículo. Este ano, como o \acrshort{ng} apenas enviou uma equipa, o \acrshort{nei}, o \acrshort{neeec} e o \acrshort{neemaac} puderam enviar, de imediato, três equipas. O torneio em si, tem uma organização demasiado simples tendo imensas falhas principalmente nas escalas e nos pormenores do evento que poderiam dar qualidade ao mesmo. Não foi feita nenhuma divulgação do torneio, tendo apenas saído uma foto das medalhas, o que permitiu, de forma demasiado simples, assinalar a existência do torneio. Em suma, esta é como que uma Liga de Campeões do Polo 2 mas feita às escondidas. As falhas da escala fazem com que não haja apanha bolas, árbitros e suplentes definidos. Assim, as falhas ao longo do torneio são uma constante. A falta de ligação entre os vários Coordenadores do desporto faz com que o evento seja praticamente organizado pelo Presidente/Vice responsável pelo torneio e este, por sua vez, tem uma falta imensa de poder para obrigar os elementos dos vários núcleos a trabalhar. Desta forma, a liga Polo 2 acaba por ser organizada, quase na totalidade, pelo Núcleo ao qual pertence o Presidente/Vice responsável. Este ano, como tivémos três equipas a participar no torneio pelo que foi necessário pagar 30€ por cada uma. O torneio teve um lucro final de 30€ por Núcleo pelo que, na realidade, o \acrshort{neeec} pagou 60€ para as suas equipas jogarem dois jogos (uma desistiu e as restantes foram logo eliminadas). Novamente, aplicando o exemplo da Liga dos Campeões, este evento teria tudo para dar visibilidade aos participantes e poderia ser um reforço financeiro aos núcleos para estes poderem, por exemplo, baixar o preço dos seus torneios internos mas funciona exatamente ao contrário. No futuro, recomendamos vivamente a emissão, pelo menos, da final em livestream e um maior envolvimento da comunidade no evento. Note-se que este ano nem se sabia o calendário das competições. Adicionalmente, através do tio do André Soares, o \acrshort{neeec} arranjou as medalhas em acrílico com o símbolo da Liga Polo 2, medalhas essas que custaram 70 cêntimos por unidade e ficaram muito bem feitas pelo que recomendamos à sua imitação no futuro.