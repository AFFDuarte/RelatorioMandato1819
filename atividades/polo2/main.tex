% ========================
% # Polo 2               #
% ========================

\section{Polo 2} \label{sec:polo2}

\subsection{Introdução}

Ao longo da história, os sete núcleos do Polo 2 foram-se unindo de forma a potenciar as suas atividades e lutas estudantis. O Somos Polo 2 acaba por ser uma associação de estudantes não oficial e, como tal, sem qualquer tipo de cargos, objetivos e missão delineados.

Como atividades, o Somos Polo 2 realiza essencialmente atividades lúdicas e desportivas: o Mega Convívio, o Beer Olympics, a Liga Polo 2 e a Mega Febrada. Além disso realiza também muitas das iniciativas da receção ao caloiro em conjunto, nomeadamente as t-shirts e os brindes inseridos nos kits de caloiro. As atividades realizadas neste ano foram do exclusivo âmbito lúdico e consistiram na repetição dos eventos já realizados em anos anteriores.

No início do mandato, foi criada uma conversa de Facebook entre todos os Presidentes e Vice-Presidentes dos sete núcleos do Polo 2 tendo-se marcado uma reunião inicial onde foram apresentados pelos novos presidentes que já tinham sido vice-presidentes nos anos anteriores a existência dos seguintes cargos: Tesoureiro do Polo 2, Secretário do Polo 2, moderador das reuniões do Polo 2 e Coordenadores do Mega Convívio. No dia em que se teve conhecimento dos cargos, decorreu de imediato a nomeação das pessoas para os cargos tendo-se verificado o seguimento de algumas tradições estipuladas como o Secretário ser sempre o Presidente do NEDEQ, os Coordenadores do mega convívio serem sempre os presidentes do \acrshort{neeec}, do \acrshort{neec} e do \acrshort{neemaac} e o moderador das reuniões ser sempre o Presidente do \acrshort{nei}. Adicionalmente, o Tesoureiro do Polo 2 teve eleições disputadas tendo, no entanto, esta nomeação sido decidida no prazo de minutos, onde os tesoureiros candidatos, Tiago Caniço e Ivo Frazão, foram surpreendidos com a hipótese de, através de uma chamada telefónica, se apresentarem. A votação favoreceu o Tiago Caniço (4-3), Tesoureiro do \acrshort{neemaac}, tendo-se mantido a tradição que diz que os tesoureiros do Polo 2 devem ser sucessivamente, os tesoureiros do \acrshort{nei}, do \acrshort{neemaac} e do \acrshort{neeec}. Adicionalmente não é delineado nenhum cargo de responsabilidade máxima (como acontece com um Presidente do Núcleo em que este é responsabilizado por todas as situações que ocorram no Núcleo).
Durante o verão, devido à organização do mega convívio e da receção ao caloiro, existiram algumas reuniões de Polo 2 mas, após o mega convívio, as reuniões passaram a ser muito esporádicas. Ao longo do mandato, surgiu a ideia de se fazer também algumas ideias novas nomeadamente um torneio de vólei Polo 2, um Open Day Polo 2 para dar a conhecer aos alunos das escolas secundárias os cursos do Polo 2 e ainda a reivindicação do mau estado em que se encontram as vias do Polo 2.

A organização das diversas atividades no Polo 2 passa sempre pelos presidentes tendo sempre uma dimensão exagerada (exemplo: febradas onde se encomendam 15 barris e são gastos 2, provocando um esforço físico enorme da organização). Este ano foi também criado um Slack do Polo 2 tendo sido introduzidos, inicialmente, os Presidentes e Vice-Presidentes bem como o Tesoureiro do Polo 2. Posteriormente, adicionaram-se ao Slack os Coordenadores dos Pelouros de Cultura, Desporto, Imagem e Relações Externas mas, uma vez que esta integração foi feita demasiado tarde, não existiu nenhuma ligação entre os Coordenadores e o Somos Polo 2, tendo os eventos organizados sido sempre feitos "porque tinham de ser".

A falta de organização e estipulação de modos de funcionamento do polo 2 fez também com que em diversos eventos houvesse imensas formas diferentes de trabalho e em situações como o Open Day do Polo 2 houvesse envio de informações através de emails novos inventados por alguém e não pelo mail do polo 2, uma enorme falta de comunicação entre os pelouros e as direções dos núcleos fazendo com que o evento não tivesse qualquer capacidade para avançar.

No que toca à gestão financeira do Polo 2, ao início, cada Núcleo deve pagar ao Núcleo ao qual o Tesoureiro pertence 500€ para ser criada uma caixa. Este valor foi depois devolvido assim que o mega convívio terminou. Contudo, o \acrshort{neeec} pagou apenas 250€ dada a sua condição financeira, na altura. A gestão de todos os valores do Polo 2 são feitos pelo Tesoureiro do Polo 2, contudo, este em vez de comunicar com os restantes Tesoureiros do Núcleo comunicou com os Presidentes dos núcleos criando uma enorme confusão desnecessária na gestão das contas do mesmo. Adicionalmente, os Tesoureiros não foram inseridos no Slack do Polo 2 o que prejudicou imenso a relação dos mesmos.

Quanto aos barris, existiram inúmeros problemas ao longo do ano. Logo em setembro, foi feita uma encomenda conjunta para a receção ao caloiro e para o mega convívio. Contudo, opiniões diferentes sobre o fornecedor a utilizar fizeram com que houvesse vários desentendimentos e as máquinas, no dia da receção ao caloiro, chegaram minutos antes do início das febradas. Desta forma, as máquinas estiveram sempre quentes o que fez com que a quantidade de finos vendida tivesse sido sempre diminuta comparando com mandatos anteriores, o que largamente não compensou os poucos euros que se pouparam com a encomenda conjunta. Adicionalmente houve várias confusões com o gás tendo sido pago gás que nunca foi gasto por vários núcleos pelo que a poupança, de todo, não compensou. Ao longo do ano não foram feitas mais encomendas em conjunto e, no BeerOlympics, dado os acontecimentos de setembro, os barris foram encomendados em separado, o que foi muito melhor. Ao longo do ano existiram, no entanto, inúmeros problemas com barris principalmente pelo facto das empresas associadas ao Pitéu entenderem o Polo 2 como um só e não separarem os núcleos e, pior, os carros da Queima das Fitas, uns dos outros. Desta forma, os principais problemas consistiram nas máquinas emprestadas entre núcleos e entre núcleos e carros bem como na contabilização de barris também trocados entre as várias entidades. Adicionalmente, como o principal sítio de armazenamento de barris no Polo 2 é a garagem do \acrshort{dei}, aberta a qualquer um, existiram alguns problemas com barris que desapareceram e máquinas que foram trocadas, sem autorização. Após consultarmos algumas atas de anos anteriores acabámos por constar que o problema dos barris é já muito antigo tendo, apesar de tudo, este sido o ano em que o problema menos se demonstrou pois em mandatos anteriores as situações só se resolveram esquecendo os problemas e distribuindo o prejuízo por todos. No futuro, recomendamos vivamente encomendas totalmente separadas, uma explicação séria às várias empresas envolvidas no negócio sobre as diferenças entre os vários núcleos e os carros da Queima das Fitas, uma centralização dos barris dos carros junto dos respetivos núcleos e um registo muito pormenorizado e rígido da entrada e saída de barris, máquinas e botijas de gás do Polo 2.

O Somos Polo 2 é, para nós, a organização onde nos inserimos que pior funciona. O melhor proveito que podia ser tirado desta organização, na nossa opinião, seria a união das lutas dos vários núcleos, o que não é de todo aproveitada. Além disso, os eventos realizados dão prejuízo aos núcleos com maior atividade e não dão lucro aos núcleos com menor atividade (por exemplo, nós como tivémos três equipas a jogar na Liga Polo 2 tivémos de pagar 90€ tendo apenas recebido cerca de 30€ da atividade pelo que a participação na atividade nos dá prejuizo, e o mesmo se passou no BeerOlympics, enquanto que um Núcleo que tenha levado uma só equipa pagou 30€ e recebeu 30€) e o mega convívio dá quase sempre prejuízo tendo este ano, pela primeira vez em muitos, dado um lucro ridículo de 11€ por Núcleo, o que não compensa, de todo, o trabalho tido (um dia inteiro, entre as 9h da manhã de um dia, após uma noitada no dia anterior, e as 10h do outro, de trabalho algo intenso). Ao longo do ano, as contas são também facilmente confusas sendo que o saldo final dado é impossível de validar, tendo-se perdido inúmeras pequenas despesas pelo caminho por envolverem dezenas e dezenas de pessoas. Para que esta associação possa funcionar, recomendamos vivamente a criação de um regimento interno onde sejam definidos os cargos a eleger, todas as suas funções, um plano de atividades do Polo 2, consequências para o caso de algum Núcleo sair ou entrar numa dada atividade, responsáveis por cada área temática (Saídas Profissionais, Relações Externas, Cultura, Desporto e Imagem) e a criação de uma equipa de coordenação do Polo 2 composta por um Coordenador, um Vice-Coordenador, um Tesoureiro, um Administrador e um Secretário, todos de núcleos, obrigatoriamente, diferentes. Estas medidas poderiam levar a um menor trabalho por parte dos presidentes/vice-presidentes do Polo 2, algo que na nossa opinião era excelente funcionando este grupo de presidentes apenas como que um senado do Polo 2, que aprovam e dão o apoio do seu Núcleo às atividades. Isto faria com os presidentes e os vice se pudessem dedicar ao seu Núcleo, como é sua responsabilidade, e os responsáveis designados, pessoas que deveriam ter poucas tarefas dentro dos seu núcleos, se pudessem dedicar com mais afinco ao Polo 2.

\subsection{Atividades}

% ========================
% # Mega Convívio        #
% ========================

\subsubsection{Mega Convívio}

O Mega Convívio é um evento lúdico, organizado pelos sete Núcleos do Polo 2, com o intuito principal de receber os caloiros e proporcionar uma atividade dentro do Polo 2, única atividade deste género realizada ao longo do ano neste local. Esta atividade, realizou-se no dia 21 de setembro e a organização era composta pelos Presidentes do \acrshort{neeec}, \acrshort{neemaac} e \acrshort{neec}, como tem acontecido todos os anos. A coordenação deste evento correu bastante mal uma vez que os três coordenadores não trabalharam bem em conjunto. Por sua vez, o João Machado esteve em Lisboa em julho e agosto a fazer um estágio de verão, o Pedro Matias esteve a trabalhar na Sertã durante o verão e o João Bento este ocupado com o \acrshort{ene3} até 8 de setembro pelo que só a partir dessa data, já demasiado tarde, os três coordenadores conseguiram trabalhar em conjunto para o evento.

Durante o desenrolar do evento, existiu uma ajuda por parte da restante Direção e Coordenadores de cada Núcleo, no entanto, a maioria apenas realizou o seu papel mínimo obrigatório ao preencher as slots na escala, ficando todo o trabalho de voltar a deixar o espaço do recinto arrumado na responsabilidade novamente sobre os Presidentes/Vices dos Núcleos, como é costume nas atividades do Polo 2. De realçar que o espaço tem de estar limpo e livre até as 9h da manhã do dia seguinte à sua realização sendo esta uma tarefa muito cansativa para os membros da organização.

O evento tem uma complexidade muito grande, bem maior do que é normal num convívio de um só dia com apenas 1000 clientes, sendo que os detalhes do mesmo se encontram descritos no Guia do Mega Convívio disponível na drive do Polo 2. De ressalvar que todos os detalhes da edição de 2017 se encontram descritos pormenorizadamente pelo que é muito importante a leitura do documento.

Este é também um evento com vários problemas logísticos, desde a venda de tabaco, à forma como é feito o pagamento das zona vip, aos eternos problemas com as máquinas de cerveja, que causam inúmeras dores de cabeça durante a noite. Por sua vez, as borlas são uma constante e não se conseguem controlar sendo imperativo um sistema fortíssimo de vigilância na caixa e no serviço de finos. É também importante haver um controlo maior em tudo, nomeadamente na zona de entrega de bebidas aos carros e na entrega de senhas de finos às entradas. Os artistas são também exagerados tendo sido contratado um DJ à última da hora, supostamente conhecido e atrativo, sem o consentimento do \acrshort{neeec} que foi caro e não teve interesse nenhum, trazendo mais despesas ao evento.

Como aspetos importantes desta edição, destaca-se, de forma positiva, a inclusão dos carros da Queima das Fitas que trouxe uma dinâmica muito positiva ao evento. De forma negativa, existiram dois problemas graves a assinalar: as bebidas brancas excedentes desapareceram causando um prejuízo de centenas de euros que poderá ter ter tido origem ou num roubo ou numa errada contabilização das bebidas gastas pelos carros durante a noite; o outro problema foi o facto de ex-dirigentes do \acrshort{neeec} terem acedido à zona VIP de forma gratuita e terem trazido bebidas gratuitas para fora desta, para dar a outras pessoas, tendo causado alguns atritos desnecessários bem como o facto de outros ex-dirigentes do \acrshort{neeec} terem roubado a lona que fazia publicidade ao Mega nos jardins da \acrshort{aac} causando mais problemas desnecessários.

De realçar que esta edição contou com lucro de 11€ por núcleo, valor ridículo para um evento desta dimensão mas, infelizmente, a primeira vez que o evento deu lucro nos últimos anos.

% ========================
% # Mega Febrada        #
% ========================

\subsubsection{Mega Febrada}

A Mega Febrada foi, mais uma vez, um evento lúdico organizado pelos Núcleos do Polo 2. Teve início pelas 12h e término às 22h entre as escadas do \acrshort{dei} e \acrshort{deec} no dia 8 de fevereiro. Esta atividade tem como principal objetivo compensar os reduzidos lucros (ou prejuízos) do Mega Convívio e costumava realizar-se no início do primeiro semestre, antes da Latada, tendo lucros na ordem dos 1000€. Contudo, com Latada a realizar-se tão cedo não foi possível realizar este evento no 1º semestre e acabou por se realizar na primeira semana do 2º semestre.

A atividade consistiu numa febrada normal com início à hora de almoço. Contudo teve uma afluência muito reduzida tendo o lucro normal de uma febrada organizada por um só núcleo, cerca de 180 euros, dando 30 euros a cada núcleo. Na logística desta edição as principais falhas prenderam-se com o facto das compras terem sido feitas demasiado tarde e de tudo ter ficado montado já muito em cima da hora de almoço provocando filas grandes que fizeram com que as pessoas indecisas optassem por almoçar na cantina. O facto de ser uma atividade por tantos núcleos só causa problemas acabando por ser um evento demasiado cansativo. Para a febrada o \acrshort{neemaac} encomendou 15 barris e 3 máquinas de cerveja tendo sido usados 2 barris e 1 máquina contudo, esta encomenda, provocou um esforço físico gigante, absolutamente desnecessário.

No futuro, recomenda-se vivamente voltar aos moldes anteriores, ou seja, a fazer um género de sunset, com música, com bom tempo e calor, numa altura em que as pessoas ainda não tem enjoado de febradas, para que haja mais sucesso na iniciativa.

% ========================
% # Liga Polo 2          #
% ========================

\subsubsection{Liga Polo 2}

A Liga Polo 2 é um evento desportivo já realizado há vários anos onde competem as várias equipas apuradas dos torneios internos de cada Núcleo. De forma a que seja possível a realização de oitavos de final, cada Núcleo envia duas equipas e os dois núcleos com mais inscritos nos seus torneios internos enviam três equipas. Este é logo um sistema com problemas à partida pois não se sabe, inicialmente, quais os núcleos com mais inscritos, não se podendo garantir aos participantes quem é ou não apurado de imediato, algo que, na nossa opinião, é ridículo. Este ano, como o \acrshort{ng} apenas enviou uma equipa, o \acrshort{nei}, o \acrshort{neeec} e o \acrshort{neemaac} puderam enviar, de imediato, três equipas. O torneio em si, tem uma organização demasiado simples tendo imensas falhas principalmente nas escalas e nos pormenores do evento que poderiam dar qualidade ao mesmo. Não foi feita nenhuma divulgação do torneio, tendo apenas saído uma foto das medalhas, o que permitiu, de forma demasiado simples, assinalar a existência do torneio. Em suma, esta é como que uma Liga de Campeões do Polo 2 mas feita às escondidas. As falhas da escala fazem com que não haja apanha bolas, árbitros e suplentes definidos. Assim, as falhas ao longo do torneio são uma constante. A falta de ligação entre os vários Coordenadores do desporto faz com que o evento seja praticamente organizado pelo Presidente/Vice responsável pelo torneio e este, por sua vez, tem uma falta imensa de poder para obrigar os elementos dos vários núcleos a trabalhar. Desta forma, a liga Polo 2 acaba por ser organizada, quase na totalidade, pelo Núcleo ao qual pertence o Presidente/Vice responsável. Este ano, como tivémos três equipas a participar no torneio pelo que foi necessário pagar 30€ por cada uma. O torneio teve um lucro final de 30€ por Núcleo pelo que, na realidade, o \acrshort{neeec} pagou 60€ para as suas equipas jogarem dois jogos (uma desistiu e as restantes foram logo eliminadas). Novamente, aplicando o exemplo da Liga dos Campeões, este evento teria tudo para dar visibilidade aos participantes e poderia ser um reforço financeiro aos núcleos para estes poderem, por exemplo, baixar o preço dos seus torneios internos mas funciona exatamente ao contrário. No futuro, recomendamos vivamente a emissão, pelo menos, da final em livestream e um maior envolvimento da comunidade no evento. Note-se que este ano nem se sabia o calendário das competições. Adicionalmente, através do tio do André Soares, o \acrshort{neeec} arranjou as medalhas em acrílico com o símbolo da Liga Polo 2, medalhas essas que custaram 70 cêntimos por unidade e ficaram muito bem feitas pelo que recomendamos à sua imitação no futuro.

% ========================
% # Beer Olympics        #
% ========================

\subsubsection{Beer Olympics}

O Beer Olympics é um evento composto por várias modalidades que envolvem sempre jogos com cerveja desde corridas de sacos, beerpong, etc.

A primeira fase deste evento é realizado em conjunto com o \acrshort{nei}. Cada Núcleo é responsável pelas provas das equipas que o representam. Para além das provas, há também uma febrada, da responsabilidade de ambos os núcleos, a decorrer em simultâneo. Esta parceria é algo que achamos que se deva manter pois não há público suficiente, na altura em que se costuma realizar o evento, para dinamizar uma febrada única por Núcleo. Apesar de ser um bocado mais complicado organizar eventos com pessoas que têm métodos de trabalho diferentes, no final acaba por haver mais pessoas a ajudar e até mesmo, caso seja necessário, realizar provas com os dois núcleos. No presente ano letivo, apenas uma, de três equipas inscritas para competir pelo \acrshort{nei}, compareceu, acabando por competir com as equipas do \acrshort{neeec} (apesar disso, esta equipa garantiu passagem direta para a final, competindo pelo \acrshort{nei}). Também este ano, uma das equipas competiu pelo \acrshort{neeec} na primeira fase e acabou a competir a fase final pelo \acrshort{nei} (com a autorização dos restantes núcleos organizadores) pois um dos elementos do grupo era aluno do \acrshort{dei}.

A segunda fase, a final do evento, é realizado em conjunto com todos os núcleos do Polo 2, onde cada Núcleo leva as equipas com melhor classificação para competir contra as equipas dos outros núcleos.

É um evento que até atraí alguns participantes, mas, no presente ano letivo, não atraiu muitas pessoas à febrada. É preciso ter em atenção às compras, particularmente na compra da cerveja porque tecnicamente são dois eventos (febrada e a competição) a decorrer. É necessário também comunicação com o Núcleo “coorganizador” para definir o que cada um precisa de tratar (material para o jogos, compras, equipamentos necessários para a febrada – grelhador, máquina de finos, etc.).

É preciso ter em atenção às equipas inscritas e garantir que o respetivo pagamento das mesmas foi efetuado. Habitualmente feito no parque de estacionamento do \acrshort{deec}, este ano decorreu no cimo das escadas entre os dois departamentos. Este não é o melhor espaço para se realizar o evento (piso irregular, proximidade com estrada). O estacionamento do Departamento é, sem dúvida, o melhor espaço para se realizar pois é de maiores dimensões e tem mais condições (apesar da proximidade com a estrada, há um muro que ajuda a proteger um bocado o espaço e as pessoas). Apesar de ser o melhor espaço, trás algumas complicações e prejudica bastante a vida das pessoas pois tem de se fechar o estacionamento, o que é uma enorme falta de educação para quem não participa no evento e necessita de ir para o Departamento. Caso este espaço não esteja disponível, o melhor local é o espaço em frente ao \acrshort{dei} (espaçoso e seguro). Caso não haja outra opção que o parque de estacionamento do \acrshort{deec}, é imperativo fazer um aviso prévio a toda a comunidade que o mesmo se vai encontrar encerrado o dia todo, contudo, na nossa opinião, essa opção nunca deve ser tomada. No final do evento, independentemente da sua localização, há que garantir que o espaço é deixado limpo (principalmente os copos de plástico usados).

No presente ano letivo, a final do evento foi realizada no \acrshort{dem}, onde toda a organização do evento foi assegurada pelos coordenadores do evento (Tiago Caniço, David Pereira e Diogo Oliveira) algo que trouxe aspetos positivos, mas também negativos. O facto de não nos termos de preocupar com a sua organização é claramente positivo pois é menos trabalho que temos, mas ao mesmo tempo acabamos por sentir que o evento não é “nosso”, mas sim do Núcleo que assegurou a organização. A escala para esta fase, foi muito pouco preenchida pelo \acrshort{neeec}, mas os que se comprometeram a ajudar, cumpriram a escala (alguns tendo de fazer turnos extras por falta de comparência de outros núcleos).

No atual ano letivo, uma das equipas apuradas desistiu de participar à última da hora. Foi então contactada uma nova equipa, por ordem de classificação, que aceitou participar. Esta situação é algo que no futuro deve ser prevenida (ou pelo menos tentar prevenir), contactando as equipas com antecedência (ou até mesmo na manhã do dia da competição) para confirmar a sua presença e ter prontas eventuais substituições.
