% ========================
% # Mega Convívio        #
% ========================

\subsubsection{Mega Convívio}

O Mega Convívio é um evento lúdico, organizado pelos sete Núcleos do Polo 2, com o intuito principal de receber os caloiros e proporcionar uma atividade dentro do Polo 2, única atividade deste género realizada ao longo do ano neste local. Esta atividade, realizou-se no dia 21 de setembro e a organização era composta pelos Presidentes do \acrshort{neeec}, \acrshort{neemaac} e \acrshort{neec}, como tem acontecido todos os anos. A coordenação deste evento correu bastante mal uma vez que os três coordenadores não trabalharam bem em conjunto. Por sua vez, o João Machado esteve em Lisboa em julho e agosto a fazer um estágio de verão, o Pedro Matias esteve a trabalhar na Sertã durante o verão e o João Bento este ocupado com o \acrshort{ene3} até 8 de setembro pelo que só a partir dessa data, já demasiado tarde, os três coordenadores conseguiram trabalhar em conjunto para o evento.

Durante o desenrolar do evento, existiu uma ajuda por parte da restante Direção e Coordenadores de cada Núcleo, no entanto, a maioria apenas realizou o seu papel mínimo obrigatório ao preencher as slots na escala, ficando todo o trabalho de voltar a deixar o espaço do recinto arrumado na responsabilidade novamente sobre os Presidentes/Vices dos Núcleos, como é costume nas atividades do Polo 2. De realçar que o espaço tem de estar limpo e livre até as 9h da manhã do dia seguinte à sua realização sendo esta uma tarefa muito cansativa para os membros da organização.

O evento tem uma complexidade muito grande, bem maior do que é normal num convívio de um só dia com apenas 1000 clientes, sendo que os detalhes do mesmo se encontram descritos no Guia do Mega Convívio disponível na drive do Polo 2. De ressalvar que todos os detalhes da edição de 2017 se encontram descritos pormenorizadamente pelo que é muito importante a leitura do documento.

Este é também um evento com vários problemas logísticos, desde a venda de tabaco, à forma como é feito o pagamento das zona vip, aos eternos problemas com as máquinas de cerveja, que causam inúmeras dores de cabeça durante a noite. Por sua vez, as borlas são uma constante e não se conseguem controlar sendo imperativo um sistema fortíssimo de vigilância na caixa e no serviço de finos. É também importante haver um controlo maior em tudo, nomeadamente na zona de entrega de bebidas aos carros e na entrega de senhas de finos às entradas. Os artistas são também exagerados tendo sido contratado um DJ à última da hora, supostamente conhecido e atrativo, sem o consentimento do \acrshort{neeec} que foi caro e não teve interesse nenhum, trazendo mais despesas ao evento.

Como aspetos importantes desta edição, destaca-se, de forma positiva, a inclusão dos carros da Queima das Fitas que trouxe uma dinâmica muito positiva ao evento. De forma negativa, existiram dois problemas graves a assinalar: as bebidas brancas excedentes desapareceram causando um prejuízo de centenas de euros que poderá ter ter tido origem ou num roubo ou numa errada contabilização das bebidas gastas pelos carros durante a noite; o outro problema foi o facto de ex-dirigentes do \acrshort{neeec} terem acedido à zona VIP de forma gratuita e terem trazido bebidas gratuitas para fora desta, para dar a outras pessoas, tendo causado alguns atritos desnecessários bem como o facto de outros ex-dirigentes do \acrshort{neeec} terem roubado a lona que fazia publicidade ao Mega nos jardins da \acrshort{aac} causando mais problemas desnecessários.

De realçar que esta edição contou com lucro de 11€ por núcleo, valor ridículo para um evento desta dimensão mas, infelizmente, a primeira vez que o evento deu lucro nos últimos anos.