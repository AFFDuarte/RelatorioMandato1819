% ===================================
% # Futurália e Qualifica           #
% ===================================

\subsubsection{Feiras de Oportunidades: Futurália e Qualifica}

No seguimento de uma reunião com o \acrfull{gad} da \acrshort{fctuc} descobrimos que os núcleos de estudantes podiam representar os respetivos departamentos nas feiras Qualifica e Futurália com os custos suportados pela \acrshort{fctuc}. As idas em representação dos departamentos são sempre coordenadas com o representante da divulgação dos mesmos, no nosso caso o Vice-Diretor, que faz a ponte entre o \acrshort{deec} e a \acrshort{fctuc}.

Devido à indisponibilidade, mais uma vez, dos membros do Pelouro das Relações Externas não foi possível estar presente na Qualifica e foram apenas dois membros da Direção no sábado da Futurália. Como referido acima o transporte e alimentação são cobertos pela \acrshort{fctuc} (alimentação em qualquer local com um custo até 9.96€, no presente ano).

É importante coordenar as visitas com o Clube de Robótica para que estes possam levar algo prático e relacionado com o curso de Engenharia Eletrotécnica e de Computadores que possa chamar a atenção das pessoas.

A visita em si correu bastante bem, mas foi o suficiente para perceber que se devia estar presente nos restantes dias, uma vez que não estando presentes, o nosso curso não é devidamente divulgado quando alguém procura por ele. Durante a semana os alunos vão com as respetivas escolas e no fim de semana com os pais. Estes são públicos alvo diferentes e que devem ser encarados de maneira diferente.

Recomenda-se a quem vai representar o nosso curso que tenha uma pequena formação sobre as várias áreas do nosso curso e também um pouco sobre a academia e ação social, embora haja pessoas destinadas para falar desta, enviadas pela \acrshort{neeec} e pelos SASUC. O guia de divulgação começado neste ano é também bastante importante, mas este necessita de ser mais completo e atualizado com a nova reforma curricular.

Um lembrete importante: é essencial exterminar os atuais panfletos do Departamento que estão desatualizados e aparecem, de surpresa, nas feiras de oportunidades.