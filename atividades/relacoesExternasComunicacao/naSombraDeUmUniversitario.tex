% ===================================
% # Na Sombra de um Universitário   #
% ===================================

\subsubsection{Na Sombra de um Universitário}

O \acrshort{neeec} foi contactado pela Câmara Municipal de Mortágua, através do Professor Paulo Peixoto, na sequência de uma iniciativa com o nome "Na Sombra de um Aluno Universitário" que consistia na vinda de um aluno do ensino secundário ao nosso Departamento com o propósito de acompanhar um estudante universitário na sua rotina diária. A proposta foi aceite de imediato, mas o programa foi um pouco mudado. Foi decidido que o aluno iria a uma, ou mais que uma, aula na parte da manhã e durante a tarde haveria uma visita aos laboratórios do Departamento, pelo que foram contactados todos os professores responsáveis pelos mesmos. Chegado o dia, o aluno foi a uma aula de Estruturas de Dados e Algoritmos, visitou todas as organizações estudantis do Departamento e de seguida os laboratórios que lhe permitiram perceber os vários ramos da Engenharia Eletrotécnica. Por fim, visitámos o Diretor do Departamento no seu gabinete e este proferiu umas palavras de incentivo. Este tipo de iniciativas são extremamente importantes para dar visibilidade ao nosso Departamento de modo a angariar mais e melhores alunos e devem ser repetidas. Recomenda-se que o aluno seja acompanhado de um estudante mais velho que saiba um pouco de tudo o que se faz no nosso Departamento.