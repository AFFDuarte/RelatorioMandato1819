% ===================================
% # Aula de Inglês                  #
% ===================================

\subsubsection{Aula de Inglês}

A aula de inglês foi um género de workshop ministrado pelo Curso Privado de Inglês, derivado do protocolo estabelecido com esta escola. Este é um evento que, na nossa opinião, não se deve repetir pois a adesão foi quase nula e não houve interesse dos alunos por este evento. O feedback dos participantes não foi muito positivo: apenas tivemos feedback de 50\% dos participantes, sendo que 25\% destes achou o evento interessante e útil para o seu futuro, e os restantes 25\%, apesar de acharem interessante a iniciativa, não a acharam útil para o seu futuro. O professor diz ter gostado da iniciativa, apesar de se inscreverem pouquíssimos alunos. O objetivo do formador era também divulgar a sua escola o que acabou por não ter o impacto esperado. Para que a aula tivesse corrido melhor talvez se pudesse ter escolhido outra hora assim como proceder à divulgação do evento com mais antecipação dando deste modo mais visibilidade ao evento. O evento também foi organizado muito em cima da hora, pelo que a organização do mesmo decorreu em apenas 4 dias, embora esta fosse uma organização muito simples. Além destas razões há uma certa vergonha por parte dos alunos em participar num evento onde têm de se expor, falando em inglês.
