% ===================================
% # Saídas Profissionais e Formação #
% ===================================

\section{Saídas Profissionais e Formação}

\subsection{Introdução}

A equipa deste Pelouro no presente mandato foi formada por 5 elementos, incluindo o Coordenador Geral. Para a realização dos eventos a equipa foi dividida em 2 grupos, onde cada grupo ficou responsável por uma atividade diferente (por exemplo, cada um dos workshops). Nas atividades maiores, tais como a \acrshort{f3e} e a Semana dos Ramos, a equipa foi dividida em grupos e cada grupo ficou responsável por organizar uma ou várias atividades do programa, sendo que no dia do evento todos os elementos estavam informados de todas as atividades e de todo o funcionamento do evento para que assim pudessem contribuir para o sucesso do mesmo

O mandato começou, com o evento de maior dimensão deste Pelouro, a \acrfull{f3e}.

Depois da \acrshort{f3e} seguiram-se, ao longo do 1º semestre, vários workshops tais como uma aula de inglês, um workshop de Autocad, um workshop de Impressão 3D em conjunto com o clube de Robótica e um workshop de Machine Learning.

No segundo semestre realizou-se a Semana dos Ramos, a visita à Ubiwhere, o workshop de Simulink, o workshop de QT, o workshop de Excel, o workshop de Arduino e o Workshop de Android. Pensamos também em fazer um workshop de primeiros socorros, um workshop de Instrumentação e Medidas e um workshop de HTML, mas o mesmo não foi possível, o que será abordado mais à frente.

\subsection{Atividades}

% ===================================
% # Visita à Ubiwhere               #
% ===================================

\subsubsection{Visita à Ubiwhere}

A visita à Ubiwhere realizou-se em Aveiro, numa tarde de quarta-feira. Para o transporte pedimos aos participantes para que levassem os seus carros, tendo sido as despesas divididas por todos de igual forma. Estas contas deram bastante confusão, assim como a viagem, pelo que sugerimos que, no futuro, quando houver algum evento deste tipo o transporte seja feito de autocarro ou similar, de forma a que os participantes possam ir todos juntos e ao mesmo tempo. De ressalvar, no entanto, que a opção por transporte individual foi muito mais económica.

% ===========================
% # Semana dos Ramos        #
% ===========================

\subsubsection{Semana dos Ramos}

A Semana dos Ramos é um evento já organizado desde 2012 que tem como objetivo principal dar a conhecer os vários ramos de mestrado do MiEEC/UC a todos os alunos do 3º ano. Ao longo dos anos, este evento tem tido vários formatos diferentes, adaptando-se a circunstâncias diferentes.

No presente ano, o tema que mais se falava em todo o \acrshort{deec} era a reestruturação do curso e os alunos estavam curiosos para saber o que iria acontecer após a mesma. Assim, decidiu-se que a semana dos ramos se focaria nesse mesmo tema.

Decidiu-se também realizar o evento em 2 dias, apesar do número de atividades da presente edição ser até maior que o número de atividades da edição anterior, que durou 4 dias. O primeiro dia foi dedicado ao percurso dentro do curso e o segundo dia dedicado ao futuro dos estudantes, após o curso.

O evento começou no dia 12 de março com o professor Humberto Jorge a expor o novo plano de estudos, a implementar com a reestruturação do curso, explicando o que iria acontecer em cada ano e respondendo a todas as questões dos alunos, numa sessão que foi bastante concorrida e que durou quase uma manhã inteira. 

Após a discussão sobre a reestruturação do curso houve um debate entre os 4 ramos com a presença de um professor e aluno de cada ramo. Os professores escolhidos e os respetivos alunos foram: Professor Humberto Jorge com o aluno André Duarte a representar o ramo de energia; Professor Paulo Peixoto com o aluno Ivo Frazão a representar o ramo de computadores; Professor Urbano Nunes com o aluno Luís Garrote a representar o ramo de automação e Professora Maria do Carmo Medeiros com o aluno Frederico Vaz a representar o ramo de telecomunicação. Cada professor e aluno teve ao seu dispor cerca de 5 minutos para expor o seu ramo, para que no final houvesse tempo para esclarecimento de dúvidas, mas este prazo não foi cumprido, havendo professores a falar mais de 30 minutos, pelo que nas próximas edições há uma enorme necessidade de arranjar uma forma de cumprir estes horários.

Na parte da tarde, começámos por ter um debate tese vs estágio \footnote{Este nome é errado e tal facto já foi alertado pelo Professor Lino Marques uma vez que o trabalho desenvolvido no final do mestrado é uma dissertação e não uma tese (esse é o trabalho desenvolvido no final do doutoramento). O nome estágio está também errado pois o que é feito numa empresa é também um projeto de investigação que deve ser devidamente documento.}, com a presença do professor Marco Gomes e o aluno André Silva para representar as dissertações no DEEC e a presença do professor Humberto Jorge para representar as dissertações em regime empresarial. Esta sessão decorreu em forma de debate, tendo sido moderado pelo João Martins. Nesta sessão houve bastante interação dos alunos tendo o mesmo estendido-se mais que o tempo estipulado, atrasando as restantes atividades.
No final do dia, houve ainda dois workshops, um dedicado à escrita de documentos em \LaTeX, com o exemplo dos trabalhos a desenvolver nas dissertações, e outro dedicado a como construir uma tese, ambos lecionados pelo professor Tony Almeida e com casa cheia. O workshop de \LaTeX correu muito bem, foi bastante interessante e os alunos ficaram com as bases necessárias para iniciar o trabalhar com a ferramenta. Quanto ao workshop de Construção de Tese, houve muitos alunos que disseram que o que tinha sido dito já sabiam e que esperavam uma coisa mais aprofundada e não tão superficial.

O segundo dia, que foi dedicado à Académica Start UC, começou com um workshop de empreendedorismo. Este evento apresentou uma componente menos teórica e expositiva de exemplos ilustrativos, como pedido aos oradores, de como se deve criar e gerir um negócio, nomeadamente as "burocracias" importantes a ter em consideração durante este processo. A adesão da parte dos participantes foi relativamente boa (cerca de 15 pessoas). O Workshop foi dividido em duas partes: a primeira, ministrada pelo Eng. Jorge Figueira, Chefe de Divisão da \acrfull{dits} e uma segunda parte pelo Dr. Miguel Gonçalves também pertencente à \acrfull{dits}. Em ambas as partes, os participantes acharam o evento bastante dinâmico e interessante.

De seguida, realizou-se um concurso de ideias de negócio. O concurso consistia em realizar um pitch com tempo limite de 3 minutos, sobre uma ideia de negócio que era avaliada por um júri, composto pelo Dr. Miguel Gonçalves e pela Dra. Deolinda Estevinho da \acrfull{dits} e pelo Dr. Jorge Pimenta do \acrfull{ipn}.

A adesão ao evento foi quase nula e não existiu interesse da parte dos alunos pelo evento, alguma parte devido a algum medo pela exposição em público. As inscrições grátis também não ajudaram na realização do concurso, pois várias pessoas inscreveram-se e acabaram por não aparecer o que, na nossa opinião, é uma falta de respeito. Aconselhamos que no futuro, as inscrições devam ter sempre um custo simbólico de forma a precaver esta situação. Apesar de o tema ser bastante interessante não aconselhamos a sua repetição.

Para finalizar esta edição da Semana dos Ramos, realizou-se uma palestra intitulada de "What's Next?" que pretendia falar do futuro dos estudantes após a conclusão do curso. O motivo de realizar uma palestra sobre o futuro universitário, deveu-se ao elevado interesse da parte dos estudantes sobre este tema num fórum de discussão no grupo do Facebook (\acrshort{mieec}/\acrshort{uc}). Relativamente à adesão a esta atividade foi média, mas de realçar que para a altura do ano letivo em que se realizou foi até elevada (estava prestes a chegar uma época crítica de frequências).  A mesa era composta por um forte grupo de oradores que conseguiu captar o interesse da plateia e criar uma conversa bastante dinâmica ao longo da sessão. Tivemos presentes o Prof. Dr. João Barreto, o Dr. Pedro Neto e o Diogo Justo, recém graduado no nosso curso. Com o papel de moderador tivemos a Dra. Ana Seguro do \acrfull{ipn}.

A palestra apenas teve a duração de 1 hora, uma vez que o Prof. João Barreto teve um compromisso, o que achamos que foi o fator mais negativo da mesma pois a conversa estava a ser mesmo muito interessante e a sessão tinha todas as considerações para continuar por mais uma hora. Esta atividade teve ainda alguns custos suportados pelo NEEEC, uma vez que o Diogo Justo teve de vir de propósito de Lisboa para participar na palestra. 

Nas próximas edições deve-se apostar ainda mais na divulgação, tornar alguns dos temas mais interessantes e dinamizar sempre as atividades, por mais que os temas possam parecer secantes. Pode-se também apostar em debates sobre coisas que se estejam a passar naquele momento no curso ou na área da Engenharia Eletrotécnica. Os workshops devem repetir-se pois são atividades que têm sempre muita adesão. De uma forma geral o modelo de 2 dias correu bastante bem sendo as atividades mais procuradas a reestruturação do curso, o debate entre os ramos e os workshops.

% ======================================
% # Workshop de Desenho e Impressão 3D #
% ======================================

\subsubsection{Workshop de Desenho e Impressão 3D}

Este workshop foi realizado em parceria com o \acrfull{cr}, onde o \acrshort{neeec} ficou responsável pela parte logística e o \acrshort{cr} pela parte de formação.

Como formador contámos com a presença do Paulo Almeida, Presidente do \acrshort{cr} no presente ano.

O local do workshop foi o próprio clube não sendo este o sítio mais indicado, pois não havia espaço suficiente para todos os alunos inscritos. Se este evento se voltar a concretizar deve ser realizado numa sala de aula, de forma a que as condições sejam melhores. 

No geral foi um evento que correu bastante bem e que contou com um número elevado de participantes.

% ===================================
% # Workshop de Unity               #
% ===================================

\subsubsection{Workshop de Unity}

O Unity é um motor de jogo 3D e IDE dedicado ao mundo dos jogos, sendo um tópico que costuma cativar bastante as pessoas. Este ano, o workshop foi organizado pelo \acrfull{cp}, que apenas pediu ajuda ao Núcleo para realizar o cartaz do evento, não tendo requisitado qualquer outro tipo de apoio. Desconhecemos o feedback geral da atividade, pelo que não podemos tirar ilações sobre o mesmo, mas, sendo um tópico tão cativante, consideramos que o mesmo deverá ter um envolvimento por parte do Núcleo muito superior ao que teve.

% ===================================
% # Aula de Inglês                  #
% ===================================

\subsubsection{Aula de Inglês}

A aula de inglês foi um género de workshop ministrado pelo Curso Privado de Inglês, derivado do protocolo estabelecido com esta escola. Este é um evento que, na nossa opinião, não se deve repetir pois a adesão foi quase nula e não houve interesse dos alunos por este evento. O feedback dos participantes não foi muito positivo: apenas tivemos feedback de 50\% dos participantes, sendo que 25\% destes achou o evento interessante e útil para o seu futuro, e os restantes 25\%, apesar de acharem interessante a iniciativa, não a acharam útil para o seu futuro. O professor diz ter gostado da iniciativa, apesar de se inscreverem pouquíssimos alunos. O objetivo do formador era também divulgar a sua escola o que acabou por não ter o impacto esperado. Para que a aula tivesse corrido melhor talvez se pudesse ter escolhido outra hora assim como proceder à divulgação do evento com mais antecipação dando deste modo mais visibilidade ao evento. O evento também foi organizado muito em cima da hora, pelo que a organização do mesmo decorreu em apenas 4 dias, embora esta fosse uma organização muito simples. Além destas razões há uma certa vergonha por parte dos alunos em participar num evento onde têm de se expor, falando em inglês.


% ===================================
% # Workshop de AutoCAD             #
% ===================================

\subsubsection{Workshop de AutoCAD}

O workshop de AutoCAD teve imensa adesão pois era uma ferramenta que se estava a utilizar no momento numa cadeira do curso em que não havia qualquer formação sobre a mesma.

Este workshop foi ministrado por um professor de Arquitetura, o professor Pedro Filipe, cujo contacto é pfmartins.c@gmail.com.

O feedback tanto do formador como dos participantes foi bastante positivo, sendo um workshop a repetir todos os anos, pois é uma ferramenta bastante útil para os alunos de quinto ano no 1º semestre, do ramo de Energia.

Os moldes do workshop foram os habituais, ou seja, com bastante componente prática ao invés de teoria com alguns exercícios para que os alunos pudessem aplicar ao máximo os conteúdos lecionados.

O workshop contou com 2 sessões sendo, por isto, muito mais fácil para o orientador coordenar o tempo e os alunos praticarem mais a ferramenta. 


% ===================================
% # Workshop de Machine Learning    #
% ===================================

\subsubsection{Workshop de Machine Learning}

Este workshop foi realizado em parceria com o \acrlong{ieeeuc}, onde o \acrshort{neeec} ficou responsável pela componente logística e o \acrshort{ieeeuc} pelo formador. Este tema é, sem dúvida, um workshop interessante e que teve muita adesão mas o formador não foi bem escolhido o que tornou o workshop bastante mau. Para voltar a repetir este workshop, deve-se apostar em várias sessões, e não apenas numa como foi o caso, pois este é um tema que tem muito conteúdo a ser explicado.

O formador do Workshop foi um Colaborador da empresa \acrfull{jest}, que cobrou publicidade e metade do valor das receitas.

% ===================================
% # Workshop de Simulink            #
% ===================================

\subsubsection{Workshop de Simulink}

O Simulink é uma ferramenta utilizada por todos os alunos em todos os ramos em ambos os semestres, os quais não têm qualquer formação nas cadeiras, pelo que deve ser um tema alvo de vários workshops a repetir e, de preferência, no primeiro semestre. O formador do Workshop foi o professor Marco Gomes, o qual teve um feedback muito bom dos alunos, e o próprio professor gostou de lecionar o workshop. Contudo, em edições futuras, caso seja dado um workshop de Simulink mais avançado, não será possível contar com o Professor Marco Gomes pois o próprio disse que não aceitava, uma vez que está apenas confortável com os passos básicos do Simulink.

% ===================================
% # Workshop de QT Creator          #
% ===================================

\subsubsection{Workshop de QT Creator}

Este workshop realiza-se há já muitos anos na semana da Queima das Fitas. Tal se deve ao facto de facilitar a aprendizagem dos alunos em algumas cadeiras, nomeadamente \acrshort{neeec} do primeiro ano e ES do quarto ano. Este ano, alterámos a data, realizando-se o mesmo em março para facilitar o trabalho dos alunos da cadeira de ES. Contudo, este teve pouca adesão pois a data e hora coincidia com uma aula. O workshop consistiu bastante na apresentação de slides com explicação do orador acerca dos conteúdos, mesmo nos momentos de desenvolvimento do jogo, o que é extremamente negativo. Neste tipo de workshop seria muito mais interessante o orador desenvolver o jogo na altura, ao mesmo tempo dos participantes, tendo já aquilo preparado.

O workshop foi lecionado pelo aluno de doutoramento, Luís Garrote, cujo contacto luissgarrote@gmail.com. Este workshop contou também com 2 sessões de 2h.

% ===================================
% # Workshop de Excel               #
% ===================================

\subsubsection{Workshop de Excel}

Este workshop foi pedido pelos alunos nos inquéritos pedagógicos mas, apesar disso, não houve muita adesão ao workshop. Não conseguimos identificar o real problema, mas pensamos que tal se deveu ao facto do Excel não ser uma ferramenta utilizada no curso.

O workshop foi lecionado pelo aluno Ivo Frazão, Tesoureiro do \acrshort{neeec} no presente ano.

% ===================================
% # Workshop de Android             #
% ===================================

\subsubsection{Workshop de Android}

Este workshop já tinha sido realizado no ano transato tendo, nesse ano, corrido bastante bem com bastante adesão. No presente ano o mesmo foi realizado em abril e apesar de haver mais de uma dezena de pessoas inscritas quando chegamos ao dia e hora do workshop apenas apareceram 3 pessoas. Tal deveu-se ao facto de muitas das inscrições não terem sido pagas pelo que não havia nenhuma condição para garantir que os participantes compareciam. O workshop realizou-se na mesma e a fonte do problema não conseguiu ser descoberta.

O workshop foi bastante prático e o facto do mesmo conter poucas pessoas, serviu para que estas conseguissem ter mais atenção por parte do orador, o que foi muito positivo.

O workshop foi ministrado pelo Ricardo Pereira, professor no \acrfull{isec} e aluno de Mestrado em Engenharia Informática, cujo contacto é ricardo.dc.pereira@gmail.com. O workshop contou com duas sessões de 2h, uma dedicada à formação inicial dos ministrantes e outra dedicada a um caso prático um pouco mais avançado.


\subsection{Disposições Finais}

Em suma, quanto à feira e à semana dos ramos, não há nada a apontar, sendo que quanto aos workshops, deve-se apostar mais na divulgação e não colocar apenas workshop de “...”, mas sim o tema real do workshop com uma breve explicação do conteúdo do mesmo e do formador. Deve também ter-se especial atenção aos horários e ao mapa de frequências para que o nosso publico alvo seja realmente atingido, e não haja nem frequências nem aulas a impedir estes de participar.

Um aspeto logístico a melhorar, e que tem de ser falado com o Departamento, é o controlo das luzes na antiga biblioteca. Se fosse possível desligar apenas metade era ótimo, porque as pessoas que ficam mais atrás têm dificuldade em ver com a luz ligada. A única hipótese é desligar todas as luzes, mas aí também fica muito escuro e pode não ser tão confortável para alguns.

As tomadas e as entradas de rede da antiga biblioteca não funcionam pelo que devem ser arranjadas o quanto antes.