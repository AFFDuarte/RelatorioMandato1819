% ===========================
% # Semana dos Ramos        #
% ===========================

\subsubsection{Semana dos Ramos}

A Semana dos Ramos é um evento já organizado desde 2012 que tem como objetivo principal dar a conhecer os vários ramos de mestrado do MiEEC/UC a todos os alunos do 3º ano. Ao longo dos anos, este evento tem tido vários formatos diferentes, adaptando-se a circunstâncias diferentes.

No presente ano, o tema que mais se falava em todo o \acrshort{deec} era a reestruturação do curso e os alunos estavam curiosos para saber o que iria acontecer após a mesma. Assim, decidiu-se que a semana dos ramos se focaria nesse mesmo tema.

Decidiu-se também realizar o evento em 2 dias, apesar do número de atividades da presente edição ser até maior que o número de atividades da edição anterior, que durou 4 dias. O primeiro dia foi dedicado ao percurso dentro do curso e o segundo dia dedicado ao futuro dos estudantes, após o curso.

O evento começou no dia 12 de março com o professor Humberto Jorge a expor o novo plano de estudos, a implementar com a reestruturação do curso, explicando o que iria acontecer em cada ano e respondendo a todas as questões dos alunos, numa sessão que foi bastante concorrida e que durou quase uma manhã inteira. 

Após a discussão sobre a reestruturação do curso houve um debate entre os 4 ramos com a presença de um professor e aluno de cada ramo. Os professores escolhidos e os respetivos alunos foram: Professor Humberto Jorge com o aluno André Duarte a representar o ramo de energia; Professor Paulo Peixoto com o aluno Ivo Frazão a representar o ramo de computadores; Professor Urbano Nunes com o aluno Luís Garrote a representar o ramo de automação e Professora Maria do Carmo Medeiros com o aluno Frederico Vaz a representar o ramo de telecomunicação. Cada professor e aluno teve ao seu dispor cerca de 5 minutos para expor o seu ramo, para que no final houvesse tempo para esclarecimento de dúvidas, mas este prazo não foi cumprido, havendo professores a falar mais de 30 minutos, pelo que nas próximas edições há uma enorme necessidade de arranjar uma forma de cumprir estes horários.

Na parte da tarde, começámos por ter um debate tese vs estágio \footnote{Este nome é errado e tal facto já foi alertado pelo Professor Lino Marques uma vez que o trabalho desenvolvido no final do mestrado é uma dissertação e não uma tese (esse é o trabalho desenvolvido no final do doutoramento). O nome estágio está também errado pois o que é feito numa empresa é também um projeto de investigação que deve ser devidamente documento.}, com a presença do professor Marco Gomes e o aluno André Silva para representar as dissertações no DEEC e a presença do professor Humberto Jorge para representar as dissertações em regime empresarial. Esta sessão decorreu em forma de debate, tendo sido moderado pelo João Martins. Nesta sessão houve bastante interação dos alunos tendo o mesmo estendido-se mais que o tempo estipulado, atrasando as restantes atividades.
No final do dia, houve ainda dois workshops, um dedicado à escrita de documentos em \LaTeX, com o exemplo dos trabalhos a desenvolver nas dissertações, e outro dedicado a como construir uma tese, ambos lecionados pelo professor Tony Almeida e com casa cheia. O workshop de \LaTeX correu muito bem, foi bastante interessante e os alunos ficaram com as bases necessárias para iniciar o trabalhar com a ferramenta. Quanto ao workshop de Construção de Tese, houve muitos alunos que disseram que o que tinha sido dito já sabiam e que esperavam uma coisa mais aprofundada e não tão superficial.

O segundo dia, que foi dedicado à Académica Start UC, começou com um workshop de empreendedorismo. Este evento apresentou uma componente menos teórica e expositiva de exemplos ilustrativos, como pedido aos oradores, de como se deve criar e gerir um negócio, nomeadamente as "burocracias" importantes a ter em consideração durante este processo. A adesão da parte dos participantes foi relativamente boa (cerca de 15 pessoas). O Workshop foi dividido em duas partes: a primeira, ministrada pelo Eng. Jorge Figueira, Chefe de Divisão da \acrfull{dits} e uma segunda parte pelo Dr. Miguel Gonçalves também pertencente à \acrfull{dits}. Em ambas as partes, os participantes acharam o evento bastante dinâmico e interessante.

De seguida, realizou-se um concurso de ideias de negócio. O concurso consistia em realizar um pitch com tempo limite de 3 minutos, sobre uma ideia de negócio que era avaliada por um júri, composto pelo Dr. Miguel Gonçalves e pela Dra. Deolinda Estevinho da \acrfull{dits} e pelo Dr. Jorge Pimenta do \acrfull{ipn}.

A adesão ao evento foi quase nula e não existiu interesse da parte dos alunos pelo evento, alguma parte devido a algum medo pela exposição em público. As inscrições grátis também não ajudaram na realização do concurso, pois várias pessoas inscreveram-se e acabaram por não aparecer o que, na nossa opinião, é uma falta de respeito. Aconselhamos que no futuro, as inscrições devam ter sempre um custo simbólico de forma a precaver esta situação. Apesar de o tema ser bastante interessante não aconselhamos a sua repetição.

Para finalizar esta edição da Semana dos Ramos, realizou-se uma palestra intitulada de "What's Next?" que pretendia falar do futuro dos estudantes após a conclusão do curso. O motivo de realizar uma palestra sobre o futuro universitário, deveu-se ao elevado interesse da parte dos estudantes sobre este tema num fórum de discussão no grupo do Facebook (\acrshort{mieec}/\acrshort{uc}). Relativamente à adesão a esta atividade foi média, mas de realçar que para a altura do ano letivo em que se realizou foi até elevada (estava prestes a chegar uma época crítica de frequências).  A mesa era composta por um forte grupo de oradores que conseguiu captar o interesse da plateia e criar uma conversa bastante dinâmica ao longo da sessão. Tivemos presentes o Prof. Dr. João Barreto, o Dr. Pedro Neto e o Diogo Justo, recém graduado no nosso curso. Com o papel de moderador tivemos a Dra. Ana Seguro do \acrfull{ipn}.

A palestra apenas teve a duração de 1 hora, uma vez que o Prof. João Barreto teve um compromisso, o que achamos que foi o fator mais negativo da mesma pois a conversa estava a ser mesmo muito interessante e a sessão tinha todas as considerações para continuar por mais uma hora. Esta atividade teve ainda alguns custos suportados pelo NEEEC, uma vez que o Diogo Justo teve de vir de propósito de Lisboa para participar na palestra. 

Nas próximas edições deve-se apostar ainda mais na divulgação, tornar alguns dos temas mais interessantes e dinamizar sempre as atividades, por mais que os temas possam parecer secantes. Pode-se também apostar em debates sobre coisas que se estejam a passar naquele momento no curso ou na área da Engenharia Eletrotécnica. Os workshops devem repetir-se pois são atividades que têm sempre muita adesão. De uma forma geral o modelo de 2 dias correu bastante bem sendo as atividades mais procuradas a reestruturação do curso, o debate entre os ramos e os workshops.