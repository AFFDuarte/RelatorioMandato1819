% ==========================
% # Sugestões              #
% ==========================

\section{Sugestões}

Deixamos agora algumas sugestões a serem feitas no futuro:
\begin{itemize}
\item No site do Núcleo, dinamizar a página inicial criando alguns acessos rápidos, popups associados às redes sociais do Núcleo, etc.;
\item Criar, no site do núcleo, uma tabela para que fiquem registadas algumas queixas feitas à pedagogia e o estado em que as mesmas se encontram (resolvido, em resolução, sem resolução e respetivo motivo). Assim os alunos teriam exemplos daquilo que a pedagogia resolve e entendiam que enviar email para o pelouro é a forma mais adequada de resolver os assuntos;
\item Na plataforma interna de gestão informática permitir a reserva da sala do \acrshort{neeec} para reuniões, inserindo automaticamente essa informação no calendário interno do núcleo, e também de outros espaços do \acrshort{deec}, enviando automaticamente um email para a secretaria identificando o dia, hora, motivo e responsável pela reserva da sala;
\item Criar uma página no site do núcleo que permita a requisição de materiais do \acrshort{neeec} através de um formulário, podendo este ter ligação direta à base de dados do inventário de materiais emprestáveis;
\item Na plataforma de gestão interna do núcleo, permitir a marcação de presenças na escala do núcleo, que só poderia ser feita no pc do núcleo identificado pelo seu endereço IP;
\item No orçamento do núcleo, garantir uma verba máxima para gastar em aquisição de materiais para papelaria, manutenção/bricolage e ferramentas ou eletrodomésticos. Desta forma, eram garantidas verbas para este tipo de compras todos os anos e, simultaneamente, impediam-se gastos superiores ao que é orçamentado no início do ano, mesmo em anos cujo saldo financeiro seja melhor do que o esperado;
\item Criar guidelines/checklists para todos os eventos semelhantes do núcleo tais como, por exemplo, workshops, visitas de escolas ao \acrshort{deec}, entre outras, para facilitar, ao máximo, o trabalho das equipas envolvidas nessas atividades, garantindo que não se esquecem de nada;
\item Incentivar o \acrshort{gri} a criar uma página, no MyDEEC, onde seja possível saber facilmente quem entrou na sala do núcleo e alterar os acessos à mesma para que seja possível dar e retirar acesso aos membros do núcleo sem ser preciso incomodar o \acrshort{gri};
\item Solicitar ao \acrshort{gri} a \acrshort{api} do mapa do DEEC, existente no MyDEEC, para este ser integrado no site do núcleo e, assim, providenciar aos visitantes do site, nomeadamente aos caloiros, um mapa interativo do departamento;
\item Tentar fazer pacotes para a \acrshort{f3e} que incentivem as empresas a participar nos workshops do resto do ano;
\item Criar um dia de visitas a empresas: por exemplo ir a Lisboa e visitar a Deloitte, a Microsoft, etc. Assim, com uma só viagem e respetivo custo era possível visitar vários locais e era mais fácil ter público nas empresas (uma vez que quem iria a uma visita, teria de ir às outras) algo que agrada bastante às empresas;
\item Realizar um workshop de capa e batina, próximo da Queima das Fitas, com a presença do Conselho de Veteranos (que já costuma fazer este tipo de sessões);
\item Realizar um workshop sobre dinâmicas de grupo, gestão de stress, técnicas de estudo, etc.
\item Definir, desde o início do mandato, quem é responsável por realizar atividades de intervenção cívica e ação social, como é o caso do Mês Solidário ou o caso das doações aos bombeiros, impedindo que este trabalho recaia sempre sobre a Direção;
\item Disponibilizar camisolas de curso no início do ano, logo na semana das matrículas ou então ceder a elaboração das mesmas aos Carros da Queima, uma vez que, atualmente, esta iniciativa não é muito lucrativa para a dimensão do \acrshort{neeec};
\item Criar cafés com cultura, trazendo personalidades interessantes do \acrshort{deec} ou externas a este (com interesse para a área de engenharia eletrotécnica) a sessões que se poderiam realizar nalgum local icónico da cidade ou num ambiente diferente, por exemplo, no bar do Sr. Vítor;
\item Criar um dia fixo por semana onde se sabe que vai sempre haver um workshop que poderá ser a continuação de outros ou um tema nova mas que é sempre marcado no mesmo dia e hora da semana, de forma a ser mais fácil orientar os horários do público interessado. Em alternativa, pode-se criar um dia mensal formativo - o Dia W (de workshops) - com a realização de vários workshops condensados nesse dia e realizado todos os meses, exceto os meses de maior atividade como, por exemplo, setembro. Este dia poderia ter vários graus de dificuldade (workshops para iniciantes, médios e avançados) e poderia ser dedicado a um tema diferente todos os meses (programação num mês, ferramentas do Office noutro mês, robótica noutro, quotidiano, etc.);
\item Após as reuniões com os núcleos de outros locais do país, apercebe-mo-nos que somos dos poucos cursos da nossa área onde não é dada aos alunos a possibilidade de adquirirem um kit de componentes eletrónicos. Vários são os núcleos que vendem kits mais completos (contendo, por exemplo, pontas de osciloscópio ou placas FPGA) pois as universidades não têm nenhum material ou então vendem kits mais simples para que os alunos, nomeadamente os caloiros, possam ir explorando em casa algumas coisas da área do curso. Desta forma, sugerimos a criação, no futuro, de um kit simples de componentes eletrónicos que possa ser vendido ao público por um custo básico. Esta iniciativa poderia dinamizar bastante os alunos e, caso fosse feita em conjunto com o Departamento, poderia ter custos reduzidos pois poderia ser inserida em encomendas maiores. Adicionalmente, poderiam ser também elaborados kits para outras situações nomeadamente para as componentes práticas das cadeiras de automação, onde atualmente as vendas são centradas no \acrlong{cr} sem que os alunos, no geral, tenham oportunidade, ou sequer conhecimento, da possibilidade de comprar os materiais;
\item Disponibilizar os membros do Núcleo para manter o laboratório multidisciplinar aberto durante períodos de maior trabalho. No mestrado, existem várias cadeiras que têm projetos que têm de ser feitos fora das aulas mas cuja utilização de materiais existentes no laboratório referido é essencial. Com o alargamento do horário do laboratório multidisciplinar, poderia não ser necessária a utilização do Clube de Robótica e do Laboratório de Sistemas Digitais, mantendo o material centralizado num só espaço. Poderia também ser útil, garantir um horário alargado de acesso ao Clube de Robótica, nestas alturas. Todos estes horários deveriam ser públicos e bem divulgados para que qualquer estudante saiba que teria acesso a estes espaços e para que os professores saibam também destas condições;
\item Providenciar sessões de pitch, estilo Pitch Bootcamp, nomeadamente na \acrshort{f3e}, ao invés das habituais sessões de recrutamento estáticas;
\item Utilizar a Mesa do Plenário para fazer reuniões abertas, tais como os fóruns pedagógicos, permitindo assim que o que é falado nestas reuniões fique sempre devidamente arquivado e possa ser utilizado para compilar informação e enviá-la às entidades competentes sem sobrecarregar o trabalho dos pelouros.
\item Uma vez que o objetivo principal de se cobrar inscrição em atividades como workshops é garantir que os participantes vão, de facto, aos eventos, poderia ser criado um género de cartão de fidelidade para os participantes que vão a quase todos os workshops, permitindo-lhes assim obter inscrições mais baratas.
\end{itemize}