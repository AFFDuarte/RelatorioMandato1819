% ========================
% # UGF                  #
% ========================

\section{UGF}

\subsection{Introdução}

 A \acrfull{ugf} é um evento que contempla torneios de vários videojogos, como CS:GO, LoL, Fifa e Hearthstone. Tem, por norma, duração de 3 dias e foi uma ideia que surgiu há dois mandatos atrás e, na sua primeira edição, foi organizado por dois núcleos, o \acrshort{neeec} e o \acrshort{nei}.

Os espaços usados foram os dois departamentos, o \acrshort{deec} e o \acrshort{dei}, distribuindo os torneios grandes (CS e LoL), pelas salas de cada edifício e com duas LAN parties, nas bibliotecas do piso 6 de cada Departamento. O formato manteve-se, em termos logísticos, na edição deste ano, apesar de se ter combinado que tal não seria feito.

Na 2ª edição entrou mais um Núcleo para a comissão organizadora, o \acrshort{neemaac}, tendo o evento crescido bastante em relação ao ano anterior, contando com cerca de 150 participantes no total dos 3 dias.

\subsection{Atividades}

Neste mandato, a equipa organizadora começou a ter reuniões bem cedo, em setembro de 2017, tendo em conta que o evento seria em abril. Distribuiu-se trabalho pelos 3 núcleos mais uma vez, ficando cada entidade com 1 Coordenador Geral do evento. Assim, tínhamos 3 Coordenadores Gerais ao todo, Carlos Abegão (\acrshort{neeec}), Tiago Caniço (\acrshort{neemaac}) e João Ferreiro (\acrshort{nei}).

No entanto, mesmo com tanta antecedência e após tantas reuniões, não se tomava nenhuma decisão em concreto e, as que eram efetivamente tomadas, eram descartadas nas seguintes reuniões, fosse por esquecimento ou por desleixo.

Em novembro de 2017, Carlos Abegão abandonou o seu cargo no \acrshort{neeec} e, por conseguinte, a comissão organizadora da UGF. Para colmatar a falha da equipa, José Pereira, Administrador do \acrshort{neeec}, entrou para o lugar do Carlos, como Coordenador Geral do evento e responsável pelas parcerias.

Nessa altura, Tiago Caniço moderava as reuniões de coordenação e embora ainda houvessem algumas reuniões periódicas não se saía do sítio, o evento não andava.

A situação das parcerias estava péssima, uma vez que após se mandarem dezenas de mails, só em dezembro se descobriu que o e-mail estava com um problema que não permitia receber emails, só enviar, daí a falta de respostas por parte de eventuais parceiros. Quando finalmente se resolveu essa falha, em meados de dezembro, o tempo já era escasso para contactar as empresas. Assim, foi feito um esforço da equipa das parcerias para tentar contactar todos os possíveis parceiros. No entanto, as respostas ou eram negativas ou não existiam, o que desmotivou bastante a equipa.

Entretanto Tiago Caniço conseguiu arranjar uma parceria com a TIS, na qual a empresa disponibilizava um simulador de corridas para o fim de semana do evento e José Pereira estava em negociações com uma empresa de jogos de tabuleiro, que estava praticamente confirmada até à última semana antes do evento, quando cancelou a sua vinda.

Em relação a patrocinadores de coffee-breaks, a situação também não estava a correr bem, sem respostas por parte dos supermercados. No total, contámos apenas com três patrocinadores, a TIS, a \acrshort{fctuc}, que nos ajudou monetariamente e a COQF, que nos confirmou a oferta de bilhetes para a Queima das Fitas 2018, a menos de uma semana do evento.

Não obstante, o número de inscritos subia, mas sempre com o receio de que estes pudessem não aparecer no evento, uma vez que Tiago Caniço estipulou que os participantes só pagavam a inscrição aquando do check-in, que era feita nos dias do evento. No entanto, ninguém, fora do \acrshort{neeec}, parecia realmente preocupado com tal facto.

A equipa só começou a ganhar forma duas semanas antes do evento, tendo sido feitas apenas duas reuniões gerais, para tentar por todas as pessoas a par do evento, o que correu mal, por ser muito em cima da hora. Até lá estavam na equipa algumas pessoas, muitas delas sem tarefas definidas ao certo, o que resultou num desinteresse pelo evento e consequente esquecimento do mesmo.

Enquanto tudo isto acontecia, a página do Facebook da UGF estava bastante boa. A imagem era excelente e cumpria sempre todos os prazos à risca e, quem via as constantes publicações tinha a sensação que a organização poderia estar a correr relativamente bem, o que foi um dos factos mais positivos desta organização.
A divulgação dos prémios tornou-se um problema, devido à tardia resposta da COQF, da qual dependíamos para lançar os mesmos. Assim, os prémios foram divulgados na própria semana do evento, o que é impensável, isto se quisermos ter inscritos. Este problema repercutiu-se no Facebook da UGF tendo havido inúmeras queixas do público sobre a forma de comentários públicos na plataforma.

Felizmente, o Tesoureiro da UGF, Gonçalo Santos (\acrshort{nei}), conseguiu, com o dinheiro das inscrições comprar alguma comida para coffee-breaks que, embora fosse insuficiente, foi complementada com mais alguns snacks que seriam para o Shift-Appens, evento grande do \acrshort{nei}, na semana a seguir. Evento esse, do qual João Ferreiro era responsável e, por isso não dedicou muito do seu tempo à UGF, o que é compreensível.

Tivemos vários problemas logísticos antes do evento começar, como é exemplo o facto das portas do \acrshort{dei} estarem todas fechadas, devido a medidas de segurança no Departamento. Teve de ser pedida uma autorização especial ao Diretor para que fosse possível deixar a porta aberta durante o fim de semana, com o comprometimento de que se acontecesse alguma coisa seria a UGF que padecia.

No resto, a logística consistiu em organizar as salas de cada Departamento, pondo duas filas de mesas, frente a frente, em cada sala. Usou-se a antiga biblioteca do \acrshort{deec} para a Zona de Lan e para o HearthStone, sendo que apenas se transmitiram 2 ou 3 jogos na stream, sempre sem qualquer público a assistir. Isto tudo só foi possível graças ao professor Humberto Jorge, que nos disponibilizou as chaves da porta de vidro, que dá acesso à biblioteca do piso 6, através da antiga biblioteca.

Quanto ao evento em si, foi feita uma escala completamente irrealista, na qual estavam pessoas que saíam do turno às 4 da manha, mas tinham de estar de novo no Departamento às 9h da manhã e 15 horas de turno seguidas. Para além disso, como não haviam tarefas definidas ao certo, também não estavam alocadas pessoas para elas na escala, ficando grande parte dos coffee-breaks por tratar, por falta de pessoas que se ocuparam com outras tarefas.

Em relação ao CS:GO, que é o torneio mais trabalhoso: para este torneio funcionar tem de se ter servidores locais dedicados para o efeito, o que pode complicar as coisas, visto ser preciso configura-los. Como tal, pediram-se servidores ao HelpDesk do \acrshort{dei}, de modo a poder-se configura-los com a devida antecedência. No entanto, é difícil falar com o HelpDesk, visto estarem sempre indisponíveis ou fora do \acrshort{dei} e portanto recebemos os servidores apenas a duas semanas do evento. Felizmente, conseguimos arranjar uma pessoa que percebia do assunto, Miguel Santos (\acrshort{neeec}), que, mesmo com pouquíssimo tempo, conseguiu tratar da configuração dos servidores, antes do torneio.

A meio do torneio os servidores do \acrshort{dei} falharam, mas, após algum tempo (horas) e após muitos participantes estarem já irritados, o Miguel arranjou uma alternativa, usando servidores dele. Chegou-se ao cúmulo de, na final do torneio, os próprios participantes terem de usar os seus próprios servidores para que a final pudesse decorrer. Tal deveu-se ao facto de todos os servidores falharem e já não haver suporte técnico por parte do Miguel que tinha ido embora, uma vez que a final deveria decorrer durante a tarde mas não decorreu pois as pessoas decidiram ver o jogo do Benfica que se realizou à mesma hora.

Tudo isto serve para concluir que não vale a pena fazer um torneio destes nestes moldes, a não ser que tenhamos a certeza que os servidores aguentam a carga.

Quando aos restantes torneios, Rocket League, Hearthstone, LoL, FIFA e DragonBall correram bastante bem, mas tiveram um problema. Problema esse que, olhando para a organização, acabou por ser a solução - inscritos online que não apareceram, resultando em muitos poucos inscritos por cada um dos torneios, individualmente. Ou seja, logisticamente foi mais fácil organizar estes jogos, mas fica a nota de que se houvessem mais inscritos seria um problema muito difícil de resolver. 

Um episódio, um pouco vergonhoso, foi ver a zona de LAN do \acrshort{deec} completamente vazia no domingo à tarde. Isto porque fechámos a sala de estudo, exclusivamente por causa do evento e chegaram lá várias pessoas para estudar nesse dia, o que era normal dado que ainda decorria a época de frequências, e foram impedidas de o fazer. Não podíamos sequer dizer que se estava a passar a UGF naquele espaço, porque, de facto não se estava a passar nada e, por isso, a determinada hora começámos a deixar as pessoas estudar.

Concluindo, muito superficialmente, não se justifica a intervenção de tantos núcleos neste evento.  Muita gente, muitas opiniões diferentes, ninguém que soubesse realmente delegar tarefas, muita confusão. É preferível fazer torneios mais pequenos, ao longo do ano, do que proporcionar esta má experiência aos participantes. Na nossa opinião este foi, sem dúvida, o pior evento do \acrshort{neeec} no presente mandato.

