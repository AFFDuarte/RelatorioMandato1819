% ===========================
% # Relatório de Contas     #
% ===========================

\ifthenelse{\boolean{biblia}}
{ % TRUE
    \part{Relatório de Contas Real}
}
{ % FALSE
    \part{Relatório de Contas}
}

\section{Introdução}

Após alguns anos atribulados a nível financeiro, resultante de uma dívida antiga que atingiu mandatos recentes, recebemos um Núcleo livre de quaisquer dívidas, pelo que, desde já, agradecemos o esforço do mandato anterior em saldar esta situação. Contudo, a situação financeira era ainda frágil, sendo o \acrshort{neeec} incapaz de suportar despesas avultadas necessárias para a realização de eventos de média/grande dimensão, apesar da responsabilidade em realizar o \acrfull{ene3} já existir e recair sobre o \acrshort{neeec} a realização deste evento, que precisava de ser revitalizado para garantir a sua continuidade num futuro próximo.

Após garantirmos, junto da Tesouraria da \acrshort{aac}, que não existiam mais dívidas pendentes sobre o nosso Núcleo, procedemos a uma gestão rigorosa que permitisse realizar o \acrshort{ene3} com a qualidade pretendida e sem comprometer novamente a situação financeira do Núcleo. Felizmente, a atividade pôde realizar-se com uma qualidade que nos orgulha e que, no fim, permitiu um desafogo financeiro enorme à tesouraria. Começámos assim o novo ano letivo com uma estabilidade renovada, que continuou a crescer com uma edição muito bem sucedida da \acrfull{f3e}, a nível financeiro.

A presença do \acrshort{neeec} na Festas das Latas e Imposição de Insígnias com a sua tradicional barraca, novamente numa gestão cuidada dessa presença, permitiu ao Núcleo angariar um novo valor significativo, assegurando a trajetória de estabilização financeira do Núcleo.

De olhos postos na quarta edição do Bot Olympics, deparámo-nos com uma situação precária do evento, com investimentos normais no evento adiados em edições anteriores, novamente, devido à situação precária do \acrshort{neeec}, e que tiveram que ser realizados nesta edição. Além disso, a necessidade de algum crescimento da dimensão do evento (destacando, neste ponto, a realização da final do evento no Alma Shopping) obrigou a investimentos não previstos pela organização, mas que, sem dúvida, podemos afirmar terem sido bem sucedidos. Contudo, a aposta muito forte nos patrocínios permitiu que o evento tivesse lucro, apesar do mesmo apenas ter sido atingido na reta final do mesmo.

A realização da já tradicional Gala Ohms D'Ouro, este ano associada à celebração dos 20 Anos do \acrshort{neeec}, procurou garantir a estabilidade procurada nos restantes eventos do mandato, suportada através de patrocínios publicitários.

Os restantes eventos do Núcleo procuraram sempre garantir a estabilidade financeira do mesmo, de forma nem sempre bem sucedida, contudo, olhando o panorama geral, consideramos ter concluído a nossa principal missão com sucesso: deixar o Núcleo capaz de assegurar todas as suas atividades sem ter que estar constantemente a contar o dinheiro para saber se será capaz de assegurar as despesas necessárias.

Na tabela \ref{tab:contas}, apresentamos um resumo das receitas e despesas executadas pelo \acrshort{neeec} durante o decorrer do presente mandato. Na tabela \ref{tab:pendentes}, apresentamos os valores que ficam pendentes deste mandato, nomeadamente, uma parte do valor do patrocínio da Altice Labs à \acrshort{f3e}, que devido ao método de pagamento (por acerto de contas com a \acrshort{aac}) atrasou constantemente os pagamentos desta entidade perante o \acrshort{neeec}, e os valores que resultam da distribuição dos lucros da Queima das Fitas pelas estruturas da \acrshort{aac} (o valor referente ao ano de 2017 é ainda desconhecido, pelo que não está contabilizado). Por fim, na tabela \ref{tab:saldos} apresentamos uma comparação geral das contas do \acrshort{neeec} do mandato passado e do presente mandato.

Em conclusão, esperamos que este novo panorama em que conseguimos deixar o \acrshort{neeec} se mantenha e que o objetivo dos próximos mandatos continue a ser a da contínua estabilização financeira, permitindo "atacar"\space projetos de cada vez maior envergadura financeira sem comprometer futuros mandatos.

\section{Relatório}

\ifthenelse{\boolean{biblia}}
{ % TRUE
\LTXtable{\textwidth}{contas/tabela-contas-biblia}

\LTXtable{\textwidth}{contas/tabela-pendentes-biblia}

\LTXtable{\textwidth}{contas/tabela-saldos-biblia}
}
{ % FALSE
\LTXtable{\textwidth}{contas/tabela-contas-aac}

\LTXtable{\textwidth}{contas/tabela-pendentes-aac}

\LTXtable{\textwidth}{contas/tabela-saldos-aac}
}