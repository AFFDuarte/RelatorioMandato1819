% ==========================
% # Estrutura Interna      #
% ==========================

\subsection{Estrutura Interna}

Tendo por base o facto do \acrshort{neeec} dispor de autorização para ter, oficialmente, apenas 7 vogais, ao elaborarmos o projeto para este mandato decidimos manter uma estrutura de pelouros muito semelhante à do mandato anterior, apenas com algumas alterações pontuais. Dessa forma foram feitas as seguintes alterações:
\begin{itemize}
\item Extinção do Pelouro da Logística\\
O Pelouro da Logística tratava de uma área que, na nossa opinião, se sobrepunha por completo ao trabalho dos pelouros e ao trabalho do Administrador do Núcleo. Em particular, notava-se que em atividades pequenas o Pelouro da Logística não era abordado pois era desnecessário chamá-los (por exemplo, para organizar mesas) e em atividades grandes o Pelouro não tinha capacidade para suportar toda a logística sem necessitar de mais apoio humano. Por sua vez, a gestão do material do \acrshort{neeec} é, na nossa opinião, competência do Administrador do Núcleo pelo que decidimos que este passaria a ter uma equipa a seu lado e que o trabalho logístico das atividades seria responsabilidade individual do Pelouro responsável por cada atividade.

\item Separação da Imagem e da Comunicação\\
O Pelouro da Comunicação e Imagem era um Pelouro que estava constantemente sobrecarregado com a criação de todas as imagens necessárias para as várias atividades do Núcleo não tendo vazão para garantir a verdadeira comunicação dessas atividades. Por sua vez, o facto do Coordenador Geral deste Pelouro ser, por regra, a pessoa mais especializada no trabalho artístico da imagem fazia, na nossa opinião, com que este não se preocupasse de todo com o trabalho de comunicação. Assim, decidimos retirar a comunicação de forma oficial do nome deste Pelouro o que, no nosso entender, mais não foi do que a constatação do que era a realidade deste Pelouro.

\item Criação do Pelouro das Relações Externas e Comunicação\\
O \acrshort{deec} sofre há já vários anos com vários problemas no que toca à divulgação do curso. Por sua vez, existem várias instituições estudantis, com diferentes dinâmicas de trabalho que tentam intervir nesta área de uma forma descoordenada entre si e entre a Direção do Departamento. Por sua vez, todo o trabalho de divulgação do curso, que, na nossa opinião, era muito aquém daquilo que o Departamento necessitava, acabava por ser feito pela Presidente do Núcleo o que, para além de não dever ser uma competência sua, não permitia ter alguém dedicado a formular um plano e uma orientação para dialogar com as outras instituições e traçar uma linha orientadora entre todos para aumentar a divulgação do curso. Assim, surgiu a ideia de criar um Pelouro exclusivo para este tópico que deveria ter o trabalho de planeamento de divulgação do curso e do \acrshort{neeec}, no primeiro semestre, e a execução do mesmo, no segundo semestre. Por sua vez, decidiu-se também que a comunicação do Núcleo deveria ser feita por este Pelouro mas tal não resultou, o que abordaremos mais à frente no tópico da comunicação.

\item Saídas Profissionais e Formação\\
As Saídas Profissionais eram já responsáveis pela maior parte dos workshops realizados, pelo que decidimos que todos os workshops, independentemente do tema, seria organizados por este Pelouro passando assim o nome Formação a constar da sua nomenclatura.

\item Cultura e Lazer\\
Uma vez que o Pelouro da Cultura tomava conta de muitas atividades de âmbito lúdico e não apenas atividades de âmbito cultural, decidiu-se acrescentar o nome Lazer à sua nomenclatura.

\item Pedagogia e \acrshort{gape}\\
A Pedagogia teve a nomenclatura de Pedagogia e Ação Social no mandato anterior, contudo, achamos que havia uma lacuna no Núcleo sobre quem deveria auxiliar o estudante sempre que havia dúvidas ou questões no decorrer do seu percurso académico. Assim decidimos que a Pedagogia passaria a ter um \acrfull{gape} a si associado, pelo que, o Pelouro deveria ser responsável por todas as questões pedagógicas bem como pela busca e divulgação de informações sobre ação social, apoio na criação dos horários, busca autónoma e disponibilização de informação para o site, criação de uma drive e de um banco físico de materiais de estudo, orientação para a escolha de cadeiras opcionais, auxilio dos estudantes sobre a reestruturação do curso, etc.
\end{itemize}

Na nossa opinião, a atribuição dos nomes aos pelouros é importante para se perceber a dinâmica do Núcleo e a sua forma de funcionamento como foi o caso das Saídas Profissionais e Formação e na Imagem. Contudo, mudanças de nome em pelouros muito enraizados (como o caso das próprias Saídas Profissionais e Formação que continuaram a ser conhecidas apenas por "Saídas") impede que os membros do Núcleo, fora dos pelouros em questão, associem corretamente as tarefas de competência desse Pelouro ao mesmo. Quanto ao \acrshort{gape}, achamos também que o Pelouro não soube dividir bem as tarefas entre os problemas pedagógicos e o \acrshort{gape}, pelo que poderia ter apostado muito mais neste assunto como, por exemplo, aconteceu com o banco de materiais de estudo e o conteúdo do site, que tiveram de ser muito incentivados pela Direção quando deveria ter sido o Pelouro a idealizar, de forma autónoma, o assunto. Dada a tradição existente em não criar mais do que 6 pelouros pois apenas temos direito a sete vogais (um deles é o Administrador), não vemos este assunto a ser tratado por outro Pelouro que não o da Pedagogia pelo elevado envolvimento que este assunto tem com os problemas com que tratam de âmbito pedagógico. No entanto, caso seja ponderada a criação de mais pelouros, sugerimos que uma maior distinção destes assuntos possa ser benéfica.