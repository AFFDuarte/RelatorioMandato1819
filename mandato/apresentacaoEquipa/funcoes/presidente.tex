% ========================
% # Presidente           #
% ========================

\subsubsection{Presidente}

O papel de Presidente fundamentou-se nas tomadas de decisão de todos os assuntos referentes ao \acrshort{neeec}, em conjunto com os restantes membros da Direção, bem como pela representação institucional do \acrshort{neeec}.

O Presidente assumiu várias funções para além das descritas no regulamento interno. Assim, foi o principal responsável por convocar, preparar e presidir as reuniões de Direção, as reuniões de Coordenadores Gerais e também algumas reuniões de pelouros, quando foi necessário. Para além disso, foi o responsável, a par do Vice-Presidente, de representar o \acrshort{neeec} nas reuniões de Núcleos Polo 2, nas \acrshort{an} e nas reuniões de Núcleos da área de Engenharia Eletrotécnica nacionais.

Adicionalmente, o Presidente foi um dos Coordenadores Gerais do \acrshort{ene3} e da Gala Ohms D'Ouro.

No que toca à área dos Núcleos nacionais, uma vez que a iniciativa de os reunir partiu do \acrshort{neeec}, visto ter sido este um dos organizadores da última edição do \acrshort{ene3}, o Presidente foi o responsável por entrar em contacto com estes, organizar a sua interação e dinamizá-la.

O Presidente foi também o principal elo de ligação entre o \acrshort{neeec} e a Direção do \acrshort{deec}, principalmente através do seu Diretor. Assim, no geral, o Presidente foi o responsável por reunir com o Diretor sempre que fosse necessário para abordar qualquer tópico no qual fosse necessário um trabalho conjunto entre as duas entidades, situação que ocorreu várias vezes.

Devido à orgânica interna, o Presidente e o Vice-Presidente assumiram também a responsabilidade de toda a comunicação do Núcleo, bem como dos protocolos a celebrar com outras entidades por parte do \acrshort{neeec}.
Por fim, o Presidente assumiu ainda um papel bastante ativo no auxílio ao Administrador do Núcleo, nomeadamente na gestão de espaços e na realização de compras e acordos.