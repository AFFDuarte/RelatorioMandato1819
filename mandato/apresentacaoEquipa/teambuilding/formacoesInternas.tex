% ========================
% # Formações Internas   #
% ========================

\subsubsection{Formações Internas}

A equipa do Núcleo é acima de tudo constituída por estudantes que se propõe serem iniciantes numa estrutura deste tipo. Por sua vez, os membros que já estão há vários anos no Núcleo são membros que, em regra, se destacaram pela positiva e, desta forma, ocupam cargos mais altos e têm mais conhecimento sobre como fazer as coisas, acabando por tendencialmente achar que tudo é básico e trivial ou não se lembrando que é normal as pessoas não saberem certos pormenores que estes já consideram ser demasiado banais e óbvios. Desta forma, a formação dos membros do Núcleo é essencial.

Antes de mais, realçamos a necessidade de deixar explícito, desde o início, as normas e funcionamentos do Núcleo, de preferência de forma escrita e o mais condensada possível para que não haja dúvidas (por exemplo, num regimento interno). Em seguida, destacamos a necessidade de formar os novos membros (e também os antigos) para coisas como:
\begin{itemize}
\item como fazer uma chamada telefónica para um parceiro do Núcleo?
\item como proceder quando se recebe uma chamada no telefone do Núcleo?
\item como estabelecer um contacto via email junto de um parceiro?
\item como organizar a caixa de entrada de email do meu Pelouro?
\item como organizar uma carta de apresentação?
\item como utilizar os formulários de pedido de imagem, relatórios de atividades e reunião de pelouros?
\item como gerir a minha equipa de Pelouro?
\item utilização de software (word, excel, illustrator, photoshop, slack, trello, etc)
\item como utilizar a OneDrive e organizar a drive do meu Pelouro?
\item fund raising
\item comunicação
\item organização dos órgãos sociais da \acrshort{aac}, do \acrshort{deec}, da \acrshort{fctuc} e da \acrshort{uc}
\item tesouraria, contabilidade e administração
\end{itemize}

Algo de salientar é que, habitualmente, há vários membros que pensam que não necessitam de ter acesso a estas formações pois nunca irão necessitar das competências que nelas podem adquirir, uma vez que elas não se aplicam aos seus pelouros. Contudo, muitos dos membros acabam por ingressar em comissões organizadoras ou até trocar de Pelouro, não sabendo depois os conceitos básicos que podem obter em formações como estas, que devem ocorrer no início do mandato.

Este ano tentámos fazer um workshop de Fund Raising logo no mês de julho, principalmente para o Pelouro das Saídas Profissionais que já estavam a preparar a \acrshort{f3e}, mas não conseguimos, devido a restrições temporais.

As restantes formações estavam planeadas para o teambuilding de setembro, mas acabaram por não se realizar, uma vez que nem todas foram preparadas e as poucas que foram não tiveram atenção necessária no teambuilding (os próprios membros da Direção não incentivaram a atenção necessária para este tipo de situações).

Na nossa opinião, estas formações são essenciais e permitem uma partilha de conhecimento geral entre os membros do \acrshort{neeec}. O que sugerimos, é que seja criado um dia, ou um fim de semana, de formações, em Coimbra, onde o Núcleo até poderá sustentar os custos da alimentação fazendo assim com que haja uma adesão maior dos seus membros. Por sua vez, a componente recreativa/lúdica poderia ficar reservada para a noite ou para uma atividade desportiva numa tarde.