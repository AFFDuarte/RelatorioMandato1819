\subsubsection{Setembro}

O teambuilding tem como objetivo principal a interação entre todos os membros do Núcleo numa fase inicial do ano. Este momento é ideal para formar uma equipa coesa e, ao mesmo tempo, prepará-la para os desafios que se aproximam com o começar do novo ano letivo. Desta forma, consideramos que deve existir uma forte componente lúdica e uma componente formativa adequada, para esses desafios, tendo em atenção que apenas uns dias nunca servirá para dar todo o tipo de informações a toda a gente. Este ano, devido ao curto espaço de tempo entre o \acrshort{ene3} e a Receção ao Caloiro, a Direção, responsável pela organização deste teambuilding, não teve tempo de o organizar como devia ser, muito embora esta atividade não requeresse muito tempo para elaborar (ficou em falta, principalmente, o planeamento das atividades de teambuilding e formação).

O local escolhido foi o mesmo local do teambuilding do mandato anterior: a Aldeia do Casal Novo, uma aldeia de xisto na serra da Lousã, numa casa de uma senhora chamada Conceição Carvalho, que é bastante acessível. O local tem um custo reduzido, sendo que o preço por noite é negociável, tendo mesmo existido um erro no pagamento este ano que a senhora não se importou pois tem uma boa impressão nossa, querendo manter as nossas idas ao seu espaço. Uma informação importante: a senhora informa que apenas poderão levar 10 pessoas, dado que supostamente não cabem mais, mas o espaço permite mais pessoas, desde que se levem sacos de cama e outros acessórios adequados. Dada a boa impressão que a senhora tem de nós, recomendamos que continuem a deixar o espaço nas melhores condições possíveis, informando de qualquer percalço que exista assim que possível. O local é bastante afastado da "civilização", existindo mesmo pouca rede móvel (apenas alguns sítios com rede 3G/4G), o que pode ser benéfico, pois impede que as pessoas se mantenham conectadas à Internet constantemente em vez de fazerem o teambuilding, contudo, caso exista algum trabalho do Núcleo pendente nessa altura, como já aconteceu com a \acrshort{f3e}, este pode ser um problema. O espaço, apesar de acolhedor e bastante interessante para um teambuilding, não apresenta, de todo, as condições necessárias para fazer a componente formativa, dado que não tem espaços adequados para as pessoas todas se sentarem, evitando que as formações possam ser mais interativas.

Novamente devido à organização do \acrshort{ene3} este ano, tivemos que adiar um pouco o teambuilding em relação ao ano anterior, tendo ocorrido este ano já depois do início das aulas, no fim-de-semana a seguir à Semana das Matrículas, tendo a partida de Coimbra sido na sexta ao fim do dia e o regresso no domingo seguinte ao fim do dia também.

O teambuilding foi dividido em momentos de formação e momentos lúdicos. As formações foram apresentadas pelo João Bento, nas quais ele mostrou alguns slides do \acrshort{aac} (in)Forma bem como alguns slides feitos por ele com o tema “Como organizar um evento”, de forma a que as pessoas tivessem uma pequena noção de alguns pontos fulcrais aquando da organização de eventos. A parte lúdica foi à base de passeios pela serra/praia fluvial e alguns momentos de confraternização. Esta parte é minimamente fácil de organizar e por isso é pouco propícia a falhas, embora seja importante garantir que o grupo se encontra sempre unido e não existem separações, algo que sucedeu este ano após se verificar que a praia fluvial escolhida para a atividade se encontrava sem água. As refeições servem também de ice breaking, assim como pequenos jogos e todo o convívio entre os demais.

No que toca à alimentação, as compras ficaram a cargo do Núcleo, apesar do custo das mesmas ter sido dividido entre todos. As compras realizadas seguiram a seguinte ideia de alimentação:
\begin{itemize}
\item Jantar de sexta: Pão e Febras
\item Almoço de sábado: Esparguete e latas de atum [Makro]
\item Jantar de sábado: Salsichas, Pão de cachorro e Molhos [Makro]
\item Almoço de domingo: Esparguete com atum/febras (sobras)
\end{itemize}

Quanto a bebidas alcoólicas foi feito um formulário para ver o que era pretendido por quem, que depois seria comprado pelo Núcleo. Este foi um ponto positivo na organização, pois permitiu comprar o estritamente necessário para o teambuilding, evitando desperdícios (apesar de terem existido na mesma, dado que algumas pessoas levaram mais algumas coisas próprias). Esse formulário serviu ainda para fazer a divisão das boleias até à Lousã, em que quem levasse carro e quem precisava de boleia preenchia a escala, respetivamente. Após as respostas, a Direção organizou as pessoas pelos carros disponíveis, evitando vários problemas no próprio dia e a utilização de carros em excesso, fulcral dado que o local não tem muito estacionamento.

Como disposições finais, consideramos que é necessário explicar melhor às pessoas a necessidade da realização deste teambuilding, incentivando-as a irem. O principal entrave à ida das pessoas tem sido a altura do ano, muito propícia à realização de férias familiares, e o custo associado à ida ao evento, que é bastante reduzido tendo em conta as atividades todas existentes, mas que nem todos poderão suportar, podendo ser interessante a comparticipação do Núcleo de algumas despesas, caso seja possível. Além disso, é preciso criar o maior número de atividades lúdicas possível para que se force as pessoas a separarem-se dos seus grupos de amigos, um problema que ocorreu ainda em demasia este ano.