% ==========================
% # Arrumos                #
% ==========================

\subsection{Arrumos}

Um dos maiores problemas da sala do Núcleo era a falta de espaço para arrumar materiais que não eram tão utilizados no dia-a-dia, como, por exemplo, as tendas e grelhadores. Sempre tivemos o anexo, do outro lado dos jardins do Núcleo, onde estavam dois armários, os carrinhos de compras e os grelhadores. No entanto, esse anexo tinha três problemas:
\begin{enumerate}
\item O acumular de coisas levou a que as portas de emergência ficassem frequentemente impedidas.
\item O espaço é muito húmido e, por mais que se varra, acumula demasiado lixo.
\item Os armários existentes não apresentavam qualquer segurança para se poder guardar neles coisas de valor.
\end{enumerate}

Tendo em conta esta situação falámos com o Professor Humberto, que compreendeu e concordou com os nossos argumentos, disponibilizando-nos um arrumo na zona técnica do piso 3A da torre S (a zona do piso 4 já é utilizada pela reprografia do Departamento). Deste modo conseguimos alocar muitos dos materiais para lá, especialmente stock de consumíveis (águas, napolitanas, café) e as tendas, uma vez que o arrumo é perto do Núcleo.

O problema do anexo, no entanto, não ficou totalmente resolvido, porque tínhamos demasiados coisas para tão pouco espaço (no 3A), o que levou a que os grelhadores e todo o material de febradas, bem como material de desporto continuasse a ser guardado no anexo.

Sempre tivemos a opção de usar o arrumo do piso 1 da torre do bar, mas, como este é bastante longe da sala do Núcleo, foi uma opção que foi sendo sempre relegada para segundo plano. Para além disso, este espaço estava bastante desarrumado, com material partido/inutilizável do Departamento, impedindo um uso adequado do espaço. Contudo, ao percebermos que a grande maioria das febradas, tanto do Núcleo, como dos Carros da Queima das Fitas, se realizam nas escadas do \acrshort{deec}, ou seja, exatamente acima desse mesmo arrumo, foi alocada uma tarde inteira para arrumar esse espaço todo, empilhando mesas e cadeiras quase até ao teto. Contudo, o Diretor do Departamento ressalvou que a organização realizada impedia que, caso fosse necessário retirar alguma coisa do espaço (o que apesar de raro, acontece), seria necessário desarrumar novamente todo o espaço, para além da arrumação de alguns materiais (como os grelhadores) causar bastante sujidade no espaço.

Com esta nova arrumação do B1, pudemos finalmente limpar o anexo, servindo atualmente para guardar apenas os carrinhos de compras e algumas das cadeiras de plástico da esplanada, pelo que os problemas supramencionados deixaram de ser preocupantes.