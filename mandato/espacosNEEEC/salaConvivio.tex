% ==========================
% # Sala de Convívio       #
% ==========================

\subsection{Sala de Convívio e Esplanada}

No início do mandato existia um problema nos jardins do \acrshort{neeec} que se prendia com o facto de haver arbustos junto às janelas que impossibilitavam a passagem de pessoas para a rua. Após falarmos com a Direção do Departamento os arbustos foram todas retirados por elementos da Direção do \acrshort{deec} e foi plantada relva. Foram também refeitos cinzeiros junto de todas as entradas da esplanada e criados bancos de jardim em cima da relva. Por sua vez, a Direção do \acrshort{neeec} informou o Núcleo de que as cadeiras e mesas de esplanada localizadas nos arrumos do Departamento pertenciam ao Núcleo pelo que o \acrshort{neeec} utilizou, de imediato, esse material para construir uma esplanada junto da sala de convívio. Neste espaço exterior pretendeu-se também colocar iluminação exterior, obra essa que se concluiu no final do mandato e que está preparada para, caso necessário, serem instalados mais pontos de luz, a partir do primeiro. Fica também em falta a aquisição de alguns chapéus de sol, algo que pode ser feito através de contactos com entidades patrocinadoras (contactámos a Super Bock, marca de cerveja oficial da \acrshort{aac} da qual fazemos parte, sem sucesso, uma vez que os contratos para material de jardim são feitos anualmente e o de 2018 já se encontrava fechado à data de contacto).

No interior da sala de convívio, começou-se por trazer um armário de prateleiras da sala de estudo, por indicação da Direção do \acrshort{deec}. Esta indicação deveu-se ao facto da parede estar altamente deteriorada devido a encostos na mesma e ao facto de ser construída em gesso sem qualquer proteção de madeira, como acontece nas restantes paredes. Devido ao \acrshort{ene3}, foi afixada um PVC de grande dimensão na parede que contém a imagem do Páteo das Escolas da \acrlong{uc}. Também a exposição dos cartazes da Queima das Fitas foi relocalizada e foram adquiridos os quadros em falta, uma vez que a exposição estava por completar desde há já três anos. Foi também criado um novo espaço de alimentação com maior capacidade que serve também para estudo ou para se poder fazer jogos tradicionais. Por sua vez, com a oferta da televisão por parte da Direção do \acrshort{deec}, foi criado um novo espaço com os sofás que se encontravam na sala de estudo do piso 6 para que fosse possível jogar PlayStation ou ligar um computador para ver séries, jogar algum jogo ou até assistir a jogos de futebol em direto. Foi também reabilitada a zona de jogos de dardos tendo sido adquirida uma nova máquina, dado que a antiga apresentava já vários problemas. Os jogos de dardos e de ping-pong tiveram a implementação de uma condicionante: os dardos, as bolas e as raquetes tinham de ser requisitadas na sala do Núcleo a troco de um cartão identificativo como caução. Desta forma e a juntar ao facto, muito importante, da administração do Núcleo apenas disponibilizar cerca de três bolas na gaveta dos jogos, fez-se uma poupança significativa no número de bolas a comprar tendo este sido quase 10 vezes inferior ao do mandato anterior. Esta medida embora tenha causado alguma estranheza no início, rapidamente se implementou e passou a ser algo básico para todos os utilizadores do serviço. Também os matrecos este ano sofreram algumas reparações tendo sido também avaliada a hipótese de se comprar uma nova mesa de matrecos, algo que não chegámos a realizar, mas que devíamos ter feito, dado que a mesa atual já se encontra em elevado estado de deterioração, devido a vários anos de falta de manutenção adequada. Devido à comemoração dos 20 anos do \acrshort{neeec} foi também instalada uma lona comemorativa sobre a data na parede da reprografia, permitindo dar o efeito de sala composta.

Fica por arranjar, à semelhança da sala do Núcleo, o chão e a pintura da parede.

É de notar que toda a calha elétrica localizada na parede lateral foi também arranjada este ano, após ter sido apresentada à Direção do Departamento para que estes vissem o elevado estado de degradação que a mesma apresentava. À semelhança da sala do Núcleo, esta calha existia devido a umas obras que se iniciaram aquando da construção da sala de convívio, há 14 anos, e que nunca tiveram fim. Desta forma, decidiu-se fechar a calha e colocar tomadas elétricas na mesma.