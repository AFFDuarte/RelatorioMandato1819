% ========================
% # Introdução           #
% ========================

\section{Introdução}

%Ao longo da história do \acrshort{neeec} muitos têm sido os avanços e também os recuos que marcam aquilo que esta instituição é nos dias de hoje. Contudo, ao longo do nosso mandato, temos verificado que a maior parte da informação é perdida com os anos fazendo com que o trabalho do \acrshort{neeec} seja muitas vezes pautado pela repetição de erros desnecessários que impedem o Núcleo de avançar mais rapidamente para patamares mais altos.

%No mandato de 2013/2014, liderado pelo João Carocha, foi criada pela primeira vez um documento, intitulado de Bíblia do \acrshort{neeec}, onde os 11 membros da Direção do respetivo mandato falaram um pouco do que era fazer parte desta instituição, nomeadamente nas equipas onde cada um se inseria. Este documento teve a sua continuidade no mandato seguinte, liderado pela Elisabete Santos, tendo, nesse mandato, sido feito um acrescento ao documento anterior. Por sua vez, nos mandatos seguintes, 2015/2016 e 2016/2017, este documento nunca mais foi atualizado. Apesar de simples, a Bíblia do \acrshort{neeec} foi um documento que muito ajudou o trabalho da nossa Direção, após ter sido ligo logo no início do mandato, tendo sido o primeiro passo para querermos investigar a história do \acrshort{neeec} por se perceber que a informação proveniente de mandatos anteriores, não restrita ao que nos era passado pelo mandato anterior, era essencial para que pudéssemos fazer um trabalho cada vez maior e melhor.

%Assim, a juntar ao facto de ser estatutariamente obrigatória a elaboração de um Relatório de Atividades e Contas no final de cada mandato, decidimos elaborar este Relatório de Mandato onde constam os nossos métodos laborais, o relacionamento que temos com as várias entidades com que trabalhos, um relato das várias atividades e iniciativas desenvolvidas ao longo do mandato tentando, neste campo, abordar sempre o que foi feito de forma positiva, para que possa ser repetido, e o que foi feito de forma negativa, para poder ser corrigido.

%Posto isto, é de notar a importância de todos os elementos do \acrshort{neeec}, desde colaboradores a membros de comissões organizadores, suplentes, coordenadores e membros da Direção, lerem o presente documento. Procurámos também dividir o mesmo por áreas, devidamente identificas no índice, contudo, várias poderão ser as informações sobre um mesmo assunto que poderão estar espalhadas por diversos locais, pelo que a leitura do documento na íntegra será a melhor forma de recolher toda a informação. Este documento pode também ser apresentado à Direção do Departamento, à Faculdade, a empresas e a demais entidades com que o \acrshort{neeec} se relaciona, como relato quer do trabalho desenvolvido pelo \acrshort{neeec} anualmente quer da importância destas entidades para esse mesmo trabalho. 

%Embora esta tenha sido a primeira vez que um documento desta dimensão foi feito, o que provocou algum trabalho mais pesado, recomendemos vivamente a sua repetição no futuro. Como tal, elaborámos o documento num modelo LaTeX que se encontra devidamente dividido por áreas pelo que, no futuro, bastará aceder ao mesmo, substituir o texto em cada um dos tópicos existentes e adicionar ou remover os tópicos de acordo com o que for necessário abordar no mandato em causa. Alertamos também para a importância de manter sempre este tipo de relatórios escrito de forma mais descritiva possível pois no início de cada mandato ninguém terá tempo, nem paciência, para ler vários relatórios desta dimensão (de todos os mandatos anteriores) prevendo-se que cada um destes ficheiros, uns anos após ter sido redigido, se acabe por tornar como um documento histórico do Núcleo.

%Por fim, de realçar que a elaboração deste documento, embora tenha sido pensado a partir do mês de abril, apenas começou a ser desenvolvido já no mês de maio, após a Queima das Fitas, o que levou a que a sua versão final apenas tenha ficado pronta já após o mandato, o que não é, de todo positivo. Para a elaboração do presente documento foi essencial a colaboração da maior parte dos Coordenadores, de todos os membros da Direção e do Presidente da Mesa do Plenário. De notar que a elaboração de textos por diversas pessoas, com diferentes formas de abordar os assuntos e de escrevê-los, implica uma leitura rigorosa do documento para a necessária uniformização do mesmo, sendo necessárias algumas semanas para tal. Assim, no futuro, recomendamos que seja mantido o formulário de relatório de cada uma das atividades mas sugerimos que o Secretário do \acrshort{neeec}, membro que estatutariamente é o responsável por elaborar o presente documento, faça uma revisão ao longo de todo o mandato, idealmente semanal, dos textos inseridos e os comece a inserir no relatório, de imediato, para que no final do mandato seja apenas necessário redigir algumas considerações finais e fazer algumas retificações finais.