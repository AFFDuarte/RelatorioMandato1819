% ===========================
% # Comissões Externas      #
% ===========================

\subsection{Comissões Organizadoras}

Para além das atividades desenvolvidas pelos pelouros ou pela Direção existem atividades de maior dimensão que necessitam de uma comissão organizadora. No presente mandato foram criadas comissões para a realização do \acrshort{ene3}, da \acrshort{ugf}, do Bot Olympics, da Gala Ohms D’Ouro e das celebrações dos 20 anos do \acrshort{neeec}. O mês solidário e respetiva componente solidária foram também organizadas por uma pequena comissão externa, de reduzida dimensão orientada pelo João Bento e pela Ana Calhau.

No mandato anterior (2016/2017), aquando do teambuilding, cada membro do Núcleo pôde dizer em que evento gostaria de participar (Bot Olympics ou Gala Ohms D'Ouro). Essa situação foi referida para quem estava na sala de estar da casa onde decorreu o teambuilding e voltou a ser mencionada dias mais tarde na sala do Núcleo. Desta forma as pessoas só puderam ocupar uma das comissões e houve comissões que tiveram várias pessoas desnecessárias e falta de pessoas com determinadas competências. Facilmente, quer numa comissão, quer noutra, as pessoas abandonaram as mesmas. No caso da gala, aquando da reta final da organização da mesma, já só a Presidente da altura estava a organizar a gala em conjunto com outra pessoa.

Este ano, optámos por criar as comissões em reunião de Direção, convidando as pessoas a fazer parte das mesmas. Achamos que esta organização deu resultados muito positivos, uma vez que permitiu uma melhor gestão de toda a equipa, divisão dos membros pelas várias comissões e atribuição de membros às competências que melhor desempenham. Podemos, no entanto, ter deixado de fora alguns membros que potencialmente poderiam ter ajudado bastante mas dos quais não tínhamos feedback suficiente para considerarmos a sua inserção nas comissões evitando, contudo, que vários membros que pretendiam juntar-se mas que não iriam fazer nada ocupassem lugares. Na totalidade dos casos, consideramos que esta medida melhorou em muito a qualidade do trabalho desenvolvido comparado com os métodos de seleção aplicados em mandatos anteriores.

\subsubsection{UGF}

Esta comissão foi presidida, ao início, pelo CG da Cultura e Lazer, Carlos Abegão, que ficou como Coordenador do evento. Após entrarmos em contacto com os outros Núcleos envolvidos (\acrshort{nei} e \acrshort{neemaac}), ainda em julho, convidámos várias pessoas a fazer parte desta comissão. O Coordenador não gostou que tivéssemos convidado pessoas sem ser ele a decidir, o que apesar de compreensível, teve como intuito uma gestão equilibrada de toda a equipa do \acrshort{neeec} para se saber como dividir a equipa para os vários eventos que iríamos ter. Contudo, ao convidarmos membros numa altura em que ainda quase ninguém tinha tido a oportunidade de mostrar trabalho para o Núcleo fez com que convidássemos algumas pessoas que não tinham as qualidades necessárias para o evento. Além disso, a saída do CG da Cultura e Lazer fez com que o Administrador do Núcleo passasse a ser Coordenador deste evento, criando entropia na equipa dado o período de ambientação que este teve de ter para se inteirar do evento. Também a própria estrutura da Comissão Organizadora, já com os vários Núcleos, em que estes (\acrshort{neemaac}) só pretendiam ter alguns pelouros em alturas mais tardias, fez com que, ao longo do ano, fosse entrando e saindo gente da organização o que é péssimo para as pessoas se ambientarem ao evento e estarem a par das várias decisões já tomadas desde o início da organização.

\subsubsection{Bot Olympics}

Em outubro, foram criadas, em simultâneo, em reunião de Direção, as comissões organizadoras do Bot Olympics e da Gala Ohms D'Ouro. Uma vez que o Bot Olympics foi realizado em conjunto com o \acrfull{cr} já sabíamos que áreas teríamos que preencher com pessoal do \acrshort{neeec}. Desta forma, ao serem selecionadas as pessoas, estas foram logo apontadas para a área onde iriam trabalhar. Em paralelo, foram escolhidos três Coordenadores que não ficaram responsáveis por nenhuma área do evento. Esta comissão resultou muito bem tendo sido apenas necessário substituir um dos responsáveis (o responsável pelo contacto às escolas) tendo este sido substituído pelo Presidente do Núcleo já bastante tarde (em dezembro) e tendo um dos responsáveis pelos patrocínios por parte do Clube de Robótica abandonado a comissão, algo que já tinha sido previsível e contemplado um plano alternativo pelo que não trouxe problemas.

\subsubsection{Gala Ohms D'Ouro / 20 Anos do NEEEC/AAC}

A criação da comissão para a gala contemplou de imediato a tarefa de criar a VI edição da Gala e planear todas as celebrações do 20º aniversário do \acrshort{neeec}, uma vez que as datas e a temática eram bastante próximas. Como é habitual na gala, o Presidente do Núcleo foi um dos Coordenadores do evento tendo, este ano, sido acompanhado pela Vânia Silva. Tendo em conta o maior trabalho que se previa foram convidadas várias pessoas, num número maior do que o habitual para este evento. Esta comissão resultou muito bem tendo apenas havido duas alterações: a Elisabete Santos, que entrou após ter saído da \acrshort{dg}, para apoiar na história do Núcleo e a entrada do Tiago Baltazar, um dos apresentadores do evento. De realçar, no entanto, que havendo uma estrutura completamente nova na gala e ainda mais diferente tendo em conta a celebração do aniversário, houve várias pessoas que tiveram de ser elucidadas sobre as suas tarefas e como funcionaria o trabalho para as mesmas, tendo a comissão um período grande de adaptação até ter entrado no seu pleno de trabalho, em algumas áreas. Dado que esta edição da gala celebrou o aniversário do Núcleo, teve uma equipa maior que o habitual que não será necessária em futuras edições da mesma.

\subsubsection{ENE3}

Este evento começou a ser organizado ainda no início do mandato anterior, tendo, por isso, uma comissão já organizada no início do mandato 2017/2018. No entanto, foi necessário acrescentar vários elementos do novo mandato do Núcleo e o Presidente e Vice-Presidente do Núcleo tiveram de passar a ser Coordenadores do evento, algo que não foi feito com tanto planeamento como aconteceu nas atividades referidas anteriormente, mas que acabou por correr bastante bem por se tratar de um evento que decorreu muito antes das restantes atividades.

\subsubsection{Outros}

Existem ainda vários outros eventos de considerada dimensão, alguns do Polo 2, outros do Pelouro das Saídas Profissionais, por exemplo, que não tiveram comissões externas, mas que envolveram o trabalho de vários pelouros em conjunto, de acordo com os métodos laborais estipulados para o Núcleo.
