% ===========================
% # Correio                 #
% ===========================

\subsection{Correio}

O envio e receção de cartas por parte do Núcleo tem sido uma realidade cada vez mais escassa, mas que continua a existir. No sentido oposto, o envio e receção de encomendas tem crescido bastante, principalmente com o crescimento acentuado de eventos de maior dimensão que o Núcleo tem tido. Existem duas "caixas de correio"\space que o Núcleo dispõe: a Secretaria do \acrshort{deec} e a \acrshort{aac}.

No caso da Secretaria do \acrshort{deec}, as senhoras costumam ligar sempre que uma nova encomenda ou carta chega dirigida para o Núcleo, contudo já aconteceu terem recebido uma carta que deixaram na caixa do \acrshort{neeec} situada na sala das impressoras, pelo que convém, esporadicamente, verificar se a caixa tem alguma carta. No caso de encomendas, caso não consigam contactar ninguém do Núcleo para a levantar na hora, as senhoras costumam ficar com ela e enviam um email para o Núcleo a avisar, para depois se ir levantar. Dada a confusão que as senhoras da Secretaria costumam ter em distinguir o Núcleo de outras associações estudantis existentes no Departamento, é também frequente contactarem-nos por correio dirigido a estes.

No caso da \acrshort{aac}, o correio costuma ser encaminhado para a Secretaria da \acrshort{aac}, situada no piso 2 do edifício. No único caso que ocorreu este ano, o aviso que recebemos foi pessoalmente pela Coordenadora dos Núcleos da \acrshort{dg}, pelo que tememos não existir nenhum método oficial de aviso. Quando o correio se refere ao Montepio Geral, as cartas são redirecionadas para o gabinete do \acrfull{ctp}, situada no piso térreo do edifício. Quando se tratou das cartas com os cartões matriz para os novos titulares, apenas fomos avisados da receção das cartas quando foi realizado o fecho de contas mensal pelo Tesoureiro, contudo, no final do mandato, chegámos a receber um aviso por email direcionado para a caixa de email da Tesouraria do Núcleo, contudo não sabemos se foi um caso pontual ou não.

Quanto ao envio de cartas, a situação mais frequente costuma envolver assuntos de Tesouraria, nomeadamente o envio de faturas referentes a patrocínios. Desta forma, dado que se tratam de assuntos bastante sérios, recomendamos o envio por correio registado para manter sempre um registo e prova das cartas que foram ou não enviadas para cada entidade.

Para o envio de cartas, existem já dois modelos à disposição: um modelo de carta, que apesar de ter conteúdo muito direcionado para o envio de faturas, permite ter uma ideia do que consideramos ser uma carta formal; e um modelo de envelope, para imprimir na impressora do Núcleo, permitindo o envio de envelopes com um ar mais profissional.