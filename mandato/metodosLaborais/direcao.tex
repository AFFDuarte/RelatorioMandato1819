% ===========================
% # Direção                 #
% ===========================

\subsection{Direção}

A Direção do \acrshort{neeec} nomeou internamente um responsável para cada Pelouro. Este responsável devia acompanhar o que se estava a passar no Pelouro, as reuniões e as suas atividades aconselhando o CG sempre que fosse necessário.
Inicialmente, os responsáveis por cada Pelouro foram distribuídos consoante o historial dos membros da Direção no Núcleo e tentando evitar também a ligação por amizade ficando a distribuição feita da seguinte forma:
\begin{itemize}
\item João Bento – Saídas Profissionais e Formação
\item João Martins –  Relações Externas e Desporto
\item Ivo Frazão – Pedagogia e \acrshort{gape}
\item Miguel Antunes – Imagem
\item José Pedro – Cultura e Lazer
\end{itemize}

Contudo, esta organização não resultou da melhor forma pois o Pelouro da Cultura e Lazer apresentava um trabalho fraco e o facto do CG da altura e do José Pedro terem uma relação de amizade próxima impossibilitou uma forte tomada de posição. Também o Pelouro do Desporto e das Relações Externas apresentaram problemas e estavam ambos em cima da mesma pessoa pelo que foi necessário reorganizar a distribuição passando a ser a seguinte:
\begin{itemize}
\item João Bento – Desporto
\item João Martins –  Relações Externas
\item Ivo Frazão – Pedagogia e \acrshort{gape} e Cultura e Lazer
\item Miguel Antunes – Imagem
\item José Pedro – Saídas Profissionais e Formação
\end{itemize}

Passado algum tempo desta alteração os CGs da Cultura e Lazer e do Desporto demitiram-se tendo ficado o João Bento como CG do Desporto temporariamente (toda a Liga \acrshort{deec} foi organizada enquanto o Pelouro estava a seu cargo) e o Ivo Frazão como CG da Cultura e Lazer temporariamente (tendo, durante este período, sido iniciada a organização do Quiz Solidário). Após a Liga \acrshort{deec}, a Direção nomeou novos CGs tendo havido uma nova reorganização:
\begin{itemize}
\item João Bento – Desporto e Saídas Profissionais e Formação
\item João Martins –  Imagem
\item Ivo Frazão – Pedagogia e \acrshort{gape}
\item Miguel Antunes – Relações Externas
\item José Pedro – Cultura e Lazer
\end{itemize}

De realçar que esta mudança se deveu a vários fatores:
\begin{itemize}
\item As Saídas Profissionais apresentavam um nível de trabalho muito profissional sendo extremamente independentes pelo que qualquer que fosse o responsável, não teria trabalho acrescido.
\item O CG das Relações Externas já apresentava um desrespeito enorme por todos os membros da Direção pelo que o Miguel Antunes assegurou a gestão do Pelouro, pois era a pessoa da Direção mais indicada para exercer as funções de CG das RE, caso necessário, o que veio, de facto, a acontecer.
\item O João Martins passou a ter um conhecimento mais aprofundado do funcionamento do Pelouro da Imagem e da gestão de equipa desta pelo que foi mais fácil coordenar-se com o Pelouro, em vez do Miguel Antunes.
\item Por ser um Pelouro com muitas atividades recreativas e não havendo já mais problemas com a gestão do Pelouro, o José Pedro, naturalmente, era a pessoa mais indicada para auxiliar o Pelouro da Cultura e Lazer.
\end{itemize}

Esta organização, na nossa opinião, é extremamente importante para que possa haver alguém mais em cima dos pelouros. Contudo, houve vários casos que não contribuíram de forma positiva como o facto de haver Colaboradores que se apoiavam mais nos representantes da Direção do que nos CGs. O facto do Presidente, do Tesoureiro e do Administrador, pessoas que têm competências de elevado trabalho a si atribuídas, terem ainda de estar preocupados com pelouros foi também extremamente negativo tendo provocado um trabalho acrescido e cansativo a estes membros.