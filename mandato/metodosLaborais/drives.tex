% ===========================
% # Drives                  #
% ===========================

\subsection{Drives}

No início do mandato a Drive utilizada pelo Núcleo era a Google Drive. Na conta principal (neeec.aac.uc@gmail.com), existia uma pasta para a Direção e outra para cada Pelouro com o material do mandato 2016/2017. Cada uma das pastas estava partilhada com os membros da respetiva edição. Para a Gala Ohms D'Ouro as informações de todas as edições encontravam-se na Google Drive respetiva. No caso do Bot Olympics existiam várias drives, não se tendo sequer conhecimento de todas as drives existentes no início do presente mandato. Por fim, a drive principal inclui ainda algumas pastas partilhadas consigo: a Drive do Polo 2 e a Drive da UGF. A Google Drive encontrava-se ainda bastante cheia, já muito perto do seu limite máximo. Descobriu-se de seguida, através das senhas guardadas no Google Chrome, através da conta principal do \acrshort{neeec}, existirem ainda mais drives (dropbox, BOX, etc.) onde se encontravam dados de outros mandatos.

Desta forma, deu-se início a uma reestruturação da organização das drives do Núcleo, começando-se por criar uma OneDrive Empresarial com 1 Tb sem custos (associada ao domínio do \acrshort{deec}: neeec.aac@deec.uc.pt). A escolha da OneDrive recaiu sobre o facto de ser um sistema que permite um melhor sistema de partilhas, possui um melhor cliente de sincronização com os computadores e não obriga à utilização de internet para edição de documentos, como acontece com o formato "Docs do Google". Contudo, esta drive não permitia partilhas entre contas cujo domínio não fosse deec.uc.pt e também não permitia a criação de pastas públicas. Assim, criou-se outra OneDrive Pessoal (com 5 Gb de armazenamento) onde se inseriu todos os dados do presente mandato (2017/2018) e na drive empresarial foram colocadas todas as informações dos restantes anos. Em ambas as Drives foi criada a seguinte estrutura de pastas:
\begin{itemize}
\item Direção
\item Geral
\item Eventos
\item Mesa do Plenário
\item Pelouros
\end{itemize}

Desta forma, na Google Drive ficaram apenas as drives partilhadas do Polo 2 e da UGF bem como documentos gerais cuja utilização da Google fosse imperativa, nomeadamente formulários Google.

Ao longo do ano foram existindo vários problemas uma vez que a OneDrive pessoal era de apenas 5 Gb o que não permitia salvaguardar os dados todos do presente mandato e levava a que as pessoas necessitassem de intercalar entre as duas drives existentes. Adicionalmente, foi criada uma OneDrive Empresarial para o Bot Olympics sendo sempre necessário aceder à mesma para gerir os documentos, uma vez que não era possível fazer partilhas com a mesma.

Já no término do mandato, tomou-se a decisão de adquirir uma conta OneDrive Pessoal com 1 Tb de armazenamento. Desta forma, foram transferidos todos os ficheiros das drives existentes para a nova drive (geral@neeec.pt) obtendo-se a seguinte estrutura de pastas:
\begin{itemize}
\item Direção: pastas de Direções dos vários mandatos, encontrados até então.
\item Geral: pasta geral com as informações gerais como contactos, regulamentos e modelos.
\item Eventos: pastas do \acrshort{ene3} 2017 e das várias edições da Gala Ohms D'Ouro, UGF e Bot Olympics.
\item Mesa do Plenário: pastas das Mesas do Plenário dos vários mandatos, encontrados até então.
\item Pelouros: pastas de cada um dos pelouros com os vários mandatos, encontrados até então.
\end{itemize}

Os vários membros das equipas de cada Pelouro, comissão organizadora, Direção ou Mesa do Plenário tinha então acesso, através de partilha com permissões totais, às pastas dos respetivos eventos (que contém as informações do presente mandato e dos mandatos anteriores) e acesso à pasta geral através de acesso por link sem permissões de edição. Este sistema parece-nos ter funcionado extremamente bem e realçamos o facto de termos reunido num só local as informações encontradas dos mandatos mais recentes permitindo um arquivo muito bom do Núcleo. Chamamos à atenção, no entanto, da importância de ao longo de cada mandato relembrar os vários membros para porem os documentos nas drives, para que estes não se percam, e, sempre que possível, comprimir o tamanho dos ficheiros para que a drive não encha rapidamente o seu espaço.