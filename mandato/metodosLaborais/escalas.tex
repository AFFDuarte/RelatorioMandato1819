% ===========================
% # Escalas                 #
% ===========================

\subsection{Escala da Sala do Núcleo}
\label{escalanucleo}

De modo a fazer cumprir o horário de atendimento da Sala do \acrshort{neeec} decidimos implementar uma escala. A escala tinha turnos de uma hora a começar às 10h e a terminar às 17h, em que a cada momento estavam presentes duas pessoas na sala do \acrshort{neeec} de modo a estimular também a interação entre todos os membros. Cada Colaborador tinha que fazer dois turnos e cada Coordenador Geral um turno. Por sua vez, os membros da Direção e da Mesa do Plenário estavam isentos de fazer qualquer turno (note-se que se a escala fosse, de facto, toda preenchida não haveriam turnos para todos os membros do Núcleo). Já no segundo semestre, após uma análise do primeiro semestre, decidiu-se não ter horário do Núcleo às quartas à tarde, uma vez que são quase sempre dias de reunião e às sextas de manhã, pelas dificuldades óbvias no preenchimento da mesma. Ainda no primeiro semestre, após se verificar que alguns membros não cumpriam a escala, foi criada uma folha de presenças, algo que vigorou até ao final do ano e funcionou bastante bem. Um dos problema com esta escala prendeu-se com o facto dos membros menos interessados do Núcleo não a preencherem e, assim, nem toda a escala ter estado preenchida. 

A escala foi também colocada online para que todos os que necessitassem de mudar o horário ao longo do semestre (devido a mudanças no seu horário de aulas, por exemplo) o pudessem fazer, algo que nunca foi muito entendido pelas pessoas. Outro dos problemas prendeu-se com o facto de várias pessoas não acharem importante avisar de quando iriam faltar e nos casos em que as duas pessoas faltavam em simultâneo o Núcleo acabava por estar fechado. A escala era semanalmente colocada numa folha junto da porta todas as segundas-feiras de manhã para que os membros pudessem assinar e confirmar a sua presença.

No início do primeiro semestre, numa reunião geral, foi feita a escala dando prioridade a quem estava presente. Os restantes membros foram de seguida informados que podiam preencher a escala. 

No início do segundo semestre, o método de preenchimento da escala mudou, dando a oportunidade às pessoas que tinham tido mais presenças no 1º semestre de preencher a escala primeiro, como forma de recompensa pelo bom trabalho.

No final foram contabilizadas as presenças que contribuíram para a avaliação de todos os membros do \acrshort{neeec}. 

Um problema com que nos deparámos foi o facto de nem todos os Colaboradores poderem preencher dois turnos dada a interseção entre os espaços livres dos seus horários e os turnos livres no horário do Núcleo. Consequentemente, no início do segundo semestre foi pedida uma justificação a quem apenas conseguia preencher um turno e foram justificadas essas faltas por impedimento horário. No caso de um Colaborador faltar foi pedido para justificarem as faltas junto do Secretário, que decidia ou não justificar essa mesma falta perante a justificação apresentada. Outro problema foi também o facto de, como sempre acontece, haver membros do Núcleo que vão deixando de pertencer ao Núcleo mas não informam ninguém disso e, como tal, acabam por ficar na escala para sempre. Algo que poderá resolver facilmente esta questão é a adição de um módulo ao sistema informático do \acrshort{neeec} que permita marcar aí as presenças e que retire os membros da escala após um dado número de faltas consecutivas. Nesse módulo poderá também ser implementado um botão simples para a justificação de faltas, que poderão depois ser aprovadas pelo Secretário ou outro a designar, e avisos para quem não tem os turnos todos preenchidos poder ir sendo alertado quando são abertas novas vagas nos horários. Dados estes problemas, é muito importante repensar exaustivamente o modo de preenchimento da escala bem como se é importante ou não todos os Colaboradores preencherem dois turnos ou se bastará um. É também importante deixar todas as regras bem estipuladas logo no início do mandato.