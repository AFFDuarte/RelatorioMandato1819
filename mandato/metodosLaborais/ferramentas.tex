% ===========================
% # Ferramentas de Trabalho #
% ===========================

\subsection{Ferramentas de Trabalho}

\begin{itemize}
\item Slack\\
A ferramenta de trabalho e diálogo entre a equipa foi sempre o Slack. Desta forma, existiu um workspace para o Núcleo e um para cada um dos eventos de maior dimensão (\acrshort{ene3}, Gala Ohms D'Ouro e Bot Olympics), tendo existido também um Slack para o Polo 2. Dentro do Slack havia salas públicas e privadas. As salas públicas tinham um off-topic, para momentos mais engraçados, uma sala geral, para assuntos normais do Núcleo, e uma sala para anúncios, principalmente para anunciar horas de divulgar iniciativas do Núcleo através de spam. Foram também criadas salas adicionais, sempre que necessário, como, por exemplo, uma sobre a plataforma informática de gestão interna do Núcleo e outra sobre problemas logísticos existentes no polo 2. Nas salas privadas havia uma sala para cada Pelouro, uma sala para os Coordenadores Gerais e uma sala para feedback de divulgação onde eram discutidos a qualidade e a informação que a divulgação iria ter. Existiram também salas privadas para assuntos diferentes como para o site, o Polo 2, a Semana dos Ramos e a NEEEC Sports \& Culture Week. Adicionalmente, cada Pelouro poderia fazer a sua gestão conforme bem entendesse, podendo criar mais salas. Em cada sala de Pelouro estavam todos os membros do Pelouro e os membros da Direção. A sala do feedback de divulgação continha todos os CGs, o Presidente da Mesa do Plenário bem como os membros da imagem para que estes pudessem ver quais os problemas nos cartazes e pudessem submeter novas versões. Esta ferramenta possibilitou uma maior organização da equipa e uma concentração de toda a informação. Adicionalmente, foi possível organizar de forma mais coerente os tópicos das conversas, separando-os em várias salas. É de realçar que, para que esta plataforma funcione, o Núcleo não pode ter qualquer tipo de espaço de trabalho no Facebook para que não haja tentações em usar as duas plataformas e consequente dispersão da informação. É também de notar que existem algumas pessoas que tinham alguma resistência em utilizar o Slack, mas após uma ajuda sobre o uso da plataforma e a um elevado incentivo em instalar as aplicações da mesma, as pessoas que continuaram a resistir ao uso desta plataforma foram as mesmas que, mesmo em conversas em outros meios (por exemplo, Facebook Messenger) não respondiam. Esta ferramenta permite várias integrações com aplicações e scripts online, facilitando a interação com outras ferramentas que sejam utilizadas, como é o caso de um aviso sempre que o formulário de pedidos de imagem recebia uma nova resposta, enviando um aviso para o Pelouro da Imagem. Esta ferramenta apresenta o problema de manutenção do histórico de mensagens, limitado a apenas 10000 mensagens, um valor muito facilmente esgotado em pouco tempo, contudo as vantagens da ferramenta ultrapassam claramente esta desvantagem. Existem contudo outras ferramentas deste estilo que podem ser interessantes de avaliar, principalmente se ultrapassarem esta desvantagem, como é o caso do Microsoft Teams, integrado no Office 365 para estudantes.\\
Esta ferramenta foi também utilizada como conversa entre participantes, voluntários e membros da organização do Bot Olympics, para permitir que toda a gente pudesse receber as informações importantes da organização para os participantes, mantendo algum espaço lúdico de conversa entre todos.\\
No futuro, recomendamos vivamente que seja mantido o mesmo Slack de mandato para mandato uma vez que existem várias salas como, por exemplo, a "pedagogia\_email" ou a "direcao\_trello" que já continham várias configurações informáticas feitas e, ao se fazer um novo Slack para cada mandato, obriga a inserir novas configurações. Desta forma, entre cada mandato bastaria, em cada Pelouro, inserir as novas pessoas e retirar as antigas, sendo possível eliminar o histórico de conversas ou não, conforme fosse pretendido. A única desvantagem é o facto do limite de 10000 mensagens não ser reposto no início do mandato. Contudo, temos verificado que este limite é facilmente alcançado pelo que a reposição, ou não, do Slack acaba por não fazer diferença.

\item Trello\\
Esta foi uma novidade deste mandato tendo sido criada uma sala para cada pelouro, Direção, Mesa do Plenário, comunicação e eventos grandes. Nesta plataforma cada membro podia acrescentar as suas tarefas, metas temporais e comentários sendo semelhante a uma parede de post-its. Esta plataforma foi utilizada com integração no Slack, algo que consideramos essencial para que não ocorra o esquecimento da mesma. Em alguns casos, como o da comunicação, foram utilizados plugins para associar o Trello ao calendário e, assim, emitir lembretes sobre a mesma a quem subscrevesse o calendário. A utilização desta plataforma foi muito positiva pois permite rapidamente perceber o que está ou não por fazer no Núcleo mas foi essencialmente utilizada em eventos grandes, pela Direção, pela comunicação e pela imagem pelo que, no futuro, aconselhamos a uma maior dinamização da mesma, para que todos a utilizem eficientemente.
À semelhança do Slack, existem várias configurações informáticas no Trello que nos fazem aconselhar a não criação de um novo Trello a cada mandato. No caso do Trello não existe absolutamente nenhuma vantagem em criar um novo Trello pelo que basta, no final dos mandatos, retirar os membros antigos e colocar os novos. É, no entanto, importante avisar as Direções futuras deste assunto o mais cedo possível, assim que estas comecem a avançar com os seus projetos, pois as mesmas têm tendência a criar as novas plataformas sem se lembrarem deste pormenor e, depois, já não têm, naturalmente, interesse em anular o trabalho que tiveram.

\item WhatsApp\\
Esta plataforma foi utilizada para a comunicação interna entre participantes e staff de grandes eventos, como foi o caso do \acrshort{ene3}, permitindo uma conversa engraçada entre todos. Contudo a sua utilização como uma conversa muito lúdica faz com que muitas das mensagens importantes da organização acabem por não ser lidas pelos participantes, pelo que não é a plataforma ideal para todos os casos.

\item Gmail\\
Apesar dos emails do Núcleo estarem todos localizados no servidor do \acrshort{deec}, a interface própria disponibilizada por estes (Zimbra) obrigaria à aprendizagem de mais uma ferramenta, o que poderia criar alguma entropia. Desta forma, reencaminhámos todos os emails através de contas GMail, servindo esta como a interface de utilização, dado que esta tende a ser uma das interfaces mais conhecidas por todos. A única desvantagem que esta plataforma apresenta é o facto de se ter de fazer login com uma conta do estilo "nomedopelouro.neeec@gmail.com" e não através do verdadeiro email, "nomedopelouro@neeec.pt".

\item OneDrive\\
Esta passou a ser a nossa ferramenta base para arquivo de documentação principalmente pelo facto de ter melhores condições na interligação com os programas do Office e pelas suas características de partilha e armazenamento de dados bastante simples.

\item Google Drive\\
Uma vez que migrámos todos os nossos documentos para a OneDrive, a Google Drive foi utilizada somente para a criação de formulários da Google e para a ligação a outros Núcleos que utilizassem a plataforma.

\item Google Forms\\
Os formulários disponibilizados pela Google foram das principais ferramentas utilizados pelo Núcleo, permitindo gerir vários tipos de informação que precisávamos para as atividades. Esta ferramenta é muito versátil e intuitiva, além de permitir o uso de plugins e scripts personalizados que facilitam ainda mais a gestão da informação. Para além disso, os forms são facilmente integrados no site do Núcleo deixando a possibilidade de qualquer pessoa sem acesso à edição do site ou sem conhecimentos informáticos que permitam a edição do site possam gerir os formulários.

\item Google Scripts\\
Associados aos formulários do Google, esta ferramenta permitiu um maior dinamismo na comunicação com as pessoas, nomeadamente na fácil personalização de respostas automáticas ao preenchimento dos mesmos e para uma comunicação mais rápida na submissão de formulários e na sua ligação ao Slack. Os scripts baseiam-se em JavaScript, uma linguagem bastante simples de aprender e com bastante documentação online, pelo que a habituação a utilizar este tipo de utilitários não será complicada.

\item Google Calendar\\
Esta foi uma ferramenta essencial tendo sido utilizados três tipos de calendários: o calendário de atividades que é acessível a todos, através do site do Núcleo; o calendário interno que permite a gestão interna do Núcleo, nomeadamente a reserva da sala para reuniões; e o calendário de divulgação, associado ao Trello, que contém toda a informação sobre as publicações a emitir.

\item Skype\\
Esta plataforma foi utilizada, principalmente, para reuniões com empresas. Existe uma conta do núcleo nesta plataforma, já devidamente configurada, e o computador do Núcleo tem uma webcam precisamente para permitir reuniões via Skype na sala do Núcleo.

\item Office\\
As ferramentas base de trabalho foram o Word e o Excel. Dado que atualmente estas ferramentas permitem a edição simultânea por múltiplos utilizadores, tal como o Google Docs, esta ferramenta tornou-se ainda mais preponderante, dado que, ao contrário dos Google Docs, permite a utilização offline desses ficheiros. Este tipo de ferramentas permite a criação facilitada de modelos, que permitem um preenchimento facilitado de documentos, evitando erros por parte de utilizadores menos experientes em realizar esses documentos.

\item Photoscape\\
Esta plataforma, gratuita, permitiu a qualquer membro do Núcleo, sem conhecimentos específicos de ferramentas de imagem, inserir as marcas de água necessárias para se poder publicar as fotos nas redes sociais.

\item Illustrator\\
Esta foi a ferramenta principalmente utilizada pela Imagem para a criação de grande parte dos materiais gráficos do Núcleo. Por produzir imagens vetoriais, os resultados conseguidos com esta plataforma possuem melhor definição.

\item Photoshop\\
Esta foi outra das plataformas utilizadas pela Imagem, em menor escala, para a criação dos materiais gráficos do Núcleo.

\item Premier\\
Esta foi uma das ferramentas utilizadas pelo Pelouro da Imagem na elaboração de vídeos.

\item I love pdf\\
Este site foi muito utilizado pois permite a manipulação rápida e gratuita de ficheiros PDF desde a sua compressão, junção, separação, recorte e conversão.

\item I love img\\
Este site foi muito utilizado pois permite a manipulação rápida e gratuita de ficheiros de imagem desde a sua compressão, junção, separação, recorte e conversão.

\item PNG 2 PDF\\
Esta ferramenta foi utilizada para a conversão de ficheiros PNG em ficheiros PDF, nomeadamente cartazes para impressão.

\item Zello\\
Esta ferramenta funciona como um walkie-talkie permitindo a criação de grupos individuais dentro de uma equipa pelo que foi muito útil em eventos de maior dimensão.

\item VP Eventos\\
Esta plataforma foi utilizada no \acrshort{ene3} e permitiu, através da atribuição de um QRCode a cada participante, gerir todo evento, nomeadamente nas inscrições de cada atividade e na entrega de senhas de refeição, ao evitar entregas de mais senhas que o suposto a cada participante.

\item Adobe Premier Pro\\
Adobe Premiere Pro é um programa que é empregado para a edição de vídeos profissionais, tendo sido também usado pelo Pelouro da Imagem.

\end{itemize}