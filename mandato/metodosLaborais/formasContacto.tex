\subsection{Formas de Contacto}

Existem diversas formas de contacto com as entidades parceiras, sendo que as principais são indubitavelmente o email e o telefone. Destes dois métodos, o email foi claramente o mais utilizado, não só pela sua praticalidade, facilidade de comunicação para as pessoas mais tímidas e pelo facto de permitir manter um registo do que foi escrito (necessário para relembrar certos pormenores que naturalmente vão sendo esquecidos), contudo o telefone foi uma chave essencial para acelerar certas tarefas, pois tende a chegar-se mais facilmente às pessoas certas e "obriga-as"\space a executar a tarefa que necessitamos na hora ou, pelo menos, de forma mais célere. Contudo a falta de registos pode criar algumas situações evitáveis, pelo que aconselhamos a concluir as chamadas telefónicas realizadas com o envio de um email para a pessoa com um resumo breve do que foi discutido para que as duas partes estejam cientes do que foi discutido.

\subsubsection{Email}

\paragraph{Contas de Email}

O \acrshort{neeec} tinha há já vários anos uma conta disponibilizada pelo \acrshort{deec}, neeec.aac@""deec.uc.pt. Esta conta encontra-se alojada nos servidores do Departamento, o que requeria um acesso pela interface disponibilizada por este (o Zimbra), contudo, esta interface era mais uma ferramenta que os Colaboradores teriam que aprender a utilizar, pelo que foi adotada uma solução de reencaminhamento permanente desta conta para uma conta do Gmail (neeec.aac.uc@gmail.com), que serviria apenas de interface para a conta do \acrshort{deec}, dado que essa seria a conta predefinida a ser utilizada. Este método permite ainda que a conta de email seja acessível pelas aplicações disponibilizadas para o Gmail de forma simples, sem configurações complexas que a ligação direta ao Zimbra obrigaria. Para além desta conta, existia ainda uma conta própria do Pelouro das Saídas Profissionais (sp.neeec@gmail.com) e outra dos Representantes dos Estudantes do \acrshort{mieec} que era utilizada também pelo Pelouro da Pedagogia (pedagogiadeec@gmail.com), dado que estes eram os pelouros que, à parte da Direção, mais utilizavam o email como plataforma de comunicação. Contudo, havia pouca uniformização destes emails, nomeadamente nas assinaturas dos emails que dependiam de cada Coordenador Geral. Além disso, estes emails eram apenas diretamente acessíveis pelos CGs e nunca pela Direção do Núcleo. Mais nenhum Pelouro possuía email próprio, dependendo do email principal do Núcleo para comunicarem e cada Direção do \acrshort{neeec} tinha uma política diferente quanto ao uso desse email (no mandato anterior, apenas a Direção tinha acesso para evitar problemas recorrentes no mandato anterior de emails que nunca eram respondidos porque eram lidos pelas pessoas indevidas, fazendo com que as pessoas certas por vezes não recebessem notificação desses emails, dependendo do Secretário para enviarem os emails que necessitavam).

Aproveitando o facto de se ter comprado um domínio próprio para o Núcleo, definiu-se que todos os pelouros deveriam ter um email próprio (estilo Pelouro@neeec.pt), mantendo a política de interface através do Gmail (cujo email de acesso seria Pelouro.neeec@""gmail.com). Cada um dos emails dos pelouros dá acesso ao email principal através do sistema de delegação do Gmail, permitindo que a Direção do Núcleo tenha acesso a esses emails caso necessite. Dessa forma, foram também criadas assinaturas padronizadas para cada um dos emails, permitindo garantir uma imagem coesa do Núcleo. Inicialmente, esta assinatura incluía dois campos que foram entretanto removidos:
\begin{itemize}
\item Nome do responsável pelo email e redação da frase "Com os melhores cumprimentos": inicialmente estava escrito em todas as assinaturas de emails a frase "Com os melhores cumprimentos, NOME DA PESSOA, Coordenador do Pelouro NOME DO PELOURO do \acrshort{neeec}". Contudo, em diversos pelouros várias eram as pessoas a mexer no email esquecendo-se de substituir o nome pré-definido pelo correto pelo que esta redação foi retirada.
\item Telefone do responsável: inicialmente nas assinaturas do email existia o número de telemóvel do responsável pelo Pelouro contudo, esta informação fazia com que as chamadas fossem feitas sempre para o Coordenador do Pelouro e não para a pessoa que estava a escrever a mensagem. Adicionalmente, vendo o caso que ocorria com a Direção, este telefone ficava registado nos emails e, nos anos seguintes, a pessoa que já não exercia funções continuava a receber várias chamadas. Além disso, algumas pessoas referiam motivos de privacidade para não pretenderem o seu número de telefone exposto no email. Desta forma colocou-se em todas as assinaturas o número de telefone fixo do Núcleo sendo que depois, através do telefone se reencaminharia a chamada para a pessoa indicada, tarefa facilitada pela instalação do sistema \acrshort{ivr} (ver secção \ref{subsubsec:telefone}). Contudo, caso seja vantajoso, durante a escrita de cada email pode ser colocado um contacto no corpo do texto mais direto para a pessoa que está a tratar do assunto.
\end{itemize}

Já perto do fim do mandato, descobrimos que existia ainda um email do \acrshort{neeec} associado ao domínio da \acrshort{aac}, neeec@academica.pt (que como é explicado na secção \ref{subsubsec:dominio}, seria um email que achávamos mais interessante de ter), contudo, dado que já tínhamos domínio próprio, apenas criámos um reencaminhamento deste email para o nosso principal.

Recomenda-se que no futuro seja criado um manual de instruções sobre como gerir os emails (incluindo os pormenores técnicos mais avançados) para permitir que os mesmos possam continuar a ser utilizados e melhorados/adaptados no futuro.

\paragraph{Alguns problemas}

O problema de um domínio próprio a nível dos emails é a reputação que o domínio tem nos serviços de bloqueamento de spam que existem, que direcionam muitas vezes os emails para as pastas de spam dos destinatários, que nem sempre a veem, ou os bloqueiam mesmo. Esta área de capturar emails de spam sofre frequentemente alterações, com técnicas novas de captura, que obrigam a um esforço adicional pelos gestores de rede para manter os seus serviços a funcionar da melhor forma, uma tarefa que nos foi simplificada pelo facto do \acrshort{gri} ter que fazer esse trabalho para os emails do Departamento, pelo que facilmente fizeram esse trabalho no nosso domínio. Desta forma, aconselhamos que analisem periodicamente os resultados de testes disponibilizados na Internet sobre a reputação do domínio (por exemplo, www.mail-tester.com) para saberem o estado atual da reputação. Outro problema recorrente, devido a termos o email alojado nos servidores do Departamento, é esses serviços bloquearem o endereço IP do servidor, um problema por vezes mais complicado de detetar, dado que nem sempre se recebe o email de regresso a informar desse bloqueio. Este caso aconteceu com os emails destinados aos servidores hotmail.com, live.com.pt ou live.com (apesar dos servidores outlook.com e afins continuarem a receber sem problema), causando alguns transtornos nas informações enviadas por email aos participantes das atividades.

\subsubsection{Telefone} \label{subsubsec:telefone}

O telefone do Núcleo é também uma peça fundamental para uma comunicação com o exterior. O Departamento fornece ao Núcleo um número de telefone próprio, permitindo comunicação direta exterior para o gabinete do Núcleo. Contudo, esta ligação com o gabinete do Núcleo tinha dois problemas principais: a timidez das pessoas que estavam em escala no Núcleo, fazendo com que muitas chamadas não fossem atendidas e os assuntos acabassem por se arrastar por mais tempo do que o que seria necessário; e a inexperiência das pessoas para encaminhamento das chamadas, uma funcionalidade chave que permite transferir uma chamada recebida para a pessoa mais correta para resolver o assunto. Enquanto que para o segundo problema resolvemos colocar instruções na secretária do Núcleo, resolvendo o problema (a inexperiência inicial pode fazer demorar um pouco nas primeiras vezes, mas seguindo as instruções, dificilmente irão errar e nada que incentivar as pessoas a fazer chamadas entre amigos para aprenderem a mexer não resolva), o primeiro problema não tem uma solução óbvia em vista, dado que cada pessoa tem a sua razão para não atender, muitas vezes tendo medo de dizer "parvoíces"\space que os façam ser gozados. Uma estratégia possível é mostrar que as pessoas do outro lado não os conhecem, pelo que não ficarão marcados pelos telefonemas, contudo esta estratégia pode fazer com que algumas pessoas acabem a comportar-se "mal"\space ao telefone, fazendo coisas pouco profissionais, o que mancha a imagem do Núcleo.

No final do mandato, com a ajuda do \acrshort{gri}, o telefone do Núcleo passou a ter um sistema \acrfull{ivr} que permite reencaminhar logo as chamadas automaticamente para as áreas, pelo que é agora possível centralizar as chamadas num só local bastando depois alterar o contacto que está nesse sistema. Além de permitir um sistema único entre diferentes mandatos, esta solução é também bastante vantajosa para quando os responsáveis por uma dada área mudam ao longo do mandato, bastando alterar esse número. Contudo, sendo um sistema que está dependente de entidades terceiras ao próprio \acrshort{gri}, estas mudanças podem demorar algum tempo, pelo que não é recomendado mudanças muito frequentes.

Nas atividades grandes que realizámos que requeriam que vários participantes ligassem para a organização dessas atividades, nomeadamente no \acrshort{ene3} e no Bot Olympics, foi frequente a instalação de uma Secretaria do evento na entrada do piso 2 do Departamento, pelo que pedíamos ao \acrshort{gri} um telefone extra, que tinha um número próprio que era sempre colocado nas credenciais desses participantes e, quando a Secretaria fechava, tinha reencaminhamento permanente para a pessoa que estava responsável durante esse período, permitindo que apenas um telefone servisse para todo o evento. Dado que não estava associado ao número do Núcleo, este número impedia que após o evento os participantes pudessem utilizar esse número para contactar a organização, contudo evitava o abuso das linhas do telefone do Núcleo que podia ter que estar a trabalhar noutros assuntos nesses momentos do evento.

O telefone do \acrshort{neeec} tem duas linhas que podem ser usadas em simultâneo, pelo que é possível (e por vezes necessário) colocar pessoas em espera numa linha e atender/ligar a outra pessoa na outra linha. É bastante fácil de identificar as linhas que estão em espera através da luz laranja intermitente nos botões que ativam/desativam as linhas no telefone. É também possível criar uma conferência entre as duas linhas e o telefone do Núcleo ativando as duas linhas em simultâneo.