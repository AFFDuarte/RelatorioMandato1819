% ===========================
% # Fotografias             #
% ===========================

\subsection{Fotografias}

Ao longo de todos os eventos um dos problemas que tem existido no \acrshort{neeec} prende-se com a aquisição de fotografias e vídeos das atividades. O paradigma que tem existido faz com que haja apenas captação de fotografias principalmente para divulgação da atividade nas redes sociais. Desta forma são captados muito poucos vídeos e existem atividades onde nem fotos são captadas. Algo que consideramos fulcral é a criação de um arquivo de fotografias e vídeos das atividades onde alguém seja designado responsável por arquivar, em local próprio, todas as imagens e vídeos das atividades para que posteriormente estas sejam utilizadas para divulgação das mesmas em futuras edições e para que sirvam, também, como relatório das atividades desenvolvidas.

Um dos pontos que já há mais tempo se discute, sem nunca se ter chegado a nenhuma conclusão, é quem deve ser o responsável pela aquisição de multimédia nos eventos. Dessa forma, desde o início definiu-se que cada Pelouro deveria informar o Pelouro da Imagem para que este enviasse um repórter fotográfico para cada evento. Acabou-se por recuar na decisão, uma vez que não resultava, voltando-se ao modelo que tinha sido decidido no mandato anterior e as atividades voltaram a ser fotografadas por cada Pelouro em específico. Desta forma, voltou-se a ter o problema em que uns pelouros fazem esta tarefa de forma correta e outros não. Contudo, este é, de facto, o molde encontrado até agora que resulta melhor. De notar que o Núcleo não dispõe de nenhuma câmara fotográfica, algo que prejudica bastante a tarefa de aquisição de fotos. Com a criação do Instagram do Núcleo, todas as atividades do Núcleo, por regra, acabaram por ser facilmente divulgadas através de insta stories ou de publicações na plataforma referida, uma vez que todos os membros do \acrshort{neeec} tinham acesso à mesma.