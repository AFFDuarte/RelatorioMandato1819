% ==========================
% # Impressões             #
% ==========================

\subsection{Impressões}

O \acrshort{neeec} dispõe de três meios onde pode fazer as suas impressões:
\begin{enumerate}
    \item Impressora do \acrshort{neeec}\\
    O \acrshort{neeec} dispõe de uma impressora a laser na sua sala onde pode fazer impressões a preto e branco. Esta impressora utiliza toneres que podem ser comprados no eBay de forma bastante barata. Esta impressora tem sido utilizada apenas para impressões simples e rápidas uma vez que das várias opções disponíveis é a única que apresenta custos para o \acrshort{neeec}, para além de apresentar a pior qualidade de todas.

    \item Impressoras do \acrshort{deec}\\
    O \acrshort{deec} dispõe de três impressoras em dois locais aos quais o \acrshort{neeec} tem acesso através da sua chave eletrónica: a sala junto à Secretaria que tem uma impressora a cores e outra a preto e branco e o gabinete 3A.4 com uma impressora a cores. Estas impressoras têm um custo muito reduzido sendo o mesmo suportado pelo Departamento. Para aceder às mesmas é necessário utilizar um programa (Paper Cut) que pode ser acedido através da conta do \acrshort{neeec} e cuja instalação se encontra descrita no site informática.deec.uc.pt e pode ser feita em qualquer computador. Estas impressoras devem ser utilizadas como muita moderação. O \acrshort{neeec} utiliza-a para impressões normais ao seu bom funcionamento e para impressões elevadas para os seus eventos (neste caso, informou sempre a Direção do Departamento para que o facto da despesa na conta do \acrshort{neeec} do Paper Cut aumentar rapidamente não fosse uma surpresa para a Direção do Departamento). É também de notar que só a Direção teve acesso à senha que permitia impressões fazendo assim com que se evitassem abusos. Contudo, acabou por haver vários atrasos desnecessários e CGs a tomar iniciativa de fazer impressões em locais desnecessários pois tinham os documentos prontos muito em cima da data em que deles necessitavam, pelo que se aconselha a uma sensibilização dos mesmos para terem estes documentos prontos mais cedo.

    \item Plafond da \acrshort{aac}\\
    A \acrshort{aac} disponibiliza um plafond de 40€ por mês para fotocópias na Reprografia da \acrshort{aac}. Este plafond tem sido muito pouco usado dado que o edifício da \acrshort{aac} se encontra longe do Polo 2 e, como tal, dadas as nossas condições, não nos é útil ir ao mesmo fazer as impressões. Contudo, recomendamos uma utilização maior deste plafond principalmente para a impressão de grandes quantidades (eventos grandes), evitando assim utilizar em demasia a impressora do \acrshort{deec}. 
\end{enumerate}

É de notar que a Direção do \acrshort{neeec} decidiu proibir impressões da campanha eleitoral quer na impressora do \acrshort{neeec}, quer na do Departamento, mas autorizou a utilização do plafond da \acrshort{aac} para este fim, algo que acabou por ser proibido pela Administração da \acrshort{dg}. Esta decisão da Administração da \acrshort{dg} procura evitar que as direções de Núcleos ainda em funções inflacionassem a capacidade de divulgação de umas listas em detrimento de outras, apesar de, no nosso caso, a situação ter sido precavida com um regulamento que procurava a igualdade entre todas as listas que viessem a existir. Contudo, dado que seria extremamente complicado manter um registo atualizado dessa situação por parte dos serviços da Reprografia da \acrshort{aac}, compreendemos esta decisão. Desta forma, é de notar que a \acrshort{aac} controla o motivo para o qual é usado o plafond de impressões pelo recomendamos o seu uso moderado e adequado às atividades do \acrshort{neeec} para que não se repercutam, no futuro, restrições desnecessárias ao uso deste plafond.

A \acrshort{aac} dispõe também de um plafond de 10€ mensais para material de reprografia, plafond esse que tem vindo a ser utilizado para reposição do stock de material de papelaria no \acrshort{neeec}. Este plafond só pode ser utilizado em certo e determinado tipo de materiais que se encontram descritos numa lista emitida pela Administração da \acrshort{aac}. É também de notar que o acesso aos plafonds da \acrshort{aac} implica, obrigatoriamente, a apresentação do cartão do Núcleo e só pode ser utilizado por membros efetivos do \acrshort{neeec}, incluindo os 11 elementos da Direção e os 3 elementos da Mesa do Plenário. De notar também que o saldo de 40€ inclui despesas de encadernação pelo que a elaboração de livros encadernados, como este, torna-se gratuita através deste meio. Este plafond é bloqueado quando não são apresentadas as contas do mês anterior no \acrshort{ctp} até ao dia 10 de cada mês. Contudo, o plafond é renovado, à mesma, no dia 1 de cada mês ficando apenas suspenso se não for cumprida a condição anterior, a partir do dia 10.
