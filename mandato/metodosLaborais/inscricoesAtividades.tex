% =============================
% # Inscrições em Atividades  #
% =============================

\subsection{Inscrições em Atividades}

As inscrições em atividades são uma peça fundamental para a organização das mesmas, dado que é necessário garantir que o número de participantes é o adequado para a atividade, nomeadamente para evitar, por exemplo, workshops com demasiadas pessoas, impedindo que o orador consiga garantir um workshop de qualidade, ou com tão poucas pessoas que o orador fique chateado por ter que gastar do seu tempo em preparar e executar um workshop para tão pouca gente.

Anteriormente, a gestão de inscrições dependia muito da gestão que cada \acrshort{cg} fazia, podendo haver atividades em que os participantes tinham que preencher um formulário online com informações, enquanto outras atividades não questionavam informações nenhumas e apenas era necessário ir ao Núcleo, o que se tornava bastante confuso tanto para os participantes como para as pessoas do Núcleo. Além disso, dado que eram realizados pagamentos em dinheiro presencialmente no gabinete do Núcleo para quase todas as atividades, a gestão dos pagamentos tendia a seguir o seguinte modelo: existia um envelope para cada atividade no gabinete do Núcleo, onde eram apontados os nomes das pessoas que já tinham pago e era colocado o pagamento dessa pessoa. Considerámos que toda esta gestão era bastante confusa, podendo facilmente originar erros, e que obrigava cada \acrshort{cg} a deslocar-se ao Núcleo sempre que precisasse de confirmar o estado das inscrições, o que nem sempre era possível.

Para evitar os problemas elencados, decidimos mudar o sistema de gestão de inscrições e pagamentos, concentrando todas as atividades num formulário único (as únicas exceções deste formulário foram o \acrshort{ene3}, o Bot Olympics e a Gala Ohms D'Ouro dado que tinham especificidades muito próprias, como, por exemplo, a existência de Early Birds, e o Beer Olympics, por imposição do Polo 2). Este formulário procurava ser o mais generalista possível, questionando de imediato todas as informações comuns a todas as atividades e, em atividades que necessitavam de mais informações, eram acrescentadas páginas novas ao formulário, específicas para atividades (tirando partido do redirecionamento em função da resposta que os formulários permitiam). Associado a um link fácil de divulgar, todos os cartazes de atividades que necessitavam de inscrição passaram a divulgar apenas esse link e que rapidamente foi interiorizado pelas pessoas, pelo que consideramos que foi uma aposta ganha.

Além disso, o Excel criado automaticamente pelo formulário tinha as colunas referentes ao pagamento de cada pessoa disponível para todas as pessoas do Núcleo poderem indicar se o pagamento tinha sido realizado ou não, indicando ainda quem foi a pessoa responsável por esse pagamento, que tinha uma coluna para outras informações que fossem necessárias de corrigir, mas que não era possível dado que as restantes colunas tinham a edição bloqueada.

Posteriormente, com a criação do site de gestão interna do núcleo (interno.neeec.pt) o processo de pagamento e registo de inscrições ficou feito de forma automatizada pelo que para receberem inscrições os membros do Núcleo apenas teriam de aceder à plataforma e registar o pagamento da inscrição e o sistema automaticamente marcava a inscrição como paga bem como adicionava o movimento na caixa.

Definimos ainda um período máximo para pagamento das inscrições de 48 horas, para impedir que as vagas fossem ocupadas indefinidamente por algumas pessoas que acabavam por não comparecer nas atividades e não as pagavam. Contudo, neste ponto fomos algo relaxados inicialmente, permitindo que os prazos fossem sucessivamente ultrapassados sem consequências para essas pessoas, o que no segundo semestre acabou por correr mal, com vários workshops supostamente lotados, mas acabaram por não ter quase ninguém presente. Desta forma procurámos apertar um pouco as regras, contudo o sistema informático que tínhamos em funcionamento ainda não permitia um controlo tão apertado sem dedicar uma pessoa a isso, pelo que, aproximando-se o fim das atividades, responsabilizámos essa tarefa a cada \acrshort{cg}.

Tivémos algumas ideias para o sistema informático nomeadamente o envio de emails automáticos por falta de pagamento indicando que a inscrição seria anulada e o envio de certificado e justificação de faltas de forma automática, após confirmação da presença dos elementos nos eventos contudo, tal ainda não foi possível devido à falta de tempo do membro responsável pela plataforma mas recomendamos, vivamente, que seja feito no futuro pois facilitará imenso o trabalho.

Dada a possibilidade do formulário ter plugins e scripts próprios associados ao mesmo, por cada inscrição em atividade passámos a enviar um email automático a cada pessoa a garantir que a sua inscrição estava realizada, voltando a indicar os dados introduzidos para que pudessem ser corrigidos caso a pessoa detetasse algum erro e voltando a relembrar os métodos de pagamentos disponíveis para essa atividade. Pensamos que este simples email, totalmente automatizado, foi uma aposta também ganha, pois permite melhorar significativamente a imagem do Núcleo para as pessoas, dando um ar muito mais profissional, pelo que recomendamos que mantenham esse email.

\paragraph{Atividades com inscrições próprias}

Tal como referido anteriormente, algumas das atividades do Núcleo tiveram um formulário próprio para inscrição. No caso do \acrshort{ene3}, o formulário de inscrição específico foi criado antes da existência do formulário do Núcleo que criámos, pelo que não foi transposto para o novo formulário. No caso do Bot Olympics, dado que havia muitas especificidades do formulário de inscrição nesta atividade, o que obrigaria a criar muitas páginas específicas para o evento e criar alguma confusão no Excel das inscrições das atividades normais do Núcleo, associado ao facto de se tratar de uma organização conjunta com o \acrlong{cr} e haver uma pessoa responsável por todos as inscrições, que definiu que todos os pagamentos seriam por transferência bancária, optámos criar um formulário próprio. O mesmo aconteceu no caso da Gala Ohms D'Ouro, pelo que se optou pela criação de formulário próprio também, além de também existir a necessidade de configurar Early Birds no evento, com scripts próprios para facilitar essa gestão, o que poderia criar problemas nas outras atividades.