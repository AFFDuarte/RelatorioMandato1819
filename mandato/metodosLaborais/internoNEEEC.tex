% ==========================
% # Interno NEEEC          #
% ==========================

\subsection{Sistema Informático de Gestão Interna}

Com a existência da cadeira de Base de Dados, frequentada por vários membros da Direção do \acrshort{neeec}, foi desenvolvida, pelo Miguel Antunes e pelo João Bento, uma plataforma informática onde era possível gerir utilizadores e clientes que podiam gerir vendas, empréstimos, produtos, fornecedores, fornecimentos e saldo de caixa. Decidiu-se depois aplicar este sistema ao \acrshort{neeec} para tentar resolver o problema existente com o registo de vendas de produtos dentro da Sala do Núcleo bem como para gerir, de melhor forma, o stock do Núcleo e os empréstimos. Com a entrada do Miguel Santos e respetiva responsabilização pela parte informática do Núcleo, o sistema interno acabou por ser todo reformulado apresentando hoje possibilidade de gerir o stock do núcleo bem como de vários outros locais, fazer vendas de produtos, registar inscrições (nomeadamente, fazendo a ligação direta ao formulário de inscrições presente no site do Núcleo, possibilitando marcar a presença dos participantes em eventos e exportar uma lista de cada evento, no final do mesmo, com informações sobre os participantes), gerir os empréstimos (o sistema produz também as declarações de empréstimos de forma automática), consultar o saldo previsto de caixa, adicionar ou retirar dinheiro de caixa, entre outros. Adicionalmente é também possível gerir o stock do banco de materiais através da plataforma. A plataforma tem ainda diferentes níveis de acesso: direção, administração e geral.

A implementação do sistema dentro da equipa decorreu de forma bastante pacífica tendo-se, até ao momento, verificado muitos poucos erros por parte dos utilizadores gerais da plataforma. No entanto, o sistema encontra-se ainda em desenvolvimento tendo alguns bugs, principalmente no que toca à área de empréstimos em que o sistema gera multas diárias automáticas. Estes problemas fazem com que haja alguns erros no dinheiro previsto de caixa o que impede um controlo total das contas. Adicionalmente, o pelouro da Administração não se habitou logo a manter o stock de cada local sempre atualizado. Quando estes dois fatores estiverem resolvidos será possível verificar, de forma muito fácil, as vendas realizadas e as discrepâncias existentes.

Esta plataforma tem muito potencial para crescer, principalmente se coordenada com o site do Núcleo. Contudo é necessário, antes de mais, resolver os bugs existentes para que deixem de haver erros nos saldos de caixa o mais depressa possível. No futuro, seria interessante interligar as queixas logísticas bem como as queixas pedagógicas com o sistema, permitindo que se possa dizer o estado de resolução dos problemas e, assim, os estudantes saibam que os formulários existentes são úteis. Seria também interessante colocar um acesso à Drive Geral bem como aos restantes formulários na página inicial do interno. Outra sugestão que deixamos é a de mudar o domínio do site para my.neeec.pt. Por fim, seria também interessante adicionar locais e caixas dinâmicas permitindo, assim, inserir o cofre e caixas como caixas de vendas de jantar de curso no sistema.