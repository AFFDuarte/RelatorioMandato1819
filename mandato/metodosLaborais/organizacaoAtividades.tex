% =============================
% # Organização de Atividades #
% =============================

\subsection{Organização de Atividades}

A interação e interligação das funções dos diversos membros quer da Direção, quer dos CGs está inerente à organização de cada atividade do Núcleo. A projeção e idealização está a cargo de cada Pelouro com o auxílio do responsável da Direção por esse Pelouro. Após definição dos objetivos para cada atividade, o CG deve estipular com o Secretário a calendarização e registos de informações relevantes bem como a abertura de inscrições, reunir com o Tesoureiro para planeamento do orçamento e custo por participante e reunir com o Administrador para eventuais compras e organização logística necessária para fazer uso do material do Núcleo. O estabelecimento de todos os contactos deve ser realizado via Slack, de modo a assegurar o registo apropriado e consequentemente proporcionar uma melhor organização da atividade. Deve ser também feito um pedido de imagem, no formulário respetivo, para que o Pelouro da Imagem e para que os responsáveis pela Comunicação possam articular o seu trabalho e divulgar o evento com a devida antecedência. Por todas estas questões, é indispensável uma boa articulação e gestão de toda a equipa. Os Colaboradores têm como papel a organização e execução de todas as tarefas no decorrer da atividade, sendo que todos os membros do Núcleo se devem disponibilizar para auxílios pontuais. Toda a articulação na organização das iniciativas é orientada pelo Presidente do \acrshort{neeec}. É também importante a criação e divulgação prévia e atempada das escalas necessárias para a execução da atividade quando o staff do Pelouro em questão não tem capacidade para, sozinhos, assegurarem o evento.

Após a concretização da atividade, o CG do Pelouro é também responsável pela emissão de certificados e justificação de faltas, caso se aplique, e a arquivar as fotografias registadas no evento para uso posterior. Os responsáveis da Comunicação devem também fazer a divulgação do evento através de uma foto ou outra ou de um álbum, caso se trate de um evento de âmbito mais social.