% ===========================
% # Organização do PC       #
% ===========================

\subsection{Organização do PC}

No início deste mandato fomos presenteados com um computador oferecido pela HP, em forma de patrocínio ao \acrshort{ene3}, que permitiu ter uma base de trabalho no Núcleo disponível a todos. Posteriormente, este computador foi reforçado com um disco SSD de 275 GB pelo que é um computador bastante fluído e de muito boa qualidade para as necessidades do Núcleo. Neste computador foram criadas duas contas: uma livre e uma protegida por senha que só a Direção possui. Em ambas foram instalados todos os programas utilizados pelo Núcleo, nomeadamente os softwares de imagem, as drives e o Office. Além disso, no explorador de Internet, foram inseridos os marcadores necessários para aceder a sites muito utilizados para o funcionamento do Núcleo. A diferença entre as duas contas reside no facto dos dados mais confidencias do \acrshort{neeec}, nomeadamente os da Direção, só estarem acessíveis na conta protegida e de os dados do explorador de Internet na conta pública serem esquecidos assim que este é encerrado, evitando contas abertas esquecidas. Inicialmente havia apenas uma conta protegida mas onde toda a gente do Núcleo tinha a senha pelo que esta medida teve de ser implementada após se verificar que qualquer pessoa, nomeadamente membros externos ao núcleo, estava a aceder a dados que não devia.