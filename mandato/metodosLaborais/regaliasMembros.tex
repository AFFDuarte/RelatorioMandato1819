% ==========================
% # Regalias dos Membros   #
% ==========================

\subsection{Regalias dos Membros}

Cada um dos Pelouros do \acrshort{neeec} tem uma equipa atribuída que, em conjunto, permitem o desenvolvimento do trabalho do \acrshort{neeec} avançar. Além das funções de cada Pelouro, toda a equipa colabora nos eventos de âmbito geral do Núcleo ou em atividades de maior dimensão como foi o caso do Encontro Nacional de Estudantes de Engenharia Eletrotécnica ou do Bot Olympics.

Desta forma, e por sugestão de antigos dirigentes da casa, a Direção do \acrshort{neeec}, tentou estudar a possibilidade de atribuir algumas regalias, principalmente aos Colaboradores que não têm qualquer tipo de benesse no seu trabalho, fazendo com que sintam que o seu trabalho seja compensado. É ainda de realçar que muitos dos Colaboradores, dado o seu trabalho, acabam por assumir funções de elevada responsabilidade e carga horária pela qualidade do trabalho que prestam.

Foram solicitados suplementos ao diploma para a organização, staff e participantes de atividades de maior dimensão tais como o Bot Olympics, o Encontro Nacional de Estudantes de Engenharia Eletrotécnica e a Feira de Emprego e Empreendedorismo. Contudo, fica em falta o reconhecimento ao trabalho geral de todo o mandato e dos pelouros em si.

Assim, após análise em reunião, a proposta da Direção do \acrshort{neeec} foi criar uma classificação que tentou, de forma o mais justa possível, identificar quem é de facto ou não merecedor de algum tipo de benesse e em que medida. Para tal, selecionaram-se vários critérios, abaixo descritos, de forma a avaliar o trabalho de todos os membros:
\begin{itemize}
\item Presenças na escala do Núcleo (30\%)\\
Desde setembro de 2017, a sala do Núcleo dispõe de um horário de atendimento, durante o período de aulas, entre as 10h e as 17h com encerramento para almoço entre as 13h e as 14h, com exceção das quartas-feiras à tarde e das sextas-feiras de manhã em que o Núcleo se encontra encerrado. Desta forma, todos os Coordenadores e Colaboradores devem ocupar um turno ou dois, respetivamente, de uma hora em cada semana. As presenças do mesmo são registadas em folha de presenças afixada na sala do Núcleo pelo Secretário. Quem não cumpriu um turno na escala tem falta e todas as semanas a escala pode ser alterada sendo que o Secretário só ao domingo à noite imprime a folha de presenças já com os nomes da escala dessa semana. Todos os Colaboradores que faltem à escala justificadamente têm a sua presença justificada desde que indiquem o motivo ao Secretário e este seja considerado válido. Os turnos que se sobrepõe a exames ou dias festivos (por exemplo, após a serenata da Queima das Fitas) não contam para as presenças.
A nota atribuída a este parâmetro é diretamente proporcional ao número de presenças na escala (Em 28 turnos, uma pessoa que tenha vindo a 14, terá 15\% nesta avaliação enquanto que quem veio aos 28 turnos terá 30\%).
\item Presenças em escalas de eventos (10\%)\\
Ao longo do mandato existem vários eventos gerais do Núcleo e eventos cuja capacidade logística transcende a capacidade do Pelouro que a organiza. Desta forma, é necessária a criação de escalas para esses eventos de forma a os organizar. Para este ponto foram analisadas as seguintes escalas: banca na semana das matrículas, dia da receção ao caloiro, \acrshort{f3e}, \acrshort{neeec} Open Day, Lanche Solidário, Mega Febrada do Polo 2, Mega Febrada Polo 2, Venda do Jantar de Curso, Noite de Fados, Visita à Ubiwhere, Bot Olympics, Nomeações dos Ohms D’Ouro, Votações dos Ohms D’Ouro, Ultra Gaming Fest, BeerOlympics Eliminatória, BeerOlympics Final e Peddy Tascas. Não foram considerados eventos onde as escolas foram restritas a apenas algumas pessoas (por exemplo, a Barraca da Festa das Latas que foi restrita a Coordenadores e quatro Colaboradores que foram convidados para tal). Foi também decidido dar-se uma percentagem de apenas 10\% a este aspeto uma vez que muitos eventos calham nos mesmos dias da semana e, dessa forma, as pessoas que não podem comparecer num também não poderão comparecer noutro que se realize no mesmo dia da semana.\\
A nota atribuída a este parâmetro é de 10\% para todas as pessoas que tenham feito 12 ou mais turnos por semestre, sendo a partir daí para baixo proporcional ao número de presenças.
\item Avaliação Individual (40\%)\\
Esta avaliação dada por cada Coordenador de Pelouro aos seus Colaboradores e dada pela Direção aos Coordenadores, pretende avaliar o trabalho dos mesmos para o bom funcionamento do Pelouro e para a qualidade do trabalho nele desenvolvido. Os CGs são avaliados pela Direção tendo em conta o seu desempenho na coordenação dos respetivos pelouros e nos resultados alcançados, no seu trabalho para o intuito do Pelouro e no respeito pelas formas de trabalho estabelecidas para a equipa no seu global.\\
A avaliação é feita numa métrica de 0\%, 20\% ou 40\%, respeitando assim três níveis. Caso um Pelouro tenha funcionado notoriamente mal, a Direção poderá intervir, alterando a classificação dada aos colaborados do Pelouro em questão.
\item Proatividade na comunicação (10\%)\\
Sendo a comunicação um dos pilares mais essenciais do Núcleo, entendemos que este deve ser também avaliado. Assim, os responsáveis da comunicação devem avaliar quem está sempre a colaborar na divulgação das atividades do Núcleo, quem o faz de vem em quando ou quem nunca o faz (quer através de passar a mensagem boca-a-boca, quer através da divulgação nas redes sociais, etc). Serão valorizadas as pessoas que são mais independentes nesta área em vez das pessoas que têm de ser constantemente chateadas para colaborarem neste campo.\\
A avaliação é feita numa métrica de 0\%, 5\% ou 10\%, respeitando assim três níveis.
\item Avaliação bónus (10\%)\\
Havendo vários aspetos não avaliados nos pontos anteriores, nomeadamente a proatividade das pessoas, a presença das mesmas para ajudar em atividades gerais como a remodelação dos espaços de estudo, a montagem das decorações de Natal, entre outros, a presença em reuniões, a proatividade na sugestão de ideias, a colaboração para o conteúdo do site do Núcleo, entre muitas outras, a Direção valoriza todos os membros que têm espírito de iniciativa e disponibilidade tentando assim avaliar os pontos que não foram avaliados nos pontos anteriores.\\
A avaliação é feita numa métrica de 0\%, 5\% ou 10\%, respeitando assim três níveis.
\end{itemize}
Tendo em conta os critérios de avaliação aqui descritos, as benesses pensadas pela Direção do \acrshort{neeec} foram as seguintes:
\begin{itemize}
	\item As avaliações inferiores a 50\% não têm qualquer tipo de benesse.
	\item As avaliações entre 51\% e 79\% têm os seguintes direitos:
      \begin{itemize}
      \item Menção na \acrshort{rga}
      \item Certificado de participação ativa assinado pelo \acrshort{neeec} e pelo \acrshort{deec}
      \end{itemize}
 	\item Avaliações superiores a 80\%:
      \begin{itemize}
      \item Menção na \acrshort{rga}
      \item Certificado de participação ativa assinado pelo \acrshort{neeec} e pelo \acrshort{deec}
      \item Inscrições nas atividades gratuitas (workshops)
      \item Cartas de recomendação (individuais e especializadas)
      \item Potenciais suplementos ao diploma
      \end{itemize}
\end{itemize}

Com estas avaliações, pretendeu-se atingir o patamar justo para todos os membros não prejudicando quem não colaborou da forma que devia para o mandato, uma vez que não deixamos de ser uma organização sem fins lucrativos, não profissional e feita de estudantes cuja principal atividade não é esta, mas valorizando todos aqueles que despenderam muito do seu tempo para trabalhar em prol desta casa.

É de notar que esta classificação foi positiva contudo as percentagens podem não ser as mais justas. No final do ano, o que verificámos é que a larga maioria dos casos foi extremamente justa tendo, no entanto, existido casos pontuais em que o resultado pode não ter sido o mais adequado, nomeadamente com a justificação de falta de preenchimentos de escala a quem tinha sobreposição de horários, fazendo com a classificação desses membros subisse de forma galopante comparando com os restantes. De notar que, de forma a não haver qualquer injustiça sobre os diversos casos, não houve uma única nota alterada, mesmo os casos que estavam perto da transição para o nível seguinte. De notar que as notas não foram divulgadas publicamente (por exemplo no canal geral do Slack) para evitar que se tratasse o Núcleo como se fosse uma cadeira mas, quem quis, pode saber as suas avaliações individuais.