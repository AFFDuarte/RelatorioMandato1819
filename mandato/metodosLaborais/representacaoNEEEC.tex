\subsection{Representação do NEEEC/AAC}

Ao longo do mandato é frequentemente necessário representar o \acrshort{neeec} em ações de carácter diverso. Uma vez que pertencemos à \acrfull{aac}, a capa e batina deve ser utilizada sempre que se adeque, não só pelo Presidente e Vice-Presidente do Núcleo, mas por todos os que o representam (como aconteceu, por exemplo, nos 20 anos do Núcleo e na manifestação "Basta"). Apesar de no nosso Núcleo não ser habitual, existem vários Núcleos que mesmo em Reuniões Gerais de Alunos, a Mesa do Plenário bem como a Direção se apresentam de capa e batina. Em outras atividades como cerimónias de abertura de eventos é também essencial a apresentação de capa e batina. Este é um hábito que, principalmente os membros fora da Presidência, preferem abdicar o que, no nosso ver, é completamente errado e desprestigia o símbolo da Academia que representamos. Noutros casos como a visita a empresas, reuniões, etc., deve ser adotado um visual correto podendo-se fazer uso do merchandising do \acrshort{neeec} para tal, de forma a promover a imagem do mesmo.

É também importante manter o \acrshort{neeec} representando em todas as situações de relevo para as entidades com que se relacione para que possam ser estreitados os relacionamentos com essas entidades, como aconteceu, por exemplo, na Tomada de Posse do Diretor do DEEC onde a Direção do \acrshort{neeec} esteve presente, em peso, de capa e batina, como sinal de respeito.