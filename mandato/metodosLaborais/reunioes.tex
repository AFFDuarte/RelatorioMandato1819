% ==========================
% # Reuniões               #
% ==========================

\subsection{Reuniões}

\paragraph{Funcionamento}

A realização de reuniões entre os diversos elementos do \acrshort{neeec} assume especial importância na dinâmica de trabalho da estrutura associativa. As reuniões de cada equipa têm um âmbito diferente e, como tal, diferentes assuntos são abordados. As reuniões de membros da Direção foram realizadas semanalmente nas quais foram debatidos assuntos de gestão interna, Tesouraria, Administração, estratégias para a área comercial, representação externa e distribuição de trabalho por cada elemento. As matérias definidas ou acordadas nestas reuniões foram posteriormente expostas nas reuniões de \acrfullpl{cgs}.

Os Coordenadores Gerais, em conjunto com a Direção, reuniram mensalmente, salvo exceções devidamente justificadas. Estas reuniões abordaram informações que a Direção tivesse a dar, uma análise do plano de atividades, um balanço das atividades desenvolvidas pelos diversos pelouros, uma análise dos eventos futuros e discussão de outros assuntos relevantes.

Tendo como foco o desenvolvimento e planeamento de atividades, cada CG reunia, em reunião de Pelouro, com os seus Colaboradores, estando a metodologia, frequência e ordem de trabalhos das reuniões a cargo do respetivo CG.

Adicionalmente, foi ainda necessária a realização de várias reuniões individuais com cada CG quer em pontos chaves do mandato para balanço dos pelouros, quer em momentos em que o trabalho nos pelouros não corria da melhor forma e era necessário resolver a situação.

\paragraph{Conclusões}

As reuniões de Direção correram muito bem e, apesar da frequência das mesmas ter sido drasticamente superior ao que antigamente se fazia no \acrshort{neeec}, o facto de tal ter sido implementado desde o início fez com que a medida corresse bem e ocorresse durante o mandato inteiro. O facto de em período de férias e exames as reuniões não terem parado foi muito positivo pois fez com que não houvesse uma acumulação de assuntos. Por sua vez, após semanas em que não houve reunião (caso do Natal e da Páscoa, por exemplo), os assuntos acumulados provocaram reuniões pouco produtivas com durações superiores a 5 horas. Foi também necessário fazer reuniões extraordinárias para preparar outro tipo de reuniões como reuniões com a Direção do \acrshort{deec} ou reuniões fulcrais do Polo 2.

As reuniões de Coordenadores Gerais decorreram quase todos os meses. Contudo, o facto de haver alguns Coordenadores pouco habituados a este tipo de trabalho e/ou desligados fez com que o espírito crítico, em algumas destas reuniões, fosse fraco, algo que deve ser evitado no futuro. Estas reuniões, na nossa opinião, deveriam ter sido mais frequentes (preferencialmente semanais) para evitar reuniões tão longas e a continuação de assuntos tratados de forma rápida (ou seja, um assunto falado numa semana poderia ser concluído na semana seguinte sem problema enquanto que com reuniões mensais isso não era possível).

As reuniões de cada Pelouro foram livres e respeitaram o critério de cada CG, mas tiveram, sempre que possível, a presença de um membro da Direção na mesma. O facto de não haver nenhum limite mínimo para o número de reuniões e para a sua periodicidade fez com que houvesse pelouros extremamente discrepantes (sem justificação) o que se repercutiu na sua atividade e forma de trabalhar pelo que sugerimos que, no futuro, seja implementada uma medida que, não impedindo cada CG de ter a sua forma de trabalho, imponha uma maior regularidade no trabalho de todos os pelouros.

\subsubsection{Reuniões Gerais}

Desde o início do mandato que a Direção sempre defendeu a realização de Reuniões Gerais para todos os membros do Núcleo. Durante o mandato anterior (2016/2017) apenas foi realizada uma primeira Reunião Geral, logo no primeiro dia de aulas dos caloiros, e uma última antes da destomada de posse dos antigos corpos gerentes, que serviu como reunião de rescaldo do mandato.

Apesar destas reuniões serem importantes, achámos que a sua realização era extremamente escassa. Uma vez que, após os primeiros meses de mandato, sentimos que não existia total troca de informação entre os tópicos decididos nas reuniões de Coordenadores Gerais com os Colaboradores de cada Pelouro (por outras palavras, alguns Coordenadores não faziam passagem de informação com o resto das suas equipas) o que, na nossa opinião, criava bastante entropia no trabalho do Núcleo, devido às várias reformas internas que fizemos durante este ano, no nosso mandato realizámos 6 Reuniões Gerais, de forma a tentar ao máximo colmatar estas falhas anteriormente referidas e envolver ao máximo a equipa no Núcleo e nas suas atividades.

Tivemos uma reunião em junho, em setembro, em dezembro, em fevereiro, em abril e a última em junho.

A reunião de junho decorreu na sala do núcleo e serviu para dar a conhecer a nova disposição da sala e as regras de funcionamento da mesma. Realizaram-se também algumas atividades de teambuilding, sendo que cada membro devia indicar o que era, para si, o Núcleo.

A reunião de setembro ocorreu na sala de convívio. A maioria dos elementos presentes estavam de pé, pois não existiam cadeiras para todos se sentarem, o que se tornou muito maçador e chato. Nessa noite tirámos também a primeira foto de equipa.

A reunião de dezembro foi durante a hora de almoço e ocorreu, novamente, na sala do Núcleo, cuja ordem de trabalhos foi a retrospetiva das atividades do 1º semestre. Desaconselhamos a utilização desta sala para este fim, pois é pequena. Desaconselhamos também a realização da mesma à hora do almoço dado que este momento de grande confusão nos corredores do \acrshort{deec} e maior parte dos membros têm aulas às 14h.

Devido à confusão das anteriores reuniões, a partir de fevereiro passámos a realizar as Reuniões Gerais numa sala de aula da torre T e dispusemos a sala ao estilo de uma Assembleia de Núcleos. Sentimos que esta mudança foi bastante positiva pois ninguém estava de costas para ninguém nem não havia tanta confusão na sala. O eco da sala é o único ponto negativo que temos a apontar, no entanto, quase não afetou o decorrer da reunião.

Recomendamos ainda o uso de algum tipo de apresentação (por exemplo, PowerPoint), pois obriga a uma preparação mais cuidada dessa mesma reunião por parte da Direção, o que beneficia notoriamente a produtividade da reunião, e melhora a captação da atenção dos participantes da reunião, que tendencialmente tendem a desligar nas reuniões quando não têm nada para acompanhar para além de apenas a pessoa que está a falar.