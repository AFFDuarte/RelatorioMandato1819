% ==========================
% # Gestão de caixa        #
% ==========================

\subsubsection{Gestão de Caixa}

Para além do cofre no \acrshort{neeec}, existe a caixa disponível para as vendas internas do Núcleo. Infelizmente, como esta é de acesso livre, ela tende a ser das maiores fontes do problema na gestão financeira do Núcleo. Esta caixa, até ao mandato anterior (2016/2017) tendia a ser um conjunto de várias caixas simples onde estavam guardadas algumas moedas para facilitar os trocos das pessoas, não havendo qualquer tipo de registo sobre o que era vendido, dependendo sempre da boa-fé das pessoas. Enquanto o único produto a ser vendido no \acrshort{neeec} era o café, era sempre possível verificar se os valores não estavam a ser aldrabados por uma contagem das cápsulas de café que ainda existiam em comparação com a contagem anterior. Contudo, mesmo nesta situação, bastava existirem cápsulas que estavam estragadas ou eventos que necessitavam de café para a contagem "descarrilar"\space e ser quase impossível manter um registo sobre o valor que devia existir. Além disso, ao misturar estes valores com os valores em dívida que eram permitidos para os membros do Núcleo, esta tarefa tendia a complicar bastante.

Inicialmente, neste mandato, mantivemos essa estratégia, tendo apenas unido tudo numa única caixa, dado que não era uma prioridade alterar isso, desde que as contagens não tivessem problemas. Contudo, como passámos a ter o frigorífico do Núcleo, começámos a vender também águas frescas e aproveitámos para vender algumas minis que tinham sobrado de eventos, pelo que as contagens simples se tornavam bastante tediosas e obrigava a um esforço redobrado para manter um registo quando alguns dos materiais desaparecia por causa de eventos. Para evitar essa situação, procurámos arranjar uma solução que permitisse as pessoas registarem o que tinham consumido e o método de pagamento, para além de permitir identificar movimentos de produtos utilizados em eventos. A solução inicial passou por uma folha de registos localizada por cima da caixa para que todas as pessoas vissem a sua existência e não se esquecessem de a preencher, contudo, mesmo assim, ainda demorou até que todas as pessoas se habituassem a este sistema, pelo que, apesar de reduzir significativamente os problemas que havia, acabou por criar outros, além do esforço que era analisar cada linha da folha de registos, ainda para mais preenchida à mão. Contudo, acreditávamos na altura que apostar num registo informático iria provocar ainda mais esquecimentos que o registo em folha, o que não iria resolver nenhum dos problemas que existiam. Mais tarde, o sistema mudou para um registo informático online (interno.neeec.pt), que, dado que as pessoas já estavam habituadas ao registo na folha, não provocou mais esquecimentos que os que já haviam com a folha. Supostamente esta plataforma facilitará bastante a gestão quando todas as funcionalidades pedidas para a parte de contabilidade estiverem concluídas, o que ainda não aconteceu, pelo que ainda mantém o mesmo nível de trabalho que a folha de registos tinha.