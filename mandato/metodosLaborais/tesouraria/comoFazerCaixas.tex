% ==========================
% # Como fazer caixas      #
% ==========================

\subsubsection{Como Fazer Caixas para Atividades}

Fazer caixa de trocos para as atividades do Núcleo é algo relativamente simples de fazer, mas que nem toda a gente é capaz, infelizmente. Esta deve ser pensada atempadamente para evitar confusões que ocorrem quando esta é feita à pressa e deve ser feita em função do preçário que esse evento terá. Por exemplo, se os preços a cobrar na atividade nunca envolvem moedas de 1, 2 e 5 cêntimos, estas serão desnecessárias para a caixa, exceto se existirem poucas moedas de 10 cêntimos, que são facilmente compensadas por estas, contudo será de evitar, pois cria um peso desnecessário na caixa.

Outra forma de analisar o precário, é pensar nas situações mais comuns de trocos que serão necessários. Por exemplo, numa febrada, as febras poderão ser vendidas a 1,20€, pelo que é frequente as pessoas pagarem os seguintes montantes: 1,20€; 1,50€; 2€; 2,20€; 5€, 10€. Desta forma, serão necessárias várias moedas de 10 e 20 cêntimos para quase todos os casos, de 50 cêntimos para o segundo e terceiro casos, de 1 euro para o quarto caso e assim sucessivamente. Ainda neste exemplo, como às vezes acontece não existirem moedas de 10 cêntimos suficientes, pode ser necessário ajustar o preço para 1,30€, evitando problemas com trocos.

Regra de ouro a fazer caixas: os montantes pequenos somados dão montantes grandes, contudo os montantes grandes não podem ser partidos para os montantes mais pequenos.