% ==========================
% # Contabilidade          #
% ==========================

\subsubsection{Contabilidade}

A contabilidade do Núcleo é a tarefa principal do Tesoureiro do Núcleo, que terá que garantir que as várias regras impostas pela Tesouraria da \acrshort{aac} são cumpridas. Estas regras nem sempre são simples, mas a principal razão destas existirem é para evitar abusos por parte de pessoas que controlam o Núcleo num certo mandato e que podem comprometer todo o trabalho feito em anos anteriores, como aconteceu a meio da história deste Núcleo e que apenas muito recentemente ficou resolvido.

A gestão contabilística do Núcleo é bastante facilitada pelo \acrshort{ctp}, principalmente por causa do fecho de contas mensal obrigatório (ver secção \ref{subsubsec:tesourariaAAC}). Contudo, é necessário manter uma gestão criteriosa desses valores para que estes possam ser apresentados, pelo que é necessário manter algum registo de todas as despesas e receitas do Núcleo, algo que não tem sido regra neste Núcleo, em mandatos anteriores, impedindo, por exemplo, que futuros mandatos possam realizar orçamentos baseados nos históricos das atividades. Desta forma, o Tesoureiro procurou realizar um registo criterioso de todos os movimentos do Núcleo através do Excel, mantendo um registo da data do movimento, a categoria, tipo e descrição dessa despesa, o montante, a pessoa responsável por esse movimento, a origem/destino do dinheiro e a fatura referente a esse movimento, caso fosse aplicável. Aliado ao facto de manter uma digitalização de todas as faturas que passaram pelo Núcleo, esta gestão permitiu manter um registo sólido do mandato. Contudo, era também interessante uma gestão de todos os movimentos contabilísticos do Núcleo, o que inclui um registo dos movimentos entre o cofre e a conta bancária, um pormenor que já não foi possível realizar neste mandato por excesso de trabalho do Tesoureiro na gestão das atividades em que este se encontrava envolvido. Este registo é importante pois facilmente permite detetar a localização de erros que possam existir no registo contabilístico do Núcleo, além de garantir que a gestão financeira do Núcleo é plena. Esta tarefa já existe com o fecho de contas mensal no \acrshort{ctp}, contudo, a existência do saco azul originou sempre vários problemas, pelo que foi uma das bandeiras do mandato terminar definitivamente com a existência deste saco azul.

\ifthenelse{\boolean{biblia}}
{ % TRUE
Apesar de se ter procurado realizar uma gestão o mais criteriosa quanto possível, foram aparecendo várias discrepâncias ao longo do mandato, cujas origens nunca foram possíveis de detetar. Contudo, dado que as mesmas ocorreram quase sempre após um período de esforço adicional na gestão das atividades em que o Tesoureiro se encontrava, nomeadamente o \acrshort{ene3} e o Bot Olympics, que o forçou a colocar as tarefas de Tesouraria para segundo plano, recomendamos, assim, que o Tesoureiro, exceto em situações muito pontuais e que não obriguem a muito esforço, não se comprometa com mais tarefas nas atividades do Núcleo, mantendo apenas a sua presença enquanto Tesoureiro dessas atividades.
}
{ % FALSE
}