% ==========================
% # Montepio Geral         #
% ==========================

\subsubsection{Montepio Geral}

A \acrshort{aac} tem há alguns anos um acordo com a Caixa Económica Montepio Geral, pelo que todas as contas da \acrshort{aac} (secções, núcleos de estudantes, Queima das Fitas, etc.) têm uma conta neste banco, sendo as estruturas impedidas de abrir conta noutro banco, sendo mesmo impedidas de receber qualquer tipo de apoio doutras instituições bancárias, mesmo que estes apoios sejam solidários, por exemplo. A conta do \acrshort{neeec} tem quatro titulares da conta: o Presidente e Tesoureiro do Núcleo e o Presidente e Administrador da \acrshort{dg}. Sendo assim, sempre que um destes titulares precisa de ser alterado (novos órgãos dirigentes de cada uma das estruturas, por exemplo), é cobrada uma taxa pelo banco.

A gestão da conta é bastante simples, estando disponível a interface web e a interface por telemóvel, sendo interfaces completamente naturais para este tipo de instituições. Cada titular tem o seu acesso personalizado e tem um cartão matriz que serve para confirmar as transações (estas são enviadas por correio e terão de ser levantadas no edifício da \acrshort{aac}). Todos os movimentos financeiros na conta do Montepio Geral têm que ser autorizados por dois dos titulares da conta (suspeitamos que esta regra se deva a abusos que tenham existido quando apenas um dos titulares pudesse fazer o que bem entendia), o que impede a existência de um cartão bancário de acesso direto à conta. No final do mandato, para evitar pagamentos através de contas de terceiros que seriam posteriormente transferidos para a conta dessa terceira entidade, pedimos o cartão pré-pago da conta (este teve um custo de 6€), que permite ser carregado com um valor até 2000€. Esse carregamento terá que ser aceite sempre por dois titulares da conta, pelo que evita (parcialmente) o problema da existência de cartão de acesso direto à conta.

Todos os meses o banco emite um extrato combinado da conta, que pode ser transferido através do site. Este extrato é o usado pela Susana do \acrshort{ctp} para verificar os valores da conta, pelo que cada movimento terá que ser justificado.