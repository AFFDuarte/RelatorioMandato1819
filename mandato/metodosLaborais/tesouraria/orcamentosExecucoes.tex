% ==========================
% # Orçamentos e Execuções #
% ==========================

\subsubsection{Orçamentos e Execuções}

A criação de orçamentos é uma das principais tarefas do Tesoureiro, permitindo estabelecer uma barreira quanto aos gastos que o evento possa ter, as receitas que é necessário atingir para esses mesmos gastos e o lucro da atividade. O lucro da atividade não precisa de ser sempre positivo, desde que haja a definição inicial de que tipo de atividades poderão, ou não, ter lucro/prejuízo em nome da sustentabilidade tanto da atividade como do Núcleo.

A criação de um orçamento, para qualquer tipo de atividade, é, provavelmente, das tarefas mais complexas, principalmente quando o histórico de gastos e receitas das atividades é inexistente ou manifestamente escasso. Nestes casos, é necessário ter alguma noção do que a atividade requer, ou não, e os custos ou receitas possíveis que é possível ter com esses requerimentos, o que é especialmente difícil quando não se sabe um valor aceitável para esses requerimentos. Para tal, é necessário conversar muito bem com o organizador da atividade para perceber o que vai ser realmente feito e este deve saber transmitir valores aceitáveis para esses requerimentos, contudo é necessário ter sempre alguma desconfiança (saudável) e realizar uma pesquisa própria sobre esses valores.

Durante este mandato, foram iniciados os passos para uma construção frequente de orçamentos, tendo sido realizados orçamentos principalmente para as atividades grandes do Núcleo, nomeadamente \acrshort{ene3}, Bot Olympics e Gala Ohms D'Ouro. Quando foi possível, foi feita uma pesquisa sobre os orçamentos que já existiam e as discrepâncias para os valores reais que apresentaram para que fosse possível discernir os valores a definir, pelo que se aconselha uma pesquisa por esses documentos feitos. Nos ficheiros das execuções financeiras dessas atividades, foi também ambição disponibilizar o valor da discrepância que se verificou para facilitar o trabalho futuro.