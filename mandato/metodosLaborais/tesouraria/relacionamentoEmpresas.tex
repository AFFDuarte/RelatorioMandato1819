% ===============================
% # Relacionamento com empresas #
% ===============================

\subsubsection{Relacionamento com Empresas}

As empresas com que lidamos normalmente têm métodos muito distintos de se relacionar connosco, pelo menos, a nível de Tesouraria. Enquanto que algumas empresas requerem que se envie a fatura do apoio por correio antes destas pagarem, outras basta apenas enviar uma digitalização dessa fatura por email. Chegou mesmo a acontecer, apesar de ter sido um caso único, ter havido uma empresa a pagar antes de ter sido sequer pedida a fatura. Neste caso, é necessário ter uma comunicação estreita com a pessoa que está a tratar dos patrocínios de cada evento, para que estas informações de como lidar com a empresa em questão não se perca, dado que acontece frequentemente as empresas mudarem o seu método de trabalho (aconteceu este mandato com a Critical Software/itGrow, que passou a requerer o envio por correio das faturas) e é preciso saber o que cada uma precisa.

Durante este mandato, houve duas formas de lidar com o envio de faturas para empresas: na maior parte das vezes, o processo de arranjar o patrocínio pertencia a uma pessoa e depois o processo de faturação caía todo sobre o Tesoureiro, enquanto que noutros casos, o Tesoureiro era apenas responsável por pedir a fatura que depois enviava à pessoa responsável pelo pedido de patrocínio que tratava de enviar para a empresa. 

Nota pessoal do Tesoureiro: preferi a primeira situação, porque permitiu manter um histórico de como lidar com cada empresa (existe mesmo um histórico dos modelos de pedidos de informação preenchidos pelas empresas na pasta da Tesouraria) e assim saber lidar diretamente com a empresa caso surgisse algum problema, que normalmente apenas surge numa fase final da preparação do evento ou até mesmo depois do evento e que a pessoa que fazia a ligação anteriormente poderá já não estar disponível para solucionar.

Um problema recorrente ao lidar com empresas é a confusão gerada pelo facto de todas as estruturas da \acrshort{aac} partilharem um \acrshort{nif} comum, principalmente nas empresas mais organizadas que têm um software de gestão de contabilidade e criam fichas de clientes que apenas permite criar um ficha por \acrshort{nif} associado. Desta forma, pode acontecer as empresas contactarem as pessoas erradas para resolver algum problema ou, mais frequentemente, transferirem os apoios monetários a atividades para outras estruturas da casa. Neste campo, são excelentes exemplos a EFAPEL que apoia a Secção de Basquetebol, transferindo sempre o dinheiro para estes, e a ITGrow que transferia sempre para o \acrshort{nei} o dinheiro e atualmente transfere sempre para nós, mesmo quando o dinheiro não é para nós. Nestas situações é necessário descobrir para onde o dinheiro foi transferido (por exemplo, colocar no multibanco o NIB de forma a saber o titular da conta para onde foi transferido o dinheiro) e contactar as estruturas em causa, solicitando a transferência do dinheiro para a conta bancária correta.