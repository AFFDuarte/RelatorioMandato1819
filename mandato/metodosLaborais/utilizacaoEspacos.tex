% ===========================
% # Utilização dos Espaços  #
% ===========================

\subsection{Utilização dos Espaços}

Para o normal decorrer das suas atividades, quer internas, quer externas, o \acrshort{neeec} utiliza variadíssimos espaços do Departamento. No início do mandato foi feita uma apresentação à Direção do Departamento para a necessidade de utilização dos espaços, algo que fez com que não fosse necessário estar sempre a solicitar autorização para o aluguer de salas. A Direção do \acrshort{deec} entendeu, também, dar acesso direto à sala de reuniões sendo, no entanto, necessário reservar a sua utilização através da Secretaria dado que esta sala é bastante utilizada por várias entidades dentro do Departamento. Para as restantes salas, basta o \acrshort{neeec} reservar as mesmas junto da Secretaria do \acrshort{deec} devendo levantar a chave antes dos eventos e devolvê-la logo a seguir. É de notar que frequentes são os casos que envolvem confusões com chaves devido a desorganizações que fazem com que estas não sejam devolvidas de imediato. Este deve ser um problema a evitar, de todo, pois a Direção do \acrshort{deec} detém toda a informação sobre as chaves que desaparecem e quem as utilizou, informação essa que pode quebrar a confiança existente entre a Direção do Departamento e o \acrshort{neeec}, de forma absolutamente desnecessária.

Quanto à utilização de espaços para febradas como a entrada do \acrshort{deec}, a esplanada do bar ou a esplanada do Núcleo, deve ser sempre feito um pedido de utilização do espaço à Direção do \acrshort{deec} e, se aplicável, ao Sr. Vítor.

Por consequência e uma vez que o Núcleo é frequentemente associado, erroneamente, a tudo o que diz respeito aos estudantes, alguns grupos de alunos não organizados, nomeadamente os carros da Queima das Fitas, recorrem ao Núcleo para ter acesso a salas sem pedirem autorização a quem devem e de forma a poder ter reuniões marcadas em cima do acontecimento. Desta forma, a posição do Núcleo tem sido sempre a de não emprestar as chaves de que dispõe para nada que não seja marcado atempadamente junto da Direção ou da Secretaria do Departamento.

A sala do Núcleo é um espaço que não é autorizado para qualquer tipo de reunião que não diga respeito ao mesmo. A sala de convívio e a sala de estudo T.4.2 têm sido também frequentemente utilizadas para acontecimentos deste tipo. A sala de convívio, sendo um espaço público, não pode ser vedada a ninguém mas deve-se sempre ressalvar a não perturbação do espaço para o seu princípio básico. Já a sala de estudo não pode ser fechada por um grupo de alunos, sem autorização da Direção do \acrshort{neeec} e da Direção do \acrshort{deec} estando em vigor um regulamento que permite a proibição do acesso a esta sala para todos os que não cumpram o regulamento.

Quanto à reserva de espaços internamente, a sala do Núcleo deve ser reservada ao Secretário sendo que, nos momentos em que esta se encontre reservada para reuniões, a escala do Núcleo é automaticamente suspensa. O Secretário marca no Google Calendar interno do Núcleo a reserva da sala pelo que se aplica a regra do "primeiro a chegar, primeiro a reservar". Adicionalmente, sempre que seja necessário utilizar a sala de reuniões do \acrshort{deec}, deve ser informado o Secretário que fará a reserva da sala. Quanto aos espaços para eventos, os membros do Núcleo devem também informar o Secretário da necessidade de reserva de espaços.

Algo que aconselhamos no futuro é a expressa indicação aos carros da Queima das Fitas da interdição de reuniões na sala de convívio bem como na sala de estudo, indicando como devem proceder para a marcação de reuniões e reserva de espaços. Internamente, aconselhamos a uma centralização das marcações de espaços junto do Secretário do \acrshort{neeec} e uma indicação, junto da Secretaria e da Direção do \acrshort{deec}, de quem pode ou não reservar salas em nome do Núcleo de forma a centralizar a informação sobre quais os espaços reservados em nome do \acrshort{neeec} e quais as chaves levantadas.