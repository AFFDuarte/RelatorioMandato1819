% ==========================
% # Máquina de Café Piso 6 #
% ==========================

\subsection{Máquina de Café Piso 6}

Ao tomarmos posse, tivemos conhecimento de um contrato celebrado pelo \acrshort{neeec} a 1 de janeiro de 2015 cujo término seria  31 de dezembro de 2017, com renovação automática, para a exploração de uma máquina de vending na sala de estudo do Piso 6. Soubemos também que a máquina se encontrava sem funcionar há alguns meses e que tal se devia ao não cumprimento do horário estipulado para a mesma. Por sua vez, no contrato era dito que a empresa exploradora pagaria ao \acrshort{neeec} 10\% das receitas provenientes da mesma, algo que foi feito apenas em alguns meses do primeiro ano de contrato.

Desta forma, a Direção do \acrshort{neeec} entrou de imediato em contacto com a Direção do \acrshort{deec} para saber a proveniência e as condições do contrato ficando a saber que a máquina estaria sob alçada do \acrshort{neeec} pois não é permitido ao Departamento celebrar contratos deste tipo sem passar pela universidade que detém um contrato único para todas as máquinas de vending da \acrshort{uc}. Outra das exigências era que, dada a proximidade da máquina com o bar do Sr. Vítor, a mesma só poderia funcionar quando o bar está encerrado, ou seja não poderia funcionar das 8h às 19h de dias úteis. Assim, uma vez que o contrato não era cumprido pelo primeiro outorgante, o \acrshort{neeec} cessou, de imediato, o contrato com a entidade tendo a máquina sido retirada, pela empresa, durante o mês de agosto.

No âmbito das remodelações da sala de estudo, era essencial a existência de uma máquina deste tipo uma vez que não há nenhuma estabelecimento no Polo 2 aberto durante a noite. Assim, o Administrador do \acrshort{neeec} contactou várias empresas de vending tendo selecionado a empresa ParkVending para celebrar contrato. Foi então celebrado um contrato de 4 anos com a empresa em questão, nas mesmas condições que a anterior (horário de funcionamento restrito e percentagem de 10\% paga pela empresa ao \acrshort{neeec}, mensalmente). A empresa dispõe de um sistema informático ligado à máquina o que permite ao \acrshort{neeec}, acedendo a um site online, obter informação sobre todas as vendas feitas na máquina e emitir automaticamente, no final de cada mês, fatura referente à percentagem que lhe é devida. Adicionalmente o \acrshort{neeec} tem também as chaves da máquina para que seja possível ajustar o relógio da mesma ou encher o depósito de água. Por sua vez, a empresa tem uma chave para aceder ao piso 6 fora do horário de funcionamento do Departamento. De realçar também que, para os alunos entenderem o motivo da máquina estar desligada durante o dia, foi colocado um aviso informando do horário de funcionamento da mesma tendo-se reduzido as questões sobre o porquê da máquina não estar sempre ligada.