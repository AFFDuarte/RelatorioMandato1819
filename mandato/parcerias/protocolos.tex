% ==========================
% # Protocolos             #
% ==========================

\subsection{Protocolos}

No presente mandato deparámo-nos com a existência de vários protocolos celebrados pelo \acrshort{neeec} em tempos e que não eram comunicados à comunidade estudantil. Reparámos também que os mesmos tinham, por norma, a duração de um ano e não eram renovados.

Desta forma, optámos por assinar novos protocolos com várias entidades, algumas repetentes outras novas de todo. Decidimos também não renovar alguns protocolos pelos motivos que referimos a seguir. Os protocolos foram tratados em conjunto pelo Presidente e pelo Vice-Presidente do Núcleo, mas, na nossa opinião, este deve ser um assunto tratado, no futuro, pelo Administrador do Núcleo podendo ter a colaboração dos pelouros (por exemplo, o desporto poderia querer ter um protocolo com uma escola de surf). No entanto, achamos que os Coordenadores não deverão tratar dos protocolos sozinhos pois estes não têm conhecimentos de toda a organização financeira e logística do \acrshort{neeec} como o Administrador deve ter.

De forma a divulgar as parcerias existentes, criámos uma secção no nosso site onde é possível consultar informação sobre a mesma. Nesta secção foi inserida informação sobre a data do término da parceria para que, caso o mandato em vigor do Núcleo não se recorde de tirar a informação do site, os estudantes saibam facilmente que se trata de um erro e que, como tal, o protocolo já não está em vigor. De realçar que tal não foi suficiente para divulgar os protocolos, sendo necessário, no futuro, adotar mais estratégias, não retirando a atual divulgação e organização no site, que achamos essencial.

De forma a organizar todos os protocolos, criámos também um dossiê só sobre este assunto, acrescido de contratos e garantias de produtos para que os mesmos não estejam associados aos mandatos mas sim a documentos em vigor e sejam, assim, de fácil consulta no armário da Direção.

Passamos agora a analisar os vários protocolos estabelecidos no presente mandato:
\begin{itemize}
\item Escola de Condução Universidade e Escola de Condução Taveiro\\
\label{parcerias_escola}
Este protocolo foi assinado em maio de 2017, aquando do anterior do mandato, mas já com a presença da nova Direção. O protocolo era entre a empresa e os vários núcleos do polo 2, tratando de todos os núcleos da mesma forma e de uma só vez. O principal objetivo da empresa era ter a sua marca presente na receção ao caloiro, cedendo para isso sacos para os kits de receção. Oferecia também a todos os estudantes que se inscrevessem ao abrigo deste protocolo 10\% de desconto imediato na carta, oferta do kit de formação e oferta de outras campanhas que pudessem vir a ser lançadas. Por sua vez, o Núcleo, além dos sacos, receberia 25€ por cada inscrição angariada. Nos sacos dos kits de caloiro, seria oferecido um panfleto com todas as condições da campanha e o \acrshort{neeec} deveria publicar mensalmente no Facebook as campanhas lançadas pela escola e afixar no Departamento um cartaz A0. Da parte da escola, o \acrshort{neeec} nunca recebeu nenhum cartaz nem nenhuma informação sobre as publicações a fazer no seu Facebook nem nenhum material a afixar no Departamento. O \acrshort{neeec} criou um documento que os alunos deveriam levantar no Núcleo para que pudessem ter o desconto. Desta forma, o \acrshort{neeec} saberia quantos estudantes tinham-se inscrito na escola através do Núcleo, contudo, não nos é garantido que a escola não pratica aos estudantes as mesmas condições promocionais sem os mesmos apresentarem nenhum documento. O \acrshort{neeec} apenas teve conhecimento de um aluno que se inscreveu na escola através desta parceria não tendo, até à data deste documento, recebido qualquer valor monetário sobre o mesmo, mesmo após o estabelecimento de vários contactos com a escola de condução.\\
Na nossa opinião, o protocolo com uma escola de condução é de elevado interesse, mas deve ser adotado um protocolo com uma escola mais próxima do polo 2 e com maior seriedade no seu trabalho. Não recomendamos também qualquer estabelecimento de protocolos deste tipo (que envolvam documentos para saber quem aderiu, etc.) com outros núcleos do polo 2 pois as formas de trabalho e gestão de Secretaria dos vários núcleos são extremamente diferentes.
\item Curso Privado de Inglês\\
Esta escola entrou em contacto connosco para o estabelecimento de uma parceria de um ano onde eles oferecem 20\% de desconto nos seus cursos. Por sua vez, o \acrshort{neeec} teria de fazer divulgação da escola através dos habituais meios de divulgação. A escola ofereceu ainda uma pequena aula de inglês, atividade essa que foi realizada no \acrshort{deec} no final de outubro. Não temos noção de quantos estudantes poderão ter feito uso desta oferta, embora achamos que poderão ter sido poucos. No entanto achamos que este tipo de protocolos é proveitoso pois permitiu ter mais uma atividade sem encargos para o \acrshort{neeec} e permitiu à escola fazer alguma divulgação. No entanto, a divulgação poderia ter sido muito mais forte de ambas as partes. De realçar também que a universidade, através da FLUC, tem vários cursos de inglês que podem também ser divulgados pelo Núcleo.
\item Drag \& Print\\
Foi estabelecido um protocolo com esta empresa para que os estudantes do MiEEC pudessem ter condições especiais na utilização das máquinas tendo uma creditação de créditos suplementar aquando dos carregamentos. Esta parceria fez com que fosse necessário gerar um código diferente para cada sócio do \acrshort{neeec}, algo que não ofereceu grande trabalho. Contudo, o envio dos emails foi ainda na altura em que os emails do \acrshort{neeec} iam parar ao spam tendo sido prejudicial este fator para a divulgação da campanha. Adicionalmente, foram feitas publicações nas redes sociais.
Este acordo parece-nos muito vantajoso uma vez que a empresa nos pareceu extremamente profissional, contudo, a adesão dos estudantes foi muito escassa, não tendo o \acrshort{neeec} nunca obtido nenhum proveito financeiro da parceria.
\item Fitness Hut\\
Tentámos estabelecer um protocolo com este ginásio que oferecia condições especiais aos estudantes (as mesmas que eles oferecem em quase todas as campanhas em vigor) mas, no momento de assinar o contrato, pedimos algumas alterações ao mesmo, algo que o ginásio nunca chegou a fazer. Por sua vez, a \acrshort{uc} já tem um protocolo com o ginásio, pelo que os estudantes da mesma estão automaticamente beneficiados pelo mesmo, pelo que decidimos não insistir neste assunto.
\item Parcerias antigas\\
Decidimos não renovar a parceria com a Coimbra Stand Up Paddle por não ter havido uso da mesma por parte dos estudantes e por as condições não nos parecerem interessantes. Decidimos também não renovar a parceria com a \acrshort{tvaac} uma vez que esta parceria era apenas uma constatação óbvia daquilo que deveria ser obrigatório através da aplicação dos estatutos da \acrshort{aac} para o trabalho conjunto entre as diversas casas da \acrshort{aac}.
\item Protocolos da Rede \acrshort{uc}\\
A \acrshort{uc} tem protocolos com inúmeras entidades que beneficiam os seus estudantes, professores e funcionários. Estes protocolos fazem com que não seja necessário estabelecermos tantas parcerias bastando sim divulgá-las, pelo que recomendamos um trabalho mais forte disto no futuro.
\end{itemize}

No futuro, recomendamos uma continua aposta nos protocolos mas uma maior divulgação dos mesmos, nomeadamente na receção ao caloiro, tentando sempre remeter para o site do Núcleo onde é fácil atualizar a informação.