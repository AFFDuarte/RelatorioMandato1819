% ==========================
% # Académica Start UC     #
% ==========================

\subsubsection{Académica Start UC}

A nomeação do embaixador da \acrfull{asuc} começou a ser pensada pela Direção do \acrshort{neeec} desde o início do mandato, tendo, após alguns convites e aberta uma candidatura interna para os membros do Núcleo onde não existiu nenhum membro a autopropor-se para integrar este projeto, sido nomeado o Vice-Presidente, João Martins, como Embaixador da ASUC para o mandato 2018/2019, com o papel de promover o empreendedorismo junto dos estudantes do \acrshort{deec}.

Por sugestão do Coordenador Área da Política para o Empreendedorismo durante o evento \acrshort{aac} (in)Forma, pensou-se em integrar o evento Bot Olympics com a ASUC, no entanto uma vez que, na opinião do Embaixador, isso não iria trazer nada de inovador para a competição, a ideia não avançou e decidiu-se integrar a ASUC com a Semana dos Ramos, como acabou por acontecer.

É de notar que um dos eventos mais adequados à \acrshort{asuc} é a \acrshort{f3e} contudo, esta realiza-se antes do início do mandato dos Embaixadores da \acrshort{asuc}.

Foi através dos contactos da \acrshort{asuc} que conseguimos estabelecer contacto com a maioria dos oradores e formadores para o dia dedicado ao empreendedorismo da Semana dos Ramos, algo que, em retrospetiva, foi bastante positivo para a qualidade do evento.

A nomeação do Vice-Presidente enquanto Embaixador foi, na nossa opinião, uma decisão errada pois as funções que este detém não permitem que o mesmo não dedique o tempo necessário para que o \acrshort{neeec} seja devidamente representado no projeto.
