% ==========================
% # Queima das Fitas       #
% ==========================

\subsection{Queima das Fitas}

A Queima das Fitas é das atividades mais carismáticas da vida académica de Coimbra, sendo organizada pela \acrfull{coqf}, com supervisão da \acrshort{dg} e do Conselho de Veteranos da Universidade de Coimbra. Existem várias iniciativas que a \acrshort{coqf} promove que podem envolver Núcleos de Estudantes, além de que os lucros da Queima das Fitas são distribuídos para a \acrshort{aac}, que depois os distribui pelas várias estruturas internas de acordo com os regulamentos em vigor.

\paragraph{Atividades Patrocinadas}

No âmbito das atividades culturais, desportivas e tradicionais da Queima das Fitas, a \acrshort{coqf} disponibiliza várias candidaturas a apoios que poderão ser monetários, logísticos ou bilhetes para as Noites do Parque. Este ano foram abertas candidaturas a eventos desportivos e outras candidaturas a eventos culturais. Cada Núcleo de Estudantes poderia submeter-se a um evento desportivo e/ou a um evento cultural. Os prazos e condições de ambas as candidaturas são diferentes e estavam descritos em regulamento próprio colocado no site da Queima das Fitas e divulgado sem a antecedência devida, via e-mail, pelo que é importante estar atento ao site da Queima das Fitas para se saber quais os projetos abertos.

Este ano decidimos não nos candidatar a projetos desportivos, uma vez que não possuíamos qualquer tipo de evento, realizado apenas pelo \acrshort{neeec}, nesta altura do ano que pudesse ter prémios atribuídos em forma de bilhetes para as Noites do Parque da Queima das Fitas. Candidatámo-nos aos projetos culturais através do Peddy Tascas, o que permitiu oferecer um bilhete geral a cada elemento da equipa vencedora, um bilhete pontual a cada elemento da equipa que ficou em segundo lugar e dois bilhetes pontuais à equipa que ficou em terceiro lugar, sendo que das principais condicionantes à existência dos prémios era a inscrição de, pelo menos, 10 equipas.
\ifthenelse{\boolean{biblia}}{ % TRUE
Apesar de não termos conseguido esse número de equipas, divulgámos os resultados com equipas fictícias para permitir esses prémios, dado que os mesmos já tinham sido anunciados e apenas não foram anunciados com mais tempo por culpa da \acrshort{coqf}.}{}
Em conjunto com os restantes núcleos do Polo 2, candidatámos a Liga Polo 2 ao projeto desportivo (pedido feito pelo \acrshort{neemaac}), o BeerOlympics ao projeto cultural (pedido feito pelo \acrshort{neec}) e a \acrfull{ugf} também como projeto cultural (\acrshort{nei}). De notar que todos os pedidos foram aceites e é possível pedir vários projetos por Núcleo, tanto que o \acrshort{nei} teve bilhetes para outros eventos seus, apesar de se ter candidatado a projetos para eventos do Polo 2.

Deixamos agora algumas considerações sobre o processo de candidaturas:
\begin{itemize}
\item Logo após as candidaturas, é importante estabelecer um contacto como o Comissário responsável pela área (cultural ou desportiva) para que se saiba se a candidatura foi recebida e tentar pressionar uma resposta. A candidatura ao Peddy Tascas tinha ido parar ao spam pelo que, aquando da libertação de resultados o \acrshort{neeec} não obteve qualquer informação. Ao sabermos que outros Núcleos tinham obtido resposta aos seus projetos, através de chamada telefónica, explicámos a situação e obtivemos, de imediato, a confirmação da aceitação do projeto.
\item As respostas às candidaturas são feitas muito em cima das atividades, ou até após estas (este ano foram dadas a 13 de abril quando a \acrshort{ugf} começava precisamente nesse dia) o que pode prejudicar imenso a divulgação das atividades. É muito importante ter isto em conta pois não faz qualquer sentido os bilhetes não serem divulgados, algo que frequentemente acontece principalmente nas atividades do Polo 2.
\item Estes bilhetes são excelentes formas de dinamizar as atividades sendo que, no caso do Peddy Tascas, se registou uma enchente de inscrições como há muitos anos não era visto, algo que acreditamos dever-se aos prémios em jogo (apesar de ter causado outros problemas na atividade, como é explicado na secção \ref{subsubsec:atividades-cultura-peddy}).
\item O historial do \acrshort{neeec} mostra que estes projetos não têm sido aproveitados, tendo este sido um dos primeiros anos onde se aproveitaram os projetos para obter bilhetes para atividades internas do \acrshort{neeec} o que, na nossa opinião, é um péssimo desperdício dado o valor destes apoios pelo que recomendamos que, no futuro, sejam até submetidas mais candidaturas.
\item Quase todos os projetos com a estrutura 1-2-1 (1 bilhete geral para o vencedor, 2 pontuais para o 2º classificado e 1 pontual para o 3º) são, quase sempre, aprovados. Este ano, no evento a que nos candidatámos, não aproveitámos este facto, por não o sabermos, tendo apenas recebido 2 pontuais para o 3º classificado (apesar da equipa ter 5 elementos), pelo que recomendamos algum cuidado com esta condição no futuro.
\end{itemize}

\paragraph{Projetos da Queima das Fitas}

Os projetos da Queima das Fitas é uma iniciativa não associada à \acrshort{coqf}, mas sim ao \acrfull{cin}. Tal como já foi referido, os lucros da Queima das Fitas são distribuídos pelas várias estruturas da \acrshort{aac} e, no caso dos Núcleos de Estudantes, uma das parcelas que lhes é destinada pretendem apoiar monetariamente atividades dos Núcleos, principalmente atividades referentes às áreas de atividade principal dos Núcleos (saídas profissionais e pedagogia), que sem esse apoio poderiam ter elevado prejuízo apesar do valor acrescentado que trazem aos estudantes. Até há uns anos, este apoio servia como forma de sustento do Núcleo, dado que costumava trazer das maiores receitas anuais em atividades, principalmente nos anos em que a Queima das Fitas garantia bons resultados financeiros, contudo, dado que a atividade do Núcleo tem sido suportada bastante pelos patrocínios de empresas às atividades e dado que a distribuição desse dinheiro costuma demorar imenso tempo (estando, neste momento, ainda pendentes a distribuição de dinheiro referente à Queima das Fitas de 2014, 2015, 2016 e 2017), a importância deste dinheiro nas contas do Núcleo tem caído bastante.

Desde a criação do Bot Olympics que esta tem sido a atividade proposta pelo Núcleo como projeto da Queima das Fitas, contudo, dado que a Queima das Fitas tem apresentado lucros baixíssimos ou até mesmo prejuízo nos últimos anos associado à política deste mandato de terminar com o saco azul do Núcleo, fazendo com que o Bot Olympics apresentasse lucro perante a \acrshort{aac}, comparado aos anos anteriores em que apresentava prejuízo (apesar de, na realidade, apresentar lucros substancialmente mais elevados que este ano), prevemos que o apoio dado a esta atividade será muito reduzido, caso chegue a existir.

A submissão de inscrição da atividade como projeto da Queima das Fitas é realizada através do preenchimento de um formulário em Excel, que nos foi disponibilizado pela Coordenadora dos Núcleos da \acrshort{dg}, Beatriz Banaco, no início de maio. Um dos problemas que existiu com o preenchimento deste formulário foi sobre a secção do orçamento da atividade dado que a atividade já tinha sido realizada em fevereiro, pelo que questionámos se seria para preencher na mesma o orçamento e mais tarde, no relatório final, apresentar a execução ou se poderíamos apresentar já a execução. Fomos informados que podíamos preencher de imediato com a execução, apesar de termos que apresentar esses valores no relatório final na mesma, contudo ressalvamos que devem novamente questionar sobre esse ponto caso o evento candidato volte a ser o Bot Olympics ou outro feito antes de maio, dado que não era uma regra estipulada e os próximos dirigentes poderão ter uma interpretação diferente. Mais tarde foi enviado o relatório final da atividade, aproveitando o relatório que também foi enviado para a \acrshort{fctuc} devido ao apoio desta entidade ao evento, contendo uma breve descrição dos objetivos da organização, um relato das atividades desenvolvidas e a execução financeira do evento.