% ===========================
% # Carros Queima das Fitas #
% ===========================

\subsubsection{Carros da Queima das Fitas}

No início do mandato, em julho, tivemos uma reunião inicial com os carros da Queima das Fitas do ano seguinte, 2018. Nesta reunião, foram apresentados os aspetos logísticos do \acrshort{neeec} e como os carros deveriam proceder para colaborar com este. Estabeleceram-se este ano várias regras novas: os carros deveriam falar atempadamente com o Administrador do Núcleo para que este pudesse providenciar o material necessário a tempo das atividades dos carros; os carros tiveram de pagar caução dos materiais emprestados (esta nova medida, embora ao início tenha sido um pouco chata de aplicar, revelou-se muito positiva pois marcou um maior respeito pelas regras e cuidado com os materiais). Houve ainda algum cuidado para que o \acrshort{neeec} fosse justo e não tendencioso com ambos os carros apesar de um dos carros ter imensos membros do Núcleo e outro dos carros só ter um membro do Núcleo.

Devido aos novos moldes do Mega Convívio do Polo 2, os carros perderam a possibilidade de fazer a febrada no início da noite desse dia, que era, tradicionalmente, dos eventos mais lucrativos para os mesmos. Dessa forma, procurámos compensá-los, cedendo algumas coisas como, por exemplo, o jantar de curso (este tópico é abordado na área dos jantares de curso). Após esta reunião, foi assinada pelas três entidades (os dois carros e o \acrshort{neeec}) a ata resultante da mesma como formalização do acordo. Como resultado da reunião, o \acrshort{neeec} ficou de enviar a tabela de cauções, que na altura ainda não estava elaborada, e foi criada uma conversa no Facebook entre as três entidades, nomeadamente com o Presidente, Tesoureiro e Administrador do \acrshort{neeec} e as direções todas dos Carros. No entanto, os carros não utilizaram muito a conversa, procurando utilizar conversas individuais com as pessoas do Núcleo, um ponto bastante negativo uma vez que uns conversavam com o Presidente, outros com o Administrador e isso criou alguma entropia (a regra era de que deveriam falar principalmente com o Administrador). É ainda de realçar que, habitualmente, os carros começam a trabalhar com o Núcleo a seguir à Queima das Fitas para a realização da primeira febrada de apresentação dos mesmos. Desta forma, os carros habituam-se às regras impostas por um mandato que termina logo de seguida podendo, desta forma, não ser fácil a adaptação às novas regras impostas no novo mandato. É também de salientar que um dos carros deste ano não cumpriu com as regras estipuladas, avisando muito em cima dos eventos, e pressupondo que tinha sempre prioridade sobre outras entidades (por exemplo, precisando da tenda quando a mesma estava emprestada). Este carro era também o que tinha maior envolvimento com o Núcleo, no que toca a pessoas amigas, pelo que pressupomos que esta falta de organização se deva à excessiva confiança que o carro achava que tinha com o Núcleo. Por sua vez, o outro carro foi absolutamente espetacular tendo resultado uma parceria que, na nossa opinião, foi de elevada qualidade e proveitosa para ambas as entidades.