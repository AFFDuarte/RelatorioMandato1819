% ==========================
% # Coordenadores de Curso #
% ==========================

\subsubsection{Coordenadores de Curso}

O Coordenador de Curso é a figura responsável pela gestão interna de cada curso, procurando garantir o seu funcionamento salutar e a sua melhoria continuada. Estes são nomeados de dois em dois anos, tomando posse no início do ano civil em que é escolhido um novo Coordenador, o que aconteceu este ano.

No início do mandato do \acrshort{neeec}, o Coordenador de curso do \acrfull{mieec} era o Professor Doutor Urbano Nunes, que se encontrava em final de mandato, o que provocou uma relação mais fraca entre este e o Pelouro da Pedagogia do Núcleo. Infelizmente, esse momento baixo na relação com a coordenação de curso ocorreu precisamente numa altura em que existia um problema pedagógico grave a resolver, uma situação que acabou por não ser tão bem gerida como seria possível caso essa relação fosse mais forte.

No início do ano de 2018, o novo Coordenador de curso passou a ser o Professor Doutor Jorge Batista, que desde o início procurou colmatar a falha nas relações entre a Coordenação de Curso, os Representantes dos Estudantes e o Pelouro da Pedagogia. Após uma série de reuniões com o professor, procurou-se dar um término saudável ao problema pedagógica pendente, culminando numa \acrfull{rga} com a presença do mesmo. Além disso, o mesmo procurou ideias para resolver um dos maiores problemas do curso, o absentismo às aulas, nomeadamente, após o início da época de frequências, que tradicionalmente implicava uma média de uma frequência por semana aos alunos em tempo de aulas. No seu entender, uma forma de resolver esse problema seria implementar uma condensação da época de frequências em duas/três semanas chave, onde existiriam todas as vagas de frequências, possibilitando a existência de semanas livres para os alunos, que permitisse aos mesmos regressar às aulas, combatendo assim o absentismo às mesmas. No seu entender, mesmo que a aposta não resultasse, algo teria que mudar, pelo que procurou o apoio do Núcleo para fazer o teste desta ideia logo no segundo semestre, tanto na combinação das frequências e trabalhos entre si, como na divulgação da nova realidade à população estudantil.

Quanto ao Coordenador de curso do Doutoramento, dada a falta de atividade pedagógica por parte do Núcleo neste curso, associada à falta de uma interligação forte da população estudantil desse curso com o Núcleo, não existiu uma relação direta com o Coordenador do mesmo.

Para além de ser uma responsabilidade inerente ao Núcleo, consideramos que este relacionamento profícuo com as mais altas instâncias do curso é uma mais-valia ao mesmo, permitindo participar no processo de decisão e de apoio à atividade pedagógica no curso, para além de permitir uma resolução mais célere dos vários problemas que possam existir.