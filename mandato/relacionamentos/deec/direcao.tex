% ==========================
% # Direção                #
% ==========================

\subsubsection{Direção}

O relacionamento com a Direção do \acrshort{deec} foi, para nós, um dos pontos mais importantes deste mandato. No início do mandato, como é habitual, promovemos a realização de uma reunião de apresentação da nova Direção do \acrshort{neeec} à Direção do \acrshort{deec} e que consideramos ter sido essencial para a relação muito proveitosa que tivemos com este órgão. Nesta reunião foram apresentados os princípios básicos deste mandato, pelo que foi necessário preparar bastante esta reunião para que fosse apresentado um plano de atividades, iniciada a conversação necessária para a remodelação de espaços no \acrshort{deec} (tais como as salas de estudo, os locais de publicidade e a remodelação de sala de convívio) e o trabalho conjunto para resolver vários problemas logísticos do Departamento. Isto tudo para além da estruturação do trabalho conjunto a desenvolver no âmbito de pedagogia e divulgação do curso (relações externas). Ao longo do resto do ano existiram mais duas reuniões oficiais entre as duas direções, uma após a tomada de posse da nova Direção do \acrshort{deec}, que serviu para apresentar ambas as Direções e fazer a manutenção dos propósitos do presente mandato, e outra no final do mandato, que serviu para apresentar o relatório de atividades e agradecer, pessoalmente, por toda a ajuda durante o mandato. O estreitar da relação entre estas duas entidades fez com que a Direção do \acrshort{deec} quisesse estar, em peso, na cerimónia de tomada de posse do novo Diretor do \acrshort{deec}, Professor Humberto Jorge.

Ao longo do ano foram também existindo várias reuniões pontuais com elementos da Direção do Departamento, nomeadamente com a professora Maria do Carmo Medeiros, da antiga Direção, sobre o vandalismo nas salas de estudo, e com o professor Paulo Peixoto, da atual Direção, sobre a divulgação do curso. Informalmente houve ainda dezenas de reuniões entre o Presidente do \acrshort{neeec} e o Diretor do Departamento, no seu gabinete para acertar alguns pormenores que fossem surgindo.

De realçar que as três reuniões entre ambas as Direções foram feitas na sala do Núcleo para que fosse possível destacar, de imediato, alguns dos pontos que queríamos mencionar referentes à sala do Núcleo, como, por exemplo, a calha que se encontrava há anos por arranjar. Desta forma é também possível dar a conhecer aos professores que esta sala serve, de facto, como uma sala de trabalho e criar assim um ambiente mais próximo entre ambas as Direções.