% ==========================
% # GRI                    #
% ==========================

\subsubsection{GRI}

Desde o início do mandato, com a compra do novo domínio do \acrshort{neeec}, foi essencial a ajuda do \acrfull{gri} para a configuração deste nos servidores do \acrshort{deec} e a configuração dos emails (inicialmente do Núcleo, mais tarde de todos os Pelouros e, por fim, dos Delegados de Ano). A partir daí, devido a alguns problemas com a configuração dos emails, as idas ao GRI passaram a ser uma constante. Por sua vez, com a chegada do \acrshort{ene3} foi também fulcral o trabalho desta entidade para nos ajudar e permitir, sem qualquer tipo de custo, ter uma elevada e moderna rede montada para acolher o evento. Daí em diante, muito por causa da acessibilidade do Engenheiro Francisco Maia bem como de toda a sua equipa, com destaque do Artur Dias e do Tiago Ribeiro, a relação entre o \acrshort{neeec} e o GRI passou a ser muito próxima tendo a colaboração desta entidade sido fulcral para a realização de eventos como o Bot Olympics e a Ultra Gaming Fest bem como para a criação de novos softwares/interfaces como é o caso da televisão do \acrshort{deec} e das plataformas de ofertas de emprego e de perdidos e achados. Foi também graças ao GRI que nos foram disponibilizadas etiquetas que permitiram fazer todo o inventário do \acrshort{neeec}, sem qualquer custo, bem como a criação do IVR que permite o sistema de atendimento automático de que dispomos agora no número de telefone do \acrshort{neeec}.
