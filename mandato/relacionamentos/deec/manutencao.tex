% ==========================
% # Manutenção             #
% ==========================

\subsubsection{Manutenção}

A Manutenção do \acrshort{deec} é o órgão que supostamente faz, como o nome indica, a manutenção dos espaços e dos materiais do Departamento, promovendo o seu bom uso, o arranja quando tal é necessário e faz as adaptações necessárias aos espaços do Departamento. Este órgão é composto pelo Sr. Carlos Coelho (eletricista) e pelo Sr. Augusto Figueiredo, mais conhecido por Tito. Contudo, o relacionamento com este órgão é complicado, mesmo por parte da Direção do \acrshort{deec}, dada a baixa eficiência deste órgão.

Todas as alterações que foram feitas na sala do \acrshort{neeec} e em particular no \acrshort{deec}, foram fruto de uma grande persistência, entre a Direção do \acrshort{neeec} e do \acrshort{deec}, em especial do Prof. Humberto Jorge, perante este grupo de funcionários, que ao fim de um mandato, foi mais um fator de bastante cansaço que podia ter sido evitado. Todas as obras realizadas, tais como as instalações elétricas nas salas de estudo ou na sala de convívio, têm o seu mérito, mas apresentaram um processo, desde o início ao fim, excessivamente complicado e demorado.

Começamos em junho a tentar fornecer condições de eletrificação na sala do Núcleo, com a instalação de pontos de energia. A instalação destas, que no máximo para um funcionário de trabalho normal, demorariam sensivelmente uma tarde de trabalho, duraram quase 3 semanas onde durante esse tempo foi mais difícil trabalhar na nossa sala.

A implementação do projeto da sala de estudo do piso 6 e do piso 3 atrasou, precisamente pela falta de material que supostamente era só "encomendar" pelo responsável da manutenção, mesmo após disponibilização total dos membros do Núcleo em cooperar em melhorar estes espaços.

Assim, para todos os trabalhos a realizar pela Manutenção do \acrshort{deec}, é necessário abordar a Direção, e insistir, diariamente, com os elementos deste gabinete para que as obras fiquem feitas. É também muito importante estabelecer uma relação de simpatia e cooperação, mesmo quando a paciência para isso, dados os atrasos, não é a maior.
