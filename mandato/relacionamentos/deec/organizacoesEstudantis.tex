% ===========================
% # Organizações Estudantis #
% ===========================

\subsubsection{Organizações Estudantis}

O \acrshort{deec} é composto por várias estruturas estudantis com diferentes dimensões, objetivos, modos de funcionamento e estruturas: o \acrfull{neeec}, o \acrfull{cr}, o \acrfull{cp}, o \acrfull{best} Coimbra e o \acrfull{uc}.

As várias organizações estudantis têm diversos âmbitos de representação: o \acrshort{neeec} representa exclusivamente os estudantes dos cursos sediados no DEEC, o \acrshort{cr} e o \acrshort{cp} abrangem todos aqueles que se interessam pelas áreas de robótica e da programação, o \acrshort{ieeeuc} representa todos os alunos da área da Engenharia Eletrotécnica (abrangendo, pelo menos, os cursos de Engenharia Eletrotécnica e de Computadores e de Engenharia Biomédica, na \acrlong{uc}) enquanto o \acrshort{best} representa todos os alunos da \acrshort{fctuc}. Contudo, pela sua localização, a maioria das atividades do \acrshort{cr} e do \acrshort{cp} vocacionam-se para o \acrshort{deec}. Há dois anos, quando a direção do IEEE era composta por alunos do DEEC, este era também muito vocacionado para o departamento, contudo, atualmente, isso não se verifica.

Das várias organizações estudantis referidas, o \acrshort{neeec}, o \acrshort{best} e o \acrshort{ieeeuc} são as únicas com organização estatutária sendo que o \acrshort{neeec} e o \acrshort{best} contam, ambos, com mais de duas décadas de existência. Essencialmente por este motivo, bem como pelo facto de, dada a sua estrutura, terem equipas com uma dimensão muito maior que as restantes estruturas, o \acrshort{neeec} e o \acrshort{best} têm assumido um papel preponderante no estabelecimento da relação entre as várias associações, no entanto, existem vários problemas que não contribuem para uma saudável relação das mesmas:
\begin{itemize}
\item Existem vários organismos com estruturas diferentes (enquanto uns têm uma direção com cargos bem definidos, noutros chega a ser impossível identificar alguém responsável pela estrutura, tal como um diretor ou presidente).
\item A data em que ocorre a mudança de equipas é bastante díspar. Por serem associações estudantis, todas apresentam, caso aplicável, mandatos de apenas um ano, contudo, o \acrshort{neeec} muda em junho, o \acrshort{best} faz eleições em junho e muda de órgãos gerentes em setembro e o \acrshort{ieeeuc} exerce este procedimento em outubro pelo que existe muito pouco tempo em que os vários órgãos podem trabalhar em conjunto, não existindo um período (como o verão) em que possa haver algum planeamento.
\item Existem órgãos (como o \acrshort{best}, \acrshort{neeec} e \acrshort{ieeeuc}) que possuem estruturas acima que os regulam e deles exigem trabalho, contudo tal não é aplicável a todos.
\end{itemize}

O \acrshort{neeec}, dando continuidade ao trabalho iniciado no mandato anterior, estabeleceu parcerias individuais com as várias estruturas estudantis da casa. O protocolo estabelecido era um modelo feito pelo \acrshort{neeec} que poderia ser adaptado consoante cada situação contemplando casos muito gerais. É de realçar que estes protocolos têm a duração de um ano não renováveis pelo que, no início do mandato 2018/2019, estarão em vigor mas terminarão de seguida, não se renovando. Os mesmos protocolos são cessáveis, desde que em acordo por ambas as entidades.

\paragraph{Clube de Robótica da \acrlong{uc}}

O protocolo foi assinado com o \acrfull{cr} em setembro de 2017, aquando da primeira reunião para a organização do Bot Olympics. Este protocolo não contemplou eventos maiores como o Bot Olympics tendo sido combinado estabelecer-se um protocolo à parte, algo que não se chegou a realizar, mas que aconselhamos vivamente a que seja feito, dado que, caso voltem a acontecer situações de tensão com este órgão, como ocorreu no mandato anterior, haverá um documento base que regulamente a relação.

A relação com este clube correu bem este ano. Começou-se por organizar alguns workshops em conjunto, que tiveram alguns problemas: houve inscrições aceites em ambos os lados (complicando bastante a gestão de inscrições e pagamentos) e o cartaz foi mandado fazer pelo \acrshort{cr} a uma entidade externa, sem autorização do \acrshort{neeec}, pagando-se assim por ele. Contudo, após uma reunião em que se apresentou a forma de trabalho interna do \acrshort{neeec}, os workshops restantes correram bem tendo o \acrshort{neeec} ficado com toda a parte logística (inscrições, cartazes, divulgação, montagem da sala) e o \acrshort{cr} com toda a parte técnica (orador, material, montagem da sala no que toca ao material, etc).

Em relação ao Bot Olympics, foi criada uma comissão à parte que envolvia membros das duas entidades criando-se assim uma equipa coesa onde não interessava quem pertencia ao quê, acabando por correr bastante melhor que em edições passadas. No entanto, a falta de protocolo para este evento é algo que consideramos muito grave uma vez que, dados os elevados montantes envolvidos, pode correr mal, como já aconteceu em anos anteriores. De realçar que a ligação entre o \acrshort{neeec} e o \acrshort{cr} correu muito bem devido à presença do Paulo Almeida enquanto Presidente do \acrshort{cr} que é uma pessoa mais acessível e não tão focada na parte técnica do \acrshort{cr}, conseguindo fazer uma fácil ligação entre as duas organizações.

\paragraph{Clube de Programação da \acrlong{uc}}

O protocolo com o \acrfull{cp} foi assinado em outubro de 2017, exatamente com as mesmas cláusulas do protocolo assinado com o \acrshort{cr}. Contudo, o \acrshort{cp} foi um clube que não apresentou praticamente atividade nenhuma ao longo do ano pelo que a única atividade em conjunto que se fez foi um workshop de Unity, de várias sessões, no qual o \acrshort{neeec} ficou apenas responsável pelas imagem e divulgação enquanto que o \acrshort{cp} ficou responsável pelas inscrições, pela sala e pela parte técnica. Na nossa opinião esta parceria correu bem, uma vez que sendo costume haver workshops em ambas as entidades, este ano não trabalhámos em sobreposição, como aconteceu em anos anteriores. Contudo é de realçar que não tendo o \acrshort{cp} realizado mais nenhuma atividade para além desta, é difícil avaliar se a parceria resultou ou se correu bem apenas devido à inércia do \acrshort{cp}.

\paragraph{IEEE UC Student Branch}

Com esta instituição foi também assinado um protocolo, em outubro, com as mesmas condições que as anteriores. Esta era uma instituição com a qual o \acrshort{neeec} trabalhou muito em conjunto no mandato anterior nomeadamente nos workshops e no \acrshort{ene3} devido ao Chair da mesma ser um ex-membro do Núcleo, o Diogo Justo. A colaboração do \acrshort{ieeeuc} nos workshops baseava-se muito em estarem presentes no mesmo de forma a poderem divulgar-se um pouco, compensando com a realização de todos os coffee breaks, sendo os mesmos de elevada qualidade (com muita comida variadas, sandes e café gratuito). A única condição que impuseram era que os membros do \acrshort{ieeeuc} não pagassem, tendo o lucro das atividades ficado para o \acrshort{neeec}. Contudo, este ano, de acordo com o protocolo estabelecido, o lucro seria dividido 50/50. No primeiro semestre realizaram-se ainda dois workshops entre as duas entidades (workshop de AutoCAD e workshop de Machine Learning). No primeiro, eles faltaram, pelo que o \acrshort{neeec} teve de arranjar o coffee break à pressa. No segundo, eles voltaram a faltar mas ficaram responsáveis pelo orador que foi pago e este era péssimo. Desta forma, só se realizou a habitual febrada de apresentação do \acrshort{ieeeuc} organizada por ambas as entidades, febrada essa onde o IEEE não apresentou qualquer tipo de atividade para se divulgar, não passando assim de uma febrada normal. Dados os acontecimentos, não voltaram a haver atividades entre ambas as entidades ao longo do mandato. De referir também que o próprio \acrshort{ieeeuc} parece ter estado muito menos ativo no mandato do presente ano e, uma vez que os estudantes que geriam o mesmo eram de Engenharia Biomédica, não teve qualquer tipo de representação/atividade no \acrshort{deec}. No futuro, não achamos que seja muito fácil voltar a ter uma parceria tão positiva como a que tivemos no ano 2016/2017 pois tal se deveu à elevada confiança que existia entre o CG das Saídas Profissionais do \acrshort{neeec} e o Chair do \acrshort{ieeeuc}, João Bento e Diogo Justo, respetivamente.

\paragraph{BEST Coimbra}

O \acrfull{best} Coimbra foi a única das entidades estudantis com a qual não assinámos nenhum protocolo. Contudo, ao longo do ano, realizaram-se algumas reuniões entre os Presidentes de ambas as entidades onde foi abordado o plano de atividades da mesma. Desta forma, a Semana dos Ramos, por exemplo, foi alterada para a semana seguinte para não coincidir com a BEW, algo que já acontecia há dois anos consecutivos. Entre o \acrshort{neeec} e o \acrshort{best} não foram realizadas qualquer tipo de atividades em conjunto. Também não houve nenhum trabalho conjunto para a divulgação dentro do Departamento, realçando, no entanto, que tendo as duas entidades públicos alvos diferentes e tipos de atividades diferentes, não consideramos isto um ponto negativo.

\paragraph{Conclusões}

No geral, consideramos que os protocolos foram positivos. Contudo, achamos que estes devem ser revistos e escritos com mais detalhe, especificando mais casos possíveis tendo em conta as falhas que ocorreram este ano. Achamos positivo continuar a ter um protocolo semelhante para todas as entidades para que a forma de trabalho seja a mais uniforme possível, não criando entropias.

Achamos também que a Direção do \acrshort{deec} deve ter um papel preponderante na coordenação das várias organizações estudantis. Na nossa opinião, esta deve funcionar como que uma entidade reguladora e deve promover reuniões com alguma periodicidade (no mínimo, uma no início de cada semestre para planeamento do mesmo, no final de cada semestre para balanço do mesmo e no início dos mandatos para apresentação das equipas) que deveriam incluir todos os organismos em simultâneo. O \acrshort{deec} deve ainda promover a interação entre os vários organismos de forma a promover o seu trabalho conjunto e não a sobreposição dos mesmos (por exemplo, estipular as responsabilidades de cada organismo na divulgação do curso). Estas reuniões deveriam também servir para estipular as formas de funcionamento do departamento no que toca a impressões, reservas de salas, arrumos, utilização de espaços comuns, utilização e requisição de material, etc. Adicionalmente, devem também ser redigidas as contrapartidas, nomeadamente a cedência de salas e regalias. Estas normas de funcionamento devem, na nossa opinião, estar redigidas num documento que deve ser assinado por todos no início de cada mandato sendo que, apesar de, no primeiro ano, ser um documento difícil de elaborar, no futuro passará a ser apenas um documento adaptado, ano após ano, de acordo com a nova realidade. Entendemos que será importante haver uma apresentação dos planos de atividades e respetivas datas para não haver sobreposições de atividades entre as várias entidades nem uma utilização exagerada de recursos, como acontece atualmente com as febradas. Adicionalmente, todas as regras acordadas devem ser informadas a todos os funcionários do departamento para que todos trabalhem em sintonia, de acordo com o estipulado. Entendemos que a criação dos vários documentos e da estrutura necessária para este bom funcionamento será uma tarefa complicada inicialmente, mas será fulcral para que o bom trabalho entre as várias entidades exista, independente da confiança que exista existente entre as mesmas consoante o ano letivo.

\paragraph{Outras entidades}

À parte destas, existiu também uma enorme ligação à tuna da nossa faculdade, a Quantunna, Tuna Mista da \acrshort{fctuc}. Esta ligação deveu-se muito ao facto de o Presidente da mesma ser o Matias Correia, estudante do nosso curso, e de haver vários elementos do \acrshort{neeec} que faziam também parte da Tuna. Assim, além de eles atuarem no dia da receção ao caloiro, como já era habitual, a Tuna participou também no Mega Convívio do polo 2 a nosso convite, na Noite de Fados e na comemoração do aniversário do Núcleo. A Tuna foi também convidada a atuar no \acrshort{ene3} e na Gala Ohms D'Ouro, mas tal não lhes foi possível por motivos logísticos (o \acrshort{ene3} realizou-se ainda em altura de férias, não sendo possível à Tuna ter um número de pessoas adequado para atuar e a Gala Ohms D'Ouro tem calhado no mesmo dia do primeiro dia do Festival VIII Badaladas, organizado pela Quantunna). Por recompensa das suas atuações, é habitual pagarmos um barril ou algumas refeições (no Mega Convívio receberam ambos, enquanto que na Noite de Fados receberam um pack de cervejas mais febra, por exemplo). Em situações como no aniversário do Núcleo, em que não lhes pudemos dar nada, a Tuna participou à mesma, tendo, no entanto, sido uma ocasião mais especial e, claro, a não repetir frequentemente. No âmbito da boa relação entre as entidades, o \acrshort{neeec} tentou sempre divulgar a Tuna, nomeadamente as suas atividades como o Festival VIII Badaladas e algumas tradições, como, por exemplo, a atuação da Quantunna na Queima das Fitas. Através do Instagram, apresentámos também algumas das suas atuações ao vivo, ao longo do ano. Esta parceria não oficial, na nossa opinião, é muito proveitosa e deve-se continuar a promover.