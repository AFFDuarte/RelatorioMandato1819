% =================================
% # Representantes dos Estudantes #
% =================================

\subsubsection{Representantes dos Estudantes}

Os Representantes dos Estudantes são uma estrutura representativa dos estudantes dos vários cursos da faculdade junto do Conselho Pedagógico da mesma. São nomeados de dois em dois anos por votação direta dos estudantes ou pelo Coordenador de Curso quando não existem listas candidatas. Entram em funções ao mesmo tempo que os novos Coordenadores de Curso, sendo os representantes máximos por parte dos estudantes dos assuntos pedagógicos em cada curso.

No nosso caso, os Representantes dos Estudantes do \acrfull{mieec} têm sido, nos últimos anos, simultaneamente membros dirigentes do \acrshort{neeec}, existindo alguma sobreposição entre o trabalho destes e do Pelouro da Pedagogia do Núcleo. Sendo membros dirigentes do Núcleo, normalmente não responsáveis pela Pedagogia do Núcleo, estes tendem a afastar-se um pouco, servindo apenas como conselheiros da ação desse Pelouro e uma mão de ajuda quando tal se justifica.

O nosso mandato começou com a seguinte estrutura dos representantes dos estudantes do \acrshort{mieec}:
\begin{itemize}
\item Daniela Temudo - Representante dos Estudantes
\item Miguel Antunes - 1º Suplente
\item Ivo Frazão - 2º Suplente
\end{itemize}

Após as eleições realizadas no final do ano de 2017, a nova estrutura passou a ser:
\begin{itemize}
\item Ivo Frazão - Representante dos Estudantes
\item André Duarte - 1º Suplente
\item Ana Calhau - 2ª Suplente
\end{itemize}

Já quanto aos representantes dos estudantes do Doutoramento não existe qualquer relação com os mesmos, tanto que nem sabemos quem são atualmente os representantes.