% ==========================
% # Núcleos de Portugal    #
% ==========================

\subsubsection{Núcleos de Portugal}

Com a realização do \acrshort{ene3} sentimos uma larga necessidade de entrar em contacto com os núcleos da nossa área do resto do país. Esta foi uma tarefa muito árdua uma vez que o contacto entre os vários núcleos era totalmente inexistente. Esta foi uma iniciativa que não correu da melhor forma durante o \acrshort{ene3} mas que acabou por ser prosseguida ao longo do mandato. Em janeiro havia já contacto entre os núcleos de Aveiro, Coimbra, Porto, Lisboa (Técnico), Beira Interior, Minho e Trás-os-Montes pelo que se criou um grupo no Facebook com os Presidentes, Vice-Presidentes e demais membros que cada Núcleo entendesse. Foi também criado um documento com os dados de cada membro e do Núcleo, nomeadamente o email, o telefone do Núcleo e a altura do ano em que existe renovação de mandatos. Com isto pretende-se obter informação suficiente para que, quando haja renovação de mandatos, exista possibilidade de se entrar, facilmente, em contacto com os novos dirigentes. Após a criação deste grupo existiu, em Aveiro, uma reunião, bastante produtiva e benéfica, entre todos os núcleos onde se decidiu onde seria o \acrshort{ene3} seguinte e se trocou impressões, informações e métodos de trabalho entre todos os núcleos para aprendizagem comum. Após esta reunião, existiu uma outra, em Coimbra, entre o \acrshort{neeec} e o \acrfull{neeet} que serviu para passagem de pasta do \acrshort{ene3}. Desta última reunião surgiu a ideia de se criar um fim de semana, ao estilo do Fórum \acrshort{aac}, a decorrer uns dias antes do \acrshort{ene3} na cidade onde este se realizaria e organizado pelo Núcleo anfitrião anterior. Este fim de semana teria como principal objetivo manter o relacionamento entre os vários núcleos e não deixar cair o grupo em esquecimento. Outra sugestão que surgiu desta reunião foi a ideia de se criar um evento do género do Beer Olympics que tivesse uma fase em cada Núcleo e depois uma fase final que, em cada ano, se realizaria num local diferente. Esta atividade faria com que todos os núcleos se vissem obrigados a trabalhar em conjunto, à semelhança do que ocorre com o Polo 2, por exemplo, e ligaria também um pouco os estudantes a este conjunto de núcleos.

Todo este processo leva-nos a crer que, futuramente, será necessária e natural a constituição de uma Associação Nacional de Estudantes de Engenharia Eletrotécnica (ANE3) para que se possa garantir a continuidade, associada a uma gestão rigorosa quer das contas, quer dos planos de atividades conjunto que permita elevar a nossa área num esforço de trabalho conjunto e permitir melhores condições a todos os estudantes da área.