% ==========================
% # Ordem dos Engenheiros  #
% ==========================

\subsubsection{Ordem dos Engenheiros}

O relacionamento com a \acrfull{oe} era, até à pouco tempo, quase inexistente, baseando-se apenas em patrocínios pontuais que resultavam na presença da \acrshort{oe} em algumas palestras ou na nossa feira de emprego. Em 2017, Cláudia Gaspar, ex-aluna do Departamento, contactou o \acrshort{neeec} para se realizar, em parceria um workshop denominado "Game Changers". Iniciou-se aqui um novo ciclo de parcerias tendo a Ordem dos Engenheiros sido uma das main sponsors do \acrshort{ene3} 2017. Através da Cláudia Gaspar tivemos uma ligação direta ao Engenheiro Pedro Carreira, Coordenador do Colégio de Engenharia Eletrotécnica da Ordem dos Engenheiros da Região Centro, o que nos possibilitou uma presença forte da OE em várias atividades diferentes das que costumávamos realizar. Prova desta relação foi a presença da mesma no aniversário do \acrshort{neeec}.

Aquando do Congresso da Ordem dos Engenheiros, que se realizou em novembro de 2017, em Coimbra, o \acrshort{neeec} entrou em contacto com a Cláudia Gaspar averiguando a possibilidade dos estudantes estarem presentes no congresso. Após algumas pequenas e simples conversações, a OE disponibilizou lugares gratuitos para todos os estudantes de engenharia, sócios da \acrshort{aac}.

Achamos que possíveis parcerias com a OE poderão proporcionar novas valências e atividades ao Núcleo, assemelhando-se a acordos anuais e/ou plurianuais, tais as que já existem entre outras ordens profissionais e outros núcleos.