% ==========================
% # FCTUC                  #
% ==========================

\subsubsection{FCTUC}

A \acrfull{fctuc} apoia de várias formas os Núcleos de Estudantes para a realização das suas atividades (por exemplo, o aluguer de espaços da Faculdade gratuitamente ou a preços mais reduzidos). Estes apoios devem ser pedidos atempadamente, dado que a estrutura da Faculdade é ainda relativamente grande, pelo que por vezes esta demora a responder a todos os pedidos.

\paragraph{Apoios monetários}
Um dos tipos de apoios que a faculdade dá é monetário. Este tipo de apoio segue um padrão relativamente standard que consiste em fazer um pedido formal de apoio, enviando para o gabinete do Diretor uma descrição do evento, das necessidades que esse evento terá e os apoios que a faculdade poderá providenciar. Caso este seja aceite, estes respondem enviando um modelo de pedido de apoio, onde é preciso descrever o orçamento da atividade, que terá de ser assinado pela pessoa indicada como requerente do apoio e entregue pessoalmente nos serviços da faculdade, situados no piso mais acima da torre central do edifício central do Polo 2. Enviam também a imagem gráfica da faculdade que deve ser colocado nos meios de divulgação do evento. Após o pagamento, a faculdade dá um prazo para preenchimento de um formulário online onde será necessário descrever as principais atividades do evento e a execução financeira do evento e uma comparação com o orçamento apresentado, pelo que devem guardar o orçamento inicial apresentado para evitar problemas. O prazo de pagamento varia ainda consideravelmente, principalmente se outras estruturas da \acrshort{aac} tiverem os formulários finais aos pedidos de apoio (quer da \acrshort{fctuc}, quer da UC) ainda por enviar e, como todas as estruturas da \acrshort{aac} partilham o \acrshort{nif} único e é a reitoria que processa o pagamento, qualquer uma das estruturas da \acrshort{aac} poderá ser responsável.

Neste mandato, a \acrshort{fctuc} apoiou o Núcleo monetariamente no \acrshort{ene3}, na receção ao caloiro (pagou parte substancial das t-shirts dos caloiros), o Bot Olympics e a \acrshort{ugf}. Além disso, concedeu Suplementos ao Diploma para os participantes e elementos da organização da \acrshort{f3e}, do \acrshort{ene3} e do Bot Olympics.

\paragraph{Suplementos ao Diploma}
Os Suplementos ao Diploma são um benefício que a faculdade dá, certificando os alunos pela sua participação em diversas iniciativas, sendo um documento bastante fácil de pedir, principalmente no caso dos eventos apoiados com pedidos monetários. Para tal, basta enviar um relatório assinado pelos representantes das entidades envolvidas na organização da atividade (presidentes de cada instituição envolvida e Presidente das comissões organizadoras) a indicar os nomes de todas as pessoas envolvidas com uma breve descrição do evento, o programa e os resultados conseguidos através de algumas disposições finais. Este pedido, deve ser solicitado ao Diretor da Faculdade, através do email gbdiretor@fct.uc.pt e, caso seja aceite, terá de ser impresso, assinado e carimbado e entregue, pessoalmente, no edifício central da faculdade. Esse documento é assinado e anexado ao diploma de cada pessoa, pelo que é necessário ter bastante cuidado com o que é escrito nesse documento. No presente mandato foram solicitados suplementos ao diploma para os voluntários, organizadores e participantes do Bot Olympics 2018, voluntários e organizadores do \acrshort{ene3} 2017, comissão organizadora da \acrshort{f3e} e para os Colaboradores do Núcleo com certificação superior a 80\%, tendo todos os pedidos sido aceites.

\paragraph{Divulgação do Curso}

Num outro âmbito, durante o mês de fevereiro foi convocada por parte da Cátia Sá, responsável pelo \acrfull{gad}, uma reunião com os representantes de todos os Núcleos da \acrshort{fctuc} de forma a esclarecer eventuais mitos acerca de métodos laborais entre a \acrshort{aac} e a \acrshort{fctuc} relativamente ao apoio à divulgação nas escolas do Ensino Secundário.

O \acrshort{neeec} fez-se representar pelo Presidente, João Bento, o Vice-Presidente, João Martins, e o Secretário, Miguel Antunes, que nesta altura já estava responsável pelas competências do Pelouro das Relações Externas.

Nesta reunião obtiveram-se várias informações sobre a imagem gráfica da \acrshort{fctuc}, os materiais a poder utilizar nos vários eventos de divulgação do curso e as condições para se poder ir a feiras de oportunidades e escolas secundárias, obtendo alimentação e transporte pagos pela Faculdade. Adicionalmente, soube-se também de um novo projeto que implicaria a colaboração dos núcleos, pretendendo-se criar uma espécie de rede de embaixadores mas da \acrshort{fctuc}. Este projeto acabou por nunca se vir a realizar. É de notar que, nesta reunião, deu para perceber dois problemas: a falta de meios da \acrshort{fctuc} alocados para a divulgação da mesma, uma vez que esta funcionária se encontra a trabalhar sozinha, não conseguindo dar vazão aos assuntos e a falta de ligação e estabelecimento de métodos laborais entre a \acrshort{fctuc} e a \acrshort{aac} bem como entre a \acrshort{fctuc}, os departamentos e os organismos estudantis. Notou-se ainda uma péssima confusão entre o \acrshort{neeec} e o Clube de Robótica bem como um desconhecimento, por parte da \acrshort{fctuc}, dos contactos dos Núcleos de Estudantes da \acrshort{fctuc}.

Essa mesma pessoa, mostrou-se complementa disponível em ajudar os Núcleos no apoio à divulgação, o que, na nossa opinião, deve ser aproveitado num próximo mandato.