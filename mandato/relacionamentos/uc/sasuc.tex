% ==========================
% # SASUC                  #
% ==========================

\subsubsection{SASUC}

Os \acrfull{sasuc} são outra entidade que pode apoiar bastante as atividades do Núcleo, contudo este apoio nunca será monetário, dado que são uma instituição de ação social. Os \acrshort{sasuc} apoiam normalmente fornecendo senhas para as suas cantinas/restaurantes e, quando o número de pessoas o justifica, poderão mesmo abrir gratuitamente e de forma excecional cantinas que não costumam estar abertas, como aconteceu no \acrshort{ene3}, em que as cantinas do Polo 2 também serviram pequenos-almoços nesses dias, quando naqueles dias apenas estaria aberta para almoço. Além disso, quando são feitos pedidos de apoio, é possível acordar com os serviços que as senhas que sobrem do evento possam ser devolvidas e o custo ressarcido, contudo, o número de senhas devolvidas não deve representar uma fatia muito significativa do número de senhas pedidas, dado que eles baseiam a comida que fazem nesse dia a contar com as pessoas desse evento. Mesmo que não peçam qualquer tipo de apoio, se o número de pessoas que levarem à cantina for significativo, devem avisar na mesma os serviços para que estes estejam a contar e façam comida suficiente para todos. Outro tipo de apoios que podem ser pedidos é o aluguer de quartos de residências que estejam disponíveis na altura dos eventos para dormidas e o aluguer dos seus espaços, como cantinas, para alguns eventos.

Atenção a uma coisa com as senhas da cantina: tem sido costume evitar comprar as senhas de 4,10€ visto que são mais caras, mesmo quando estas se destinam para pessoas não-estudantes da \acrshort{uc}, contudo, ultimamente, eles têm apertado o controlo a essas senhas, pelo que podem correr o risco das senhas serem rejeitadas e acabarem por ter que gastar o dinheiro da senha de 4,10€, pelo que aconselho a começarem a comprar as senhas corretas para cada pessoa. No caso do Bot Olympics, os alunos do ensino secundário, como não pertencem à comunidade da \acrshort{uc}, poderão ser excluídos, contudo isso é facilmente resolvido marcando uma reunião com os \acrshort{sasuc}, combinando bem essa situação, dado que, neste caso, pelo menos, o evento serve de divulgação à \acrshort{uc} pelo que eles apoiarão sem problema algum.