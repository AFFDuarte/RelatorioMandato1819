% ==========================
% # Atendimento NEEEC      #
% ==========================

\subsection{Atendimento na Sala do Núcleo}

Algo com que nos deparamos no início do mandato, nomeadamente aquando do preenchimento de documentos na tomada de posse, foi a falta de um horário de atendimento definido para a sala do Núcleo. Por sua vez, tínhamos já reparado que a sala do Núcleo era o local onde era feito o pagamento de inscrições, a compra de produtos como as camisolas de curso e que em algumas situações, como por exemplo os early birds da gala, quem chegasse primeiro seria o primeiro a ter acesso a um dado serviço mas que alguém que viesse a uma hora poderia ter o azar de encontrar o Núcleo fechado e quem viesse mais tarde já poderia ter a sorte de ser o primeiro a encontrar o Núcleo aberto. Simultaneamente, caso alguém precisasse de indicar a uma dada pessoa que deveria ir ao Núcleo nunca saberia informar em que horas isso era possível.

Verificámos também que o Núcleo acabava por estar aberto em períodos como as horas de almoço, em que toda a gente quer tomar o seu café, ou os buracos livres das pessoas acabando por ser mais uma sala de convívio de quem pertence ao Núcleo do que um verdadeiro local de trabalho e atendimento para todos aqueles que são, de facto, membros do \acrshort{neeec}, os alunos do \acrshort{deec}.

Assim, criámos um horário de atendimento mínimo durante o período de aulas entre as 10h e as 17h com interrupção para almoço entre as 13h e as 14h. Este horário foi afixado na porta para que fosse público. Para o cumprir, foram implementados os métodos já referidos em \ref{escalanucleo}. Já no segundo semestre, após uma análise do primeiro semestre, decidiu-se não ter horário do Núcleo às quartas à tarde, uma vez que são quase sempre dias de reunião e às sextas de manhã, pelas dificuldades óbvias no preenchimento da mesma.

A implementação deste horário trouxe duas coisas muito boas: em primeiro lugar, todos os Colaboradores passaram a ser obrigados a ir ao Núcleo, promovendo assim teambuilding entre a equipa. Este foi um facto que se notou bastante comparando com outros anos, uma vez que mesmo os membros mais tímidos acabaram por se habituar a ir ao Núcleo. O segundo benefício foi poder-se implementar serviços como a obrigatoriedade de levantar as bolas de ping pong ou os jogos de cartas e uno na sala do Núcleo, algo que permitiu poupar muitos os gastos que havia com estes produtos em anos anteriores. Desta forma, achamos essencial e renovação desta medida no futuro.