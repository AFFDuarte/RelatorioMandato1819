% ==========================
% # Banco de Apontamentos  #
% ==========================

\subsection{Banco de Apontamentos}

Uma das apostas do nosso mandato foi a existência do \acrfull{gape}, cujo propósito seria desenvolver um conjunto de iniciativas que permitisse ajudar os alunos do curso. Uma dessas iniciativas foi o banco de apontamentos que procurou pegar num simples facto que já ia existindo pontualmente: a entrega de sebentas, apontamentos, livros, etc. por parte de alunos (recém-formados, na sua grande maioria) para que os mesmos fossem distribuídos por quem os quisesse. Contudo, estes eram distribuídos apenas pelos elementos do Núcleo, não sendo uma iniciativa que servia quem o Núcleo representa. Procurámos então oficializar essa iniciativa de forma a que a mesma crescesse e se tornasse uma tradição no curso, tornando o Núcleo num intermediário entre quem quer entregar e quem precisa. A ideia base seria de doar os materiais sem limite a cada pessoa que quisesse, incentivando, contudo, a sua devolução.

Dessa forma, pegámos no conjunto de materiais que já existiam no Núcleo e arrumámo-los num dos armários disponíveis no gabinete do \acrshort{neeec}, de forma a que fosse de fácil acesso para todos e que pudessem ser facilmente entregues e/ou rececionados por qualquer pessoa que estivesse no Núcleo. A dedicação de um armário inteiro permitiu uma divisão dos materiais pelos anos a que diziam respeito a cadeira do material, simplificando a sua procura.

O principal problema detetado com esta iniciativa residia em saber quais os materiais que existiam no Núcleo a cada momento, sem se ter que fazer uma pesquisa pelos materiais todos de cada vez que fosse pedido um material, o que tornaria o processo bastante moroso e impraticável. Propusémo-nos então a criar uma base de dados simples que permitisse gerir essa informação e partilhá-la publicamente no site do \acrshort{neeec}, tendo tido uma versão preliminar de toda essa interface disponível no início do segundo semestre, contudo, com a aparecimento do interno.neeec.pt, com o propósito de fazer toda a gestão interna do \acrshort{neeec}, a implementação final foi adiada para ser integrada nessa plataforma.

A estrutura base inicial de toda a plataforma era composta por Materiais que pertenciam a Cadeiras e eram doados a Pessoas. A presença das pessoas teria o propósito de se recolher os contactos das mesmas quando se doavam materiais para que, quando terminasse o semestre, fosse enviado um email para a pessoa relembrando que poderia devolvê-lo. A presença das cadeiras permitiria realizar uma pesquisa dos materiais por cadeiras, através da associação dos materiais às cadeiras, contudo, apercebemo-nos que alguns materiais recaem sobre várias cadeiras diferentes, pelo que alterámos a estrutura para que a lista de cadeiras continuasse a existir (impedia que diferentes pessoas escrevessem a mesma cadeira de forma diferente, facilitando a pesquisa), mas deixaria de existir a relação direta, passando a existir apenas um atributo do material com o nome das cadeiras envolvidas após seleção da lista já existente.

A ideia deste banco de apontamentos é uma ideia simples de implementar e que permite uma interação forte da comunidade com o Núcleo, melhorando a imagem do mesmo. Na nossa opinião, a plataforma encontra-se agora completamente concluída, sem requerer grandes alterações futuras para que funcione corretamente, necessitando apenas de ser bem divulgada, principalmente junto dos caloiros no início de cada ano letivo, para que a ideia comece a ficar assente nas suas cabeças, e junto dos finalistas no final de cada ano letivo, para que estes entreguem os seus materiais antigos. É também muito importante manter o conteúdo que o \acrshort{neeec} possui organizado e atualizado para que as pessoas quando doem o material, sintam que o mesmo é usado para o fim devido.