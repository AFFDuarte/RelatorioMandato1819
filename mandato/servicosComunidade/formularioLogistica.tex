% ==========================
% # Formulário de Logística  #
% ==========================

\subsection{Formulário de Logística}

Um dos problemas mais antigos do Departamento são as várias falhas logísticas, desde cadeiras partidas, lâmpadas fundidas, etc. Estes são problemas de simples resolução mas que demoram muito tempo a ser solucionados quer pela dimensão exagerada do edifício, quer pela ineficácia da equipa da manutenção do \acrshort{deec}. Assim, já em tempos se tinha implementado algumas medidas para resolver este tópico nomeadamente com um formulário mensal de logística, medida que foi implementada no ano de João Freitas como Vice-Presidente do \acrshort{neeec} e que acabou por cair logo no mandato seguinte.

Assim, após a reformulação dos espaços de estudo, lançámos um formulário de logística que está disponível no nosso site e publicitado em todos os espaços de estudo do \acrshort{deec} onde os alunos podem, facilmente, lançar uma queixa logística indicando o problema e a localização do mesmo. Estas queixas transformam-se, automaticamente, num email enviado pelo sistema ao \acrshort{neeec} e depois o email pode ser reencaminhado para as pessoas competentes (por regra ou para a manutenção ou para o GRI). É de realçar que o ideal seriam as queixas serem redirecionadas automaticamente, principalmente para que esta medida não caia em esquecimento por parte do \acrshort{neeec}, mas tal não pode acontecer pois de vez em quando existem algumas "não queixas", embora seja de salientar que estas são uma percentagem extremamente diminuta, tendo-se verificado apenas uma este ano.

Esta nova medida provou-se ser um verdadeiro sucesso tendo-se resolvido vários problemas principalmente no que toca a lâmpadas fundidas e a falhas de internet de forma bastante célere. Por sua vez, pretende-se agora implementar um sistema no nosso site que indique as queixas que já foram feitas e qual o seu estado de resolução. Desta forma, pretende-se dar a conhecer aos estudantes que, de facto, é compensador preencher este formulário bem como pressionar o \acrshort{neeec} para não deixar cair esta medida em esquecimento.