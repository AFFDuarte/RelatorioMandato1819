% ==========================
% # Organogramas Piso 2  #
% ==========================

\subsection{Organogramas do Piso 2} \label{subsec:organogramasPiso2}

Ao longo do mandato surgiu a ideia de, através dos placares de cortiça do Piso 2, dar a conhecer à comunidade quais os membros da equipa do Núcleo. Desta forma, entrou-se em contacto com a Direção do Departamento que rapidamente concordou com a ideia e decidiu alargá-la também ao próprio Departamento bem como às restantes associações estudantis.

Para levar esta ação a cabo, o Pelouro da Imagem encarregou-se de criar um organograma para os membros do Departamento bem como para a área pedagógica (Coordenador de curso, representante dos estudantes e delegados de ano). Para os organogramas do \acrshort{neeec} foi inserido o organograma da lista, feito aquando das eleições. A Direção do \acrshort{deec} pretendia também obter os organogramas das restantes associações estudantis e pediu ao \acrshort{neeec} que tratasse desse assunto, contudo, achámos por bem apenas notificar as outras associações estudantis disso uma vez que não era nossa responsabilidade fazer organogramas também para eles (até à conclusão deste relatório, o BEST e o CR responderam dizendo que iriam fazer os organogramas enquanto que o IEEE \acrshort{uc} Student Branch e o Clube de Programação não deram qualquer tipo de resposta).

Ao serem colocados os organogramas nos placares de cortiça, na presença do Diretor do Departamento, foi feita uma reformulação total na organização de todos os placares da zona da Secretaria. Desta forma, cada placar passou a estar destinado a um assunto em específico (editais, empregos, bolsas, calendários de avaliações, organogramas, entre outros). Adicionalmente, um dos placares ficou também destinado aos editais das associações estudantis sendo, assim, possível colocar os vários comunicados, convocatórias e editais emitidos pelo \acrshort{neeec}, nomeadamente pela Mesa do Plenário.