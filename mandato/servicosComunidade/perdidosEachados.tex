% ==========================
% # Perdidos e Achados     #
% ==========================

\subsection{Perdidos e Achados}

Desde sempre que a grande parte dos alunos que acham objetos perdidos os levam ao \acrshort{neeec} ou à Secretaria do \acrshort{deec}.

Em temos, tinha sido divulgado devidamente que o Núcleo tinha Perdidos e Achados, mas essa divulgação perdeu-se com o tempo e muito menos tinha sido implementado um sistema para controlo dos objetos.

Este ano apostou-se muito na parte da divulgação, anunciando nas redes sociais e por cartazes a existência dos P\&A. Os objetos eram guardados numa caixa, criada este ano para o efeito, dentro de um armário, na sala do Núcleo.

Quando a caixa estava muito cheia, ou de tempos a tempos, era publicado no \acrshort{mieec}/\acrshort{uc}, no Facebook, os objetos que estavam lá, com uma breve descrição dos mesmos, para que não fossem parar às mãos erradas.

No último mês de abril notou-se um choque maior entre os perdidos e achados do Núcleo e os da Secretaria. Discutiu-se em reunião de Direção, se seria vantajoso uniformizar o sistema: ou tudo ser entregue no Núcleo ou tudo ser entregue na Secretaria. Não se chegou a um consenso, por isso manteve-se o sistema como está agora.

Por sugestão do Diretor do Departamento, o Gabinete de Rede Informática está agora a implementar uma página online onde quer o \acrshort{neeec} quer a Secretaria poderão informar que receberam um dado item, encontrado num dado local a uma dada data e hora por alguém, e que este se encontra disponível para levantamento na Secretaria ou no \acrshort{neeec}. Quando o \acrshort{neeec} levar o item para a Secretaria, bastará selecionar uma opção para mudança do local. Esta informação estará disponível num site (que deverá ser o My \acrshort{deec} e também o site do \acrshort{neeec}) e, futuramente, poderá também passar na televisão.