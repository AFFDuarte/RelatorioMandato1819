% ==========================
% # Site                   #
% ==========================

\subsection{Site}

Uma das apostas do nosso mandato foi impulsionar a divulgação do Núcleo junto da comunidade do \acrshort{deec}, sendo uma lacuna grave nesse campo a falta do site do Núcleo.

Este site já tinha existido em tempos, pelo menos até ao mandato da Elisabete Santos (2014/2015). Mais ou menos nesta altura, foi proposta uma remodelação do site para o tornar mais interativo e mais funcional, o que começou a quebrar partes do site até este se tornar impraticável e ter sido desativado. A remodelação começou, pelo menos, no mandato do Filipe Cavaleiro (2015/2016), onde o Eduardo Preto procurou realizar um tema em Wordpress especificamente para o Núcleo, pelo que, no final desse mandato, faltava, essencialmente, começar a introduzir conteúdo no site. Essa tarefa recaiu para o mandato seguinte, nomeadamente sobre o Daniel Chichorro, que não possuía os mesmos conhecimentos técnicos que o Preto para realizar sites, nem os construir de raiz. Pegando no tema que o Preto tinha construído, o Chichorro teve sérias dificuldades em introduzir o conteúdo no site, dado que o tema não era compatível com imensos plugins e construtores automáticos de páginas no Wordpress, obrigando a um trabalho longo (nomeadamente de programação PHP, conhecimento raro no nosso curso) por cada página que se pretendia inserida no site, pelo que ele nos sugeriu que construíssemos o site de raiz, utilizando temas disponibilizados online que permitem editar facilmente as páginas, praticamente sem conhecimentos de programação em Web.

Desta forma, quando tomámos posse, procurámos resolver o problema do site que já se arrastava há demasiado tempo de uma vez por todas e, pegando na sugestão do Chichorro, reconstruímos o site com um novo tema. Dividindo as tarefas entre a nova Direção, procurámos lançar o site renovado e com o máximo de informação pertinente possível, nomeadamente, para os caloiros, que seriam preferencialmente reencaminhados para o site para obter essas informações. Após alguns atrasos, a base do site ficou lançada a tempo dos caloiros, faltando, contudo, muita da informação que desejaríamos ter. Após vários atrasos, forçámos o lançamento oficial do site no final do primeiro semestre de aulas, tendo o contributo do Miguel Santos ajudado imenso a personalizar vários aspetos do site e acelerar o processo de lançamento do mesmo.

A nossa intenção era que o site não servisse apenas como um reservatório de informação relativamente estática, mas servisse sim de porta de caminho a uma interação mais dinâmica com todo o Núcleo, pelo que introduzimos uma série de funcionalidades que permitem isso mesmo, desde o calendário de atividades (com possibilidade de subscrição) que procurámos deixar o mais atualizado possível, à inscrição em todas as atividades do Núcleo que passaram a ser feitas exclusivamente por um formulário no site, às informações pedagógicas do curso (como calendários de avaliações, resultados de inquéritos, etc.), a documentos resultantes das Reuniões Gerais de Alunos, entre outros.

Consideramos que, atualmente, o site atualmente está fácil o suficiente de ser gerido por qualquer pessoa, requerendo pouco trabalho em aprender como trabalhar com ele.

Quanto a melhorias futuras, consideramos que os textos atualmente no site têm ainda muito por evoluir, principalmente a nível de conteúdos para caloiros, que muitas vezes chegam a Coimbra sem conhecerem nada da cidade e das suas tradições (por exemplo, a nível da Queima das Fitas, as atividades mais tradicionais que não a Serenata e as Noites do Parque tendem a ser muito confusas para os caloiros, podendo ajudar a esclarecer muito). Seria também interessante uma reorganização dos conteúdos e dos menus do site para facilitar a navegação do mesmo, sendo também interessante a introdução de uma ferramenta de pesquisa no site.

\subsubsection{Domínio próprio} \label{subsubsec:dominio}

O site do \acrshort{neeec} foi inicialmente feito com a ajuda do \acrshort{gri}, que o alojou nos servidores do \acrshort{deec} e criou um subdomínio próprio para este site (neeecaac.deec.uc.pt). Contudo, dado que queríamos reformular o site, achámos por bem reformular também o seu domínio, que era bastante extenso e, consequentemente, difícil de divulgar. Além disso, outros subsites, como era o caso do site da Gala Ohms D'Ouro, tornavam-se ridiculamente extensos (ohmsdouro.neeecaac.deec.uc.pt), pelo que considerámos essencial reformular este aspeto do site. Inicialmente, ponderámos associarmo-nos ao site da \acrshort{aac} (academica.pt), criando algo como neeec.academica.pt, significativamente mais simples e que permitiria melhorar a coesão da imagem de toda a academia, algo com que nos debatemos internamente com a gestão dos pelouros, pelo que faria também sentido em toda a \acrshort{aac}. Contudo, os embaixadores da Rede de Divulgação no mandato anterior, Presidente e Tesoureiro deste mandato, foram informados, aquando da formação para esta rede de embaixadores, erroneamente, que seria preciso falar com a Secção de Informática da \acrshort{aac}, composta por muita pouca gente, cujo membro mais ativo estaria bastante sobrecarregado, podendo atrasar todo o processo, que queríamos resolver rapidamente. Ao verificar que os preços cobrados pelo domínio neeec.pt seriam bastante reduzidos, optámos então por ter um domínio próprio. Contudo, achamos que a batalha por uma imagem coesa de toda a \acrshort{aac}, apesar de difícil, seria benéfica para todos e é apoiada por alguns membros da \acrshort{dg}, nomeadamente, a Mariana Gaspar, ex-Coordenadora da Comunicação e Imagem da \acrshort{dg} e atual Vice-Presidente, pelo que, mais tarde ou mais cedo, deverá acontecer, pelo que poderão ter que ser feitos os ajustes necessários para tal. Apesar disso, como o alojamento do site poderá continuar a ser local, isto é, no \acrshort{deec}, a sua gestão será bastante simplificada, dado o bom funcionamento que o \acrshort{gri} tem apresentado.

O domínio foi comprado ao dominios.pt, que, no primeiro ano, oferecia ainda um serviço de armazenamento do site, que seria posteriormente pago se não fosse cancelado. Nesta altura, a fatura do pagamento do domínio foi dirigida para o email do \acrshort{nei}, dado que estes já eram clientes da empresa e o \acrshort{nif} era o mesmo, contudo, ao renovarmos o site este ano as faturas foram dirigidas corretamente para o nosso email, pelo que este problema não se deve voltar a levantar. Um problema que surgiu na renovação do domínio no final do mandato foi que o serviço de armazenamento e o domínio foram faturados separadamente e o primeiro pagamento realizado, por lapso, foi destinado a pagar o armazenamento que já não pretendíamos, contudo, como o pagamento já tinha sido realizado já não tínhamos outra opção senão manter o serviço, ficando o aviso para evitar novamente este problema na próxima renovação.

Recomenda-se que no futuro seja criado um manual de instruções sobre como gerir o site (incluindo os pormenores técnicos mais avançados) para permitir que o mesmo possa continuar a ser utilizado e melhorado no futuro.

\subsubsection{Calendário de Atividades}

Com o lançamento do site, foi criada uma página onde se encontra o calendário do Núcleo. Este calendário está associado diretamente ao Google Calendar da conta master do Núcleo. Neste calendário estão dispostas todas as atividades do Núcleo tendo o mesmo sido preenchido no início de cada semestre com o planeamento que foi feito e alterado sempre que houve alguma modificação. Este é um trabalho que passou sempre pelo Secretário do \acrshort{neeec} exigindo alguma proatividade do mesmo, contudo, o seu trabalho foi prejudicado pelo facto de alguns CGs decidirem alterar a data das atividades sem qualquer aviso. A existência deste calendário que, de forma pública, manteve atualizada a agenda do \acrshort{neeec} facilitou bastante o cumprimento dos protocolos estipulados, uma vez que bastou informar as entidades que todo o planeamento feito para o semestre seguinte já se encontrava inserido no calendário. Por sua vez, verificou-se que o calendário tem sido consultado pelas pessoas uma vez que quando houve um erro no mesmo, algumas pessoas nos questionaram sobre o mesmo. Este calendário pode também ser subscrito por quem quiser, tendo assim no seu calendário pessoal todas as atividades do \acrshort{neeec}. Este é um serviço muito fácil de gerir que achamos essencial para a boa imagem do \acrshort{neeec} e também para uma organização interna pelo que recomendamos vivamente a sua continuidade.