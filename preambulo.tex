% ========================
% # Introdução           #
% ========================

\section{Introdução}

%O \acrfull{neeec} é a estrutura representativa de todos os estudantes dos cursos lecionados no \acrfull{deec} representando cerca de sete centenas de estudantes. O seu trabalho diário passa pela representação dos seus associados, na área da Pedagogia e das Saídas Profissionais setoriais. Somos como que uma voz dos estudantes perante todas as organizações externas bem como perante o Departamento, a faculdade e a Associação Académica de Coimbra.

%Com consciência do legado que nos era transmitido e do trabalho das gerações passadas em nos deixarem uma instituição livre de problemas financeiros, iniciámos um projeto que tinha como mote \textit{Ligar Electro}. Pretendíamos ligar os estudantes, os professores e os funcionários ao Núcleo bem como ligar esta associação a todos aqueles que a rodeiam e que são essenciais para o seu trabalho e crescimento.

%Passando por uma ligação mais forte e contínua às empresas, uma representação sempre presente junto da \acrfull{aac}, um trabalho conjunto fortíssimo com a Direção do Departamento e uma maior proximidade de trabalho junto dos vários órgãos sociais da Faculdade, este foi um mandato pautado pela reestruturação na relação com todas estas entidades. Destacamos, principalmente, o trabalho feito com o \acrlong{cr} que permitiu levar o nome do nosso curso mais longe, divulgando-o e unindo sempre esforços para batalhar pelas causas necessárias, sempre em conjunto.

%Simultaneamente, este foi um mandato pautado pela reestruturação e organização interna do \acrshort{neeec}. A criação de várias diretrizes internas, necessárias para o bom funcionamento do Núcleo, bem como a mudança de paradigma nos métodos laborais do mesmo permitiu criar uma equipa mais profissional e apresentar ao público um trabalho mais eficaz, proveitoso e constantemente valorizado. O trabalho conjunto da Tesouraria e da Administração do Núcleo permitiram um crescimento financeiro exponencial e um aumento enorme do ativo do Núcleo,  quer através dos protocolos celebrados, quer através da aquisição direta de bens. Desta forma, o Núcleo detém agora uma independência muito grande no seu trabalho diário bem como elevadas condições em diversos locais do Departamento, geridos pelo \acrshort{neeec}, trazendo, assim, melhores condições aos estudantes.

%Fomos anfitriões do maior evento estudantil da nossa área, o \acrlong{ene3}; reformulámos toda a imagem e estrutura do Bot Olympics, aumentando o número de participantes, dentro das nossas possibilidades, e escalando a imagem do evento com a realização da final num dos centros comerciais mais conhecidos de Coimbra; fizemos a segunda edição da \acrlong{f3e}, uma edição de consolidação que deixou mais de uma dezena de estudantes do nosso curso com emprego assegurado após o final deste ano letivo; criámos um novo conceito para a Semana dos Ramos compilando em dois dias um maior número de atividades do que aquele que tinha sido feito em qualquer das edições anteriores garantido, quase sempre, lotação esgotada nas mesmas; e, já perto do final do mandato, celebrámos num ambiente diferente junto da nossa família, a família de Eletro, os 20 anos do \acrshort{neeec}, data essa que nos permitiu ter uma nova visão do Núcleo por nos ter dado a conhecer tanto da história destas duas últimas décadas.

%Para além disto, realizámos, ao longo do ano, mais de uma centena de atividades distribuídas pelos diversos pelouros, fruto de um trabalho árduo de toda a equipa de Coordenadores e Colaboradores que não deitaram a toalha ao chão e permitiram preencher todas as semanas, de ambos os semestres, com atividades de diferentes âmbitos.

%De realçar, também, que fomos o ano da revisão de regulamentos, processo iniciado com a conclusão da revisão dos Estatutos da Associação Académica de Coimbra. Desta forma, a Mesa do Plenário do \acrshort{neeec} pautou o seu trabalho por um elevado rigor no cumprimento das normas estatutárias e na revisão de regulamentos do Núcleo com especial detalhe para que este passe a ser, cada vez mais, uma instituição marcada pelo rigor e transparência no seu trabalho.

%Desta forma, resta-nos agradecer, em nome do Núcleo, aos nossos Colaboradores, aos funcionários do Departamento, que colaboraram sempre connosco naquilo que podiam, à Direção do Departamento, muito em particular ao Professor Doutor Humberto Jorge, pelo trabalho conjunto desenvolvido, aos restantes núcleos da \acrshort{aac}, nomeadamente aos do Polo 2, aos restantes núcleos da área de Engenharia Eletrotécnica do resto do país, às nossas empresas associadas, mas, acima de tudo, aos nossos sócios, os estudantes de Engenharia Eletrotécnica e de Computadores por continuarem a provar que todas as nossas gotas de suor valem a pena.

%A equipa do \acrshort{neeec} 2017/2018,\\
%\vspace{200pt}\\
%\begin{center}
%\textit{“O sucesso nasce do querer, da determinação e persistência em chegar a um objetivo. Mesmo não atingindo o alvo, quem busca e vence obstáculos, no mínimo, fará coisas admiráveis.”}
%\end{center}
